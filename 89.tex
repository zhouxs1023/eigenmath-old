\parindent=0pt
\magnification=1200

\beginsection{Solution to Mandl and Shaw Problem 3.1}

Given the field
%
$$
\phi(x)=\sum_{\bf k}
\left(
{1\over2V\omega({\bf k})}
\right)^{1/2}
a({\bf k})
\exp(-ikx)
+
\sum_{\bf k}
\left(
{1\over2V\omega({\bf k})}
\right)^{1/2}
a^\dagger({\bf k})
\exp(ikx)
$$
%
and the equal-time field commutator equation
%
$$
[\phi({\bf x},t),\dot\phi({\bf x^\prime},t)]=i\delta({\bf x}-{\bf x^\prime})
$$
%
derive the following operator commutator equation
%
$$
[a({\bf k})a^\dagger({\bf k^\prime})]=\delta_{\bf kk^\prime}
$$
%
where
%
$$
\delta_{\bf kk^\prime}=\cases{
1,&${\bf k}={\bf k^\prime}$\cr
0,&${\bf k}\ne{\bf k^\prime}$\cr
}
$$
%
{\bf Step 1.}
The trick is to do two Fourier transforms of the field commutator equation.
We start by working on the right-hand side.
To do the Fourier transform of a delta function we use the
``sifting property,''
$$
\int\delta(t)f(t)\,dt=f(0)
$$
%
Since we have $\delta({\bf x}-{\bf x^\prime})$ we use the change of variable technique.
%
$$\eqalign{
\int i\delta({\bf x}-{\bf x^\prime})\exp(-i{\bf k}\cdot{\bf x})\,d{\bf x}&=
i\int\delta({\bf y})\exp(-i{\bf k}\cdot({\bf y}+{\bf x^\prime}))
\,d{\bf y},\quad{\bf x}={\bf y}+{\bf x^\prime}\cr
&=i\exp(-i{\bf k}\cdot({\bf y}+{\bf x^\prime}))\big|_{{\bf y}=0}\cr
&=i\exp(-i{\bf k}\cdot{\bf x^\prime})\cr
}$$
%
{\bf Step 2.}
Do a second Fourier transform.
%
$$\eqalign{
\int i\exp(-i{\bf k}\cdot{\bf x^\prime})\exp(i{\bf k^\prime}\cdot{\bf x^\prime})
&=i\int\exp(-i({\bf k}-{\bf k^\prime})\cdot{\bf x^\prime})\,d{\bf x^\prime}\cr
&=iV\delta_{\bf kk^\prime}
}$$
%
This result is due to the fact that when ${\bf k}={\bf k^\prime}$ the exponential
disappears and the integral becomes $\int d{\bf x^\prime}=V$.
When ${\bf k}\ne{\bf k^\prime}$ the integral is zero so we need to include a delta
function in the result.
We have now completed two Fourier transforms of the right-hand side of the field
commutator equation.
Overall we have
%
$$
\int\int i\delta({\bf x}-{\bf x^\prime})
\exp(-i{\bf k}\cdot{\bf x})
\exp(i{\bf k^\prime}\cdot{\bf x^\prime})\,d{\bf x}\,d{\bf x^\prime}
=iV\delta_{\bf kk^\prime}
$$
%
We now turn to the left-hand side of the field commutator equation.
%
\vfill
\break
%
{\bf Step 3.}
Write down $\phi$ and $\dot\phi=\partial\phi/\partial t$.
%
$$\phi({\bf x},t)=
\overbrace{
\sum_{\bf k}
{
a({\bf k})\exp(-i\omega({\bf k})t+i{\bf k}\cdot{\bf x})
\over
\sqrt{2V\omega({\bf k})}
}
}^A
+
\overbrace{
\sum_{\bf k}
{
a^\dagger({\bf k})\exp(i\omega({\bf k})t-i{\bf k}\cdot{\bf x})
\over
\sqrt{2V\omega({\bf k})}
}
}^B
$$
%
$$\dot\phi({\bf x^\prime},t)=
\overbrace{
-i
\sum_{\bf k^\prime}
{
a({\bf k^\prime})\exp(-i\omega({\bf k^\prime})t
+i{\bf k^\prime}\cdot{\bf x^\prime})
\over
\sqrt{2V/\omega({\bf k^\prime})}
}
}^C
+
\overbrace{
i
\sum_{\bf k^\prime}
{
a^\dagger({\bf k^\prime})\exp(i\omega({\bf k^\prime})t
-i{\bf k^\prime}\cdot{\bf x^\prime})
\over
\sqrt{2V/\omega({\bf k^\prime})}
}
}^D
$$
%
{\bf Step 4.}
Expand the product $\phi\dot\phi$.
%
$$\eqalignno{
\phi\dot\phi=
-&i\sum_{{\bf k},{\bf k^\prime}}
{
a({\bf k})a({\bf k^\prime})
\exp(-i\omega({\bf k})t+i{\bf k}\cdot{\bf x}
-i\omega({\bf k^\prime})t+i{\bf k^\prime}\cdot{\bf x^\prime})
\over
2V\sqrt{\omega({\bf k})/\omega({\bf k^\prime})}
}
&AC
\cr
+&i\sum_{{\bf k},{\bf k^\prime}}
{
a({\bf k})a^\dagger({\bf k^\prime})
\exp(-i\omega({\bf k})t+i{\bf k}\cdot{\bf x}
+i\omega({\bf k^\prime})t-i{\bf k^\prime}\cdot{\bf x^\prime})
\over
2V\sqrt{\omega({\bf k})/\omega({\bf k^\prime})}
}
&AD
\cr
-&i\sum_{{\bf k},{\bf k^\prime}}
{
a^\dagger({\bf k})a({\bf k^\prime})
\exp(i\omega({\bf k})t-i{\bf k}\cdot{\bf x}
-i\omega({\bf k^\prime})t+i{\bf k^\prime}\cdot{\bf x^\prime})
\over
2V\sqrt{\omega({\bf k})/\omega({\bf k^\prime})}
}
&BC
\cr
+&i\sum_{{\bf k},{\bf k^\prime}}
{
a^\dagger({\bf k})a^\dagger({\bf k^\prime})
\exp(i\omega({\bf k})t-i{\bf k}\cdot{\bf x}
+i\omega({\bf k^\prime})t-i{\bf k^\prime}\cdot{\bf x^\prime})
\over
2V\sqrt{\omega({\bf k})/\omega({\bf k^\prime})}
}
&BD
\cr
}$$
%
{\bf Step 5.}
Expand the product $\dot\phi\phi$.
It is the same as $\phi\dot\phi$ except the order of the operators is swapped.
%
$$\eqalignno{
\dot\phi\phi=
-&i\sum_{{\bf k},{\bf k^\prime}}
{
a({\bf k^\prime})a({\bf k})
\exp(-i\omega({\bf k})t+i{\bf k}\cdot{\bf x}
-i\omega({\bf k^\prime})t+i{\bf k^\prime}\cdot{\bf x^\prime})
\over
2V\sqrt{\omega({\bf k})/\omega({\bf k^\prime})}
}
&CA
\cr
+&i\sum_{{\bf k},{\bf k^\prime}}
{
a^\dagger({\bf k^\prime})a({\bf k})
\exp(-i\omega({\bf k})t+i{\bf k}\cdot{\bf x}
+i\omega({\bf k^\prime})t-i{\bf k^\prime}\cdot{\bf x^\prime})
\over
2V\sqrt{\omega({\bf k})/\omega({\bf k^\prime})}
}
&DA
\cr
-&i\sum_{{\bf k},{\bf k^\prime}}
{
a({\bf k^\prime})a^\dagger({\bf k})
\exp(i\omega({\bf k})t-i{\bf k}\cdot{\bf x}
-i\omega({\bf k^\prime})t+i{\bf k^\prime}\cdot{\bf x^\prime})
\over
2V\sqrt{\omega({\bf k})/\omega({\bf k^\prime})}
}
&CB
\cr
+&i\sum_{{\bf k},{\bf k^\prime}}
{
a^\dagger({\bf k^\prime})a^\dagger({\bf k})
\exp(i\omega({\bf k})t-i{\bf k}\cdot{\bf x}
+i\omega({\bf k^\prime})t-i{\bf k^\prime}\cdot{\bf x^\prime})
\over
2V\sqrt{\omega({\bf k})/\omega({\bf k^\prime})}
}
&DB
\cr
}$$
\vfill
\break
{\bf Step 6.}
Do two Fourier transforms of $[\phi,\dot\phi]$.
%
$$\eqalignno{
&\int\int(\phi\dot\phi-\dot\phi\phi)
\exp(-i{\bf k}\cdot{\bf x})
\exp(i{\bf k^\prime}\cdot{\bf x^\prime})
\,d{\bf x}\,d{\bf x^\prime}=\cr
\cr
&\quad-X a({\bf k})a(-{\bf k^\prime})
\exp(-i\omega({\bf k})t-i\omega({\bf k^\prime})t)			&AC\cr
&\quad+X a({\bf k})a^\dagger({\bf k^\prime})
\exp(-i\omega({\bf k})t+i\omega({\bf k^\prime})t)			&AD\cr
&\quad-X a^\dagger(-{\bf k})a(-{\bf k^\prime})
\exp(i\omega({\bf k})t-i\omega({\bf k^\prime})t)			&BC\cr
&\quad+X a^\dagger(-{\bf k})a^\dagger({\bf k^\prime})
\exp(i\omega({\bf k})t+i\omega({\bf k^\prime})t)			&BD\cr
\cr
&\quad+X a(-{\bf k^\prime})a({\bf k})
\exp(-i\omega({\bf k})t-i\omega({\bf k^\prime})t)			&-CA\cr
&\quad-X a^\dagger({\bf k^\prime})a({\bf k})
\exp(-i\omega({\bf k})t+i\omega({\bf k^\prime})t)			&-DA\cr
&\quad+X a(-{\bf k^\prime})a^\dagger(-{\bf k})
\exp(i\omega({\bf k})t-i\omega({\bf k^\prime})t)			&-CB\cr
&\quad-X a^\dagger({\bf k^\prime})a^\dagger(-{\bf k})
\exp(i\omega({\bf k})t+i\omega({\bf k^\prime})t)			&-DB\cr
}$$
%
$$
X={iV\over2\sqrt{\omega({\bf k})/\omega({\bf k^\prime})}}
$$
%
{\bf Step 7.}
Rewrite as commutators and set equal to the right-hand side of the field commutator
equation.
%
$$\eqalignno{
iV\delta_{\bf kk^\prime}=
-&X
[a({\bf k}),a(-{\bf k^\prime})]
\exp(-i\omega({\bf k})t-i\omega({\bf k^\prime})t)	&(1)
\cr
+&X
[a({\bf k}),a^\dagger({\bf k^\prime})]
\exp(-i\omega({\bf k})t+i\omega({\bf k^\prime})t)	&(2)
\cr
-&X
[a^\dagger(-{\bf k})a(-{\bf k^\prime})]
\exp(i\omega({\bf k})t-i\omega({\bf k^\prime})t)	&(3)
\cr
+&
X [a^\dagger(-{\bf k})a^\dagger({\bf k^\prime})]
\exp(i\omega({\bf k})t+i\omega({\bf k^\prime})t)	&(4)
\cr
}$$
%
{\bf Step 8.}
Since all of this is equal to $iV\delta_{\bf kk^\prime}$ we can take
${\bf k}={\bf k'}$.
This eliminates the exponentials in (2) and (3) and
eliminates (1) and (4) entirely (${\bf k}\ne-{\bf k}$).
It also eliminates the
$\omega({\bf k})/\omega({\bf k^\prime})$ term. We have
%
$$
{iV\over2}[a({\bf k}),a^\dagger({\bf k^\prime})]
-{iV\over2}[a^\dagger(-{\bf k})a(-{\bf k^\prime})]
=iV\delta_{\bf kk^\prime}
$$
%
Cancel $i$ and $V$.
%
$$
{1\over2}[a({\bf k}),a^\dagger({\bf k^\prime})]
-{1\over2}[a^\dagger(-{\bf k})a(-{\bf k^\prime})]
=\delta_{\bf kk^\prime}
$$
%
The second commutator can be transposed if the sign is reversed.
%
$$
{1\over2}[a({\bf k}),a^\dagger({\bf k^\prime})]
+{1\over2}[a(-{\bf k})a^\dagger(-{\bf k^\prime})]
=\delta_{\bf kk^\prime}
$$
%
We can sum the two commutators because ${\bf k}={\bf k^\prime}$ implies
$-{\bf k}=-{\bf k^\prime}$.
%
$$
[a({\bf k}),a^\dagger({\bf k^\prime})]=\delta_{\bf kk^\prime}
$$

\vfill
\break

\beginsection{Notes}

In Step 6, where did the $-\bf k$'s come from?
Remember that the sum over $\bf k$ includes both positive and negative $\bf k$.
What is happening is that the Fourier transform is keeping the terms where the
exponential cancels out, that is, where the integral reduces to a constant
times $\int d{\bf x}\,d{\bf x^\prime}$.
In some cases $-\bf k$ causes the exponential to cancel out.

\vfill
\break

\beginsection{Demo}

In Eigenmath we can use a tensor to represent a Klein-Gordon field.
The field consists of operators and each operator is multiplied by a coefficient.
We can take the operators to be tensor indices in which case all we need are
the coefficients to represent the field.
For example, the field
$$
\phi(x)=\sum_k C(k)a(k)+\sum_k D(k)a^\dagger(k)
$$
can be written in tensor notation as
$$
\phi(x)=
\bordermatrix{
&&k=-2&k=-1&k=0&k=1&k=2\cr
a&\ldots&C(-2)&C(-1)&C(0)&C(1)&C(2)&\ldots\cr
a^\dagger&\ldots&D(-2)&D(-1)&D(0)&D(1)&D(2)&\ldots\cr
}
$$
In this simple example we used a scalar $k$ but what we really need is a vector $\bf k$.
This promotes the tensor to rank-4.
We need one index to select $a$ or $a^\dagger$ and three indices to select the mode,
$n_1,n_2,n_3$. Also, the Eigenmath index numbering starts at 1 so we have to bias the
index as {\tt B = 2 N + 1} to accommodate an index range of {\tt -N} to {\tt +N}.
$$
\matrix{
&a \hbox{\ or\ } a^\dagger&n_1&n_2&n_3\cr
&\downarrow&\downarrow&\downarrow&\downarrow\cr
\hbox{\tt T[}&\hbox{\tt 1..2,}&\hbox{\tt 1..B,}
&\hbox{\tt 1..B,}&\hbox{\tt 1..B}&\hbox{\tt ]}\cr
}$$
The product of two fields such as $\phi\dot\phi$ is computed by using the outer product
which yields the following tensor indices.
$$
\hbox{\tt T[}\overbrace{\tt2,B,B,B,}^\phi\overbrace{\tt2,B,B,B}^{\dot\phi}\hbox{\tt]}
$$
Next we compute $\phi\dot\phi-\dot\phi\phi$ which produces a tensor of the same rank.
Since we only want terms where ${\bf k}={\bf k^\prime}$ we contract across pairs of
{\tt B} to get
$$\hbox{\tt T[2,2]}$$
We then do a dual Fourier transform that produces the final result
$$T=\left(\matrix{0&iV\cr -iV&0}\right)$$
The result is more meaningful if we add index labels.
$$T=\bordermatrix{
&a&a^\dagger\cr
a&0&iV\cr
a^\dagger&-iV&0\cr
}$$
So what we have is
$$T=iV(aa^\dagger-a^\dagger a)$$
which corresponds to the left-hand side of our result in Step 8.

\vfill
\break

\tt\obeylines

clear()
N = 2
x = (x1,x2,x3)
xp = (xp1,xp2,xp3)
phi = zero(2,2N+1,2N+1,2N+1)
for(n1, -N, N, for(n2, -N, N, for(n3, -N, N, do(
\ k = 2 pi / L (n1,n2,n3),
\ omega = sqrt(mu\char94 2 + k\char94 2),
\ C = 1 / sqrt(2 V omega),
\ phi[1,n1+N+1,n2+N+1,n3+N+1] = C exp(-i omega t + i dot(k,x)),
\ phi[2,n1+N+1,n2+N+1,n3+N+1] = C exp(i omega t - i dot(k,x))
))))
phidot = d(phi, t)
phidot = subst(xp1,x1,phidot)
phidot = subst(xp2,x2,phidot)
phidot = subst(xp3,x3,phidot)
phidot = eval(phidot)
T = outer(phi,phidot) - outer(phidot,phi)
T = contract(T,2,6)
T = contract(T,2,5)
T = contract(T,2,4)
\# Poor man's dual Fourier transform for an arbitrary k.
k = 2 pi / L (-1,1,2)
T = T V\char94 2 exp(-i dot(k,x)) exp(i dot(k,xp))
T = filter(T,x1,x2,x3)
T = filter(T,xp1,xp2,xp3)
display(T)

\end