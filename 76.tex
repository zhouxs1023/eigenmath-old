\parindent=0pt

Abstract Algebra

\bigskip
{\bf Problem A.} Prove the following corollary, using the lemma:
There is no integer between 0 and 1.
\medskip
{\bf Corollary.} Let $a\in Z$. Then there is no integer
between $a$ and $a+1$.

\bigskip
{\bf Proof.} Let $a\in Z$.
Suppose there is a $b\in Z$ such that $a<b<a+1$.
Then by Lemma 2.11 (Unit 7) we can add $-a$ to obtain
$0<b-a<1$.
We conclude that $b-a$ is an integer by closure.
However, by the lemma ``there is no integer between 0 and 1,''
we also conclude that $b-a$ does not exist.
This is a contradiction, therefore the premise must be false.
Hence there is no integer between $a$ and $a+1$

\bigskip
This proof uses the corollary ``1 is the smallest element of $N$.''
\medskip
{\bf Proof.} Let $a\in Z$.
Suppose there is a $b\in Z$ such that $a<b$ and $b<a+1$.
By the first less-than we have $b-a\in N$.
Adding $-a$ to the second less-than yields $b-a<1$.
Then by Corollary 1.2, ``1 is the smallest element of $N$,''
we have $b-a\not\in N$.
This is a contradiction, therefore the premise must be false.
Hence there is no integer between $a$ and $a+1$

\bigskip
\bigskip
{\bf Problem B.} Prove that
$$2+4+\cdots+2n=n(n+1)$$
for any $n\in N$.

\bigskip
{\bf Proof.}
(i) Substitute 1 for $n$ to obtain $2=1(1+1)$ hence the equality is
true for $n=1$.
(ii)
Assume that the equality holds for a natural number $n$.
Does it also hold for $n+1$?
For $n+1$ we have
$$\eqalign{
2+4+\cdots+2n+2(n+1)&=(n+1)((n+1)+1)\cr
&=(n+1)(n+2)\cr
&=n^2+3n+2\cr
}$$
Subtract $2n+2$ from both sides of the equation.
$$\eqalign{
2+4+\cdots+2n&=n^2+3n+2-2n-2\cr
&=n^2+n\cr
&=n(n+1)\cr
}$$
This result is the original equality.
Therefore when the equality is true for $n$ it is also true for $n+1$.
Hence by induction the equality is true for all $n\in N$.


\end
