\noindent
Let us begin by considering the following passage from Vladimir Nabokov's
autobiography {\it Speak, Memory.}
\begin{quote}
A foolish tutor had explained logarithms to me much too early, and I had
read (in a British publication, the {\it Boy's Own Paper}, I believe)
about a certain Hindu calculator who in exactly two seconds could find the
seventeenth root of, say,
3529471145 760275132301897342055866171392
(I am not sure I have got this right; anyway the root was 212).
\end{quote}
We can check Nabokov's arithmetic by typing the following into Eigenmath.

\medskip
\verb$212^17$

\medskip
\noindent
After pressing the return key, Eigenmath displays the following result.
$$3529471145760275132301897342055866171392$$
So Nabokov did get it right after all.
We can enter {\it float} or click on the float button to scale the number
down to size.

\medskip
\verb$float$

$$3.52947\times10^{39}$$

