\section{Introduction}

\noindent
The following is an excerpt from Vladimir Nabokov's
autobiography {\it Speak, Memory.}
\begin{quote}
A foolish tutor had explained logarithms to me much too early, and I had
read (in a British publication, the {\it Boy's Own Paper}, I believe)
about a certain Hindu calculator who in exactly two seconds could find the
seventeenth root of, say,
3529471145 760275132301897342055866171392
(I am not sure I have got this right; anyway the root was 212).
\end{quote}
We can check Nabokov's arithmetic by typing the following into Eigenmath.

\medskip
\verb$212^17$

\medskip
\noindent
After pressing the return key, Eigenmath displays the following result.
$$3529471145760275132301897342055866171392$$
So Nabokov did get it right after all.
We can enter {\it float} or click on the float button to scale the number
down to size.

\medskip
\verb$float$
$$3.52947\times10^{39}$$

\subsection{Symbols}
Let us see if Eigenmath can find the
seventeenth root of a number, like the Hindu calculator could.
The following example defines a symbol to augment the calculation.

\medskip
\verb$N=212^17$

\verb$N$
$$3529471145760275132301897342055866171392$$

\verb$N^(1/17)$
$$212$$

\medskip
\noindent
It is worth mentioning that when a symbol is assigned a value,
no result is printed.
To see the value of a symbol, just evaluate it by putting it on a line by
itself.

\medskip
\verb$N$
$$3529471145760275132301897342055866171392$$

\medskip
\noindent

