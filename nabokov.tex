\chapter{The Big Picture}
Let us begin by considering the following passage from Vladimir Nabokov's
autobiography {\it Speak, Memory.}

\medskip
\noindent
``A foolish tutor had explained logarithms to me much too early, and I had
read (in a British publication, the {\it Boy's Own Paper}, I believe)
about a certain Hindu calculator who in exactly two seconds could find the
seventeenth root of, say,
3529471145760275132301897342055866171392
(I am not sure I have got this right; anyway the root was 212).''

\medskip
\noindent
We can check Nabokov's arithmetic by typing the following into Eigenmath.

\medskip
\verb$212^17$

\medskip
\noindent
After pressing the return key we should get the following result.
$$3529471145760275132301897342055866171392$$
So Nabokov did get it right after all.

