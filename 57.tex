\parindent=0pt

{\bf Problem A.} Consider the sets
$$A=\{n\in{\rm N}:n=15m\;\;\hbox{for some}\;\;m\in{\rm N}\}
\quad\hbox{and}\quad
B=\{n\in{\rm N}:n=3m\;\;\hbox{for some}\;\;m\in{\rm N}\}
$$
Show that (1) $A\subset B$, but (2) $B\not\subset A$.

\bigskip
(1) {\bf Proof.} Let $A$ and $B$ be sets as defined above.
\medskip
Let $x\in A$.
Then by the membership condition for $A$ we have
$x=15m=3(5m)$ where $m$ is a natural number.
Since $5m$ is also a natural number we have $x\in B$ by the
membership condition for $B$.
Because $x$ is arbitrary, every member of $A$ has the properties of $x$.
Therefore every member of $A$ is also a member of $B$.
Hence $A\subset B$.

\bigskip
(2) {\bf Proof.} Let $A$ and $B$ be sets as defined above.
\medskip
We have $3\in B$ and $3\not\in A$. Therefore $B\not\subset A$.

\bigskip
\bigskip
{\bf Problem B.} Prove the following lemma.
Let $A$, $B$ and $C$ be sets. If $B\subset C$,
then $A\backslash C\subset A\backslash B$.
\bigskip
{\bf Proof.} Let $A$, $B$ and $C$ be sets. Assume that $B\subset C$.
\medskip
Let $x\in A\backslash C$. Then $x\in A$ and $x\not\in C$.
Since $B\subset C$, $x\not\in C$ implies that $x\not\in B$.
Because $x\in A$ and $x\not\in B$ we have $x\in A\backslash B$.
Since $x$ is arbitrary, the properties of $x$ hold for every member
of $A\backslash C$.
Therefore every member of $A\backslash C$ is also a member of $A\backslash B$.
Hence $A\backslash C\subset A\backslash B$.

\bigskip
\bigskip
{\bf Problem C.} Let $A$, $B$, $C$ and $D$ be sets.
If $A\subset C$ and $B\subset D$, then $A\times B\subset C\times D$.
\bigskip
{\bf Proof}. Let $A$, $B$, $C$ and $D$ be sets.
Assume $A\subset C$ and $B\subset D$.
\medskip
Let $(a,b)\in A\times B$. Then $a\in A$ and $b\in B$.
Since $A\subset C$, $a\in A$ implies that $a\in C$.
Also, since $B\subset D$, $b\in B$ implies that $b\in D$.
Therefore we have $(a,b)\in C\times D$.
Since $(a,b)$ is arbitrary, the properties of $(a,b)$ hold for every
member of $A\times B$.
Therefore every member of $A\times B$ is also a member of $C\times D$.
Hence $A\times B\subset C\times D$.

\end
