\parindent=0pt

\beginsection 1. Combination lock

The cardinality of the set $\{0,1,2,\ldots,50\}$ is 51.

a) The numbers may be repeated.
$$N=51^3=132{,}651$$

b) The numbers may not be repeated.
$$N=51\times50\times49=124{,}950$$

\beginsection 2. Surgical experiment

$S=\{(SS),(SDS),(SDD),(DSS),(DSD),(DDSS),(DDSD),(DDDS),(DDDD)\}$

\medskip
Map each event to a set of outcomes.

First animal survives: $A=\{(SS),(SDS),(SDD)\}$

Two animals survive: $B=\{(SS),(SDS),(DSS),(DDSS)\}$

Third animal dies: $C=\{(DDDS),(DDDD)\}$

At least two animasl survive: $D=\{(SS),(SDS),(DSS),(DDSS)\}$

$$A\cap B=\{(SS),(SDS)\}$$

$$A^\prime\cup C=\{(DSS),(DSD),(DDSS),(DDSD),(DDDS),(DDDD)\}$$

$$B\cup D=\{(SS),(SDS),(DSS),(DDSS)\}$$

$$C\cap D=\{\}$$

\beginsection 3. Proofs

a)
$$\eqalign{
P\{(A\cap B)^\prime\}&=1-P\{A\cap B\}\cr
&=1+P\{A\cup B\}-P\{A\}-P\{B\}\cr
}$$
The last step is due to the theorem $P\{A\cup B\}=P\{A\}+P\{B\}-P\{A\cap B\}$.

\medskip
b)
$$\eqalign{
P\{(A\cup B)\cap C\}&=P\{(A\cap C)\cup(B\cap C)\}\cr
&=P\{A\cap C\}+P\{B\cap C\}-P\{(A\cap C)\cap(B\cap C)\}\cr
&=P\{A\cap C\}+P\{B\cap C\}-P\{A\cap B\cap C\}
}$$

\beginsection 4. Poker hands

The total number of poker hands is ${}_{52}C_5=2{,}598{,}960$.

\medskip
a) Three of a kind, the hand is $AAABC$.
There are ${}_{13}C_1=13$ ways to choose the rank of $A$.
There are ${}_4C_3=4$ ways to choose the suits of $A$.
(The suits must be different for all three cards since the ranks
are the same.)
There are ${}_{12}C_2=66$ ways to choose the ranks for $B$ and $C$.
(The ranks must be different to exclude a full house.)
There are ${}_4C_1=4$ ways to choose the suit for $B$
and ${}_4C_1=4$ ways to choose the suit for $C$.
Therefore there are $13\times4\times66\times4\times4=54{,}912$ hands of $AAABC$.
$$P\{AAABC\}={54{,}912\over2{,}598{,}960}=0.0211285$$

\medskip
b) Full house, the hand is $AAABB$.
There are ${}_{13}C_1=13$ ways to choose the rank of $A$.
There are ${}_4C_3=4$ ways to choose the suits of $A$.
There are ${}_{12}C_1=12$ ways to choose the rank of $B$.
There are ${}_4C_2=6$ ways to choose the suits for $B$.
Therefore there are $13\times4\times12\times6=3744$ hands of $AAABB$.
$$P\{AAABB\}={3744\over2{,}598{,}960}=0.00144058$$

\vfill
\eject

\beginsection 5. Two balls drawn from an urn

On the third draw the urn can be in either one of two states.
It can either contain $M$ or $M-1$ balls.
The probability that the urn contains $M$ balls is the probability that
ball 1 was not drawn in the first two draws.
$$P\{M\}=\left(M-1\over M\right)^2$$
$$P\{M-1\}=1-\left(M-1\over M\right)^2$$
The probability of drawing ball number 3 on the third draw is
$$P={1\over M}\times P\{M\}+{1\over M-1}\times P\{M-1\}$$

\beginsection 6. Craps

$$P\{2\}=P\{12\}={1\over36}$$
$$P\{3\}=P\{11\}={1\over18}$$
$$P\{4\}=P\{10\}={1\over12}$$
$$P\{5\}=P\{9\}={1\over9}$$
$$P\{6\}=P\{8\}={5\over36}$$
$$P\{7\}={1\over6}$$
The probability of rolling point $n$ and then winning is
$$P\{win|n\}=P\{n\}P\{n|(n\cup7)\}=P\{n\}{P\{n\}\over P\{n\}+1/6}$$
Note that $(n\cup7)$ means that we only consider rolls of the
dice that result in $n$ or 7.
$$P\{win|6\}=P\{win|8\}=(5/36)(5/11)=25/396$$
$$P\{win|5\}=P\{win|9\}=(1/9)(2/5)=2/45$$
$$P\{win|4\}=P\{win|10\}=(1/12)(1/3)=1/36$$
Overall probability of winning:
$$
P\{win\}=P\{7\}+P\{11\}+\sum_n P\{win|n\}
={244\over495}=0.492929
$$

\end