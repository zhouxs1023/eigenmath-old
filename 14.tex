\parindent=0pt

First, define two vector functions F and G.

\bigskip

{\tt\obeylines
> F=(FX(),FY(),FZ())
> G=(GX(),GY(),GZ())
}

\bigskip

Now verify the following vector identities.


$$\mathop{\rm div}(\mathop{\rm curl}F)=0\eqno1$$

{\tt\obeylines
> div(curl(F))
0
}

$$\mathop{\rm curl}(\mathop{\rm grad}f)=0\eqno2$$

{\tt\obeylines
> curl(grad(f()))
(0,0,0)
}

$$\mathop{\rm div}(\mathop{\rm grad}f)=\nabla^2f\eqno3$$

{\tt\obeylines
> div(grad(f()))-laplacian(f())
0
}

$$\mathop{\rm curl}(\mathop{\rm curl}F)=\mathop{\rm grad}
(\mathop{\rm div}F)-\nabla^2F\eqno4$$

{\tt\obeylines
> curl(curl(F))-grad(div(F))+laplacian(F)
(0,0,0)
}

$$\mathop{\rm grad}(fg)=f\mathop{\rm grad}g+g\mathop{\rm grad}f\eqno5$$

{\tt\obeylines
> grad(f()*g())-f()*grad(g())-g()*grad(f())
(0,0,0)
}

$$\mathop{\rm grad}(F\cdot G)=(G\cdot{\rm grad})F+(F\cdot{\rm grad})G
+G\times\mathop{\rm curl}F+F\times\mathop{\rm curl}G\eqno6$$

{\tt\obeylines
> grad(dot(F,G))-dot(grad(F),G)-dot(grad(G),F)-cross(G,curl(F))-cross(F,curl(G))
(0,0,0)
}

\bigskip

It turns out that the notation $(G\cdot{\rm grad})F$ actually means
$({\rm grad}\,F)\cdot G$.
Note: {\tt dot(grad(F),G)} is different from {\tt dot(G,grad(F))}! Why?
Because {\tt grad(F)} is a square matrix, not a vector.
Only the dot product of vectors is guaranteed to be commutative.

$$\mathop{\rm div}(fF)=f\mathop{\rm div}F+\mathop{\rm grad}f\cdot F\eqno7$$

{\tt\obeylines
> div(f()*F)-f()*div(F)-dot(grad(f()),F)
0
}

$$\mathop{\rm div}(F\times G)=G\cdot\mathop{\rm curl}F-F\cdot\mathop{\rm curl}G\eqno8$$

{\tt\obeylines
> div(cross(F,G))-dot(G,curl(F))+dot(F,curl(G))
0
}

$$\mathop{\rm curl}(fF)=f\mathop{\rm curl}F+\mathop{\rm grad}f\times F\eqno9$$

{\tt\obeylines
> curl(f()*F)-f()*curl(F)-cross(grad(f()),F)
(0,0,0)
}

$$\mathop{\rm curl}(F\times G)
=F\mathop{\rm div}G
-G\mathop{\rm div}F
+(G\cdot\mathop{\rm grad})F
-(F\cdot\mathop{\rm grad})G
\eqno10$$

{\tt\obeylines
> curl(cross(F,G))-F*div(G)+G*div(F)-dot(grad(F),G)+dot(grad(G),F)
(0,0,0)
}

\end
