{\it Homework \#4 from George Weigt}

\beginsection 1.

Consider the continuous random variable $X$ with probability density function as
$$f(x)={xe^{-x^2}\over C},\qquad\forall x\in R=(0,\infty)$$

\bigskip
1a) Find the value of $C$. Solution: Integrate $f$ over $R$ then choose
$C$ so that the result is 1.
$${1\over C}\int_0^\infty xe^{-x^2}\,dx=-{1\over2C}e^{-x^2}\bigg|_0^\infty
=-0+{1\over2C}=1$$
Hence
$$C={1\over2}$$

\bigskip
1b) Find the distribution function of $X$.
Solution: Integrate $f$ from 0 to $t$.
$$F(t)=2\int_0^txe^{-x^2}\,dx=-e^{-x^2}\bigg|_0^t
=-e^{-t^2}+1,\qquad0\le t<\infty$$
$$F(t)=\cases{
0, & $t<0$\cr
1-e^{-t^2}, & $0\le t$\cr
}$$

\bigskip
1c) The expected value of $X$.
Solution: Integrate $xf$ over $R$.
$$E[X]=2\int_0^\infty x^2e^{-x^2}\,dx
={\sqrt\pi\over2}$$
The definite integral is number 665 in {\it CRC Handbook of Mathematical Sciences,} 6th edition.

\vfill
\eject

\beginsection 2.

Consider the continuous random variable $X$ with
probability density function as
$$f(x)={C\over1+x^2},\qquad\forall x\in R=(-\infty,+\infty)$$

\bigskip
2a) Find the value of $C$.
Solution: Integrate $f$ over $R$ then choose $C$ so that the result is 1.
$$C\int_{-\infty}^{+\infty}{1\over1+x^2}=C\tan^{-1}x\bigg|_{-\infty}^{+\infty}
=C\pi=1$$
Hence
$$C={1\over\pi}$$

\bigskip
2b) Find the distribution function of $X$.
Solution: Integrate $f$ and evaluate from 0 to $t$.
$$F(t)={1\over\pi}\int_{-\infty}^t{1\over1+x^2}\,dx={1\over\pi}\tan^{-1}x\bigg|_{-\infty}^t
={1\over\pi}\tan^{-1}x+{1\over2}$$
Check: $F(-\infty)=(1/\pi)(-\pi/2)+1/2=0$,
$F(+\infty)=(1/\pi)(\pi/2)+1/2=1$.

\bigskip
2c) Find the second moment about the origin of $X$.
$$E[X^2]={1\over\pi}\int_{-\infty}^{+\infty}{x^2\over1+x^2}\,dx
={1\over\pi}(x-\tan^{-1}x)\bigg|_{-\infty}^{+\infty}
={1\over\pi}\lim_{x\rightarrow\infty}\left(x-\tan^{-1}x-x+\tan^{-1}x\right)=0
$$

\vfill
\eject

\beginsection 3.

Consider the continuous random variable $X$ with
probability density function as
$$f(x)=2(1-x),\qquad\forall x\in R=(0,C)$$

\bigskip
3a) Find the value of $C$.
Solution: Find the integral of $f$ then choose $C$.
$$\int_0^C(2-2x)\,dx=(2x-x^2)\bigg|_0^C=2C-C^2=1$$
Hence
$$C=1$$

\bigskip
3b) Find the moment generating function of $X$.
Solution: Evaluate $\int e^{tx}f(x)$.
$$\eqalign{
E[e^{tX}]&=\int_0^1e^{tx}(2-2x)\,dx\cr
&=\left(
{2e^{tx}\over t}-{2e^{tx}(tx-1)\over t^2}
\right)\bigg|_0^1\cr
&={2e^t\over t}-{2e^t(t-1)\over t^2}-{2\over t}-{2\over t^2}\cr
&={2te^t\over t^2}-{2e^t(t-1)\over t^2}-{2t\over t^2}-{2\over t^2}\cr
&={2e^t-2t-2\over t^2}
}$$
Since $E[e^{tX}]$ is not defined for $t=0$,
the moment generating function $M(t)$ does not exist.

\bigskip
3c) Find the second moment about the origin of $X$.
Solution: Evaluate $\int x^2f(x)$.
$$\eqalign{
E[X^2]&=\int_0^1x^2(2-2x)\,dx=\left({2\over3}x^3-{1\over2}x^4\right)\bigg|_0^1={1\over6}\cr
}$$

\vfill
\eject

\beginsection 4.

Consider the continuous random variable $X$ with probability density function as
$$f(x)=\cases{
1+x, & $-1<x<0$\cr
1-x, & $0<x<1$\cr
0, & otherwise\cr
}$$

4a) Find the distribution function of $X$.
Solution: Integrate $f(x)$ piecewise.
$$F(t)=\int_{-1}^t(1+x)\,dx=\left(x+{1\over2}x^2\right)\bigg|_{-1}^t
={1\over2}t^2+t+{1\over2},\qquad-1<t<0$$
$$F(t)=F(0)+\int_0^t(1-x)\,dx={1\over2}+\left(x-{1\over2}x^2\right)\bigg|_0^t
=-{1\over2}t^2+t+{1\over2},\qquad0<t<1$$
Hence
$$F(t)=\cases{
0, & $t<-1$\cr
\cr
\displaystyle{{1\over2}t^2+t+{1\over2}}, & $-1\le t<0$\cr
\cr
\displaystyle{-{1\over2}t^2+t+{1\over2}}, & $0\le t<1$\cr
\cr
1, & $1\le t$
}$$
A cursory check reveals that $F(-1)=0$ and $F(1)=1$ which
is what we expect.

4b) Find the variance value of $X$.
Solution: Compute using $E[X]$ and $E[X^2]$.
The integrals must be done piecewise.
$$\eqalign{
E[X]&=\int_{-1}^0x(1+x)\,dx+\int_0^1x(1-x)\,dx\cr
&=\left({1\over2}x^2+{1\over3}x^3\right)\bigg|_{-1}^0
+\left({1\over2}x^2-{1\over3}x^3\right)\bigg|_0^1\cr
&=0-{1\over2}+{1\over3}+{1\over2}-{1\over3}-0\cr
&=0
}$$
$$\eqalign{
E[X^2]&=\int_{-1}^0x^2(1+x)\,dx+\int_0^1x^2(1-x)\,dx\cr
&=\left({1\over3}x^3+{1\over4}x^4\right)\bigg|_{-1}^0
+\left({1\over3}x^3-{1\over4}x^4\right)\bigg|_0^1\cr
&=0+{1\over3}-{1\over4}+{1\over3}-{1\over4}-0\cr
&=1/6
}$$
$$\sigma^2=E[X^2]-(E[X])^2=1/6=0.1667$$

4c) Find the third moment about the mean of $X$.
Solution: Calculate $E[X^3]$ by integrating piecewise.
Then calculate $E[(X-\mu)^3]=E[X^3]-3E[X]E[X^2]+2(E[X])^3$.
$$\eqalign{
E[X^3]&=\int_{-1}^0x^3(1+x)\,dx+\int_0^1x^3(1-x)\,dx\cr
&=\left({1\over4}x^4+{1\over5}x^5\right)\bigg|_{-1}^0
+\left({1\over4}x^4-{1\over5}x^5\right)\bigg|_0^1\cr
&=0-{1\over4}+{1\over5}+{1\over4}-{1\over5}-0\cr
&=0
}$$
$$E[(X-\mu)^3]=E[X^3]-3E[X]E[X^2]+2(E[X])^3=0$$

\vfill
\eject

\beginsection 5.

Consider the continuous random variable $X$ with distribution
function as
$$F(t)=\cases{
0, & $t<1$\cr
(t^3+1)/2, & $-1\le t<1$\cr
1, & $1\le t$\cr
}$$

5a) Find the probability density function of $X$.
$$f(x)={\partial(x^3+1)/2\over\partial x}={3\over2}x^2,
\qquad-1<x<1$$

5b) Find the moment generating function of $X$.
$$E[e^{tX}]={3\over2}\int_{-1}^1e^{tx}x^2\,dx
={3e^{tx}(t^2x^2-2tx+2)\over2t^3}\bigg|_{-1}^1$$
Since $E[e^{tX}]$ is not defined for $t=0$,
the moment generating function $M(t)$ does not exist.

5c) Find the variance of $X$.
$$E[X]={3\over2}\int_{-1}^1x^3\,dx={3\over8}x^4\bigg|_{-1}^1=0$$
$$E[X^2]={3\over2}\int_{-1}^1x^4\,dx={3\over10}x^5\bigg|_{-1}^1=0.6$$
$$\sigma^2=E[X^2]-(E[X])^2=0.6$$


\end