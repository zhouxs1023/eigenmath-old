The two main forces acting on a skydiver are gravity and drag.
$$\eqalign{
F_g&=mg\cr
F_d&=-{1\over2}\rho v^2AC_d\cr
}$$
where $\rho$ is air density, $A$ is the area of the skydiver and $C_d$ is the
coefficient of drag.
Hence the acceleration of a skydiver can be written as
$$a(t)={F_g+F_d\over m}=g-{\rho v^2AC_d\over2m}=
g-{\rho AC_d\over2m}\left(\int_0^t a(t)\,dt\right)^2\eqno(1)$$
According to Wikipedia, the velocity of an object falling through
a non-dense medium is
$$v(t)=\sqrt{2mg\over\rho AC_d}\tanh\left(t\sqrt{g\rho AC_d\over2m}\right)$$
If we differentiate this we obtain
$$a(t)={g\over\cosh^2\left(t\displaystyle{\sqrt{g\rho AC_d\over2m}}\right)}
\eqno(2)$$
Let us verify that the $a(t)$ given in (2) is a solution to the
differential equation (1).

\end

