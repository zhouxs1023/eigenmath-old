\item{\S1.} Isaac Newton (1642--1727)

\itemitem{1.}
Born to parents who owned property so they were wealthy.
His father died before Isaac was born.
At the age of three, his mother remarried and moved away.
From then on he was raised by his grandmother.
He was a top student in grammar school.

\itemitem{2.}
He went to college at Cambridge starting in 1661 at the age
of eighteen.
He graduated four years later in 1665.
Shortly thereafter Cambridge was closed due to the plague.
For the next 18 months Newton educated himself at home.
It was during this time that Newton began making
discoveries in the areas of mathematics and physics.

\itemitem{3.}
In 1667 Newton became a professor at Trinity College, Cambridge.
In 1689 he was elected to Parliament as the Cambridge representative.
According to Boyer, Newton suffered a nervous breakdown in 1692.
Four years later, Newton retired from Cambridge and moved to London.
He took a government job as Warden of the Mint.

\itemitem{4.}
In 1703 Newton was elected president of the Royal Society and
held the post for the rest of his life.
He never married.

\item{\S2.} His work

\itemitem{1.}
Newton's most famous work, Principia Mathematica, was
first published in 1687.
It is currently available for download and the 1687
edition consists of 494 pages.
The Burndy Library web site describes Newton's Principia as
``...the founding treatise in the domain of rational mechanics...''

\itemitem{2.}
Newton discovered the generalized binomial theorem.
He also showed how infinite series could safely be used in mathemetics.
He also came up with a method of approximating the zeroes of a function.

\itemitem{3.}
He explained the conservation of linear and angular momentum.
These conservation laws greatly expanded the number of physics
problems that could be solved.
Newton developed many other physical laws as well.

\itemitem{4.}
Invented calculus.

\itemitem{5.}
By using a lens and a second prism, Newton was the first to demonstrate 
conclusively that white light was a mixture of colors.
This spelled the end of the theory that a prism somehow added
color to light.

\itemitem{6.}
Invented the refracting telescope and made many other
contributions to the field of optics.



\item{\S3.} References

\itemitem{1.} {\it A History of Mathematics,} by Carl B. Boyer.
\itemitem{2.} {\it Wikipedia: Isaac Newton,} {\tt http://en.wikipedia.org/wiki/Isaac\_newton}
\itemitem{3.} {\it Newton and $\pi$,} {\tt chan.hei-chi@uis.edu}

\vfill
\eject

\beginsection Newton-Leibniz Controversy

Let us first review the facts.
According to Boyer, Newton invented calculus 10 years before Leibniz (400).
However, Leibniz published first, in 1684.
Newton's calculus was published three years later in his
{\it Principia Mathematica.}
In the 1684 edition of {\it Principia,} Newton acknowledged that Leibniz
had devised a similar method.
However, as the priority dispute wore on, Newton evetually deleted the
reference to Leibniz in the 1726 edition of {\it Principia.}
Finally, it is known that Leibniz, while in London in 1676, was shown
some of Newton's unpublished work.

There are two distinct aspects to the dispute.
First, there is the personal dispute between the
men themselves.
Apparently, someone gave Newton the idea that Leibniz
must have stolen Newton's calculus idea in 1676.
Newton became convinced that it was true, that's why he
removed Leibniz from {\it Principia.}
However, since Leibniz was already working on quadrature,
it seems far more likely that Leibniz did in fact
independently discover calculus.

There is a second aspect to the dispute that requires a
little more imagination.
What we have to do is imagine what it was like back then,
over 300 years ago.
Back then, there were not that many people working on advanced
mathematics.
Priority was not an issue because everyone knew everyone else
and there was a kind of gentleman's agreement that governed things.
Newton normally got the word out about his new ideas by sending letters to friends.
For example, Newton figured out the binomial theorem in 1664.
In 1676, he sent two letters to Oldenburg describing the idea.
Finally, Wallis published the result in 1685 with credit to Newton.
So what we have is the Newton-Leibniz dispute being the cusp of
something new.
The pace of mathematics was starting to pick up.
It was becoming more likely that two people would dream up
the same idea independently.
Also, getting the word out was shifting from personal letters
to journals.
Consequently, publishing priority became a new idea that eventually replaced
the gentleman's agreement.
In our modern view we recognize Leibniz' publishing priority.
However, Newton was operating under the gentlemen's agreement.
The priority dispute, aside from the personal conflict, is evidence
of the cultural change that was underway at the time. 

\bigskip
References

\itemitem{1.} {\it A History of Mathematics,} by Carl B. Boyer.
\itemitem{2.} {\it Wikipedia: Isaac Newton,} {\tt http://en.wikipedia.org/wiki/Isaac\_newton}
\itemitem{3.} {\it Wikipedia: Gottfired, Leibniz,} {\tt http://en.wikipedia.org/wiki/Leibniz}

\vfill
\eject

$${3\sqrt3\over4}+24\left(
{1\over12}-{1\over5\cdot2^5}-{1\over28\cdot2^7}-{1\over72\cdot2^9}\right)=3.14169$$


\bigskip
$$(1+x)^{1/3}=1+{1\over3}x-{1\over9}x^2+{5\over81}x^3-{10\over243}x^4+\cdots$$




\end
