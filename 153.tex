\magnification=1200
\centerline{The Life and Work of Srinivasa Ramanujan}

\beginsection Introduction

Srinivasa Ramanujan (1887-1920) was a mathematician who made
many significant contributions in the areas of hypergeometric series,
divergent series, modular functions, the partition function,
elliptic integrals and prime
number theory.
He was a prolific theorist and his results are still being studied
today.

\beginsection Outline

\item{1.}
Biography
\item{2.}
Work
\itemitem{2.1} Divergent series
\itemitem{2.2} A puzzle
\itemitem{2.3} The partition function
\itemitem{2.4} Highly composite numbers
\itemitem{2.5} Continued fractions
\itemitem{2.6} Rogers-Ramanujan Identity
\itemitem{2.7} Selected formulas for $\pi$
\itemitem{2.8} Ramanujan's tau function
\itemitem{2.9} Other formulas
\item{3.}
References

\beginsection 1. Biography

In the winter of 1913, a distinguished mathematician in England
received a letter from a clerk in India.
The letter was ten pages long and was filled with handwritten
mathematics.
The English mathematician, Godfrey Hardy, showed the letter to his friends in the
mathematics department at Cambridge.
It caused quite a sensation.
Many of the theorems in the letter, although unproved,
were astonishing in their originality.
It was generally agreed that the letter writer, S. Ramanujan,
must be a mathematical genius.
A month later, Hardy wrote back to Ramanujan.
He praised Ramanujan's work and asked Ramanujan to send him
proofs of the theorems.
What Hardy did not put in the letter was the fact that he
had already decided to bring Ramanujan to Cambridge.
Back in India, the letter from the famous Mr. Hardy immediately
boosted Ramanujan's career.
On the basis of Hardy's letter,
Ramanujan was able to secure a two year scholarship at the University of
Madras and was able to quit his job as a shipping clerk.

One thing that is important to know about Ramanujan is that, up until 1914, he
had no formal training in advanced mathematics.
He had taught himself mathematics by studying a book by Carr
that he had aquired when he was fifteen years old.
Ramanujan was twenty-five years old when he wrote to Hardy.
He also wrote to H. F. Baker and E. W. Hobson.
Hardy was the only one who replied.
Ramanujan sent the letter because he had become somewhat desperate.
He had attended college twice but did not finish because
the only thing he wanted to study was mathematics.
Unfortunately, his lack of a degree was holding him back.
He was married and needed to work but was stuck as a shipping clerk.
He dreamed of working as an academic so he could study
mathematics full time.
By writing to Hardy, he hoped to gain recognition in the mainstream mathematics
community and thereby promote his career,
which is in fact what happened.
Despite not being a professional mathematician at the time,
Ramanujan was actually well-known to mathematicians
in Madras where he lived.
In 1911 he published ``Some Properties of Bernoulli Numbers'' in the
{\it Journal of the Indian Mathematical Society.}
The paper was seventeen pages long.

In the spring of 1914, at Hardy's behest, Ramanujan traveled by ship to England.
He traveled alone and left his wife behind in India.
It had taken Hardy a year to make all the arrangements.
The plan was for Hardy and Ramanujan to work together at Cambridge University,
and for Ramanujan to study for a degree.
The book by Carr that Ramanujan had used to teach himself
mathematics was about fifty years out of date.
Ramanujan had to learn all of the new mathematics that had
been discovered since then.
In 1914, the year of his arrival in England, Ramanujan published one
paper, ``Modular Equations and Approximations to Pi.''
It was twenty-three pages long.
In the following year (1915) he published nine papers.
His most important paper that year was a fifty-two page treatise on highly composite numbers.
This paper led to his award in 1916 of a Bachelor of Science by Research degree.
Despite its title, this was actually an advanced degree.
Also in 1916, Hardy and Ramanujan collaborated on a study of the partition function.
The full account of their results was published in 1918.
This was the most famous paper on which they collaborated because it
established a new technology for attacking problems.
Between the years 1915 and 1919, Hardy and Ramanujan collaborated
on a total of four papers officially.

In the spring of 1917, Ramanujan became sick with tuberculosis.
He could not fully recover and he was in and out of hospitals over the next two years.
In 1918 Ramanujan received three distinguished honors.
He was elected Fellow of the Royal Society, Fellow of Trinity College, and Fellow of the
Cambridge Philosophical Society.
These honors boosted his spirts and he improved somewhat.
However, he became sick again and decided, in 1919, to return to his
home in India to recuperate.
Unfortunately he did not recover and died in India in 1920 at the age of thirty-two.

Throughout his life, Ramanujan accumulated all of his theorems in several large
notebooks.
When he arrived in England at the age of twenty-five, it is estimated that the notebooks
already contained three to four thousand theorems.
(Dr. Bruce C. Berndt, Professor of Mathematics at the University of Illinois,
puts the figure at exactly $3,542$ theroems [5].)
Since Ramanujan's death, the notebooks have taken on a life of their own.
The majority of the results in the notebooks were unproven.
Over the years, mathematicians have set out to prove Ramanujan's results and
discovered new mathematics in the process.
Hardy was in possession of the notebooks until 1928 when he gave them to G. N. Watson.
Watson and B. M. Wilson ended up publishing over
two dozen papers during the next ten years, all based on Ramanujan's original
research.
In 1977 the American mathematician Bruce C. Berndt took over the task.
It took him one year to prove all of the results in just one of Ramanujan's chapters.
As of 1991, Berndt was still at work investigating what Ramanujan had been working
on more than seventy years earlier.


%By 1940, over 100 papers based on Ramanujan's work had been published.

\beginsection 2.1 Divergent series

In his first letter to Hardy, Ramanujan included the following identity.
$$1+2+3+\cdots=-{1\over12}$$
Of course, this result cannot be obtained by summation.
Rather, this result comes from two premises.
First, if the sum of a divergent series is to have any meaning at all then the series
itself must admit linear operations.
Second, it is not admissible to reorder the terms of the series.
This second premise guarantees the uniqueness of the result.
At this point we need to establish an intermediate result. Let
$$s=1-1+1-1+\cdots$$
Then
$$1-s=1-(1-1+1-1+\cdots)=1-1+1-1+\cdots=s$$
Solving for $s$ we have $1=2s$ or $s=1/2$.
Now let
$$s=1+2+3+4+\cdots$$
We are allowed to use linear operators so we can put
$$-3s=s-4s$$
Now $4s$ has an interesting property. We have $4s=2\cdot2s$ and so when subtracted
from $s$ the result is that the signs of the even terms are flipped. Hence
$$-3s=1-2+3-4+5-6+\cdots$$
The right hand side can be written as the sum of two series as follows
$$-3s=1-(1-2+3-4+\cdots)-(1-1+1-1+\cdots)=1+3s-1/2$$
Solving for $s$ we have $-6s=1/2$ hence
$$s=-{1\over12}$$
from which it follows
$$1+2+3+4+\cdots=-{1\over12}$$


\beginsection 2.2 A puzzle

In 1911, Ramanujan published the following puzzle in
the {\it Journal of the Indian Mathematical Society.}
The reader was asked to evaluate
$$\sqrt{1+2\sqrt{1+3\sqrt{1+\cdots}}}$$
After six months, no reader could solve the problem so Ramanujan
provided the answer.
$$\sqrt{1+2\sqrt{1+3\sqrt{1+\cdots}}}=3$$
Ramanujan had crafted the puzzle from his theorem
$$x+n+a=\sqrt{ax+(n+a)^2+x\sqrt{a(x+n)+(n+a)^2+(x+n)\sqrt{\ldots}}}$$
In the puzzle, $x=2$, $n=1$, and $a=0$.

\beginsection 2.3 The partition function

Ramanujan spent a lot of time investigating the partition function.
The partition function $p(n)$ is the number of ways the number
$n$ can be expressed as the sum of addends.
For example, $p(4)=5$ because there are five ways to partition 4.
$$\eqalign{
&4\cr
&3+1\cr
&2+2\cr
&2+1+1\cr
&1+1+1+1\cr
}$$
Note that the order of the addends is not significant.
The addends are
normally written from largest to smallest.
Ramanujan discovered that the partition function has a $5k+4$
congruence.
In other words, if $n$ can be expressed as $5k+4$ (numbers that end in 4 or 9)
then $p(n)$
is divisible by 5.
Ramanujan also discovered congruences for numbers that can be expressed
as $7k+5$ and $11k+6$.
Ramanujan and Hardy collaborated on a forty-page paper, published in 1918,
that included a formula for computing the value of $p(n)$ from $n$.
The formula was accurate up to $p(200)$.
The exact formula for $p(n)$ was discovered twenty-one years later
in 1937 by Hans Rademacher.

\beginsection 2.4 Highly composite numbers

In 1915 Ramanujan published a paper on highly composite numbers.
Ramanujan himself invented the term ``highly composite.''
A highly composite number is a number that has more factors than any
number less than it. For example, the first six highly composite numbers are
$$1,\quad2,\quad4,\quad6,\quad12,\quad24.$$
The number 24 has eight factors.
This is more factors than any number less than 24 has.
While any composite number can be thought of as the ``opposite'' of a prime,
a highly composite number is in a sense the most extreme opposite
of a prime.
(Except for 2 which is both prime {\it and} highly composite!)
Ramanujan discovered that if you take a highly composite number
and write down its prime factors from smallest to largest, then the
powers of the primes {\it never} increase as you read from
left to right. For example,
$$24=2^3\times3^1$$
In this case we have $3\ge1$.
Ramanujan also discovered that, with the exception of 4 and 36,
the final power is {\it always} 1, as it is in the above example.
Ramanujan's published proof on this subject was fifty-two pages long.

\beginsection 2.5 Continued fractions

This formula involves the Golden Ratio $\phi$.
$$\sum_{k=1}^\infty{1\over2^{\lfloor k\phi\rfloor}}={1\over2^0+\displaystyle{1\over2^1+\cdots}}$$

\beginsection 2.6 Rogers-Ramanujan Identity

$$1+\sum_{k=1}^\infty{q^{k^2+k}\over(1-q)(1-q^2)\cdots(1-q^k)}
=\prod_{j=0}^\infty{1\over(1-q^{5j+2})(1-q^{5j+3})},\qquad|q|<1$$

\beginsection 2.7 Selected Formulas for $\pi$

The following Ramanujan formula for $\pi$ is accurate to eight decimal places.
$$\pi\approx\left(97+{1\over2}-{1\over11}\right)^{1/4}$$
The following Ramanujan formula was used to calculate $\pi$ to one billion digits in 1989.
$${1\over\pi}={\sqrt8\over9801}\sum_{n=0}^\infty
{(4n)!(1103+26{,}309n)\over(n!)^4(396)^{4n}}$$

\beginsection 2.8 Ramanujan's tau function

$$\sum_{n=1}^\infty\tau(n)x^n=x\prod_{n=1}^\infty(1-x^n)^{24}$$
The coefficients for $\tau$ can be calculated as follows.
$$\sum_{n=1}^\infty\tau(n)x^n=x(1-3x+5x^3-7x^6+9x^{10}-\cdots)^8$$

\beginsection 2.9 Other formulas

In 1916 Ramanujan published ``On certain arithmetical functions'' that contained
many identities involving the functions
$$L(q)=1-24\sum_{n=1}^\infty{nq^n\over1-q^n},\quad
M(q)=1+240\sum_{n=1}^\infty{n^3q^n\over1-q^n},\quad
N(q)=1-504\sum_{n=1}^\infty{n^5q^n\over1-q^n}$$
which are defined for $|q|<1$.
A few of the identities are
$$
q{dL\over dq}={L^2-M\over12},\qquad
q{dM\over dq}={LM-N\over3},\qquad
q{dN\over dq}={LN-M^2\over2}.
$$
These identities are applicable to the series
$$\sum_{m,n=-\infty}^\infty q^{m^2+mn+2n^2}$$

%\beginsection 2.4 Other formulas

\noindent
This formula equates the infinite sum on the left with a finite sum on the right.
$$\sum_{k=1}^\infty{1\over n+k}={n\over2n+1}+\sum_{k=1}^n{1\over(2k)^3-2k}$$

\beginsection 3. References

\item{1.}
Kanigel, Robert. {\it The Man Who Knew Infinity.} New York: Washington Square Press,
1991.
\item{2.}
Chan and Ong. {\it On Eisenstein Series.}
{\tt http://www.ams.org/proc/\hfill\break
1999-127-06/S0002-9939-99-04832-7/S0002-9939-99-04832-7.pdf}
\item{3.}
Cais, Bryden. {\it Divergent Series.}\hfill\break
{\tt http://www.math.uiuc.edu/$\sim$berndt/articles/rrcf.pdf}
\item{4.}
{\it Srinivasa Ramanujan.} Wikipedia.\hfill\break
{\tt http://en.wikipedia.org/wiki/Srinivasa\_Ramanujan}
\item{5.}
{\it Ramanujan's Tau Function.}\hfill\break
{\tt http://www.users.globalnet.co.uk/\hfill\break
$\sim$perry/maths/ramanujantau/ramanujantau.htm}
\item{6.}
{\it Rediscovering Ramanujan.} Frontline.
{\tt http://www.hinduonnet.com/\hfill\break
fline/fl1617/16170810.htm}
\end
