\parindent=0pt

Abstract Algebra

\bigskip
{\bf Problem A.}
Let ${\rm R}^2$ denote the set $\{(a,b):a,b\in {\rm R}\}$.
Let the operation $\oplus:{\rm R}^2\times{\rm R}^2
\rightarrow{\rm R}^2$ be defined as $(a,b)\oplus(c,d)=(a+c,b\cdot d)$
where $+$ and $\cdot$ denote the usual operations of real numbers.

\bigskip
\bigskip
Is $\oplus$ associative?
To find out, we need to test the following equality.
$$\bigg[(x_1,x_2)\oplus(y_1,y_2)\bigg]\oplus(z_1,z_2)
=(x_1,x_2)\oplus\bigg[(y_1,y_2)\oplus(z_1,z_2)\bigg]\eqno{\rm (A1)}$$
Step 1. Expand the left side of (A1).
$$\eqalign{
\bigg[(x_1,x_2)\oplus(y_1,y_2)\bigg]\oplus(z_1,z_2)
&=(x_1+y_1,x_2\cdot y_2)\oplus(z_1,z_2)\cr
&=(x_1+y_1+z_1,x_2\cdot y_2\cdot z_2)
}$$
Step 2. Expand the right side of (A1).
$$\eqalign{
(x_1,x_2)\oplus\bigg[(y_1,y_2)\oplus(z_1,z_2)\bigg]
&=(x_1,x_2)\oplus(y_1+z_1,y_2\cdot z_2)\cr
&=(x_1+y_1+z_1,x_2\cdot y_2\cdot z_2)
}$$
Both sides of (A1) are identical after expanding $\oplus$.
Since arbitrary symbols were used, the equality holds for all members
of ${\rm R}^2$. Consequently, the operation $\oplus$ is associative.

\bigskip
\bigskip
Is $\oplus$ commutative?
To find out, we need to test the following equality.
$$(x_1,x_2)\oplus(y_1,y_2)=(y_1,y_2)\oplus(x_1,x_2)\eqno{\rm (A2)}$$
Step 1. Expand the left side of (A2).
$$(x_1,x_2)\oplus(y_1,y_2)=(x_1+y_1,x_2\cdot y_2)$$
Step 2. Expand the right side of (A2).
$$\eqalign{
(y_1,y_2)\oplus(x_1,x_2)&=(y_1+x_1,y_2\cdot x_2)\cr
&=(x_1+y_1,x_2\cdot y_2)
}$$
The last step is allowed by the commutative property of addition
and multiplication of real numbers.
Both sides of (A2) are identical after expanding $\oplus$.
Since arbitrary symbols were used, the equality holds for all members
of ${\rm R}^2$. Consequently, the operation $\oplus$ is commutative.

\vfill
\eject

{\bf Problem B.}
Let ${\rm R}^2$ denote the set $\{(a,b):a,b\in {\rm R}\}$.
Let the operation $\oplus:{\rm R}^2\times{\rm R}^2
\rightarrow{\rm R}^2$ be defined as $(a,b)\oplus(c,d)=(a+d,b+c)$
where $+$ denotes the usual sum of real numbers.

\bigskip
\bigskip
Is $\oplus$ associative?
To find out, we need to test the following equality.
$$\bigg[(x_1,x_2)\oplus(y_1,y_2)\bigg]\oplus(z_1,z_2)
=(x_1,x_2)\oplus\bigg[(y_1,y_2)\oplus(z_1,z_2)\bigg]\eqno{\rm (B1)}$$
Step 1. Expand the left side of (B1).
$$\eqalign{
\bigg[(x_1,x_2)\oplus(y_1,y_2)\bigg]\oplus(z_1,z_2)
&=(x_1+y_2, x_2+y_1)\oplus(z_1,z_2)\cr
&=(x_1+y_2+z_2,x_2+y_1+z_1)
}$$
Step 2. Expand the right side of (B1).
$$\eqalign{
(x_1,x_2)\oplus\bigg[(y_1,y_2)\oplus(z_1,z_2)\bigg]
&=(x_1,x_2)\oplus(y_1+z_2,y_2+z_1)\cr
&=(x_1+y_2+z_1,x_2+y_1+z_2)
}$$
The results are different, hence $\oplus$ is not associative.

\bigskip
\bigskip
Is $\oplus$ commutative?
To find out, we need to test the following equality.
$$(x_1,x_2)\oplus(y_1,y_2)=(y_1,y_2)\oplus(x_1,x_2)\eqno{\rm (B2)}$$
Step 1. Expand the left side of (B2).
$$(x_1,x_2)\oplus(y_1,y_2)=(x_1+y_2,x_2+y_1)$$
Step 2. Expand the right side of (B2).
$$(y_1,y_2)\oplus(x_1,x_2)=(y_1+x_2,y_2+x_1)$$
The results are different, hence $\oplus$ is not commutative.

\vfill
\eject

{\bf Problem C.}
As in Problem A, except define $\oplus$ to be
$(a,b)\oplus(c,d)=(a+c-7,b\cdot d)$.

\bigskip
\bigskip
We want to find the identity. Let $(x_1,x_2)$ be the identity.
Then for arbitrary $(y_1,y_2)$ we have
$$(x_1,x_2)\oplus(y_1,y_2)=(x_1+y_1-7,x_2\cdot y_2)=(y_1,y_2)$$
Solving for $x_1$ and $x_2$ we have
$$\eqalign{
x_1&=7\cr
x_2&=1\cr
}$$
Verify that $(7,1)$ is indeed the identity.
$$\eqalign{
(7,1)\oplus(y_1,y_2)=(7+y_1-7,1\cdot y_2)&=(y_1,y_2)\cr
(y_1,y_2)\oplus(7,1)=(y_1+7-7,y_2\cdot1)&=(y_1,y_2)
}$$

\bigskip
\bigskip
Next, we want to find the $\oplus$-inverse of $(-2,5)$.
Let $(x_1,x_2)$ be the $\oplus$-inverse of $(-2,5)$.
We have
$$(x_1,x_2)\oplus(-2,5)=(x_1-9,5x_2)=(7,1)$$
Solving we have $(x_1,x_2)=(16,1/5)$.
Verify that it is indeed the inverse.
$$\eqalign{
(16,1/5)\oplus(-2,5)=(16-2-7,1/5\cdot5)&=(7,1)\cr
(-2,5)\oplus(16,1/5)=(-2+16-7,5\cdot1/5)&=(7,1)
}$$

\end
