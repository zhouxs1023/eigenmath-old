\magnification=1200

\noindent
{\it George Weigt --- Geometry Homework \#9}

\beginsection Page 109, problem 11.

In $\cal H$ find the angle bisector of $\angle ABC$ if $A=(0,5)$, $B=(0,3)$
and $C=(2,\sqrt{21})$.

\bigskip
\noindent
Solution: The angle bisector is the ray $\overrightarrow{BD}$ such that
$D\in\mathop{\rm int}(\angle ABC)$ and $m(\angle ABD)=m(\angle DBC)$.
For $\overleftarrow B\overrightarrow C$ we have
$$c={21-9+4-0\over2(2-0)}=4$$
Then for the tangent vectors we have
$$\eqalign{
T_{BA}&=(0,2)\cr
T_{BC}&=(3,4)\cr
}$$
We need to solve the following for $T_{BD}$.
$$
{\langle T_{BA},T_{BD}\rangle\over\|T_{BA}\|\cdot\|T_{BD}\|}
=
{\langle T_{BD},T_{BC}\rangle\over\|T_{BD}\|\cdot\|T_{BC}\|}
$$
Let $T_{BD}=(y_B,c-x_B)=(3,c)$ where this is a new $c$. Then
$${2c\over2\sqrt{9+c^2}}={9+4c\over5\sqrt{9+c^2}}$$
which reduces to
$$c=9$$
Solving for $r$ we have
$$r=\sqrt{(0-9)^2+3^2}=3\sqrt{10}$$
Therefore the line that bisects the angle is ${}_9L_{3\sqrt{10}}$.

\bigskip
\noindent
Finally, we need a point $D$ on ${}_9L_{3\sqrt{10}}$ such that
the ray $\overrightarrow{BD}$ is the angle bisector.
How do we find $D\in{}_9L_{3\sqrt{10}}$ such that $D\in\mathop{\rm int}(\angle ABC)$?
Well, we know that $D$ has to be on the same side of $\overleftarrow A\overrightarrow B$
as $C$.
Since $\overleftarrow A\overrightarrow B$ is the line $x=0$ and $x_C>0$,
any point $D=(x,y)$ such that $(x,y)\in{}_9L_{3\sqrt{10}}$ and $x>0$ should do.
The easiest choice is $D=(c,r)=(9,3\sqrt{10})$. Hence
$$D=(9,3\sqrt{10})$$

\vfill
\eject

\beginsection Page 109, problem 12.

In $\cal H$ find the angle bisector of $\angle ABC$ if $A=(1,3)$, $B=(1,\sqrt3)$
and $C=(\sqrt3,1)$.

\bigskip
\noindent
Solution: The angle bisector is the ray $\overrightarrow{BD}$ such that
$D\in\mathop{\rm int}(\angle ABC)$ and $m(\angle ABD)=m(\angle DBC)$.
For $\overleftarrow B\overrightarrow C$ we have
$$c={1-3+3-1\over2(\sqrt3-1)}=0$$
Then for the tangent vectors we have
$$\eqalign{
T_{BA}&=(0,\sqrt3-3)\cr
T_{BC}&=(\sqrt3,-1)\cr
}$$
We need to solve the following for $T_{BD}$.
$$
{\langle T_{BA},T_{BD}\rangle\over\|T_{BA}\|\cdot\|T_{BD}\|}
=
{\langle T_{BD},T_{BC}\rangle\over\|T_{BD}\|\cdot\|T_{BC}\|}
$$
Let $T_{BD}=(y_B,c-x_B)=(\sqrt3,c-1)$ where this is a new $c$. Then
$${(\sqrt3-3)(c-1)\over(\sqrt3-3)\sqrt{3+(c-1)^2}}={3-(c-1)\over2\sqrt{3+(c-1)^2}}$$
which reduces to
$$c=2$$
Solving for $r$ we have
$$r=\sqrt{(1-2)^2+3}=2$$
Therefore the line that bisects the angle is ${}_2L_{2}$.

\bigskip
\noindent
Finally, we need a point $D$ on ${}_2L_2$ such that
the ray $\overrightarrow{BD}$ is the angle bisector.
How do we find $D\in{}_2L_2$ such that $D\in\mathop{\rm int}(\angle ABC)$?
Well, we know that $D$ has to be on the same side of $\overleftarrow A\overrightarrow B$
as $C$.
Since $\overleftarrow A\overrightarrow B$ is the line $x=1$ and $x_C>1$,
any point $D=(x,y)$ such that $(x,y)\in{}_2L_2$ and $x>1$ should do.
The easiest choice is $D=(c,r)=(2,2)$. Hence
$$D=(2,2)$$

\vfill
\eject

\beginsection Page 109, problem 16.

In the Taxicab Plane let $A=(0,2)$, $B=(0,0)$, $C=(2,0)$, $Q=(-2,1)$, $R=(-1,0)$
and $S=(0,1)$.
Show that $\overline{AB}\cong\overline{QR}$, $\angle ABC\cong\angle QRS$,
and $\overline{BC}\cong\overline{RS}$.
Is $\overline{AC}\cong\overline{QS}$?

\bigskip
\noindent
Solution: In the Taxicab Plane
$$\eqalign{
AB&=|x_1-x_2|+|y_1-y_2|=|0-0|+|2-0|=2\cr
QR&=|x_1-x_2|+|y_1-y_2|=|-2-(-1)|+|1-0|=2\cr
}$$
Therefore $\overline{AB}\cong\overline{QR}$.

$$\eqalign{
m(\angle ABC)&=\cos^{-1}\left({
\langle A-B,C-B\rangle
\over
\|A-B\|\cdot\|C-B\|
}\right)=\cos^{-1}0=90\cr
m(\angle QRS)&=\cos^{-1}\left({
\langle Q-R,S-R\rangle
\over
\|Q-R\|\cdot\|S-R\|
}\right)=\cos^{-1}0=90\cr
}$$
Therefore $\angle ABC\cong\angle QRS$.

$$\eqalign{
BC&=|x_1-x_2|+|y_1-y_2|=|0-2|+|0-0|=2\cr
RS&=|x_1-x_2|+|y_1-y_2|=|-1-0|+|0-1|=2\cr
}$$
Therefore $\overline{BC}\cong\overline{RS}$.

$$\eqalign{
AC&=|x_1-x_2|+|y_1-y_2|=|0-2|+|2-0|=4\cr
QS&=|x_1-x_2|+|y_1-y_2|=|-2-0|+|1-1|=2\cr
}$$
Therefore $\overline{AC}\not\cong\overline{QS}$.

\vfill
\eject

\beginsection Page 109, problem 19.

Suppose that in the Poincare Plane the line $\ell$ is perpendicular to the line
${}_\alpha L$.
Prove that $\ell$ is a type II line and that its ``c'' parameter is equal to $\alpha$.

\bigskip
\noindent
Solution: Let $A=(\alpha,y_A)$, $B=(\alpha,y_B)$ and $C\in\ell$.
By the hypothesis that ${}_\alpha L\perp\ell$ we have
$$\langle T_{BA},T_{BC}\rangle=0$$
Since ${}_\alpha L$ is a type I line we have $T_{BA}=(0,y_A-y_B)$.
Let $T_{BC}=(x,y)$. Then
$$\langle T_{BA},T_{BC}\rangle=0\cdot x+(y_A-y_B)y=0$$
Since $y_A$ and $y_B$ are on the same vertical line, we have $y_A\ne y_B$ hence
we must have
$$y=0\eqno(1)$$
If $\ell$ were a type I line then we would have $T_{BC}=(0,y)=(0,0)$
which is not allowed since in that case $\|T_{BC}\|=0$ and the angle measure would fail.
Therefore $\ell$ must be a type II line.
Hence
$$T_{BC}=\pm(y_B,c-x_B)\eqno(2)$$
where the sign depends on whether or not $x_B<x_C$. By (1) and (2) we have
$$c-x_B=0$$
Hence
$$c=x_B=\alpha$$

\vfill
\eject

\beginsection Page 109, problem 21.

Prove that any two right angles in a protractor geometry are equal.

\bigskip
\noindent
Counterexample: $A=(1,0)$, $B=(0,0)$, $C=(0,1)$, $D=(0,-1)$.
We have $m(\angle ABC)=m(\angle ABD)=90$.
However, $\overrightarrow{BC}\ne\overrightarrow{BD}$ hence
$\angle ABC\ne\angle ABD$.

\end
