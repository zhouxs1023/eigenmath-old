\chapter{How it works}

Eigenmath is written in the C programming language.
Its central data structure is
an array of one million 16-byte blocks of memory.
\begin{verbatim}
         _______ 
      0 |_______|  \
      1 |_______|   |
      2 |_______|   |
        |       |   |
        .       .   | 16,000,000 bytes
        .       .   |
        .       .   |
        |_______|   |
999,999 |_______|  /
\end{verbatim}
If necessary, Eigenmath will allocate additional memory in increments of one
million blocks.
The maximum amount of memory allocated is 160 million bytes,
or ten times what is shown above.

\newpage

This data structure is fundamentally different from a normal linked
list.

\begin{verbatim}

For example, (a b + c) is built like this:

 _______      _______                                _______
|CONS   |--->|CONS   |----------------------------->|CONS   |
|       |    |       |                              |       |
|_______|    |_______|                              |_______|
    |            |                                      |
 ___v___      ___v___      _______      _______      ___v___
|ADD    |    |CONS   |--->|CONS   |--->|CONS   |    |SYM c  |
|       |    |       |    |       |    |       |    |       |
|_______|    |_______|    |_______|    |_______|    |_______|
                 |            |            |
              ___v___      ___v___      ___v___
             |MUL    |    |SYM a  |    |SYM b  |
             |       |    |       |    |       |
             |_______|    |_______|    |_______|
\end{verbatim}
