\chapter{How it works}

Eigenmath is written in the C programming language.
Its central data structure is
an array of one million 16-byte blocks of memory.

\begin{verbatim}
         _______ 
      0 |_______|  \
      1 |_______|   |
      2 |_______|   |
        |       |   |
        .       .   | 16,000,000 bytes
        .       .   |
        .       .   |
        |_______|   |
999,999 |_______|  /
\end{verbatim}

\medskip
\noindent
If necessary, Eigenmath will allocate additional memory in increments of one
million blocks.
The maximum amount of memory allocated is 160 million bytes,
or ten times what is shown above.

\newpage

\noindent
Each block can contain different types of information.
Blocks are linked together to store mathematical expressions.
For example, the following shows how the expression
{\it A times B plus C} is stored.

\begin{verbatim}
 _______      _______                                _______
|CONS   |--->|CONS   |----------------------------->|CONS   |
|       |    |       |                              |       |
|_______|    |_______|                              |_______|
    |            |                                      |
 ___v___      ___v___      _______      _______      ___v___
|ADD    |    |CONS   |--->|CONS   |--->|CONS   |    |SYMBOL |
|       |    |       |    |       |    |       |    |C      |
|_______|    |_______|    |_______|    |_______|    |_______|
                 |            |            |
              ___v___      ___v___      ___v___
             |TIMES  |    |SYMBOL |    |SYMBOL |
             |       |    |A      |    |B      |
             |_______|    |_______|    |_______|
\end{verbatim}

\medskip
\noindent
Only CONS blocks contain pointers to other blocks.
Every other kind of block is a terminal node.
Fundamentally, this is the same as the S-expression data structure invented by
John McCarthy for the
LISP programming language.

\medskip
\noindent
S-expressions differ from linked lists in a significant way.
In a linked list, blocks contain both data and pointers simultaneously.
For example, this is how one might store {\it A times B} using a linked list.

\begin{verbatim}
 _______       _______       _______
|       |---->|       |---->|       |
|TIMES  |     |SYMBOL |     |SYMBOL |
|       |     |A      |     |B      |
|_______|     |_______|     |_______|
\end{verbatim}

\medskip
\noindent
Now, suppose we want to store an additional expression, {\it A times C.}
Using a linked list, we have to make copies of the TIMES and SYMBOL-A blocks
because of the pointer data.
If we were to rewrite the pointer data we would destroy the original
expression.
However, with S-expressions all we have to do is allocate new CONS
blocks to create new expressions.
With S-expressions, terminal nodes never have to be copied.
We can have a thousand different expressions involving SYMBOL-A and there
will only be a single block containing the data SYMBOL-A.

