\documentclass[12pt,openany]{report}
\usepackage{graphicx}
\begin{document}

\noindent
April 20, 2007

\bigskip
\noindent
The following problem cropped up in geometry homework.

\bigskip
\noindent
Suppose we have
$$(W\ge X)\rightarrow(Y\ge Z)$$
The question is, under what circumstances can we write
$$(W>X)\rightarrow(Y>Z)$$
In other words, what is the implication of
$$((W\ge X)\rightarrow(Y\ge Z))\leftrightarrow((W>X)\rightarrow(Y>Z))$$
%
Here is a naive wff.
$$((A\vee B)\rightarrow(C\vee D))\leftrightarrow(A\rightarrow C)$$
The problem with this wff is that we cannot have both $W>X$ and $W=X$ at the same time.
Ditto for $Y$ and $Z$.
If we throw out those cases then we get the following truth table.

\begin{verbatim}
W>X W=X Y>Z Y=Z
 0   0   0   0     1
 0   0   0   1     1
 0   0   1   0     1
 0   1   0   0     0
 0   1   0   1     1
 0   1   1   0     1
 1   0   0   0     1
 1   0   0   1     0
 1   0   1   0     1
\end{verbatim}

\noindent
So the wff fails for the following two cases.

\begin{itemize}

\item[1.]
$W=X$ and $Y<Z$

\item[2.]
$W>X$ and $Y=Z$.

\end{itemize}

\noindent
It turns out that if we add the additional constraint
$$W=X\quad\hbox{if and only if}\quad Y=Z$$
then the wff becomes a tautology.

\newpage

\noindent
Here is the C program.

\begin{verbatim}
#define IMPLIES(x, y) (!(x) || (y))
main()
{
        int a, b, c, d, p, q;
        printf("W>X W=X Y>Z Y=Z\n");
        for (a = 0; a < 2; a++)
        for (b = 0; b < 2; b++)
        for (c = 0; c < 2; c++)
        for (d = 0; d < 2; d++) {
                if (a == 1 && b == 1)
                        continue;
                if (c == 1 && d == 1)
                        continue;
                p = IMPLIES(a || b, c || d);
                q = IMPLIES(a, c);
                printf(" %d   %d   %d   %d     %d\n", a, b, c, d, p == q);
        }
}
\end{verbatim}

\end{document}
