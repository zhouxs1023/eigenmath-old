\documentclass{article}
\author{G.~Weigt}
\title{Note on SMP safety}
\date{\today}
\begin{document}
\maketitle

\noindent
Here is a simple example of some driver code and the changes that should be made
to ensure SMP safety.
The following driver demo simply increments a counter on interrupt.
The driver also has an ioctl that can be called to clear the counter.

\begin{verbatim}
int counter;

// interrupt service routine
irqreturn_t counter_isr(int irq, void *p, struct pt_regs *r)
{
     counter++;
     return IRQ_HANDLED;
}

// to here from a driver ioctl
void clear_counter(void)
{
     counter = 0;
}
\end{verbatim}

\noindent
On the i386 as well as many other machines, \verb$counter = 0$ is atomic but
\verb$counter++$ is not.
On many machines, \verb$counter++$ compiles to several machine instructions that do
a read-modify-write.
This works OK on a single processor because the interrupt
occurs either before or after the atomic operation \verb$counter = 0$.
However, under SMP there is a race condition because
\verb$counter++$ and \verb$counter = 0$ can occur simultaneously on different
processors.
For example, suppose the ISR is serviced on Processor 1 and the counter clear
occurs on Processor 2. Then at the machine instruction level we can have the
following, where \verb$mem$ corresponds to the memory contents of
\verb$counter$.

\bigskip
\begin{tabular}{lll}
Processor 1 & Processor 2 \\
\verb$reg <- mem$ & \verb$reg <- 0$ \\
\verb$reg <- reg+1$ & \verb$mem <- reg$ \\
\verb$mem <- reg$ & & The clear is undone! \\
\end{tabular}

\newpage

\noindent
The solution is to use Linux's \verb$atomic_t$ data type as shown below.

\begin{verbatim}
#include <asm/atomic.h>

atomic_t counter;

// interrupt service routine
irqreturn_t counter_isr(int irq, void *p, struct pt_regs *r)
{
     atomic_add(1, &counter);
     return IRQ_HANDLED;
}

// to here from a driver ioctl
void clear_counter(void)
{
     atomic_set(&counter, 0);
}
\end{verbatim}

\noindent
The fact that \verb$atomic.h$ is under the \verb$asm$ directory attests to the fact
that atomic operations are processor-specific.
One other detail, I have used \verb$atomic_add$ instead of \verb$atomic_inc$
because I see the following at the bottom of \verb$atomic.h$:

\begin{verbatim}
250 /* Atomic operations are already serializing on x86 */
251 #define smp_mb__before_atomic_dec()     barrier()
252 #define smp_mb__after_atomic_dec()      barrier()
253 #define smp_mb__before_atomic_inc()     barrier()
254 #define smp_mb__after_atomic_inc()      barrier()
\end{verbatim}
I don't understand the barrier issue, so the safe thing to do is use \verb$atomic_add$.

\newpage

\noindent
The following test program demonstrates that \verb$counter++$ compiles to
read-modify-write code on x86 machines.

\begin{verbatim}
int counter;
main()
{
        counter++;
}
\end{verbatim}
After compiling with \verb$gcc -S test.c$ we find the following code snippet in
the resulting file \verb$test.s$:

\begin{verbatim}
.LCFI1:
        movl    counter(%rip), %eax
        addl    $1, %eax
        movl    %eax, counter(%rip)
        leave
        ret
\end{verbatim}



\end{document}