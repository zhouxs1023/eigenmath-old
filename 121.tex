\beginsection Study Guide for Exam \#2


\bigskip
\item{1.} What is the composition of ``normal air?''
\item{} 79\% Nitrogen
\item{}20\% Oxygen
\item{}1\% CO${}_2$ and other gases

\bigskip
\item{2.} Be able to describe the troposphere.
Where is the troposphere located?

\item{} The troposphere extends from the Earth's surface
up to an altitude of 12--15 kilometers.
This is the region of the atmosphere that contains the air we breathe.

\bigskip
\item{3.} Where is the stratosphere located? Why is it important?
\item{} The stratosphere is located from the upper limit of
the troposhere up to an altitude of approximately 50 kilometers.
The stratosphere is where the ozone layer is located.
The ozone layer protects plants and animals from the
intense ultraviolet radiation from the Sun.

\bigskip
\item{4.} Be able to list the chemicals (products) emitted from
combustion (burning).
Be specific relative to the chemicals (products) released to the air.
\item{} carbon dioxide
\item{} carbon monoxide
\item{} nitrogen oxides
\item{} sulfur oxides
\item{} particulates
\item{} hydrocarbons
\item{} water vapor

\bigskip
\item{5.} Be able to describe the sources of air pollutants other
than combustion. Be able to give a general description of the chemicals
emitted to the air from agricultural activities, the chemical
industry and the mining industry.
\item{} Agriculture:
\item{} Chemical industry:

\bigskip
\item{6.} Be able to describe the anatomy of the respiratory system
and the mechanics of respiration.
What are the roles of the upper airways and the alveoli?
\item{} The primary function of the respiratory system is to exchange gases.
\item{} The respiratory system consists of the oral and nasal cavities,
the trachea, the bronchial tree and the alveoli.

\bigskip
\item{7.} Be able to discuss the respiratory defense systems against:
\item{} Large particulates --- Trapped by mucous and transported out by cilia.
\item{} Very small particulates --- Organic ones are trapped by alveolar macrophages.
\item{} Water soluble gases --- Trapped by mucous and transported out by cilia.
\item{} Water insoluble gases --- No defense except for dilution in the air already in the lungs.
Nitrogen dioxide (${\rm NO}_2$) is not water soluble.
\item{} Organic vapors --- No defense.

\vfill
\eject

\item{8.} Be able to describe the following respiratory illnesses
and the effect of each on the efficiency of the respiratory process.
\item{} Respiratory irritation (acute and chronic) ---
Irritation is caused by inhaling irritant gases.
Acute exposure results in coughing and sneezing.
Chronic exposure impairs respiratory defenses.
\item{} Fibrosis --- Alveolar tissues become stiff from scarring.
Efficiency of gas exchange is reduced.
\item{} Emphysema --- Alveoli are destroyed. Efficiency of gas exchange is reduced.
\item{} Respiratory (pulmonary) edema --- Abnormal collection of fluids in the lungs.
Can cause death due to asphyxiation.
\item{} Bronchitis --- Infection and inflammation of the bronchi causing
shortness of breath.
\item{} Asthma --- Is a result of constriction of the bronchioles.
Causes a decrease in air flow and hence a decrease in gas exchange.

\bigskip
\item{9.} Be able to describe the sources and health effects of the
following asphyxiating agents:
\item{} Carbon monoxide ---
\item{} Hydrogen sulfide ---
\item{} Cyanide ---












\end
