\beginsection Study Guide for Exam \#2


\bigskip
\item{1.} What is the composition of ``normal air?''
\item{} 79\% Nitrogen
\item{}20\% Oxygen
\item{}1\% CO${}_2$ and other gases

\bigskip
\item{2.} Be able to describe the troposphere.
Where is the troposphere located?

\item{} The troposphere extends from the Earth's surface
up to an altitude of 12--15 kilometers.
This is the region of the atmosphere that contains the air we breathe.

\bigskip
\item{3.} Where is the stratosphere located? Why is it important?
\item{} The stratosphere is located from the upper limit of
the troposhere up to an altitude of approximately 50 kilometers.
The stratosphere is where the ozone layer is located.
The ozone layer protects plants and animals from the
intense ultraviolet radiation from the Sun.

\bigskip
\item{4.} Be able to list the chemicals (products) emitted from
combustion (burning).
Be specific relative to the chemicals (products) released to the air.
\item{} carbon dioxide
\item{} carbon monoxide
\item{} nitrogen oxides
\item{} sulfur oxides
\item{} particulates
\item{} hydrocarbons
\item{} water vapor

\bigskip
\item{5.} Be able to describe the sources of air pollutants other
than combustion. Be able to give a general description of the chemicals
emitted to the air from agricultural activities, the chemical
industry and the mining industry.
\item{} Agriculture: grain dust, obnoxious odors.
\item{} Chemical industry:

\bigskip
\item{6.} Be able to describe the anatomy of the respiratory system
and the mechanics of respiration.
What are the roles of the upper airways and the alveoli?
\item{} The primary function of the respiratory system is to exchange gases.
\item{} The respiratory system consists of the oral and nasal cavities,
the trachea, the bronchial tree and the alveoli.

\bigskip
\item{7.} Be able to discuss the respiratory defense systems against:
\item{} Large particulates --- Trapped by mucous and transported out by cilia.
\item{} Very small particulates --- Organic ones are trapped by alveolar macrophages.
\item{} Water soluble gases --- Trapped by mucous and transported out by cilia.
\item{} Water insoluble gases --- No defense except for dilution in the air already in the lungs.
Nitrogen dioxide (${\rm NO}_2$) is not water soluble.
\item{} Organic vapors --- No defense.

\vfill
\eject

\item{8.} Be able to describe the following respiratory illnesses
and the effect of each on the efficiency of the respiratory process.

\itemitem{8.1}
{\it Respiratory irritation (acute and chronic).}
Irritation is caused by inhaling irritant gases.
Acute exposure results in coughing and sneezing.
Chronic exposure impairs respiratory defenses.

\itemitem{8.2}
{\it Fibrosis.}
Alveolar tissues become stiff from scarring.
Efficiency of gas exchange is reduced.

\itemitem{8.3}
{\it Emphysema.}
Alveoli are destroyed. Efficiency of gas exchange is reduced.

\itemitem{8.4}
{\it Respiratory (pulmonary) edema.}
Abnormal collection of fluids in the lungs.
Can cause death due to asphyxiation.

\itemitem{8.5}
{\it Bronchitis.}
Infection and inflammation of the bronchi causing
shortness of breath.

\itemitem{8.6}
{\it Asthma.}
Is a result of constriction of the bronchioles.
Causes a decrease in air flow and hence a decrease in gas exchange.

\bigskip
\item{9.} Be able to describe the sources and health effects of the
following asphyxiating agents:

\itemitem{9.1}
{\it Carbon monoxide.}
The source is combustion.
Carbon monoxide is also present in coal mines.
Carbon monoxide binds to hemoglobin and blocks the transport
of oxygen in the body.

\itemitem{9.2}
{\it Hydrogen sulfide.}
Hydrogen sulfide is released by decaying human and animal waste.
Hydrogen sulfide can paralyze the respiratory control
center in the brain.

\itemitem{9.3}
{\it Cyanide.}
Cyanide blocks cell respiration.
It is found in plants, including apple seeds.
Hydrogen cyanide is produced by internal combustion engines.

\bigskip
\item{10.}
Be able to list the sources of the following pollutants and the
health effects resulting from exposure to each.

\itemitem{10.1}
{\it Sulfur dioxide.}
From sulfur in fossil fuels, mainly coal.
Is water soluble, trapped by mucous.
Chronic exposure can cause fibrosis, emphysema.
Acute exposure can cause pulmonary edema.

\itemitem{10.2}
{\it Nitrogen dioxide.}
From combustion.
Not water soluble, not trapped by mucous.
Cuases corrosion of the lungs, damages alveoli.

\itemitem{10.3}
{\it Hydrocarbon vapors (petroleum distillates).}
From evaporated fuels and solvents, incomplete
combustion of petroleum fuels.
No respiratory defense.
Depresses central nervous system function.

\itemitem{10.4}
{\it Particulates.}
From combustion, agriculture, construction and mining.
Metallic particulates are toxic.
If a particle is not toxic it may still catalyze gases
and create toxic chemicals in the lungs.
Very small airborne particles can cause childhood asthma.

\itemitem{10.5}
{\it Ozone (ground level).}
From lightning and photochemical reactions that generate smog.
Not water soluble, no respiratory defense.
Can destroy cilia, alveolar membranes and macrophages.
Damage to alveoli can lead to fibrosis and emphysema.

\itemitem{10.6}
{\it Silicates.}

\itemitem{10.7}
{\it Asbestos.}
Naturally occurring mineral fiber.
Causes asbestosis, plaque in the thoracic cavity, lung cancer.

\bigskip
\item{11.}
Be able to describe the process of electrical generation in a
coal-fired power plant.
Be able to discuss control of the primary emissions from this
type of electrical power plant (particulates, sulfur dioxide,
nitrogen oxides).

\item{}
The coal is burned to heat water in pipes.
The water turns to steam which drives turbines.
The turbines drive generators.
The generator spins a magnet that induces electric
current in a coil.
The electric current passes through transformers that
distribute electricity throughout the service area.

\item{}
{\it particulates} Controlled through the use
of electrostatic precipitators.

\item{}
{\it sulfur dioxide}
Can be controlled by using western (low sulfur) coal,
washing the coal, or using scrubbers.

\item{}
{\it nitrogen oxides}
Catalytic reduction.

\bigskip
\item{12.}
Be able to describe the health effects related to smoking.

\item{} Emphysema, cancer, heart disease.

\vfill
\eject

\item{13.} Be able to describe the environmental and health
effects of photochemical smog.

\item{}
What are the precursors for smog formation?
Nitrogen oxides,volatile organic compounds (VOCs), sunlight, heat.

\item{}
What are the components of smog that are health concerns?
Ozone and oxidized hydrocarbons are very strong respiratory and eye irritants.

\item{}
Why is smog a summer event in most of the US?
Sunlight and heat are needed for form smog.
Temperature inversions trap the smog near the ground causing
the smog concentration to increase.

\item{}
Describe the process of smog formation.
Nitrogen oxides react with VOCs in the presence of sunlight
to produce ozone and oxidized hydrocarbons.

\item{}
Be able to describe the purpose of the Air Quality Index (AQI).
The AQI rates air pollution on a daily basis
of many metropolitan areas in the US.
It uses a scale of 0 to 500.
Zero is good air, 500 is bad.

\bigskip
\item{14.}

\bigskip
What is the greenhouse effect?
Visible light from the Sun is absorbed by the earth's surface.
The earth's surface radiates heat which is trapped by the troposphere.
The trapped heat keeps the earth warm.

\bigskip
Describe the sources of the greenhouse gases carbon dioxide, methane
and CFCs.
Carbon dioxide comes from combustion and plant and animal respiration.
Methane comes from biological processes such as decay and cattle digestion.
CFCs used to come from aerosol propellants until they were banned for that use.
CFCs also come from refrigerant gases but they are being phased
out of this application.

\bigskip
What are the predicted problems for global warming?
Melting polar ice caps will increase ocean levels
and flood coastal areas.
Many temperate areas will become deserts thus reducing food production.
Increased incidence in tropical diseases.

\bigskip
What is the role of stratospheric ozone?
Prevents harmful UV radiation from reaching the earth's surface.

\bigskip
What chemicals are responsible for ozone depletion?
Chlorofluorocarbons (CFCs).

\bigskip
What is the ozone hole?
A temporary decrease in ozone concentration in the stratosphere over
Antartica.

\bigskip
Where is the ``hole'' located and at what time of year?
Over Antartica during late winter and early spring.
There is no sunlight so no new ozone is generated.
Chlorine atoms are frozen in ice crystals.
Air circulation is a vortex so no new air is added.
As Spring arrives temperatures increase.
Chlorine atoms are freed from the ice.
They react with the ozone and destroy it by converting it to oxygen.
Hence CFCs gain the upper hand in depleting the ozone.

\bigskip
Why is there concern over the possible depletion of
stratospheric ozone?
Skin cancer, damage to plant life and marine plankton.

\end