\magnification=1200

\noindent
{\it George Weigt -- Advanced Calculus Homework \#5}

\beginsection Q1

Obtain the Jacobian determinant $\displaystyle{\partial(u,v)\over\partial(x,y)}$ for
$$\eqalign{
u&=x^3-3xy^2\cr
v&=3x^2y-y^3\cr
}$$
We have
$$\eqalign{
{\partial(u,v)\over\partial(x,y)}
&=\left|\matrix{
{\partial u\over\partial x} & {\partial u\over\partial y}\cr
\cr
{\partial v\over\partial x} & {\partial v\over\partial y}\cr
}\right|
=\left|\matrix{
3x^2-3y^2 & -6xy\cr
\cr
6xy & 3x^2-3y^2\cr
}\right|
=(3x^2-3y^2)^2+36x^2y^2\cr
\cr
&=9x^4+18x^2y^2+9y^4
}$$

\beginsection Q2A

Given the surface
$$x^2+y^2+z^2=9$$
find the normal vector at the point $P=(2,2,1)$.
\medskip\noindent
We have
$$F(x,y,z)=x^2+y^2+z^2-9=0,\qquad
\nabla F=\left(
{\partial F\over\partial x},
{\partial F\over\partial y},
{\partial F\over\partial z}
\right)
=(2x,2y,2z)$$
The normal vector at $P=(2,2,1)$ is
$${\bf n}=\nabla F\bigg|_P=(4,4,2)$$

\beginsection Q2B

Given the surface
$$e^{x^2+y^2}-z^2=0$$
find the normal vector at $P=(0,0,1)$.
\medskip\noindent
We have
$$F(x,y,z)=e^{x^2+y^2}-z^2=0,\qquad
\nabla F=\left(
{\partial F\over\partial x},
{\partial F\over\partial y},
{\partial F\over\partial z}
\right)
=(2xe^{x^2+y^2},2ye^{x^2+y^2},-2z)$$
The normal vector at $P=(0,0,1)$ is
$${\bf n}=\nabla F\bigg|_P=(0,0,-2)$$

\beginsection Q3A

Show that the gravitational field
$${\bf F}=-k{Mm\over r^2}{{\bf r}\over r}$$
is the gradient of the scalar
$$f={kMm\over r}$$
The first step is to work out the partial derivatives for $1/r=(x^2+y^2+z^2)^{-1/2}$.
We have
$${\partial(x^2+y^2+z^2)^{-1/2}\over\partial x}=(-1/2)(2x)(x^2+y^2+z^2)^{-3/2}=-{x\over r^3}$$
For the remaining variables $y$ and $z$ we have something similar. Therefore
$$
{\partial\over\partial x}\left({1\over r}\right)=-{x\over r^3},\qquad
{\partial\over\partial y}\left({1\over r}\right)=-{y\over r^3},\qquad
{\partial\over\partial z}\left({1\over r}\right)=-{z\over r^3}
$$
Consequently for $\nabla f$ we have
$$\nabla f=\left(
{\partial f\over\partial x},{\partial f\over\partial y},{\partial f\over\partial z}
\right)=-{kMm\over r^3}(x,y,z)
$$
We have ${\bf r}=(x,y,z)$ therefore
$${\bf F}=-k{Mm\over r^2}{{\bf r}\over r}=-{kMm\over r^3}(x,y,z)=\nabla f$$

\beginsection Q3B

Show that the force field
$${\bf F}={
2(x^2-y^2-1){\bf i}+4xy{\bf j}
\over
[(x+1)^2+y^2][(x-1)^2+y^2]
}$$
is the gradient of the scalar
$$f=\log{
\sqrt{(x-1)^2+y^2}
\over
\sqrt{(x+1)^2+y^2}
}$$
Let
$$\eqalign{
u&=(x-1)^2+y^2\cr
v&=(x+1)^2+y^2\cr
}$$
Then
$$\eqalign{
{\partial\over\partial x}(\hbox{$1\over2$}\log u-\hbox{$1\over 2$}\log v)
&={1\over2u}{\partial u\over\partial x}-{1\over2v}{\partial v\over\partial x}\cr
&={x-1\over u}-{x+1\over v}\cr
&={(x-1)v-(x+1)u\over uv}\cr
&={2x^2-2y^2-2\over uv}\cr
%&={(x-1)(x+1)(x+1)+(x-1)y^2-(x+1)(x-1)(x-1)-(x+1)y^2\over uv}\cr
%
{\partial\over\partial y}(\hbox{$1\over2$}\log u-\hbox{$1\over 2$}\log v)
&={1\over2u}{\partial u\over\partial y}-{1\over2v}{\partial v\over\partial y}\cr
&={y\over u}-{y\over v}\cr
&={y(v-u)\over uv}\cr
%&={y((x+1)^2-(x-1)^2))\over uv}\cr
&={4xy\over uv}
}$$
Therefore
$$\nabla f=\left(
{2x^2-2y^2-2\over uv},
{4xy\over uv}
\right)
=
{
2(x^2-y^2-1){\bf i}+4xy{\bf j}
\over
[(x-1)^2+y^2][(x+1)^2+y^2]
}={\bf F}
$$

\beginsection Q4A

Find the divergence of the following vector field
$${\bf v}=xy{\bf i}+yz{\bf j}+zx{\bf k}$$
We have
$${\bf v}=(xy,yz,zx)$$
$$\nabla\cdot{\bf v}
={\partial xy\over\partial x}+{\partial yz\over\partial y}+{\partial zx\over\partial z}
=y+z+x
$$

\beginsection Q4B

Find the divergence of the following vector field
$${\bf v}=\nabla\times{\bf u}$$
where ${\bf u}=(P,Q,R)$ with $P$, $Q$ and $R$ having continuous second-order
partial derivatives.
\medskip\noindent
For a rectangular coordinate system we have
$${\bf v}=\nabla\times{\bf u}=
\left({\partial R\over\partial y}-{\partial Q\over\partial z}\right){\bf i}+
\left({\partial P\over\partial z}-{\partial R\over\partial x}\right){\bf j}+
\left({\partial Q\over\partial x}-{\partial P\over\partial y}\right){\bf k}
$$
Then
$$\eqalign{
\nabla\cdot{\bf v}&=
{\partial\over\partial x}
\left({\partial R\over\partial y}-{\partial Q\over\partial z}\right)+
{\partial\over\partial y}
\left({\partial P\over\partial z}-{\partial R\over\partial x}\right)+
{\partial\over\partial z}
\left({\partial Q\over\partial x}-{\partial P\over\partial y}\right)\cr
&=
{\partial^2 R\over\partial x\partial y}-{\partial^2 Q\over\partial x\partial z}+
{\partial^2 P\over\partial y\partial z}-{\partial^2 R\over\partial y\partial x}+
{\partial^2 Q\over\partial z\partial x}-{\partial^2 P\over\partial z\partial y}\cr
&=0\cr
}$$

\beginsection Q5A

Find the curl of the following vector field.
$${\bf v}=xy{\bf i}+yz{\bf j}+zx{\bf k}$$
Solution:
$$\nabla\times{\bf v}=
\left({\partial zx\over\partial y}-{\partial yz\over\partial z}\right){\bf i}+
\left({\partial xy\over\partial z}-{\partial zx\over\partial x}\right){\bf j}+
\left({\partial yz\over\partial x}-{\partial xy\over\partial y}\right){\bf k}=
-y{\bf i}-z{\bf j}-x{\bf k}
$$

\beginsection Q5B

Find the curl of the following vector field.
$${\bf v}=\nabla f$$
where $f$ is $f(x,y,z)$ with continuous second-order partial derivatives.
\medskip\noindent
We have
$$\nabla f=
\left({\partial f\over\partial x},{\partial f\over\partial y},{\partial f\over\partial z}\right)$$
hence
$$\nabla\times(\nabla f)=
\left({\partial^2 f\over\partial y\partial z}-{\partial^2 f\over\partial z\partial y}\right){\bf i}+
\left({\partial^2 f\over\partial z\partial x}-{\partial^2 f\over\partial x\partial z}\right){\bf j}+
\left({\partial^2 f\over\partial x\partial y}-{\partial^2 f\over\partial y\partial x}\right){\bf k}
=(0,0,0)
$$

\beginsection Q6

Why can't we find the normal vector for the surface
$$x^2+y^2-z^2=0$$
at the point $(0,0,0)$?
\medskip\noindent
The surface is a cone and the point $(0,0,0)$ is a singular point.
This can be seen if we ignore the $y$-coordinate for a moment.
We have $x^2-z^2=0$, or in other words, $z=\sqrt{x^2}=|x|$.
The plot of absolute value is a V.

\beginsection Q7

Show that
$$\eqalignno{
\nabla\times(\nabla\times\bf H)&=-{\partial^2 {\bf H}\over\partial t^2}&(5)\cr
\cr
\nabla^2{\bf H}&={\partial^2{\bf H}\over\partial t^2}&(6)\cr
}$$
\medskip\noindent
For (5) we are given
$$\nabla\times{\bf H}={\partial {\bf E}\over\partial t},\qquad
\nabla\times{\bf E}=-{\partial{\bf H}\over\partial t}
$$
hence
$$\eqalign{
\nabla\times(\nabla\times{\bf H})&=
\left({\partial^2E_z\over\partial y\partial t}-{\partial^2E_y\over\partial z\partial t}\right){\bf i}+
\left({\partial^2E_x\over\partial z\partial t}-{\partial^2E_z\over\partial x\partial t}\right){\bf j}+
\left({\partial^2E_y\over\partial x\partial t}-{\partial^2E_x\over\partial y\partial t}\right){\bf k}\cr
&={\partial\over\partial t}\left[
\left({\partial E_z\over\partial y}-{\partial E_y\over\partial z}\right){\bf i}+
\left({\partial E_x\over\partial z}-{\partial E_z\over\partial x}\right){\bf j}+
\left({\partial E_y\over\partial x}-{\partial E_x\over\partial y}\right){\bf k}
\right]\cr
&={\partial\over\partial t}\bigg[\nabla\times{\bf E}\bigg]\cr
&={\partial\over\partial t}\left[-{\partial{\bf H}\over\partial t}\right]\cr
&=-{\partial^2{\bf H}\over\partial t^2}\cr
}$$
For (6) we are given
$$\nabla\times(\nabla\times{\bf v})=\nabla(\nabla\cdot{\bf v})-\nabla^2{\bf v},\qquad
\nabla\cdot{\bf H}=0$$
Consequently, we have
$$\nabla^2{\bf H}=\nabla(\nabla\cdot{\bf H})-\nabla\times(\nabla\times{\bf H})=
{\partial^2{\bf H}\over\partial t^2}$$
where the last step follows from $\nabla(0)=(0,0,0)$ and from equation (5) above.

\end
