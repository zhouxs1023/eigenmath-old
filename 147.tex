\beginsection 4.

A box contains 20 red, 10 white and 30 blue balls.
Eight balls are randomly selected without replacement.

\beginsection 4. (a)

Find the probability that 4 red, 1 white and 3
blue balls are selected.
\medskip\noindent
Solution: See Note 5, page 7.
$$P={C(20,4)\times C(10,1)\times C(30,3)\over C(60,8)}=
{23800\over 309573}=0.0769$$

\beginsection 4. (b)

Find the expected number of red balls.
\medskip\noindent
Solution: This is a hypergeometric distribution $n=8$, $D=20$ and $N=60$.
From Note 3 page 8 we have
$$E[red]=n\times{D\over N}={8\over3}=2.6667$$
%Method 2: First write down the joint probability mass function.
%$$p(r,w,b)={C(20,r)\times C(10,w)\times C(30,b)\over C(60,8)}$$
%with
%$$r,w,b\in Z,\quad
%0\le r\le20,\quad
%0\le w\le10,\quad
%0\le b\le30,\quad
%r+w+b=8
%$$
%Next, with a computer, sum over all random variables to get the expectation value.
%$$E[r]=\sum_{r=0}^8\sum_{w=0}^{8-r}rp(r,w,8-r-w)={8\over3}=2.6667$$

\beginsection 4. (c)

Find the correlation coefficient between the number of red balls and the
number of blue balls.
\medskip\noindent
Solution:
From Note 5 we have
$$\rho=-\sqrt{
D_{red}\times D_{blue}
\over (N-D_{red})\times(N-D_{blue})
}=-\sqrt{
20\times30\over(60-20)\times(60-30)
}=-\sqrt{600\over1200}=-0.7071
$$

\beginsection 4. (d)

Find the expected number of red balls given 2 white balls in the sample.
\medskip\noindent
Solution: For exactly 2 white balls
$$P(w{=}2)={C(10,2)\times C(50,6)/C(60,8)}$$
$$
P(r|w{=}2)
={P(r\cap w{=}2)\over P(w{=}2)}
={
{C(20,r)\times C(10,2)\times C(30,6-r)/C(60,8)}
\over
{C(10,2)\times C(50,6)/C(60,8)}
}={C(20,r)\times C(30,6-r)\over C(50,6)}$$
Actually, this result can be obtained by simple inspection.
If the sample contains exactly 2 white, then the remaining 6
must have been drawn from the population of 20 red and 30 blue.
We have
$$n=6,\quad D=20,\quad N=50$$
hence
$$E[red]=6\times{20\over50}={12\over5}=2.4$$


\beginsection 4. (e)

Find the expected number of red balls given 2 blue balls in the sample.
\medskip\noindent
Solution: As above...
$$P(b{=}2)={C(30,2)\times C(30,6)/C(60,8)}$$
$$
P(r|b{=}2)
={P(r\cap b{=}2)\over P(b{=}2)}
={
{C(20,r)\times C(10,6-r)\times C(30,2)/C(60,8)}
\over
{C(30,2)\times C(30,6)/C(60,8)}
}
={C(20,r)\times C(10,6-r)\over C(30,6)}$$
Again, this result can be obtained by simple inspection.
If the sample contains exactly 2 blue, then the remaining 6
must have been drawn from the population of 20 red and 10 yellow.
We have
$$n=6,\quad D=20,\quad N=30$$
hence
$$E[red]=6\times{20\over30}=4$$


\end