\parindent=0pt
\nopagenumbers

{\bf Problem A.}
Let $a\in C$. Prove: If $a$ is an algebraic number over $Q$,
then $a+1$ is also an algebraic number over $Q$.
\medskip
{\bf Proof.}
Let $a\in C$. Assume that $a$ is an algebraic number over $Q$.
Then there is a nonzero rational polynomial $p(x)$
such that $p(a)=0$.
By lemma 1 there is also an integral polynomial $q(x)\in Z$
such that $q(a)=0$.
Then by lemma 2 the polynomial $q(x-1)$ is an integral polynomial.
Note that $Z[x]\subset Q[x]$ so it follows that $q(x-1)\in Q[x]$.
We have $q(a+1-1)=q(a)=0$ therefore $a+1$ is an algebraic number over $Q$.

\bigskip
\bigskip

{\bf Problem B.}
For each polynomial below, determine whether it is irreducible over $Q$.
\bigskip
{\bf 1.} $p(x)=x^2-5x+4$.
By the rational zero theorem the possible roots
are $\pm1$, $\pm2$, $\pm4$.
We have $p(1)=1-5+4=0$ therefore the polynomial is reducible, i.e.
$p(x)=(x-1)(x-4)$.
\bigskip
{\bf 2.} $a(x)=x^7-3x^4+12$.
By Eisenstein's criterion with the prime $p=3$
we have $p$ does not divide 1, $p$ divides $-3$,
$p$ divides 12 and $p^2$ does not divide 12.
Therefore $a(x)$ is irreducible over $Q$.
\bigskip
{\bf 3.} $p(x)=x^3+9$. By the rational zeroes theorem the possible roots
are $\pm1$, $\pm3$, $\pm9$.
We have $p(1)=10$, $p(-1)=8$,
$p(3)=36$, $p(-3)=-18$, $p(9)=738$, $p(-9)=-720$.
Therefore $p(x)$ is irreducible.

\bigskip
\bigskip

{\bf Problem C.}
Find the degree of $1-\root3\of2$ over $Q$.
\medskip
First, construct a polynomial that has a root at $1-\root3\of2$.
$$x-1=-\root3\of2$$
$$(x-1)^3=-2$$
$$(x-1)(x^2-2x+1)=-2$$
$$x^3-2x^2+x-x^2+2x-1=-2$$
$$x^3-3x^2+3x-1=-2$$
$$x^3-3x^2+3x+1=0$$
Let $p(x)=x^3-3x^2+3x+1$.
By the rational zeroes theorem the possible roots are $\pm1$.
We have $p(1)=2$ and $p(-1)=-6$ therefore $p(x)$ is not reducible.
Since by construction $1-\root3\of2$ is a zero of $p(x)$ and $p(x)$
is not reducible, the degree of $1-\root3\of2$ is 3.


\end
