\magnification=1200

\noindent
{\it George Weigt -- Geometry Homework \#8}

\beginsection Page 96, problem 1.

Let $A=(0,1)$, $B=(0,5)$, and $C=(3,4)$ be points in the Poincare Plane.
Find the sum of the measures of the angles $\triangle ABC$.

\bigskip
\noindent
Solution: For $\overleftarrow A\overrightarrow B$ we have $x_A=x_B$ therefore
$\overleftarrow A\overrightarrow B$ is a type I line.

\medskip
\noindent
For $\overleftarrow A\overrightarrow C$ we have $x_A\ne x_C$ therefore
$\overleftarrow A\overrightarrow C$ is a type II line with
$$c_{AC}={y_C^2-y_A^2+x_C^2-x_A^2\over2(x_C-x_A)}=4$$

\medskip
\noindent
For $\overleftarrow B\overrightarrow C$ we have $x_B\ne x_C$ therefore
$\overleftarrow B\overrightarrow C$ is a type II line with
$$c_{BC}={y_C^2-y_B^2+x_C^2-x_B^2\over2(x_C-x_B)}=0$$

\medskip
\noindent
For the tangent lines we have
$$\eqalign{
T_{AB}&=(0,y_B-y_A)=(0,4)\cr
T_{BA}&= -T_{AB}=(0,-4)\cr
T_{AC}&=(y_A,c_{AC}-x_A)=(1,4)\qquad\qquad\hbox{(note: $x_A<x_C$)}\cr
T_{CA}&=-(y_C,c_{AC}-x_C)=(-4,-1)\cr
T_{BC}&=(y_B,c_{BC}-x_B)=(5,0)\qquad\qquad\hbox{(note: $x_B<x_C$)}\cr
T_{CB}&=-(y_C,c_{BC}-x_C)=(-4,3)\cr
}$$

\medskip
\noindent
For the angles we have
$$\eqalign{
\theta_A&=\cos^{-1}\left(
{\langle T_{AB},T_{AC}\rangle\over\|T_{AB}\|\cdot\|T_{AC}\|}
\right)=\cos^{-1}{16\over4\sqrt{17}}\approx 14.04\cr
%
\theta_B&=\cos^{-1}\left(
{\langle T_{BA},T_{BC}\rangle\over\|T_{BA}\|\cdot\|T_{BC}\|}
\right)=\cos^{-1}{0\over20}=90\cr
%
\theta_C&=\cos^{-1}\left(
{\langle T_{CA},T_{CB}\rangle\over\|T_{CA}\|\cdot\|T_{CB}\|}
\right)=\cos^{-1}{13\over\sqrt{17}\cdot5}\approx50.91\cr
}$$

\medskip
\noindent
Finally,
$$\theta_A+\theta_B+\theta_C\approx154.95$$

\vfill
\eject

\beginsection Page 96, problem 3.

Let $A=(5,1)$, $B=(8,4)$, and $C=(1,3)$ be points in the Poincare Plane.
Find the sum of the measures of the angles $\triangle ABC$.

\bigskip
\noindent
$$\eqalign{
c_{AB}&={y_B^2-y_A^2+x_B^2-x_A^2\over2(x_B-x_A)}=9\cr
c_{AC}&={y_C^2-y_A^2+x_C^2-x_A^2\over2(x_C-x_A)}=2\cr
c_{BC}&={y_C^2-y_B^2+x_C^2-x_B^2\over2(x_C-x_B)}=5\cr
}$$

\bigskip
\noindent
$$\eqalign{
T(A,B)&=(1,4)\cr
T(B,A)&=(-4,-1)\cr
T(A,C)&=(-1,3)\cr
T(C,A)&=(3,1)\cr
T(B,C)&=(-4,3)\cr
T(C,B)&=(3,4)\cr
}$$

\bigskip
$$\eqalign{
\theta_A&\approx32.47\cr
\theta_B&\approx50.91\cr
\theta_C&\approx34.70\cr
}$$

\bigskip
$$\theta_A+\theta_B+\theta_C\approx118.1$$

\vfill
\eject

\beginsection Page 96, problem 4.

Let $m$ be an angle measure for $\{S,L,d\}$ based on $\alpha$.
Let $t>0$ and define $m_t$ by
$$m_t(\angle ABC)=t\cdot m(\angle ABC)$$
Prove that $m_t$ is an angle measure for $\{S,L,d\}$.
What value is $m_t$ based on?

\bigskip
\noindent
Solution: We need to show that $m_t$ satisfies the three conditions of Definition 9.1.

\medskip
\item{i.} 
By hypothesis we have $m$ is an angle measure based on $\alpha$ hence $0<m(\angle ABC)<\alpha$.
Then by the hypothesis that $m_t(\angle ABC)=t\cdot m(\angle ABC)$ we have
$$0<{m_t(\angle ABC)\over t}<\alpha$$
Since $t>0$ we can multiply through by $t$ and not affect the less-than relationship.
Hence
$$0<m_t(\angle ABC)<t\alpha$$
Therefore $m_t$ satisfies the condition that $0<m_t(\angle ABC)<r_0$ for $r_0=t\alpha$.

\medskip
\item{ii.}
By the hypothesis that $m$ is an angle measure we have the existence of
a unique ray such that $m(\angle ABC)=\theta$
for any $\theta$ in the range $0<\theta<\alpha$.
Since $t>0$ we can multiply everything by $t$ and not affect the less-than relationship.
Hence there is a unique ray
such that $t\cdot m(\angle ABC)=t\theta$ for any $t\theta$ in the range $0<t\theta<t\alpha$.
If we let $\phi=t\theta$ then by $m_t=t\cdot m$ and $r_0=t\alpha$
we can say that there is a unique ray
such that $m_t(\angle ABC)=\phi$ for any $\phi$ in the range $0<\phi<r_0$.
Hence condition (ii) is satisfied.

\medskip
\item{iii.}
By the hypothesis that $m$ is an angle measure we have for $D\in int(\angle ABC)$ the
following.
$$m(\angle ABD)+m(\angle DBC)=m(\angle ABC)$$
We can multiply this equation through by $t$ and have the same for $m_t$.
Hence condition (iii) is satisfied.

\medskip
\noindent
Conditions (i), (ii) and (iii) are satisfied hence $m_t$ is an angle measure based on
$r_0=t\alpha$.

\vfill
\eject

\beginsection Page 96, problem 5.

Assume that $m_E$ is an angle measure for Euclidean metric geometry $\{R^2,L_E,d_E\}$.
Prove that $m_E$ is an angle measure for the Taxicab Plane $\{R^2,L_E,d_T\}$.

\bigskip
\noindent
Solution: Let us check each condition of Definition 9.1 is true in the Taxicab Plane for $m_E$.

\medskip
\item{i.} We have
$$m_E(\angle ABC)=\cos^{-1}\left({\langle A-B,C-B\rangle\over\!A-B\|\cdot\|C-B\|}\right)$$
The function $d_T$ does affect the set $\{A,B,C\}$, i.e., the $x$ and $y$ values are not affected.
Hence $m_E(\angle ABC)$ is the same for both the Euclidean Plane and the Taxicab Plane.
Therefore the condition $0<m_E<r_0$ is true for the Taxicab Plane.

\medskip
\item{ii}
We have already shown that $m(\angle ABC)$ is the same for both planes.
By Problem 5 of HW 6 we have $P{-}Q{-}R$ in the Euclidean Plane if and only if $P{-}Q{-}R$
in the Taxicab Plane.
Hence the set $H_1$ of $\overrightarrow{BC}$ is the same in both planes.
We have already shown that $d_t$ does not change the $x$ and $y$ values in the set $\{A,B,C\}$.
Hence the ray $\overrightarrow{BA}$ such that
$m(\angle ABC)=\theta$ is identical in both planes.

\medskip
\item{iii.}
By Problem 5 of HW 6 we have $P{-}Q{-}R$ in the Euclidean Plane if and only if $P{-}Q{-}R$
in the Taxicab Plane.
Hence the set $int(\angle ABC)$ is the same for both planes.
Let $D\in int(\angle ABC)$.
We have already shown that $m(\angle ABC)$ is the same for both planes.
Hence we have
$$m(\angle ABD)+m(\angle DBC)=m(\angle ABC)$$
in the Taxicab Plane since it is true in the Euclidean Plane.
Therefore condition (iii) is true for the Taxicab Plane.

\medskip
\noindent
Conditions (i), (ii) and (iii) are satisfied hence $m_E$ is an angle measure for the
Taxicab Plane.

\vfill
\eject

\beginsection Page 96, problem 7.

Let $\theta$ be a real number with $0<\theta<180$ and let $\overrightarrow{BC}$
lie in the edge of a half plane $H_1$ in a protractor geometry.
Prove that there is a unique point $A\in H_1$ with $m(\angle ABC)=\theta$.

\bigskip
\noindent
Counterexample:
By condition (ii) of Definition 9.1, there is a unique ray $\overrightarrow{BA}$ that
satisfies this condition.
However, $\overrightarrow{BA}=\overrightarrow{BP}$
for any $P\in int(\overrightarrow{BA})$.
Hence $m(\angle ABC)=m(\angle PBC)$.
Therefore $A$ is not unique.

\vfill
\eject

\beginsection Page 108, problem 3.

Prove the following theorem (Theorem 5.3.5).

\medskip
\noindent
Given a line $\ell$ and a point $B\in\ell$ in a protractor geometry, there
exists a unique line $\ell'$ that contains $B$ such that $\ell\perp\ell'$.

\bigskip
\noindent
Solution:

\medskip
\item{$\scriptstyle1$}
Let $C\in\ell$.

\medskip
\item{$\scriptstyle2$}
Define $A\in H_1$ of $\ell$ such that $m(\angle ABC)=90$.
The point $A$ is guaranteed to exist by condition (i) of Definition 9.1.

\medskip
\item{$\scriptstyle3$}
Define $D\in H_2$ of $\ell$ such that $m(\angle CBD)=90$.
The point $D$ is guaranteed to exist by condition (i) of Definition 9.1.

\medskip
\item{$\scriptstyle4$}
Let $\ell'=\overleftarrow A\overrightarrow D$.

\medskip
\item{$\scriptstyle5$}
By theorem 5.3.4 in the book, we have $A{-}B{-}D$ hence $B\in\ell'$.

\medskip
\item{$\scriptstyle6$}
By (2) and $A,B\in\ell'$ we have $\ell\perp\ell'$.

\medskip
\item{$\scriptstyle7$}
By condition (ii) of Definition 9.1, $\overrightarrow{BA}$ is unique hence $\ell'$ is unique.

\vfill
\eject

\beginsection Page 108, problem 4.

Prove the following corollary (Corollary 5.3.7).

\medskip
\noindent
In a protractor geometry, every line segment $\overline{AB}$ has a unique
perpendicular bisector; that is, a  line $\ell\perp\overline{AB}$
with $\ell\cap\overline{AB}=\{M\}$ where $M$ is the midpoint of $\overline{AB}$.

\bigskip
\noindent
Solution: The midpoint $M$ was proved to be unique in problem 12 (b) of HW4.
By Theorem 9.17 there is a unique line, call it $\ell$, such that
$\ell\perp\overleftarrow A\overrightarrow B$ such that $M\in\ell$.
Since unique lines can intersect in at most one point we have
$\ell\cap\overline{AB}=\{M\}$.

\vfill
\eject

\beginsection Page 108, problem 7.

Let $\angle ABC$ and $\angle A'BC'$ form a vertical pair in a protractor
geometry. Prove that if $\angle ABC$ is a right angle so are
$\angle A'BC$, $\angle A'BC'$ and $\angle ABC'$.

\bigskip
\noindent
Solution:

\medskip
\item{$\scriptstyle1$}
By vertical pair we have $A{-}B{-}A'$ and $C{-}B{-}C'$.

\medskip
\item{$\scriptstyle2$}
By $A{-}B{-}A'$ we have $\angle ABC$ and $A'BC$ are a linear pair then by
Theorem 9.14 we have $m(\angle A'BC)=90$.

\medskip
\item{$\scriptstyle3$}
By $C{-}B{-}C'$ we have $\angle A'BC$ and $\angle A'BC'$ are linear pair then by
Theorem 9.14 we have $m(\angle A'BC')=90$.

\medskip
\item{$\scriptstyle4$}
By $A{-}B{-}A'$ again we have $\angle A'BC'$ and $ABC'$ are a linear pair then by
Theorem 9.14 we have $m(\angle ABC')=90$.

\vfill
\eject

\beginsection Page 108, problem 9.

Show that if $\triangle ABC$ is as given in Example 5.1.3, then
$(AC)^2\ne(AB)^2+(BC)^2$.
Thus the Pythagorean Theorem need not be true in a protractor geometry.

\bigskip
\noindent
Solution: We have $A=(0,1)$, $B=(0,5)$, and $C=(3,4)$ in the Poincare Plane.

\medskip
\noindent
$\overline{AB}$ is a type I line with
$$d(A,B)=|\ln5-\ln1|=\ln5$$

\medskip
\noindent
$\overline{AC}$ is a type II line with
$$c={y_C^2-y_A^2+x_C^2-x_A^2\over2(x_C-x_A)}=4,\qquad
r=\sqrt{(x_A-c)^2+y_A^2}=\sqrt{17}$$
$$d(A,C)=\left|\ln\left(
{
{x_A-c+r\over y_A}
\over
{x_C-c+r\over y_C}
}
\right)\right|
=\left|\ln\left({4(-4+\sqrt{17})\over-1+\sqrt{17}}\right)\right|
$$

\medskip
\noindent
$\overline{BC}$ is a type II line with
$$c={y_C^2-y_B^2+x_C^2-x_B^2\over2(x_C-x_B)}=0,\qquad
r=\sqrt{(x_B-c)^2+y_B^2}=5$$
$$d(B,C)=\left|\ln\left(
{
{x_B-c+r\over y_B}
\over
{x_C-c+r\over y_C}
}
\right)\right|
=\ln2
$$

We have
$$(AC)^2\approx3.4123,\qquad(AB)^2+(BC)^2\approx3.0707$$

\end
