\magnification=1200

\noindent
{\it George Weigt --- Advanced Calculus Homework \#9}

\beginsection Question 1.

Evaluate by Green's theorem
$$I=\oint_C ay\,dx + bx\, dy\eqno(1)$$
on any path.

\bigskip
\noindent
Solution: We have
$$P=ay,\qquad Q=bx$$
hence
$${\partial P\over\partial y}=a,\qquad{\partial Q\over\partial x}=b$$
By Green's theorem we have
$$I=\int\!\!\!\int_D
\left({\partial Q\over\partial x}-{\partial P\over\partial y}\right)\,dx\,dy
=(b-a)\int\!\!\!\int_D dx\,dy
=(b-a)\times(\hbox{Area of $D$})$$

\bigskip
\noindent
Let us check this result by computing the line integral
around a circle of radius $r$ centered at the origin.
The line integral can be parameterized as follows.
$$x=r\cos t,\qquad dx=-r\sin t\,dt,\qquad y=r\sin t,\qquad dy=r\cos t\,dt,\qquad
0\le t\le 2\pi$$
Hence
$$\eqalign{
\oint_C ay\,dx + bx\, dy&=\int_0^{2\pi}-ar^2\sin^2 t\,dt+\int_0^{2\pi}br^2\cos^2 t\,dt\cr
&=-ar^2\left({t\over2}-{\sin2t\over4}\right)\bigg|_0^{2\pi}
+br^2\left({t\over2}+{\sin2t\over4}\right)\bigg|_0^{2\pi}\cr
&=(b-a)\pi r^2\cr
&=(b-a)\times(\hbox{Area of a circle of radius $r$})\cr
}$$
We can do another check by noting that for $a=1$ and $b=-1$, formula (1)
at the top of the page is identical to Example 2 in Lecture 8.
For $a=1$, $b=-1$ and $C$ a unit circle centered at the origin we have
$$(b-a)\times(\hbox{Area of circle})=(-1-1)\pi=-2\pi$$
which agrees with the result in Lecture 8.

\vfill
\eject

\beginsection Question 2.

Evaluate by Green's theorem
$$I=\oint(2x^3-y^3)\,dx+(x^3+y^3)\,dy$$
around the circle $x^2+y^2=1$.

\bigskip
\noindent
Solution: We have
$$P=2x^3-y^3,\qquad{\partial P\over\partial y}=-3y^2,
\qquad Q=x^3+y^3,\qquad{\partial Q\over\partial x}=3x^2$$
By Green's theorem we have
$$\eqalign{
I&=\int\!\!\!\int_D
\left({\partial Q\over\partial x}-{\partial P\over\partial y}\right)\,dx\,dy\cr
&=\int\!\!\!\int_D(3x^2+3y^2)\,dx\,dy\cr
}$$
In polar coordinates we have
$$x=r\cos\theta,\qquad y=r\sin\theta,\qquad dx\,dy=r\,dr\,d\theta,
\qquad x^2+y^2=r^2$$
Therefore
$$I=3\int_0^{2\pi}d\theta\int_0^1r^3\,dr
=3\int_0^{2\pi}d\theta\left({1\over4}r^4\bigg|_0^1\right)
={3\pi\over2}$$

\bigskip
\noindent
Check using a line integral.
$$x=\cos t,\qquad dx=-\sin t\,dt,\qquad y=\sin t,\qquad dy=\cos t\,dt,\qquad
0\le t\le 2\pi$$
Hence
$$I=\int_0^{2\pi}(2\cos^3t-\sin^3 t)(-\sin t)\,dt
+\int_0^{2\pi}(\cos^3t+\sin^3t)(\cos t)\,dt$$
By brute force ({\tt integrals.wolfram.com}) we obtain
$$\int_0^{2\pi}(2\cos^3t-\sin^3 t)(-\sin t)\,dt
=\left({3t\over8}+{\cos2t\over4}+{\cos4t\over16}-{\sin2t\over4}+{\sin4t\over32}\right)
\bigg|_0^{2\pi}={3\pi\over4}$$
$$\int_0^{2\pi}(\cos^3t+\sin^3t)(\cos t)\,dt
=\left({3t\over8}-{\cos2t\over8}+{\cos4t\over32}+{\sin2t\over4}+{\sin4t\over32}\right)
\bigg|_0^{2\pi}={3\pi\over4}$$
Therefore
$$I={3\pi\over4}+{3\pi\over4}={3\pi\over2}$$
which agrees with the result using Green's theorem.

\vfill
\eject

\beginsection Question 3.

Evaluate by Green's theorem
$$I=\oint_C f(x)\,dx+g(y)\,dy$$
on any path.

\bigskip
\noindent
Solution: Let
$$P=f(x),\qquad{\partial P\over\partial y}=0,
\qquad Q=g(y),\qquad{\partial Q\over\partial x}=0$$
Then by Green's theorem we have
$$I=\int\!\!\!\int_D
\left({\partial Q\over\partial x}-{\partial P\over\partial y}\right)\,dx\,dy=0$$

\beginsection Question 4A.

Evaluate $I_3$.
$$I_3=\int_{C_3} (P\,dx+Q\,dy)$$
where $C_3$ is the straight line from $(A+a,B+b)$ to $(A,B+b)$.

\bigskip
\noindent
Solution: Let $y=B+b$. Then $dy=0$ hence
$$I_3=\int_{A+a}^A P(x,B+b)\,dx+\int_{A+a}^A Q(x,B+b)\cdot 0=-\int_A^{A+a} P(x,B+b)\,dx$$

\beginsection Question 4B.

Evaluate $I_4$.
$$I_4=\int_{C_4}(P\,dx+Q\,dy)$$
where $C_4$ is the straight line from $(A,B+b)$ to $(A,B)$.

\bigskip
\noindent
Solution: Let $x=A$. Then $dx=0$ hence
$$I_4=\int_{B+b}^B P(A,y)\cdot0+\int_{B+b}^B Q(A,y)\,dy=-\int_B^{B+b} Q(A,y)\,dy$$

\vfill
\eject

\beginsection Question 4C.

Add all four line integrals together and simplify.

\bigskip
\noindent
Solution:
$$
I_1+I_2+I_3+I_4
=\int_A^{A+a}\left[P(x,B)-P(x,B+b)\right]\,dx
+\int_B^{B+b}\left[Q(A+a,y)-Q(A,y)\right]\,dy$$
Reordering the terms we have
$$I_1+I_2+I_3+I_4
=\int_B^{B+b}\left[Q(A+a,y)-Q(A,y)\right]\,dy
-\int_A^{A+a}\left[P(x,B+b)-P(x,B)\right]\,dx$$
At this point it is easier to work from the right hand side of Green's Theorem.
Here is the integral over $D$ when $D$ is the square as described in Question 4.
$$\eqalign{
\int\!\!\!\int_D \left({\partial Q\over\partial x}-{\partial P\over\partial y}\right)\,dx\,dy
&=\int_B^{B+b}dy\int_A^{A+a}{\partial Q\over\partial x}\,dx
-\int_A^{A+a}dx\int_B^{B+b}{\partial P\over\partial y}\,dy\cr
}$$
By the Fundamental Theorem of Calculus we have
$$\eqalign{
\int_A^{A+a}{\partial Q\over\partial x}\,dx&=Q(A+a,y)-Q(A,y)\cr
\cr
\int_B^{B+b}{\partial P\over\partial y}\,dy&=P(x,B+b)-P(x,B)\cr
}$$
Hence
$$\eqalign{
\int\!\!\!\int_D &\left({\partial Q\over\partial x}-{\partial P\over\partial y}\right)\,dx\,dy\cr
&=\int_B^{B+b}\left[Q(A+a,y)-Q(A,y)\right]\,dy
-\int_A^{A+a}\left[P(x,B+b)-P(x,B)\right]\,dx\cr
}$$
which is identical to $I_1+I_2+I_3+I_4$ above.

\end
