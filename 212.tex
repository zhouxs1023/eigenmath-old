\noindent
6.1. Consider $s,t\in Q$. Show that
\medskip
\itemitem{(a)} $s\le t$ if and only if $s^*\subseteq t^*$;
\itemitem{(b)} $s=t$ if and only if $s^*=t^*$;
\itemitem{(c)} $(s+t)^*=s^*+t^*$. Note that $s^*+t^*$ is a sum of Dedekind
cuts.

\bigskip
\noindent
(a)
First we prove the implication.
Let $r\in s^*$. Then by definition of a Dedekind cut we have $r<s$.
From $r<s$ and $s\le t$ we have $r<t$ by the transitive law.
From $r<t$ it follows that $r\in t^*$.
Therefore $r\in s^*$ implies $r\in t^*$. Hence $s^*\subseteq t^*$.
Next we prove the converse.
Let $t<s$. Then $t\in s^*$ and $t\not\in t^*$.
It follows that $s^*\not\subseteq t^*$, a contradiction.
Since $t<s$ leads to a contradiction we must have $s\le t$.

\bigskip
\noindent
{\it The transitive law is used loosely here, is there a better way?
The proof mixes reals and rationals, is that ok?
Is there a simple way to prove the converse directly instead of by
contradiction?}

\bigskip
\noindent
(b)
First we prove the implication.
Let $s=t$.
Then for all $r\in s^*$ we have $r<t$ hence $s^*\subseteq t^*$.
In addition, for all $r\in t^*$ we have $r<s$ hence $t^*\subseteq s^*$.
Therefore $s^*=t^*$.
Next we prove the converse.
Let $s^*=t^*$.
Then for all $r\in s^*$ we have $r\in t^*$ hence $s\le t$.
In addition, for all $r\in t^*$ we have $r\in s^*$ hence $t\le s$.
Therefore $s=t$.

\bigskip
\item{1.} Let $s=t$.

\item{2.} For all $r\in s^*$ we have $r<s$ by definition.
\item{3.} For $r<s$ we have $r<t$ by hypothesis hence $r\in t^*$.
\item{4.} Therefore $r\in s^*$ implies $r\in t^*$ hence $s^*\subseteq t^*$.

\item{5.} For all $r\in t^*$ we have $r<t$ by definition.
\item{6.} For $r<t$ we have $r<s$ by hypothesis hence $r\in s^*$.
\item{7.} Therefore $r\in t^*$ implies $r\in s^*$ hence $t^*\subseteq s^*$.

\item{8.} We have $s^*=t^*$ by (4) and (7).
\item{9.} Therefore $s=t$ implies $s^*=t^*$.

\item{10.} Let $s^*=t^*$.

\item{11.} Assume $s<t$.
\item{12.} We have $s\not\in s^*$ and $s\in t^*$, a contradiction of (10), hence $s\ge t$.
\item{13.} Assume $t<s$.
\item{14.} We have $t\in s^*$ and $t\not\in t^*$, a contradiction of (10), hence $t\ge s$.
\item{15.} We have $s=t$ by (12) and (14).
\item{16.} Therefore $s^*=t^*$ implies $s=t$.

\item{17.} Finally, by (9) and (16) we have  $s=t$ if and only if $s^*=t^*$.

\bigskip
\noindent
{\it I like the way the contradictions in steps 10--16 turned out.
In particular, the way $<$ leads to $\ge$ and then to equality.
I can't think of a direct proof that is more compelling.}
\end
