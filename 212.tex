\noindent
6.1. Consider $s,t\in Q$. Show that
\medskip
\itemitem{(a)} $s\le t$ if and only if $s^*\subseteq t^*$;
\itemitem{(b)} $s=t$ if and only if $s^*=t^*$;
\itemitem{(c)} $(s+t)^*=s^*+t^*$. Note that $s^*+t^*$ is a sum of Dedekind
cuts.

\bigskip
\noindent
(a)
First we prove the implication.
Let $r\in s^*$. Then by definition of a Dedekind cut we have $r<s$.
From $r<s$ and $s\le t$ we have $r<t$ by the transitive law.
From $r<t$ it follows that $r\in t^*$.
Therefore $r\in s^*$ implies $r\in t^*$. Hence $s^*\subseteq t^*$.
Next we prove the converse.
Let $t<s$. Then $t\in s^*$ and $t\not\in t^*$.
It follows that $s^*\not\subseteq t^*$, a contradiction.
Since $t<s$ leads to a contradiction we must have $s\le t$.

\bigskip
\noindent
{\it The transitive law is used loosely here, is there a better way?
The proof mixes reals and rationals, is that ok?
Is there a simple way to prove the converse directly instead of by
contradiction?}


\end
