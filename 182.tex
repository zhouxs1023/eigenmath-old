\noindent
{\it George Weigt -- Geometry Homework \#5}

\beginsection Page 58, problem 7.

In the Taxicab Plane show that if $A=(-{5\over2},2)$,
$B=({1\over2},2)$, $C=(2,2)$, $P=(0,0)$, $Q=(2,1)$
and $R=(3,{3\over2})$ then $A{-}B{-}C$ and $P{-}Q{-}R$.
Show that $\overline{AB}\simeq\overline{PQ}$,
$\overline{BC}\simeq\overline{QR}$ and
$\overline{AC}\simeq\overline{PR}$.
Sketch an appropriate picture.
\medskip\noindent
Solution: First, compute all the distances.
$$
\eqalign{
AB&=\left|-{5\over2}-{1\over2}\right|+|2-2|=3\cr
BC&=\left|{1\over2}-2\right|+|2-2|={3\over2}\cr
AC&=\left|-{5\over2}-2\right|+|2-2|={9\over2}\cr
}
$$
$$
\eqalign{
PQ&=|0-2|+|0-1|=3\cr
QR&=|2-3|+\left|1-{3\over2}\right|={3\over2}\cr
PR&=|0-3|+\left|0-{3\over2}\right|={9\over2}\cr
}
$$
We have $AB+BC=AC$ therefore $A{-}B{-}C$.
We have $PQ+QR=PR$ therefore $P{-}Q{-}R$.
We have $AB=PQ$ therefore $\overline{AB}\simeq\overline{PQ}$.
We have $BC=QR$ therefore $\overline{BC}\simeq\overline{QR}$.
We have $AC=PR$ therefore $\overline{AC}\simeq\overline{PR}$.

\beginsection Page 58, problem 8.

Let $A=(0,0)$, $B=({1\over10},1$) and $C=(1,1)$ be points in $R^2$
with the max distance $d_S(P,Q)=max\{|x_1-x_2|,|y_1-y_2|\}$.
Prove that $\overline{AB}\simeq\overline{AC}$.
Sketch the two segments. Do they look congruent?
\medskip\noindent
Solution: First, compute the distances.
$$\eqalign{
AB=d_S(A,B)&=max\left\{\left|0-{1\over10}\right|,|0-1|\right\}=1\cr
AC=d_S(A,C)&=max\{|0-1|,|0-1\}=1\cr
}$$
We have $AB=AC$ therefore $\overline{AB}\simeq\overline{AC}$.
They look congruent when projected onto the $y$-axis.

\beginsection Page 58, problem 9.

In the Poincare Plane let $P=(1,2)$ and $Q=(1,4)$.
If $A=(0,2)$ and $B=(1,\sqrt3)$, find $C\in\overrightarrow{AB}$ with
$\overline{AC}\simeq\overline{PQ}$.
\medskip\noindent
Solution: First, compute $PQ$, the distance from $P$ to $Q$.
For $P$ and $Q$ we have $x_1=x_2$ therefore $P$ and $Q$ are on a type I line.
We have
$$PQ=\left|\ln{4\over2}\right|=\ln2$$
Next, $A$ and $B$ are on a type II line. We have
$$c={y_2^2-y_1^2+x_2^2-x_1^2\over2(x_2-x_1)}
={3-4+1-0\over2(1-0)}=0,\qquad
r=\sqrt{(x_1-c)^2+y_1^2}=2$$
Next, apply the standard ruler to $A$ and $B$.
$$\eqalign{
f(A)&=\ln\left({0-0+2\over 2}\right)=\ln 1=0\cr
f(B)&=\ln\left({1-0+2\over\sqrt3}\right)=\ln\sqrt3\cr
}$$
Next, we want to find $C$. We have
$$AC=|f(A)-f(C)|=\ln2$$
therefore
$$f(C)=f(A)\pm\ln2$$
Note that $f(B)>f(A)$.
In order to have $C\in\overrightarrow{AB}$ we must choose $f(C)>f(A)$ therefore
$$f(C)=f(A)+\ln2=\ln2$$
hence
$$t=\ln2,\qquad C=(r\tanh t+c,r\mathop{\rm sech} t)=(1.2,1.6)$$
Check:
$$AC=\left|\ln\left({{0-0+2\over2}\over{1.2-0+2\over1.6}}\right)\right|
=\left|\ln{1\over2}\right|=\ln2$$

\beginsection Page 58, problem 10.

In the Taxicab Plane let $P=(1,-2)$, $Q=(2,5)$,
$A=(4,-1)$ and $B=(3,2)$.
Find $C\in\overrightarrow{AB}$ with $\overline{AC}\simeq\overline{PQ}$.
\medskip\noindent
Solution:
$$PQ=|1-2|+|-2-5|=8$$
$$m={y_2-y_1\over x_2-x_1}={-1-2\over4-3}=-3,
\qquad b=y_1-mx_1=11$$
$$\eqalign{
f(A)&=x(1+|m|)=4(1+|-3|)=16\cr
f(B)&=x(1+|m|)=3(1+|-3|)=12\cr
}$$
We have
$$AC=|f(A)-f(C)|=PQ=8$$
therefore
$$f(C)=f(A)\pm8$$
Since $f(B)<f(A)$ we must have $f(C)<f(A)$. Therefore
$$f(C)=f(A)-8=8$$
hence $x=2$, $y=mx+b=5$, i.e.
$$C=(2,5)$$
Check:
$$AC=|4-2|+|-1-5|=8$$

\beginsection Page 62, problem 1.

Prove that $\angle ABC=\angle CBA$ in a metric geometry.
\medskip\noindent
Solution:
The objects $\angle ABC$ and $\angle CBA$ are sets of points.
Their equivalance is proven by showing that
each one is a subset of the other.
\item{i.} Let $P\in\angle ABC$.
%Then either $P\in\overrightarrow{BA}$ or $P\in\overrightarrow{BC}$.
Then $P\in(\overrightarrow{BA}\cup\overrightarrow{BC})$
which implies $P\in\angle CBA$.
Therefore $\angle ABC\subset\angle CBA$.
\item{ii.} Let $P\in\angle CBA$.
%Then either $P\in\overrightarrow{BC}$ or $P\in\overrightarrow{BA}$.
Then $P\in(\overrightarrow{BC}\cup\overrightarrow{BA})$
which implies $P\in\angle ABC$.
Therefore $\angle CBA\subset\angle ABC$.
\par\noindent
By (i) and (ii) above, $\angle ABC=\angle CBA$.
Note that asserting membership in a ray
requires a betweeness property that is fulfilled by the ruler
of a metric geometry.

\beginsection Page 62, problem 2.

Let $D$, $E$, and $F$ be three noncollinear points of a metric geometry
and let $\ell$ be a line that contains at most one of $D$, $E$, and $F$.
Prove that each of
$\overleftarrow D\overrightarrow E$,
$\overleftarrow D\overrightarrow F$ and
$\overleftarrow E\overrightarrow F$ intersects $\ell$ in at most one point.
\medskip\noindent
Solution:
By hypothesis
$$\ell=\overleftarrow P\overrightarrow Q,\qquad P\ne D,E,F$$
In an incidence geometry, two points determine a unique line.
Therefore
$$
\overleftarrow P\overrightarrow Q\ne\overleftarrow D\overrightarrow E,\qquad
\overleftarrow P\overrightarrow Q\ne\overleftarrow D\overrightarrow F,\qquad
\overleftarrow P\overrightarrow Q\ne\overleftarrow E\overrightarrow F
$$
In an incidence geometry two lines are either parallel, or they intersect in
exactly one point.
If $\overleftarrow P\overrightarrow Q$ is parallel to
$\overleftarrow D\overrightarrow E$ then the fact that
$\overleftarrow P\overrightarrow Q\ne\overleftarrow D\overrightarrow E$
implies that
$$\overleftarrow P\overrightarrow Q\cap\overleftarrow D\overrightarrow E=\emptyset$$
Therefore $\overleftarrow P\overrightarrow Q$ intersects
$\overleftarrow D\overrightarrow E$ at one point or not at all.
Repeat for $\overleftarrow D\overrightarrow F$ and
$\overleftarrow E\overrightarrow F$.

\beginsection Page 62, problem 3.

Prove that if $\triangle ABC=\triangle DEF$ in a metric geometry then
$\overleftarrow A\overrightarrow B$ contains exactly two of the points
$D$, $E$ and $F$.
\medskip\noindent
Solution:
By $\triangle ABC=\triangle DEF$ we have
$$\overline{AB}\subset\overline{DE}\cup\overline{EF}\cup\overline{FD}$$
Consequently, since $D$, $E$ and $F$
are not collinear, exactly one of the following is true.
$$
\overline{AB}\subset\overline{DE}
\quad\hbox{or}\quad
\overline{AB}\subset\overline{EF}
\quad\hbox{or}\quad
\overline{AB}\subset\overline{FD}
$$
%Hence, $A$, $B$ and exactly two of the points $D$, $E$ and $F$ are collinear.
Therefore $\overleftarrow A\overrightarrow B$ contains exactly two of the points
$D$, $E$ and $F$.


\beginsection Page 62, problem 5.

In a metric geometry, prove that if $A$, $B$ and $C$ are not collinear
then $\overline{AB}=\overleftarrow A\overrightarrow B\cap\triangle ABC$.
\medskip\noindent
Solution:
We have
$$\eqalign{
\overleftarrow A\overrightarrow B\cap\triangle ABC&=
(\overleftarrow A\overrightarrow B\cap\overline{AB})\cup
(\overleftarrow A\overrightarrow B\cap\overline{BC})\cup
(\overleftarrow A\overrightarrow B\cap\overline{CA})\cr
&=\overline{AB}\cup\{B\}\cup\{A\}\cr
&=\overline{AB}
}$$

\beginsection Page 68, problem 1.

If $S_1$ and $S_2$ are convex subsets of a metric geometry, prove that
$S_1\cap S_2$ is convex.
\medskip\noindent
Solution:
Assume $S_1\cap S_2$ contains at least one line segment.
Let $P,Q\in S_1\cap S_2$.
Then $P,Q\in S_1$ and $P,Q\in S_2$.
Hence by hypothesis we have $\overline{PQ}\subset S_1$ and $\overline{PQ}\subset S_2$.
This implies that $\overline{PQ}\subset S_1\cap S_2$.
Therefore $S_1\cap S_2$ is convex.

\beginsection Page 68, problem 3.

If $H_1$ is a half plane determined by $\ell$ prove that $H_1\cup\ell$ is convex.
\medskip\noindent
Solution:
We want to show that for every $P,Q\in(H_1\cup\ell)$ we have
$\overline{PQ}\subset(H_1\cup\ell)$.
There are three cases.
\item{i.} $P,Q\in\ell$ implies $\overline{PQ}\subset\ell$.
\item{ii.} $P,Q\in H_1$ implies $\overline{PQ}\subset H_1$.
\item{iii.} The remaining case is $P\in\ell$ and $Q\in H_1$.
Let $X$ be defined such that $P{-}X{-}Q$.
Note that $P\not\in\overline{XQ}$ therefore $X$ and $Q$ are in the same
half plane, hence $X\in H_1$.
Because $H_1$ is convex we have $\overline{XQ}\subset H_1$.
Hence $\overline{PQ}\subset(H_1\cup\ell)$.
\par\noindent
Therefore $H_1\cup\ell$ is convex.

\beginsection Page 68, problem 4.

If $H_1$ and $H_2$ are the half planes determined by the line $\ell$,
prove that neither $H_1$ nor $H_2$ is empty.
\medskip\noindent
Solution:
By definition we have $S-\ell=H_1\cup H_2$.
An incidence geometry has at least three points
that are not collinear therefore $H_1\cup H_2\ne\emptyset$.
Let $P\in\ell$ and $Q\in(H_1\cup H_2)$.
Since $Q\not\in\ell$ there is a unique line $\overleftarrow P\overrightarrow Q\ne\ell$.
Let $X$ be defined such that $X{-}P{-}Q$. Then $X$ is in one half plane
and $Q$ is in the other.
Therefore $H_1\ne\emptyset$ and $H_2\ne\emptyset$.

\beginsection Page 68, problem 5.

If $H_1$ is a half plane determined by the line $\ell$, prove that
$H_1$ has at least three noncollinear points.
\medskip\noindent
Solution:
Let $A$ and $B$ be defined such that $A,B\in\ell$, $A\ne B$.
Let $C$ be defined such that $C\in H_1$.
%Since $A\ne B$ we have
%$\overleftarrow A\overrightarrow C\ne\overleftarrow B\overrightarrow C$.
%Furthermore, we have
%$\overleftarrow A\overrightarrow C\cap\overleftarrow B\overrightarrow C=\{C\}$
%since two distinct lines can intersect in at most one point.
Let $X$ and $Y$ be defined such that $A{-}X{-}C$ and $B{-}Y{-}C$.
Then we have $X,Y\in H_1$
along with $Y\not\in\overleftarrow A\overrightarrow C$.
Therefore we have $X,Y,C\in H_1$ and $X$, $Y$ and $C$ are not collinear.
\medskip\noindent
More explanation about $Y\not\in\overleftarrow A\overrightarrow C$.
Because $A\ne B$ we have $\overleftarrow A\overrightarrow C\ne\overleftarrow B\overrightarrow C$.
Since two distinct lines can intersect in at most one point, and that point is $C$,
we have $\overleftarrow A\overrightarrow C\cap\overleftarrow B\overrightarrow C=\{C\}$.
Since $Y\in\overleftarrow B\overrightarrow C$ and $Y\not\in\{C\}$,
we must have $Y\not\in\overleftarrow A\overrightarrow C$.

\beginsection Page 68, problem 13.

If $A$, $B$, $C$ are noncollinear in a metric geometry,
prove that $\triangle ABC$ is convex.
\medskip\noindent
Solution:
We have
$$\triangle ABC=\overline{AB}\cup\overline{BC}\cup\overline{CA}$$
Let $P$ be defined such that $B{-}P{-}C$.
Since distinct lines can intersect in at most one point we have
$$\eqalign{
\overleftarrow A\overrightarrow P\cap\overleftarrow A\overrightarrow B=\{A\}\cr
\overleftarrow A\overrightarrow P\cap\overleftarrow A\overrightarrow C=\{A\}\cr
\overleftarrow A\overrightarrow P\cap\overleftarrow B\overrightarrow C=\{P\}\cr
}$$
Therefore $\overleftarrow A\overrightarrow P\cap\triangle ABC=\{A,P\}$.
Hence $\overline{AP}\not\subset\triangle ABC$.
Therefore $\triangle ABC$ is not convex.

\end
