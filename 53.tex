\hsize=4in
\nopagenumbers
\parindent=0pt

3.4 A black and white television signal has a bandwidth of about 4.2 MHz.
What bit rate is required if this signal is to be digitized with uniform
PCM at an SQR of 30 dB?
Use a sampling-rate to Nyquist-rate ratio comparable to that used for PCM
voice encoding.

\bigskip

The Nyquist criterion tells us that the sampling rate must be greater
than twice the bandwidth. The excess sampling factor for voice PCM is 4000/3400
(see page 107).
Therefore the sampling rate is
$$f_s=2\times4.2\,{\rm MHz}\times{4000\over3400}=9.88\,{\rm MHz}$$
Equation 3.4 gives us the signal to noise ratio.
$${\rm SQR}=7.78+20\log_{10}(A/q)$$
Solve for $A/q$ for an SQR of 30 dB.
$$\log_{10}(A/q)={30-7.78\over20}=1.11$$
$$A/q=10^{1.11}=12.9$$
We see that the quantization interval $q$ must divide the input amplitude
$A$ into at least 13 intervals.
Therefore four bits are required to encode the sample.
In addition, another bit is required to represent the sign of the amplitude.
(The signal swings between $+A$ and $-A$.)
The overall bit rate is
$$5\,{\rm bits}\times9.88\,{\rm MHz}
=49.4\,{\rm Mbits/sec}$$

\end
