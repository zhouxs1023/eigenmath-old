\chapter{Complex Numbers}

\noindent
When Eigenmath starts up, it defines the symbol $i$ as $i=\sqrt{-1}$.
Other than that, there is nothing special about $i$.
It is just a regular symbol that can be redefined and used for some other purpose if need be.

\medskip
\noindent
Complex quantities can be entered in rectangular or polar form.

\medskip
\verb$a+i*b$
$$a+ib$$

\verb$exp(i*pi/3)$
$$\exp({1\over3}i\pi)$$

\medskip
\noindent
Converting to rectangular or polar coordinates simplifies mixed forms.

\medskip
\verb$A=1+i$

\verb$B=sqrt(2)*exp(i*pi/4)$

\verb$A-B$
$$1+i-2^{1/2}\exp({1\over4}i\pi)$$

\verb$rect$
$$0$$

\newpage

\noindent
Rectangular complex quantities, when raised to a power, are multiplied out.

\medskip
\verb$(a+i*b)^2$
$$a^2-b^2+2iab$$

\medskip
\noindent
When $a$ and $b$ are numerical and the power is negative, the evaluation is done as follows.
$$(a+ib)^{-n}=\left[{a-ib\over(a+ib)(a-ib)}\right]^n=\left[{a-ib\over a^2+b^2}\right]^n$$
This removes $i$ from the denominator.
%For $n=1$ we have
%$${1\over a+ib}={a-ib\over a^2+b^2}$$
Here are a few examples.

\medskip
\verb$1/(2-i)$
$${2\over5}+{1\over5}i$$

\verb$(-1+3i)/(2-i)$
$$-1+i$$

\newpage

\noindent
The absolute value of a complex number returns its magnitude.

\medskip
\verb$abs(3+4*i)$
$$5$$

\medskip
\noindent
In light of this, the following result might be unexpected.

\medskip
\verb$abs(a+b*i)$
$$\mathop{\rm abs}(a+ib)$$

\medskip
\noindent
The result is not $\sqrt{a^2+b^2}$ because that would assume that
$a$ and $b$ are real.
For example, suppose that $a=0$ and $b=i$.
Then
$$|a+ib|=|-1|=1$$
and
$$\sqrt{a^2+b^2}=\sqrt{-1}=i$$
Hence
$$|a+ib|\ne\sqrt{a^2+b^2}\quad\hbox{for some $a,b\in\cal C$}$$

\medskip
\noindent
The {\it mag} function is an alternative.
It treats symbols like $a$ and $b$ as real.

\medskip
\verb$mag(a+b*i)$

$$(a^2+b^2)^{1/2}$$


