\noindent{\it George Weigt -- Advanced Algebra Homework \#3}

\beginsection Problem 1.

Let
$$I_n=\int_0^1\,e^tt^n\,dt$$
Show that, for all $n\ge1$,
$${1\over n+1}\le I_n$$

\medskip\noindent
Solution: From the hint we have
$$\int_0^1t^n\,dt\le\int_0^1t^ne^t\,dt$$
Hence
$${t^{n+1}\over n+1}\bigg|_0^1={1\over n+1}\le I_n$$

\beginsection Problem 2.

Show that, for all $n\ge1$,
$$I_n\le\int_0^1et^n\,dt$$
and conclude that
$$I_n\le{3\over n}$$
\medskip\noindent
Solution: From the hint we have for $0\le t\le 1$
$$e^t\le e$$
Multiply both sides by $t^n$ and integrate.
$$\int_0^1e^tt^n\,dt\le\int_0^1et^n\,dt$$
We have
$$I_n\le{e\over n+1}$$
From
$${1\over n+1}<{1\over n}$$
we can multiply by $e$ and obtain
$${e\over n+1}<{e\over n}$$
Then from $e<3$ we obtain
$${e\over n}<{3\over n}$$
Therefore
$$I_n<{3\over n}$$

\beginsection Problem 4.

Show that
$$I_n=e-nI_{n-1}$$
\medskip\noindent
Solution: From the hint, let
$$u=t^n,\qquad dv=e^t\,dt$$
Then
$$du=nt^{n-1}\,dt,\qquad v=e^t$$
Hence
$$\eqalign{
I_n=\int_0^1 t^ne^t\,dt&=\int_0^1 u\,dv\cr
&=uv\bigg|_0^1-\int_0^1 v\,du\cr
&=t^ne^t\bigg|_0^1-n\int_0^1e^tt^{n-1}\,dt\cr
&=e-nI_{n-1}\cr
}$$

\beginsection Problem 5.

Evaluate $I_n$ for $n=0,1,2,3,4$.
\medskip\noindent
Solution: From the hint, $I_0=e-1$. Then
$$\eqalign{
I_1&=e-1I_0=e-1(e-1)=1\cr
I_2&=e-2I_1=e-2(1)=e-2\cr
I_3&=e-3I_2=e-3(e-2)=-2e+6\cr
I_4&=e-4I_3=e-4(-2e+6)=9e-24\cr
}$$

\beginsection Problem 6.

Show that
$$I_n=A_n+eB_n$$
where
$$A_n=(-1)^{n+1},\qquad B_n=\sum_{k=0}^n(-1)^k{n!\over(n-k)!}$$

\medskip\noindent
Solution: For an inductive proof we need the following which will be proved later.
$$A_{n+1}=-(n+1)A_n,\qquad B_{n+1}=1-(n+1)B_n$$
Inductive Step 0:
$$A_0+eB_0=(-1)^{0+1}0!+e(-1)^0{0!\over0!}=e-1=I_0$$
Inductive Step 1: Assume $I_n=A_n+eB_n$ for some $n$. Then
$$\eqalign{
A_{n+1}+eB_{n+1}&=-(n+1)A_n+e(1-(n+1)B_n)\cr
&=e-(n+1)(A_n+eB_n)\cr
&=e-(n+1)I_n\cr
&=I_{n+1}\cr
}$$
Therefore by induction we have
$$I_n=A_n+eB_n$$
for all $n$.

\vfill\eject\noindent
Intermediate proofs for Problem 6.
$$A_{n+1}=(-1)^{n+2}(n+1)!=(-1)(-1)^{n+1}(n+1)n!=-(n+1)A_n$$
Here is a proof that $B_{n+1}=1-(n+1)B_n$ in six easy steps.
\medskip\noindent
Step 1. Write down the expression for $B_{n+1}$.
$$B_{n+1}=\sum_{k=0}^{n+1}(-1)^k{(n+1)!\over(n+1-k)!}$$
Step 2. Pull out the $k=0$ term and start the summation at $k=1$.
$$B_{n+1}=1+\sum_{k=1}^{n+1}(-1)^k{(n+1)!\over(n+1-k)!}$$
Step 3. Factor out $(n+1)$.
$$B_{n+1}=1+(n+1)\sum_{k=1}^{n+1}(-1)^k{n!\over(n+1-k)!}$$
Step 4.
Subtract 1 from the index and balance it
by adding 1 to $k$ in the summand.
Note that $n+1-k$ becomes $n+1-(k+1)=n-k$.
$$B_{n+1}=1+(n+1)\sum_{k=0}^n(-1)^{k+1}{n!\over(n-k)!}$$
Step 5. Factor out a $(-1)$.
$$B_{n+1}=1-(n+1)\sum_{k=0}^n(-1)^k{n!\over(n-k)!}$$
Step 6. Substitute with $B_n$.
$$B_{n+1}=1-(n+1)B_n$$

\end