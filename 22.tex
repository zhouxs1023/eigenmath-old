\hsize=4in
\nopagenumbers
\parindent=0pt
2. From Quantum magazine, p. 18.
Prove the following theorem (the ``horse'' theorem).
If a function is defined on a circle
and is continuous, there exist
two diametrically opposite points
on the circle where this function
takes on equal values.

\medskip
Let $v$ be a continuous function defined on a circle.
$$v(\theta)$$
Here are its endpoints.
$$v(0)=v(2\pi)=A$$
Now consider the following function $u$.
$$u(\phi)=v(\phi+\pi)-v(\phi)$$
Evaluate $u$ at zero and $\pi$.
$$\eqalign{
u(0)&=v(\pi)-v(0)=v(\pi)-A\cr
u(\pi)&=v(2\pi)-v(\pi)=A-v(\pi)
}$$
We see that $u(0) = -u(\pi)$. The significance of this is that the function
$u(\phi)$
goes from positive to negative, so $u(\phi)=0$ must exist
somewhere.
Since $u(\phi)=0$ exists then by our definition of $u(\phi)$ we have
$$v(\phi+\pi)=v(\phi)$$
for some $\phi$.

\end
