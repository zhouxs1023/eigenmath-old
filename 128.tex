1. Suppose 70\% of rocks in a riverbed are sedimentary in type.
A sample of 20 rocks are collected in random locations.

\bigskip
{\bf (a) Find the probability that exactly 12 rocks are
sedimentary in type.}

\bigskip
Solution: There are 20 trials with probability
of success 0.7 per trial.
Looks like a binomial distribution with $n=20$, $p=0.7$.
$$p(12)={}_{20}C_{12}(0.7)^{12}(0.3)^{20-12}=0.1144$$

\bigskip
{\bf (b) Find the probability that more than 6 rocks
are sedimentary in type.}

\bigskip
Solution:
$$P(X>6)=\hbox{\tt 1-BINOMDIST(6,20,0.7,TRUE)}=0.9997$$

\bigskip
{\bf (c) Find the probability that less than 16 rocks
are sedimentary in type.}

\bigskip
Solution:
$$P(X\le15)=\hbox{\tt BINOMDIST(15,20,0.7,TRUE)}=0.7625$$

\bigskip
{\bf (d) Find the probability that more than 12 and less
than 15 rocks are sedimentary in type.}

\bigskip
Solution: The trick here is to pay close attention to the sample number
used for {\tt BINOMDIST}.
$$P(12<X<15)=\hbox{\tt BINOMDIST(14,20,0.7,TRUE)-BINOMDIST(12,20,0.7,TRUE)}=0.3560$$

\vfill
\eject

2. An urn contains two red and three white balls.
Sample successively and independently with replacement from the urn.

\bigskip
{\bf (a) Find the probability that a second red ball occurs on the fourth draw.}

\bigskip
Solution: We need two successes so this looks like a Pascal distribution with $k=2$
and $p=2/5$.
$$p(4)={}_{4-1}C_{2-1}(0.4)^2(0.6)^{4-2}
=\hbox{\tt NEGBINOMDIST(2,2,0.4)}=0.1728$$

\bigskip
{\bf (b) Find the probability that at most five draws are needed to observe
the third red ball.}

\bigskip
Solution: In terms of {\tt NEGBINOMDIST}, the parameters are 0, 1, 2 failures
and 3 successes.
$$\hbox{\tt NEGBINOMDIST(0,3,0.4)+NEGBINOMDIST(1,3,0.4)+NEGBINOMDIST(2,3,0.4)}=0.3174$$

\bigskip
{\bf (c) Find the probability that at least five draws are needed to observe
the second white ball.}

\bigskip
Solution: Note that we have switched from red to white, $p=3/5$.
In terms of {\tt BINOMDIST}, we need 0--2 failures and 2 successes.
$$P(X\ge5)=
\hbox{\tt 1-NEGBINOMDIST(0,2,0.6)-NEGBINOMDIST(1,2,0.6)-NEGBINOMDIST(2,2,0.6)}
=0.1792$$

\vfill
\eject

3. Suppose twelve independent identical laboratory experiments are to be
undertaken that are extremely sensitive to the environmental conditions, and there
is only a probability 0.4 that an experiment will be completed
successfully.

\bigskip
{\bf (a) Find the probability that exactly five successful experiments occur.}

\bigskip
Solution: 5 successes, 12 trials.
$$p(5)=\hbox{\tt BINOMDIST(5,12,0.4,FALSE)}=0.2270$$

\bigskip
{\bf (b) Find the probability that at least six experiments fail.}

\bigskip
Solution: In other words, six or fewer successes.
$$P(X\le6)=\hbox{\tt BINOMDIST(6,12,0.4,TRUE)}=0.8418$$

\bigskip
{\bf (c) Find the probability that less than four experiments are successful.}

\bigskip
Solution:
$$P(X\le3)=\hbox{\tt BINOMDIST(3,12,0.4,TRUE)}=0.2253$$

\vfill
\eject

4. The ZZZ Company plans to visit potential customers until a substantial
sale is made. Each sales presentation costs \$1000.
It costs \$3000 to travel to the next customer and set up a new
presentation.
The probability of making a sale after a presentation is known to be 0.1.

\bigskip
{\bf (a) What is the expected cost of making a sale?}
\bigskip
Solution: Use the negative binomial distribution with $k=1$.
The expected number of presentations is
$$\mu={k\over p}={1\over0.1}=10$$
The expected cost is therefore $10\times\$4000=\$40000$.

\bigskip
{\bf (b) If the expected profit at each sale is \$15000, should the
trips be undertaken? Why?}
\bigskip
No, because the average cost of each sale is \$40000, therefore ZZZ
loses \$25000 per sale.

\bigskip
{\bf (c) If the budget for advertising is only \$100000, what is the
probability that this sum will be spent without getting an order?}
\bigskip
First we have to figure out how many sales presentations $x$ will cost
a total of \$100000. We have
$$4000x=100000$$
$$x=25$$
The probability of 25 failures is
$$p(25)=(0.9)^{25}=0.0718$$
Hence the probability is 7.2\% of spending \$100000 and not getting an order.

\vfill
\eject

5. Suppose that the number of accidents to employees working on high-explosive shells
over a period of time is taken to follow a Poisson distribution with average number of
accidents 2 per five weeks.

\bigskip
{\bf (a) Find the probability of exactly two accidents for the next five weeks.}
\bigskip
Solution: The average number of accidents is 2 per five weeks, hence $\lambda=2$.
$$p(2)={2^2e^{-2}\over2!}=0.2707$$

\bigskip
{\bf (b) Find the probability of more than three accidents for the next five weeks.}
\bigskip
Solution:
$$P(X>3)=\hbox{\tt 1-POISSON(3,2,TRUE)}=0.1429$$

\bigskip
{\bf (c) Find the probability of exactly four accidents for the next ten weeks.}
\bigskip
Solution: The average for ten weeks is twice the average for five weeks, $\lambda=4$.
$$p(4)={4^4e^{-4}\over4!}=0.1954$$

\bigskip
{\bf (d) Find the probability of less than six accidents for the next fifteen weeks.}
\bigskip
Solution: $\lambda=6$
$$P(X<6)=\hbox{\tt POISSON(5,6,TRUE)}=0.4457$$

\vfill
\eject

6. Two players each put one dollar into a pot.
They then decide to throw a pair of dice alternately.
The first one who throws a sum of 5 wins the pot.
How much should the player who starts add to the pot to make
this a fair game?
\bigskip
Solution: The chance of rolling a 5 is 1/9.
$$
P(\hbox{A win})={1\over9}+\left({8\over9}\times{8\over9}\times{1\over9}\right)+\cdots
={1\over9}\sum_{n=0}^\infty\left({64\over81}\right)^n
={1\over9}\times{1\over(1-64/81)}={1\over9}\times{81\over17}={9\over17}
$$
$$
P(\hbox{B win})=\left({8\over9}\times{1\over9}\right)
+\left({8\over9}\times{8\over9}\times{8\over9}\times{1\over9}\right)+\cdots
=\left({8\over9}\times{1\over9}\right)\sum_{n=0}^\infty\left({64\over81}\right)^n
={8\over81}\times{1\over(1-64/81)}={8\over81}\times{81\over17}={8\over17}
$$
To equalize the payoff we should have
$${8\over17}(\$2+x)-\$1=0$$
$$x=1/8$$
Therefore the first player should add 12.5 cents to the pot.

\end
