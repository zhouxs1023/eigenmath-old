\noindent{\it George Weigt -- Geometry Homework \#4}

\beginsection Page 51, problem 3.

If $A=(4,7)$, $B=(1,1)$, and $C=(2,3)$ prove that $A{-}C{-}B$ in the Taxicab Plane.
\medskip\noindent
Solution:
First, show that $A$, $B$, and $C$ are collinear by showing
$C\in\overleftarrow A\overrightarrow B$.
$$m={y_2-y_1\over x_2-x_1}={1-7\over1-4}=2,\qquad b=y_2-mx_2=1-2=-1$$
Hence
$\overleftarrow A\overrightarrow B=L_{2,-1}$.
Verify that $C=(2,3)$ satisfies the equation of the line $L_{2,-1}$.
$$y=mx+b,\qquad 3=2(2)-1$$
Therefore $C\in\overleftarrow A\overrightarrow B$.
Next, use the standard ruler for the Taxicab plane $f(x,y)=(1+|m|)x$ to obtain
$$\eqalign{
f(A)&=f(4,7)=12\cr
f(C)&=f(2,3)=6\cr
f(B)&=f(1,1)=3\cr
}$$
We have $12*6*3$ therefore $A{-}C{-}B$.

\beginsection Page 52, problem 6.

If $A{-}B{-}C{-}D$ in a metric geometry, prove that
$\{A,B,C,D\}$ is a collinear set.
\medskip\noindent
Solution: Let $\ell=\overleftarrow A\overrightarrow B$.
By definition we have $A{-}B{-}C$ and $A{-}B{-}D$
hence $C\in\ell$ and $D\in\ell$.
Therefore $\{A,B,C,D\}\in\ell$ which means that it is a collinear set.

\beginsection Page 52, problem 7.

Prove that if $A{-}B{-}C$ and $B{-}C{-}D$ in a metric geometry,
then $A{-}B{-}D$ and $A{-}C{-}D$ also so that $A{-}B{-}C{-}D$.
\medskip\noindent
Solution: Let $\ell=\overleftarrow B\overrightarrow C$.
Then $A{-}B{-}C$ and $B{-}C{-}D$ imply that $A,D\in\ell$
therefore $A$, $B$, $C$, and $D$ are collinear.
Next, there are two cases to consider.
\item{i.} If $f(A)<f(B)<f(C)$ then by $B{-}C{-}D$ we have $f(B)<f(C)<f(D)$.
\item{ii.} If $f(A)>f(B)>f(C)$ then by $B{-}C{-}D$ we have $f(B)>f(C)>f(D)$.

\noindent
Hence $f(A)*f(B)*f(D)$ and $f(A)*f(C)*f(D)$.
Therefore $A{-}B{-}D$ and $A{-}C{-}D$.
Now we have all four betweeness cases hence $A{-}B{-}C{-}D$.

\beginsection Page 52, problem 10.

In the Taxicab Plane, find three points $A$, $B$, $C$ which are not collinear
but $d_T(A,C)=d_T(A,B)+d_T(B,C)$. This problem shows why the definition of
between requires collinear points.

\medskip\noindent
Solution: $A=(0,1)$, $B=(0,0)$, $C=(1,0)$.

\beginsection Page 58, problem 1.

Complete the solution of Example 3.3.1.
\medskip\noindent
Solution:
Let $A=(x_1,y_1)$ and $B=(x_2,y_2)$ be on the type II line ${}_cL_r$
in the Poincare plane.
Show that $C=(x,y)\in{}_cL_r$ and $x_1\le x\le x_2$ implies that $C\in\overline{AB}$.
\medskip\noindent
There are three cases to consider.
\item{i.} If $x=x_1$ then $C=A$ hence $C\in\overline{AB}$.
\item{ii.} If $x=x_2$ then $C=B$ hence $C\in\overline{AB}$.
\item{iii.} Otherwise let $f(A)=t_1$, $f(B)=t_2$, and $f(C)=t$.
We have
$$x_1=c+r\tanh t_1,\qquad x=c+r\tanh t,\qquad x_2=c+r\tanh t_2$$
Then $x_1<x<x_2$ implies
$$(c+r\tanh t_1)<(c+r\tanh t)<(c+r\tanh t_2)$$
Since $\tanh(t)$ is strictly increasing we have $t_1<t<t_2$.
Therefore we have $A{-}C{-}B$ which implies $C\in\overline{AB}$.

\beginsection Page 58, problem 2.

Prove Proposition 3.3.3, i.e. in the Euclidean plane, line segments and rays
are given by
$$\eqalign{
\overline{AB}
&=\{C\in R^2\mid C=A+t(B-A)\quad\hbox{for some $t$ with $0\le t\le 1$}\}\cr
\overrightarrow{AB}
&=\{C\in R^2\mid C=A+t(B-A)\quad\hbox{for some $t\ge 0$}\}\cr
}$$

\medskip\noindent
{\bf Solution for the line segment.}
Let $C\in\overline{AB}$. There are three cases to consider.
\item{i.} If $C=A$ then $C=A+0(B-A)$.
\item{ii.} If $C=B$ then $C=A+1(B-A)$.
\item{iii.} Otherwise we have $A-C-B$ which implies $C=A+t(B-A)$ with $0<t<1$ by
Proposition 3.2.5 in the book or by Proposition 4.5 in Course Notes.
\par\noindent
Therefore
$$\overline{AB}\subset\{C\in R^2\mid C=A+t(B-A)\quad\hbox{for some $t$ with $0\le t\le 1$}\}$$
\par
\noindent
Now prove the converse. Let $C=A+t(B-A)$, $0\le t\le 1$.
There are three cases to consider.
\item{i.} If $t=0$ then $C=A$ hence $C\in\overline{AB}$.
\item{ii.} If $t=1$ then $C=B$ hence $C\in\overline{AB}$.
\item{iii.} Otherwise we have $0<t<1$ hence $C\in\overline{AB}$
by Proposition 3.2.5 in the book or by Proposition 4.5 in Course Notes.
\par\noindent
Therefore
$$\overline{AB}\supset\{C\in R^2\mid C=A+t(B-A)\quad\hbox{for some $t$ with $0\le t\le 1$}\}$$
Hence
$$\overline{AB}=\{C\in R^2\mid C=A+t(B-A)\quad\hbox{for some $t$ with $0\le t\le 1$}\}$$

\medskip\noindent
{\bf Solution for the ray.} Let $C\in\overrightarrow{AB}$.
There are two cases to consider.
\item{i.} If $C\in\overline{AB}$ then $C=A+t(B-A)$, $0\le t\le 1$.
\item{ii.} Otherwise we have $A{-}B{-}C$ which implies $B=A+s(C-A)$,
$0<s<1$.
This can be rewritten as $C=A+(B-A)/s$.
Let $t=1/s$. Then $C=A+t(B-A)$, $t>1$.
\par
\noindent
Therefore
$$\overrightarrow{AB}\subset\{C\in R^2\mid C=A+t(B-A)\quad\hbox{for some $t\ge 0$}\}$$
\par
\noindent
Now prove the converse.
Let $C=A+t(B-A)$, $t\ge0$.
There are two cases to consider.
\item{i.} If $0\le t\le 1$ then $C\in\overline{AB}$ hence $C\in\overrightarrow{AB}$.
\item{ii.} Otherwise we have $t>1$. Let $s=1/t$.
Then $0<s<1$ and $B=A+s(C-A)$ hence $A{-}B{-}C$.
\par\noindent
Therefore
$$\overrightarrow{AB}\supset\{C\in R^2\mid C=A+t(B-A)\quad\hbox{for some $t\ge 0$}\}$$
Hence
$$\overrightarrow{AB}=\{C\in R^2\mid C=A+t(B-A)\quad\hbox{for some $t\ge 0$}\}$$

\beginsection Page 58, problem 11.

Suppose that $A$ and $B$ are distinct points in a metric geometry.
$M\in\overleftarrow A\overrightarrow B$ is called a midpoint of
$\overline{AB}$ if $AM=MB$.

\medskip\noindent
{\bf Part a.} If $M$ is a midpoint of $\overline{AB}$ prove that $A{-}M{-}B$.
\medskip\noindent
Solution: $A$, $B$, and $M$ are collinear by $M\in\overleftarrow A\overrightarrow B$.
Note that $M\ne A$ and $M\ne B$ because otherwise $AM=MB$ would imply
$A=B$ which contradicts the hypothesis.
Next, there are four cases to consider.
\item{i.} If $f(A)<f(M)$ and $f(M)<f(B)$ then $A{-}M{-}B$.
\item{ii.} If $f(A)>f(M)$ and $f(M)>f(B)$ then $A{-}M{-}B$.
\item{iii.} If $f(A)<f(M)$ and $f(M)>f(B)$ then $AM=MB$ implies
$f(M)-f(A)=f(M)-f(B)$ which implies $A=B$ which contradicts the hypothesis.
\item{iv.} If $f(A)>f(M)$ and $f(M)<f(B)$ then $AB=MB$ implies
$f(A)-f(M)=f(B)-f(M)$ which implies $A=B$ which contradicts the hypothesis.
\par\noindent
Therefore by (i) and (ii) we have $A{-}M{-}B$.

\medskip\noindent
{\bf Part b.}
If $A=(0,4)$ and $B=(0,1)$ are points on the Poincare Plane find a midpoint,
$M$, of $\overline{AB}$. Sketch $A$, $B$ and $M$ on a graph.
Does $M$ look like a midpoint?
\medskip\noindent
Solution: We have $x_1=x_2$ therefore it is a type I line with
$$f(0,y)=\ln y$$
We have
$$\ln 4-\ln y=\ln y-\ln 1$$
hence $y=2$ and $M=(0,2)$. It looks like a midpoint in the logarithmic sense.
\medskip\noindent
Here is a sketch.

{\obeylines\tt
y=4 * A
y=3 *
y=2 * M
y=1 * B
y=0 *
}

\beginsection Page 58, problem 12.

If $A$ and $B$ are distinct points of a metric geometry, prove that
\item{a.} the segment $\overline{AB}$ has a midpoint $M$.
\item{b.} the midpoint $M$ of $\overline{AB}$ is unique.

\medskip\noindent
{\bf Part a.}
Define a $t\in R$ such that
$$t=f(A)+{f(B)-f(A)\over 2}$$
Because $f$ is surjective there must be a point $M\in\overleftarrow A\overrightarrow B$
such that $f(M)=t$ hence
$$f(M)=f(A)+{f(B)-f(A)\over 2}$$
This can be rewritten as
$$f(A)-f(M)=f(M)-f(B)$$
The equality is not lost by taking the absolute value of both sides.
$$|f(A)-f(M)|=|f(M)-f(B)|$$
Hence we have $AM=MB$.
In problem 11 it was proved that $AM=MB$ implies $A{-}M{-}B$.
Therefore $M\in\overline{AB}$.

\medskip\noindent
{\bf Part b.}
Assume that there is another point $X\in\overline{AB}$ such that $AX=XB$.
Then $AM=MB=AX=XB=AB/2$.
There are two cases to consider.
\item{i.} If $A{-}M{-}X{-}B$ then $AM+MX+XB=AB$ hence $MX=0$.
\item{ii.} If $A{-}X{-}M{-}B$ then $AX+MX+MB=AB$ hence $MX=0$.
\par\noindent
Therefore $X=M$ hence $M$ is unique.

\beginsection Page 52, problem 13 (extra credit)

Let $A$, $B$, and $C$ be three points on a great circle on the Riemann Sphere.
Doesn't it look like any one of them is between the other two?
How would you explain this apparent contradiction of Corollary 3.2.4?
There is a fundamental concept of betweeness involved here.

\medskip\noindent
Solution: The paradox is resolved by the ruler concept of a metric geometry.
The ruler assigns to each point a distinct real number, $f(A)$, $f(B)$, and $f(C)$.
Only one of the real numbers is between the other two.
The middle number corresponds to the unique point that is between the other two.

\beginsection Page 59, problem 20 (extra credit)

If $D\in\overleftarrow A\overrightarrow B$ in a metric geometry,
prove that either $\overrightarrow{AD}=\overrightarrow{AB}$ or
$\overrightarrow{AD}\cup \overrightarrow{AB}=\overleftarrow A\overrightarrow B$.

\medskip\noindent
Counterexample: If $D=A$ then $\overrightarrow{AD}$ does not exist,
hence $\overrightarrow{AD}\ne\overrightarrow{AB}$
and $\overrightarrow{AD}\cup \overrightarrow{AB}\ne\overleftarrow A\overrightarrow B$.

\medskip\noindent
If we add the condition that $D\ne A$ then there are a number of cases to consider.
\item{i.} If $A{-}D{-}B$ or $A{-}B{-}D$ or $D=B$ then $D\in\overrightarrow{AB}$.
Then by Theorem 4.19 (Theorem 3.3.4 in the book)
we have $\overrightarrow{AD}=\overrightarrow{AB}$.
\item{ii.} Otherwise, $D{-}A{-}B$.
We have by construction $\overrightarrow{AD}\subset\overleftarrow A\overrightarrow B$
and $\overrightarrow{AB}\subset\overleftarrow A\overrightarrow B$,
hence $\overrightarrow{AD}\cup \overrightarrow{AB}\subset\overleftarrow A\overrightarrow B$.
Now prove the converse.
Let $C\in\overleftarrow A\overrightarrow B$.
If $C{-}D{-}A$ or $D{-}C{-}A$ or $C=D$ or $C=A$ then $C\in\overrightarrow{AD}$.
If $A{-}C{-}B$ or $A{-}B{-}C$ or $C=A$ or $C=B$ then $C\in\overrightarrow{AB}$.
Hence $\overleftarrow A\overrightarrow B\subset\overrightarrow{AD}\cup \overrightarrow{AB}$.
\noindent
Therefore $\overrightarrow{AD}\cup \overrightarrow{AB}=\overleftarrow A\overrightarrow B$.

\medskip\noindent
Therefore in a metric geometry, $D\in\overleftarrow A\overrightarrow B$ and $D\ne A$
implies that either $\overrightarrow{AD}=\overrightarrow{AB}$ or
$\overrightarrow{AD}\cup \overrightarrow{AB}=\overleftarrow A\overrightarrow B$.







\end