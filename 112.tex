\beginsection 1. Viete Biography

\bigskip
\item{\S1.} Life of Viete
\itemitem{1.} Born in France, 1540.
\itemitem{2.}
Graduated from the University of Poitiers with a
law degree in 1560.

\itemitem{3.}
Published his first mathematical treatise in 1571.

\itemitem{4.}
In 1573 was appointed a councillor of the
Parliament of Brittany at Rennes. Afterwards he was
promoted to the parliament in Paris in 1580.

\itemitem{5.}
In 1584 Viete was forced to flee Paris due to
political and religious unrest.
He settled in a small coastal town and worked
exclusively on mathematics for the next
five years.

\itemitem{6.}
Returned to parliament, now in Tours, in 1589.
Gave lectures on mathematics while living in Tours.
Also worked as a cryptographer during this time.

\itemitem{7.}
Returned to Paris in 1594.

\itemitem{8.} Died in Paris, 1603.

\bigskip
\item{\S2.} Work of Viete
\itemitem{1.}
Promoted the use of decimal fractions in computations.
At that time, sexagesimals (base 60) were widely used.
In order to separate the integer and fractional parts
of the decimal, Viete used either a vertical line or
wrote the integer part in boldface.

\itemitem{2.}
In his book {\it In artem analyticam isagoge}
(Introduction to the Analytical Art) published in 1591,
Viete introduced the idea that symbols could be
used to represent both known and unknown quantities.
Since this is the fundamental notation used in algebra,
Viete is often called ``the
father of algebra.''
He also coined the term ``coefficient.''

\itemitem{3.}
Calculated $\pi$ to 10 places.
This required a polygon with 393216 sides.
Viete also devised an infinite product that converges to $\pi$.
This is believed to be the earliest analytical representation
for $\pi$.

\itemitem{4.}
Wrote {\it Recensio canonica effectionum geometricarum}
which contained many solutions of geometric problems.

\itemitem{5.}
Published {\it  Canon mathematicus seu ad triangula}
(Mathematical Laws Applied to Triangles) in 1579.
This book contained tables for all six trigonometric
functions.
Viete derived many useful trigonometric identities.
At this time trigonometry tables and identities were
used like logarithms to do multiplication and
division of multidigit quantities.

\itemitem{6.}
Wrote {\it De aequationum recognitione et emendatione}
(Concerning the Recognition and Emendation of Equations)
which was published posthumously in 1615.
It contains methods for solving second,
third and fourth degree equations.

\bigskip
\item{\S3.} References
\itemitem{1.} {\tt http://en.wikipedia.org/wiki/Viete}
\itemitem{2.} {\tt http://www-history.mcs.st-andrews.ac.uk/Biographies/Viete.html}
\itemitem{3.} {\tt http://www.britannica.com/eb/article-9075315/\break
Francois-Viete-seigneur-de-la-Bigotiere}
\itemitem{4.} {\tt http://www.princeton.edu/~mike/articles/mathnat/mathnatnotes}
\itemitem{5.} ``A History of Mathematics,'' by Carl B. Boyer.

\vfill
\eject

\beginsection 2.

Prove the following:
$$\sin\theta=2\cos{\theta\over2}\sin{\theta\over2}$$

\bigskip
We make use of the following sum and difference formulas from geometry.
$$\eqalign{
\sin(\alpha+\beta)&=\sin\alpha\cos\beta+\cos\alpha\sin\beta\cr
\sin(\alpha-\beta)&=\sin\alpha\cos\beta-\cos\alpha\sin\beta\cr
}$$
Adding the two we obtain
$$\sin(\alpha+\beta)+\sin(\alpha-\beta)=2\sin\alpha\cos\beta$$
Now let $\alpha,\beta=\theta/2$.
Since $\sin0=0$ We obtain
$$\sin\theta=2\sin{\theta\over2}\cos{\theta\over2}$$

\vfill
\eject

\beginsection 3.

Show the following:
$$\cos{\pi\over16}={\sqrt{2+\sqrt{2+\sqrt2}}\over2}\eqno(9)$$

\bigskip
Let $\theta=\pi/8$. Then by the half angle formula we have
$$\cos{\pi\over16}={\sqrt{2+2\cos(\pi/8)}\over 2}\eqno(a)$$
From equation (8) in the lecture we have
$$\cos{\pi\over8}={\sqrt{2+\sqrt2}\over 2}\eqno(8)$$
Substituting (8) into ($a$) yields (9).

\vfill
\eject

\beginsection 4.

Show the following:
$$\cos{\pi\over32}={\sqrt{2+\sqrt{2+\sqrt{2+\sqrt2}}}\over2}\eqno(10)$$

\bigskip
Let $\theta=\pi/16$. Then by the half angle formula we have
$$\cos{\pi\over32}={\sqrt{2+2\cos(\pi/16)}\over 2}\eqno(b)$$
Restate (9) proved in the previous problem:
$$\cos{\pi\over16}={\sqrt{2+\sqrt{2+\sqrt2}}\over2}\eqno(9)$$
Substituting (9) into ($b$) yields (10).

\vfill
\eject

\beginsection 5.

$$2
\cdot{2\over\sqrt2}
\cdot{2\over\sqrt{2+\sqrt2}}
\cdot{2\over\sqrt{2+\sqrt{2+\sqrt2}}}
\cdot{2\over\sqrt{2+\sqrt{2+\sqrt{2+\sqrt2}}}}
=3.13655
$$

\end