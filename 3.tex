\magnification=1200
\raggedright
\parindent=0pt

1. Find an exact solution for the following.
$$\cos\left({5\pi\over8}\right)$$

We see that there is an 8 in the denominator.
We know how to do a cosine when there is a 4 in the denominator
so it seems appropriate to use the half angle formula.
Using the half angle formula we have
$$
\cos\left({5\pi\over8}\right)
=
\cos\left({1\over2}\cdot{5\pi\over4}\right)
=
\pm\sqrt{{1\over2}+{1\over2}\cos\left({5\pi\over4}\right)}
$$

We deduce that $5\pi/4$ is $45^\circ$ in the 3rd quadrant.
Imagine a small isosceles triangle with hypotenuse $\sqrt2$.
Then we have $\cos45^\circ$ as adjacent over hypotenuse or $1/\sqrt2$.
The 3rd quadrant means the sign is negative so we use $-1/\sqrt2$
and obtain.

$$
\pm\sqrt{{1\over2}-{1\over2\sqrt2}}
$$

The final step is to determine which sign to use in front
of the radical. Note that the original angle $5\pi/8$ is less
that $\pi$ and greater that $\pi/2$ so it occurs in
the 2nd quadrant.
Therefore we should choose minus from $\pm$ to obtain the
final result

$$
-\sqrt{{1\over2}-{1\over2\sqrt2}}
$$

\end
