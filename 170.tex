\parindent=0pt

{\it George Weigt -- Geometry Homework \#1}

\beginsection Prove Corollary 1.32

In an incidence geometry two lines are either parallel,
or they intersect in exactly one point.

\medskip
Solution: Let $\ell_1$ and $\ell_2$ be two lines in an incidence geometry.
Consider the cardinality of the set $\ell_1\cap\ell_2$.
\medskip
Case 1: $|\ell_1\cap\ell_2|=0$. By Definition 1.30 the lines are parallel.
\medskip
Case 2: $|\ell_1\cap\ell_2|\ge2$. By Theorem 1.31 we have $\ell_1=\ell_2$
and by Definition 1.30 the lines are parallel.
\medskip
Case 3: $|\ell_1\cap\ell_2|=1$.
By Definition 1.13 (ii) every line has at least two points hence
$\ell_1\ne\ell_2$. Then by Definition 1.30 the lines are not parallel.
\medskip
Hence two lines in an incidence geometry are either parallel (case 1 or 2)
or they intersect in exactly one point (case 3, not parallel).

\beginsection 1.

Find the Poincare line through $(1,2)$ and $(3,4)$.
\medskip
Solution: Let $(x_1,y_1)=(1,2)$ and $(x_2,y_2)=(3,4)$.
Then
$$c={y_2^2-y_1^2+x_2^2-x_1^2\over2(x_2-x_1)}=5$$
and
$$r^2=(x_1-c)^2+y_1^2=20$$
Hence
$$\ell=\{(x,y)\in H\mid(x-5)^2+y^2=20\}$$

\beginsection 8.

Show by example that there are (at least) three non-collinear points in
the Poincare Plane.
\medskip
Solution: Let $A=(0,1)$, $B=(0,2)$ and $C=(1,1)$.
We need to prove that no line in the Poincare Plane contains $A$, $B$ and $C$.
Let $\ell$ be a type I line in the Poincare Plane such that
$$\ell=\{(x,y)\in H\mid x=0\}$$
Then we have $A,B\in\ell$ and $C\not\in\ell$.
There is a proof in Course Notes 1 that every two distinct points
in the Poincare Plane lie on a unique line.
Therefore $l$ is the only line that contains both $A$ and $B$.
Since $C\not\in\ell$ and $\ell$ is the only line that contains
$A$ and $B$, there can be no line that contains $A$, $B$ and $C$.
Hence $A$, $B$ and $C$ are non-collinear.

\beginsection 11.

Find all lines in the Poincare Plane through $(0,1)$ which are
parallel to the type I line ${}_6L$.
\medskip
Solution: The solution set is
$${}_0L\cup\{{}_cL_r\mid c+r\le6,\,c^2+1=r^2\}$$
Proof: First prove that all lines in the solution set contain $(0,1)$
and are parallel to ${}_6L$. We have
$${}_0L=\{(x,y)\in H\mid x=0\}$$
hence $(0,1)\in{}_0L$. In addition, ${}_0L\cap{}_6L=\emptyset$
hence ${}_0L$ and ${}_6L$ are parallel.
Type II lines ${}_cL_r$ such that $c^2+1=r^2$ contain $(0,1)$ by the
Pythagorean Theorem.
In addition, for ${}_cL_r$ we have $x\le c+r$. When $c+r\le 6$ there can be no
intersection with ${}_6L$.
Hence the ${}_cL_r$ in the solution set are parallel to ${}_6L$.
\medskip
Next, let us prove that the solution set is complete.
We have $(0,1)\not\in{}_aL$ for all $a\ne0$.
Hence ${}_0L$ is the only type I line that is a solution.
Type II lines such that $c^2+1\ne r^2$ do not contain $(0,1)$.
Finally, type II lines for which $c^2+1=r^2$ and $c+r>6$ intersect ${}_6L$ at $y^2=12c-35$ and hence
are not parallel to ${}_6L$.

\beginsection 15.

Let $S=R^2-\{(0,0)\}$ and $L$ be the set of all Cartesian lines which lie
in $S$.
Show that $\{S,L\}$ is not an incidence geometry.
\medskip
Solution: Cartesian lines that include $\{0,0\}$ do not lie in $S$
and hence are excluded from $L$.
Consequently there exist distinct points in $S$, such as $(1,0)$ and $(2,0)$,
that do not lie on a line in $L$ because the line passes through the origin.
Hence $\{S,L\}$ is not an abstract geometry which implies that it is not
an incidence geometry either. 


\beginsection 19.

(i) In each example list the set of lines.
(a) $PQ$, $PR$, $QS$, $RS$.
(b) $PQ$, $PS$, $QS$, $RS$.
(c) $PQ$, $PR$, $PS$, $QS$, $RS$.
(d) $PQR$, $PTV$, $RUV$, $PSU$, $RST$, $QTU$.
(e) $PR$, $PQR$, $PS$, $QS$, $RS$.
(f) $PQ$, $PR$, $PS$, $RS$, $QS$.

\medskip
(ii) Which of these geometries are abstract geometries?
Pictures (d), (e) and (f) are abstract geometries because in
these pictures every two points lie on some line and there is
no line with only one point.
Picture (a) is not an abstract geometry because there is no line
for $PS$ and no line for $QR$.
Picture (b) is not an abstract geometry because there is no line
for $PR$ and no line for $QR$.
Picture (c) is not an abstract geometry because there is no line
for $QR$.

\medskip
(iii) Which of these geometries are incidence geometries?
Only (d) and (f) are incidence geometries.
Pictures (a), (b) and (c) are not abstract geometries therefore
they are not incidence geometries.
In (e), $PR$ lies on two lines therefore $PR$ does not
define a unique line hence (e) is not an incidence geometry.

\beginsection 20.

Let $\{S,L\}$ be an abstract geometry and assume that
$S_1\subset S$.
We define an $S_1$-line to be any subset of $S_1$ of the form
$l\cap S_1$ where $l$ is a line of $S$ and where $l\cap S_1$ has
at least two points.
Let $L_1$ be the collection of all $S_1$-lines.
Prove that $\{S_1,L_1\}$ is an abstract geometry.
\medskip
Solution: First show that for every two points $A,B\in S_1$ there is
a line $l_1\in L_1$ with $A,B\in l_1$.
Note that since $\{S,L\}$ is an abstract geometry and $S_1\subset S$,
there must exist an $l\in L$
that contains both $A$ and $B$.
Hence there is a corresponding $l_1$ of the form
$l_1=l\cap S_1$.
Since $A,B\in l$ and $A,B\in S_1$, it follows that $A,B\in l_1$.
\medskip
Next, show that every line $l_1\in L_1$ has at least two points.
By definition every $S_1$-line has at least two points.
Since $L_1$ is the collection of all $S_1$-lines, every
$l_1\in L_1$ has at least two points.

\beginsection 23.

Let $\{S_1,L_1\}$ and $\{S_2,L_2\}$ be abstract geometries.
If $S=S_1\cap S_2$ and $L=L_1\cap L_2$
prove that $\{S,L\}$ is an abstract geometry.
\medskip
Counterexample: $S_1,S_2=\{A,B,C\}$, $L_1=\{\{A,B\},\{A,C\},\{B,C\}\}$,
$L_2=\{\{A,B,C\}\}$.
We have
$$S=\{A,B,C\}$$
$$L=\{\}$$
Hence $\{S,L\}$ is not an abstract geometry.

\beginsection 25.

There is a finite geometry with 7 points such that each line has exactly 3
points on it.
Find this geometry.
How many lines are there?
\medskip
Solution:
$$S=\{A,B,C,D,E,F,G\}$$
$$L=\{\{A,B,C\},\{A,D,E\},\{A,F,G\},
\{B,D,F\},\{B,E,G\},\{C,D,G\},\{C,E,F\}\}$$
There are 7 lines.
Proof:
There are $C(7,2)=21$ pairs of distinct points.
We have
$$\eqalign{
AB,AC,BC&\in\{A,B,C\}\cr
AD,AE,DE&\in\{A,D,E\}\cr
AF,AG,FG&\in\{A,F,G\}\cr
BD,BF,DF&\in\{B,D,F\}\cr
BE,BG,EG&\in\{B,E,G\}\cr
CD,CG,DG&\in\{C,D,G\}\cr
CE,CF,EF&\in\{C,E,F\}\cr
}$$

\end