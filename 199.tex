\documentclass[12pt,openany]{report}
\usepackage{graphicx}
\begin{document}

\noindent
{\it George Weigt --- Geometry Homework \#12}

\section*{Page 148, problem 2.}

In a neutral geometry, if $D$ is the foot of the altitude of $\triangle ABC$
from $C$ and $A{-}B{-}D$, then prove $\overline{CA}>\overline{CB}$.

\includegraphics[scale=0.5]{102.png}

\noindent
Solution:

\begin{itemize}

\item[]
By Theorem 13.3 the hypotenuse $\overline{CA}$ is the longest side of $\triangle ADC$.

\item[]
Then by $A{-}B{-}D$ and Theorem 12.13 we have $\overline{CA}>\overline{CB}\ge\overline{CD}$.

\item[]
Therefore $\overline{CA}>\overline{CB}$.

\end{itemize}

\newpage

\section*{Page 148, problem 4.}

Show that the conclusion of Theorem 6.4.2 is not valid in the Taxicab Plane
by taking $P=(-1,1)$, $\ell=\{(x,y)\mid y=x\}$ and $Q=(1,1)$.

\includegraphics[scale=0.5]{104.png}

\noindent
Solution:

\begin{itemize}

\item[]
$PQ=|-1-1|+|1-1]=2$

\item[]
Let $R=(x,x)$.

\item[]
If $x>1$ then $PR=|-1-x|+|1-x|=(x+1)+(x-1)=2x$.

\item[]
If $x<1$ then $PR=|-1-x|+|1-x|=(-x-1)+(-x+1)=-2x$

\item[]
Hence $PR\ge PQ$ for all $R\in\ell$.

\item[]
Therefore by Theorem 6.4.2 we have $\overleftarrow P\overrightarrow Q\perp\ell$.

\item[]
However, $\angle PQO$ is not a right angle.
$$m(\angle PQO)=\cos^{-1}{\langle P-Q,O-Q\rangle\over\|P-Q\|\cdot\|O-Q\|}
=\cos^{-1}{2\over2\sqrt2}=45$$

\item[]
Therefore Theorem 6.4.2 is not valid in the Taxicab Plane.

\end{itemize}

\newpage

\section*{Page 148, problem 5.}

Show that the conclusion of the Pythagorean Theorem is not valid in the Taxicab Plane.

\includegraphics[scale=0.5]{105.png}

\noindent
Solution:

\begin{itemize}

\item[]
Let $A=(0,1)$, $B=(0,0)$ and $C=(1,0)$.

\item[]
Then
$$m(\angle ABC)=\cos^{-1}{\langle A-B,C-B\rangle\over\|A-B\|\cdot\|C-B\|}
=\cos^{-1}{0\over1}=90$$
hence $\angle ABC$ is a right angle and $\overline{AC}$ is the hypotenuse.

\item[]
In the Taxicab Plane we have $AB=1$, $BC=1$ and $AC=2$.

\item[]
It follows that $AC^2\ne AB^2+BC^2$.

\item[]
Therefore the Pythagorean Theorem is not valid in the Taxicab Plane.

\end{itemize}

\newpage

\section*{Page 149, problem 10.}

Prove the following theorem (Theorem 6.4.7).

\medskip
\noindent
In a neutral geometry, if $\overline{BD}$ is the bisector of $\angle ABC$ and
if $E$ and $F$ are the feet of the perpendiculars from $D$ to $\overline{BA}$ and
$\overline{BC}$ then $\overline{DE}\cong\overline{DF}$.

\includegraphics[scale=0.5]{110.png}

\noindent
Solution:

\begin{itemize}

\item[]
By hypothesis $\angle DBE\cong\angle DBF$.

\item[]
Then by HA ($\overline{BD}$ is the hypotenuse) we have $\triangle BDE\cong\triangle BDF$.

\item[]
Therefore $\overline{DE}\cong\overline{DF}$.

\end{itemize}

\newpage

\section*{Page 149, problem 13.}

In a neutral geometry, let $\ell_1$, $\ell_2$, and $\ell_3$ be the perpendicular
bisectors of the three sides of $\triangle ABC$.
If $D\in\ell_1\cap\ell_2$, prove that $D\in\ell_3$.

\includegraphics[scale=0.5]{113.png}

\noindent
Solution:

\begin{itemize}

\item[]
Let $\ell_1$ be the perpendicular bisector of $\overline{AB}$
and $\ell_2$ be the perpendicular bisector of $\overline{BC}$.

\item[]
Then by Theorem 13.13 $\overline{AD}\cong\overline{BD}\cong\overline{CD}$.

\item[]
Since $\overline{AD}\cong\overline{CD}$, by Theorem 13.13
$D\in\ell_3$.

\end{itemize}

\newpage

\section*{Page 149, problem 16.}

Find the error or errors in the following alleged ``proof'' that in a neutral geometry
any triangle is isosceles.

\begin{itemize}

\item[]
Let $M$ be the midpoint of $\overline{AC}$ and let $\ell$ be the perpendicular
to $\overline{AC}$ at $M$.

\item[]
Let $\overrightarrow{BQ}$ be the angle bisector of $\angle ABC$ and let
$D\in\ell\cap\overrightarrow{BQ}$.

\item
{\it The proof has not established $B{-}D{-}Q$ or $B{-}Q{-}D$.
In other words, the proof has not established $\ell\cap\overrightarrow{BQ}\ne\emptyset$,
although it does appear to be true in general.}

\item[]
If $E$ is the foot of the perpendicular from $D$ to $\overleftarrow B \overrightarrow C$
and if $F$ is the foot of the perpendicular from $D$ to $\overleftarrow B\overrightarrow A$,
then $\overline{FD}\cong\overline{ED}$ by Theorem 6.4.7.

\item
{\it Because of the previous comment, the proof has not established that
$D\in\mathop{\rm int}(\angle ABC)$, although it does appear to be true
in general.}

\item[]
$\overline{AD}\cong\overline{CD}$ by Theorem 6.4.6.

\item[]
Hence $\triangle AFD\cong\triangle CED$ by HL and $\overline{AF}\cong\overline{CE}$.

\item[]
Since $\triangle BDF\cong\triangle BDE$ (by HA), $\overline{BF}\cong\overline{BE}$.

\item[]
Hence $BA=BF+FA=BE+EC=BC$ and $\overline{BA}\cong\overline{BC}$.

\item
{\it The proof has not established $B{-}F{-}A$ or $B{-}E{-}C$.
Therefore $BA=BF+FA$ and $BE+EC=BC$ are not true in general.}

\end{itemize}


\end{document}
