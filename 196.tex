\documentclass[12pt,openany]{report}
\usepackage{graphicx}
\begin{document}

\section*{Page 141, problem 2.}

Prove the following theorem (Theorem 6.3.2).

\medskip
\noindent
In a protractor geometry, $\angle ABC<\angle DEF$ if and only if
there is a point $G\in\mathop{\rm int}(\angle DEF)$ with
$\angle ABC\cong\angle DEG$.

\bigskip
\noindent
Solution:

\includegraphics[scale=0.5]{101.png}

\medskip
\noindent
First prove the implication.
That is, $G\in\mathop{\rm int}(\angle DEF)$ such that 
$\angle ABC\cong\angle DEG$ implies $\angle ABC<\angle DEF$.

\begin{itemize}

\item[]
By the definition of angle measure we have
$m(DEG)+m(GEF)=m(DEF)$.

\item[]
Hence $m(DEG)<m(DEF)$.

\item[]
By hypothesis $\angle ABC\cong\angle DEG$ therefore $m(ABC)<m(DEF)$.

\item[]
Therefore $\angle ABC<\angle DEF$.

\end{itemize}

\noindent
Now prove the converse.
That is, prove $\angle ABC<\angle DEF$ implies that there is a $G\in\angle DEF$
such that $\angle ABC\cong\angle DEG$.

\begin{itemize}

\item[]
Let $H_1$ be the half plane of $\overleftarrow E\overrightarrow D$ that contains $F$.

\item[]
By the angle construction axiom there is a ray $\overrightarrow{EG}$ such
that $G\in H_1$ and $m(\angle DEG)=m(\angle ABC)$.

\item[]
By $\angle ABC<\angle DEF$ we have $m(DEG)<m(DEF)$.

\item[]
Points $G$ and $F$ are on the same side of $\overleftarrow E\overrightarrow D$.

\item[]
Therefore by Theorem 9.13 we have $G\in\mathop{\rm int}(DEF)$.

\end{itemize}

\newpage

\section*{Page 141, problem 4.}

In a neutral geometry prove that the base angles of an isosceles triangle are acute.

\includegraphics[scale=0.5]{104.png}

\noindent
Solution:

\begin{itemize}

\item[]
Let $\triangle ABC$ be an isosceles triangle with $\overline{AB}\cong\overline{BC}$.

\item[]
Define $D\in\overleftarrow A\overrightarrow C$ such that $D{-}A{-}C$.

\item[]
By the Exterior Angle Theorem $\angle BAD>\angle BCA$.

\item[]
By Pons Asinorum $\angle BCA\cong\angle BAC$ hence $\angle BAD>\angle BAC$

\item[]
By the Linear Pair Theorem $m(\angle BAD)+m(\angle BAC)=180$.

\item[]
Therefore $\angle BAC<90$.

\item[]
Also $\angle BCA<90$ since $\angle BCA\cong\angle BAC$.

\end{itemize}

\newpage

\section*{Page 141, problem 5.}

Prove the following theorem (Theorem 6.3.7).

\medskip
\noindent
In a neutral geometry, if two angles of a triangle are not congruent,
neither are the opposite sides. Furthermore, the longer side
is opposite the larger angle.

\bigskip
\noindent
Solution: Assume for a moment that the opposite sides are congruent.
Then it is an isosceles triangle and by Pons Asinorum the angles
are congruent. This contradicts the hypothesis.
Therefore the opposite sides are not congruent.

\medskip
\noindent
By Theorem 12.9, if two sides of a triangle are not congruent,
then the larger angle is opposite the longer side.
Since ``opposite'' is reflexive, the longer side is opposite the larger angle.

\newpage

\section*{Page 142, problem 7.}

In a neutral geometry, if $D\in\mathop{\rm int}(\triangle ABC)$ prove that
$$AD+DC<AB+BC\qquad\hbox{and}\qquad\angle ADC>\angle ABC$$
(Hint: $\overrightarrow{AD}$ intersects $\overline{BC}$ at a point $E$.)

\includegraphics[scale=0.5]{107.png}

\noindent
Solution:

\begin{itemize}

\item[]
By the Crossbar Theorem, $\overrightarrow{AD}$ intersects $\overline{BC}$ at
$E$ with $B{-}E{-}C$.

\item[]
By the definition of a triangle, $A$ and $D$ are on the same side of
$\overleftarrow B\overrightarrow C$ hence $A{-}D{-}E$.

\item[]
By the Triangle Inequality Theorem we have $AE<AB+BE$.

\item[]
By the same theorem we have $CD<CE+ED$.

\item[]
Adding them together we have $AE+CD<AB+BE+CE+ED$.

\item[]
Subtract through by $ED$ to obtain $AE-ED+CD<AB+BE+CE$.

\item[]
Note that $AD=AE-ED$ and $BC=BE+CE$.

\item[]
Therefore $AD+DC<AB+BC$.

\end{itemize}

\newpage

\includegraphics[scale=0.5]{107.png}

\noindent
Now prove the second part.

\begin{itemize}

\item[]
Let $F\in\overrightarrow{AE}$ such that $A{-}E{-}F$.

\item[]
Let $\alpha=\angle BEF$ and $\beta=\angle CEF$.

\item[]
By the Exterior Angle Theorem we have $\alpha>\angle ABC$ and $\beta>\angle CDE$.

\item[]
Adding them together we have $m(\alpha)+m(\beta)>m(\angle ABC)+m(\angle CDE)$.

\item[]
By the Linear Pair Theorem $180>m(\angle ABC)+180-m(\angle ADC)$.

\item[]
In other words $m(\angle ADC)+180>m(\angle ABC)+180$.

\item[]
Therefore $\angle ADC>\angle ABC$.

\end{itemize}

\newpage

\section*{Page 142, problem 8.}

Prove the following theorem (Theorem 6.3.10).

\medskip
\noindent
In a neutral geometry, a line segment joining a vertex of
a triangle to a point on the opposite side is shorter than
the longer of the remaining two sides.
More precisely, given $\triangle ABC$ with $\overline{AB}\le\overline{CB}$,
if $A{-}D{-}C$ then $\overline{DB}\le\overline{CB}$.

\includegraphics[scale=0.5]{108.png}

\noindent
Solution:

\begin{itemize}

\item[]
If $\overline{AB}=\overline{CB}$ then $\triangle ABC$ is isosceles and $\angle A\cong\angle C$.

\item[]
Otherwise, by Theorem 12.9, $\angle A>\angle C$.

\item[]
Hence $\angle A\ge\angle C$.

\item[]
Let $E\in\overrightarrow{BD}$ such that $B{-}D{-}E$.

\item[]
Let $\alpha=\angle ADE$.

\item[]
By the Exterior Angle Theorem $\alpha>\angle A$ therefore $\alpha>\angle C$.

\item[]
By the Vertical Angle Theorem $\angle BDC\cong\alpha$ therefore $\angle BDC>\angle C$.

\item[]
$\overline{DB}$ is opposite the smaller angle $\angle C$.

\item[]
Therefore by Theorem 12.10 $\overline{DB}<\overline{CB}$.

\end{itemize}

\newpage

\section*{Page 142, problem 9.}

Prove the converse of Theorem 6.3.9: In a neutral geometry, given $\triangle ABC$ and
$\triangle DEF$, if $\overline{AB}\cong\overline{DE}$,
$\overline{BC}\cong\overline{EF}$,
and $\overline{AC}>\overline{DF}$, then $\angle B>\angle E$.

\bigskip
\noindent
Solution:

\begin{itemize}

\item[]
By the Open Mouth Theorem, if $\angle B<\angle E$ then $\overline{AC}<\overline{DF}$.

\item[]
The contrapositive is: if $\overline{AC}\ge\overline{DF}$ then $\angle B\ge\angle E$.

\item[]
Now all we have to do is get rid of the ``or equal.''
\item[]
If $\overline{AC}\cong\overline{DF}$ then by SSS $\triangle ABC\cong\triangle DEF$
hence $\angle B\cong\angle E$.

\item[]
If $\angle B\cong\angle E$ then by SAS $\triangle ABC\cong\triangle DEF$
hence $\overline{AC}\cong\overline{DF}$.

\item[]
It follows that $\overline{AC}\cong\overline{DF}$ if and only if $\angle B\cong\angle E$.

\item[]
Therefore if $\overline{AC}>\overline{DF}$ then $\angle B>\angle E$.

\end{itemize}

\newpage

\section*{Page 142, problem 10.}

In a neutral geometry prove that a triangle with an obtuse angle must have
two acute angles.

\bigskip
\noindent
Solution:

\begin{itemize}

\item[]
Define $\triangle ABC$ with $m(\angle B)>90$.

\item[]
Let $\theta$ be an external angle of $\angle B$.

\item[]
By the Linear Pair Theorem $m(\theta)=180-m(\angle B)$ hence $\theta<90$.

\item[]
By the Exterior Angle Theorem $\theta>\angle A$ and $\theta>\angle B$.

\item[]
Therefore $\angle A$ and $\angle B$ are acute.

\end{itemize}

\end{document}
