\subsection{Integral}
\index{integral}
\noindent
$integral(f,x)$ returns the integral of $f$ with respect to $x$.
The $x$ can be omitted for expressions in $x$.
A multi-integral can be obtained by extending the argument list.

\medskip
\verb$integral(x^2)$
$${1\over3}x^3$$

\verb$integral(x*y,x,y)$
$${1\over4}x^2y^2$$

\medskip
\noindent
$defint(f,x,a,b,\ldots)$
computes the definite integral of $f$ with respect to $x$ evaluated from $a$ to $b$.
The argument list can be extended for multiple integrals.

\medskip
\noindent
The following example computes the integral of $f=x^2$
over the domain of a semicircle.
For each $x$ along the abscissa, $y$ ranges from 0 to $\sqrt{1-x^2}$.

\medskip
\verb$defint(x^2,y,0,sqrt(1-x^2),x,-1,1)$

$${1\over8}\pi$$

\medskip
\noindent
As an alternative, the $eval$ function can be used to compute a definite integral step by step.

\medskip
\verb$I=integral(x^2,y)$

\verb$I=eval(I,y,sqrt(1-x^2))-eval(I,y,0)$

\verb$I=integral(I,x)$

\verb$eval(I,x,1)-eval(I,x,-1)$
$${1\over8}\pi$$
