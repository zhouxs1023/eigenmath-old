\magnification=1200

\noindent
{\it George Weigt --- Geometry Homework \#10}

\beginsection Page 130, problem 4.

Let $\triangle ABC$ be an isosceles triangle in a neutral geometry
with $\overline{AB}\cong\overline{CA}$.
Let $M$ be the midpoint of $\overline{BC}$.
Prove that $\overleftarrow A\overrightarrow M\perp\overleftarrow B\overrightarrow C$.

\bigskip
\noindent
Solution:

\medskip
Since $M$ is the midpoint we have $BM=MC$, hence $\overline{BM}\cong\overline{MC}$.

\medskip
By Pons Asinorum we have $\angle B\cong\angle C$.

\medskip
Then by $\overline{AB}\cong\overline{CA}$ and SAS we have
$\triangle AMB\cong\triangle AMC$.

\medskip
Hence $\angle AMB\cong\angle AMC$ and $m(\angle AMB)=m(\angle AMC)$.

\medskip
By $B{-}M{-}C$ we have $\angle AMB$ and $\angle AMC$ form a linear pair.

\medskip
By the Linear Pair Theorem, $m(\angle AMB)+m(\angle AMC)=180$.

\medskip
It follows that $m(\angle AMB)=m(\angle AMC)=90$.

\medskip
Therefore $\overleftarrow A\overrightarrow M\perp\overleftarrow B\overrightarrow C$.

\beginsection Page 130, problem 5.

Prove that in a neutral geometry every equilateral triangle is equiangular;
that is, all its angles are congruent.

\bigskip
\noindent
Solution:

\medskip
Let $\triangle ABC$ be an equilateral triangle.

\medskip
Since $\overline{AB}\cong\overline{AC}$, by Pons Asinorum $\angle B\cong\angle C$.

\medskip
Since $\overline{BA}\cong\overline{BC}$, by Pons Asinorum $\angle A\cong\angle C$.

\medskip
Therefore $\angle A\cong\angle B\cong\angle C$.

\beginsection Page 130, problem 10.

Suppose there are points $A$, $B$, $C$, $D$, $E$ in a neutral geometry with
$A{-}D{-}B$ and $A{-}E{-}C$ and $A$, $B$, $C$ not collinear.
If $\overline{AD}\cong\overline{AE}$ and $\overline{DB}\cong\overline{EC}$
prove that $\angle EBC\cong\angle DCB$.

\bigskip
\noindent
Solution:

\medskip
By the congruence relations we have $AD=AE$ and $DB=EC$.

\medskip
By $A{-}D{-}B$ we have $AD+BC=AB$.

\medskip
By $A{-}E{-}C$ we have $AE+EC=AC$.

\medskip
Therefore $\overline{AB}\cong\overline{AC}$.

\medskip
By Pons Asinorum we have $\angle ABC\cong\angle ACB$.

\medskip
Then by SAS we have $\triangle BCE\cong\triangle CBD$.

\medskip
Therefore $\angle EBC\cong\angle DCB$.

\beginsection Page 130, problem 12.

Give an example in the Taxicab Plane of an isosceles triangle whose
base angles are not congruent.

\bigskip
\noindent
For example, $A=(0,0)$, $B=(1,1)$, $C=(2,0)$.
For $\triangle ABC$ in the Taxicab Plane we have $AB=2$ and $AC=2$.
However, $\angle B=90$ and $\angle C=45$,
hence $\angle B\not\cong\angle C$.

\beginsection Page 134, problem 1.

Prove theorem 6.2.2 (Converse of Pons Asinorum).

\bigskip
\noindent
Solution:

\medskip
We are given $\triangle ABC$ with $\angle A\cong\angle C$ in a neutral geometry.

\medskip
By ASA we have $\triangle ABC\cong\triangle CBA$.

\medskip
It follows that $\overline{AB}\cong\overline{CB}$.

\medskip
Therefore $\triangle ABC$ is an isosceles triangle.

\beginsection Page 134, problem 4.

Prove theorem the following theorem (theorem 6.2.4).
\medskip
\noindent
If a protractor geometry satisfies ASA then it also satisfies SAS
and is thus a neutral geometry.

\bigskip
\noindent
Solution: Define two triangles with an SAS congruence
relationship. Then using only ASA, show that the two triangles
are congruent.

\medskip
Let $\triangle ABC$ and $\triangle DEF$ be defined such that
$\angle B\cong\angle E$,
$\overline{BA}\cong\overline{ED}$ and
$\overline{BC}\cong\overline{EF}$.

\medskip
By the angle construction axiom there is a ray $\overrightarrow{DG}$
such that $\angle A\cong\angle EDG$.

\medskip
By the segment construction theorem there is an $H\in\overrightarrow{DG}$
such that $\overline{DH}\cong\overline{AC}$.

\medskip
By ASA ($\angle B\cong\angle E$, $\overline{BA}\cong\overline{ED}$,
$\angle A\cong\angle EDH$) we have $\triangle ABC\cong\triangle DEH$.

\medskip
By the uniqeness of angle measure for $\angle E$
it follows that $\overrightarrow{EF}=\overrightarrow{EH}$ hence $H\in\overrightarrow{EF}$.

\medskip
We have $\overline{EF}\cong\overline{BC}\cong\overline{EH}$ hence $F=H$.

\medskip
It follows that $\triangle ABC\cong\triangle DEF$.

\medskip
Therefore ASA implies SAS.

\beginsection Page 134 problem 6.

In a neutral geometry, given $\triangle ABC$ with $A{-}D{-}E{-}C$,
$\overline{AD}\cong\overline{EC}$, and $\angle CAB\cong\angle ACB$,
prove that $\angle ABE\cong\angle CBD$.

\bigskip
\noindent
Solution:

\medskip
By the Converse of Pons Asinorum we have $\overline{BA}\cong\overline{BC}$.

\medskip
By $A{-}D{-}E{-}C$ we have $AE=AD+DE$ and $DC=DE+EC$.

\medskip
Then by $\overline{AD}\cong\overline{EC}$ we have $\overline{AE}\cong\overline{DC}$.

\medskip
By SAS it follows that $\triangle ABE\cong\triangle CBD$.

\medskip
Therefore $\angle ABE\cong\angle CBD$.

\beginsection Page 134, problem 10.

Prove that if a protractor geometry satisfies SSS then the base angles of any
isosceles triangle are congruent.

\bigskip
\noindent
Solution: Define an isosceles triangle and then using only SSS show that
the base angles are congruent.

\medskip
Define $\triangle ABC$ such that $\overline{BA}\cong\overline{BC}$.

\medskip
Let $M$ be the midpoint of $\overline{AC}$, hence $\overline{AM}\cong\overline{CM}$.

\medskip
Then by SSS we have $\triangle BAM\cong\triangle BCM$.

\medskip
Hence $\angle A\cong\angle C$.

\medskip
Therefore SSS implies that the base angles of an isosceles triangle are congruent.

\beginsection Page 134, problem 12.

In a neutral geometry, if $C$ and $D$ are on the same side of
$\overleftarrow A\overrightarrow B$ and if $\overline{AC}\cong\overline{AD}$
and $\overline{BC}\cong\overline{BD}$, prove that $C=D$.

\bigskip
\noindent
Solution:

\medskip
By SSS we have $\triangle CAB\cong\triangle DAB$.

\medskip
Hence $\angle CAB\cong\angle DAB$.

\medskip
By uniqueness of angle measure and $C$ and $D$ in the same half plane, we have
$D\in\overrightarrow{AC}$.

\medskip
Then $\overline{AC}\cong\overline{AD}$ implies $C=D$.

\beginsection Page 134, problem 13.

Given $\triangle ABC$ in a protractor geometry, the angle bisector $\angle A$
intersects $\overline{BC}$ at a unique point $D$.
$\overline{AD}$ is called the internal bisector of $\angle A$.
Prove that in a neutral geometry, the internal bisectors of the base angles
of an isosceles triangle are congruent.

\bigskip
\noindent
Solution:

\medskip
Let $\triangle ABC$ be an isosceles trinagle with $\angle A\cong\angle C$ and
$\overline{BA}\cong\overline{BC}$.

\medskip
Let $D\in\overline{BC}$ such that $\overline{AD}$ bisects $\angle A$.

\medskip
Let $E\in\overline{BA}$ such that $\overline{CE}$ bisects $\angle C$.

\medskip
Bisection of equal angle measures yields $\angle BAD\cong\angle BCE$.

\medskip
Then by ASA we have $\triangle BAD\cong\triangle BCE$.

\medskip
Therefore $\overline{AD}\cong\overline{CE}$.









\end
