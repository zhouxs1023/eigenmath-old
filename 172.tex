{\it George Weigt -- Geometry Homework \#2}

\beginsection Proposition 2.27

The Cartesian plane ${\cal C}=(R^2,{\cal L}_E)$ with the taxicab
distance is a metric geometry.

\medskip
Proof: A metric geometry is an incidence geometry
together with a distance function such that every line has a ruler.
It has been already proven that the Cartesian plane is an
incidence geometry.
Therefore, what remains to be proved is that, using the
taxicab distance, every line in the Cartesian Plane has a ruler.
There are two cases that we need to consider.

\medskip
Case 1: For the vertical line $L_a$ let $f(a,y)=y$.
As a first step, let us show that $f$ is surjective.
The range of $f$ is $R$ so let
$t\in R$ be an arbitrary real number.
We want to show that for every $t$ we can find a $y_1$
such that $f(a,y_1)=t$.
The obvious choice is $y_1=t$.
For $y_1=t$ it follows that $f(a,y_1)=f(a,t)=t$ hence $f$ is surjective.
%
Next we want to show that
$$|f(P)-f(Q)|=d_T(P,Q)$$
%
For $P=(x_1,y_1)$ and $Q=(x_2,y_2)$ we have
$$|f(P)-f(Q)|=|f(x_1,y_1)-f(x_2,y_2)|=|y_1-y_2|$$
%
The taxicab distance is
$$d_T(P,Q)=|x_1-x_2|+|y_1-y_2|$$
For a vertical line we have $x_1=x_2$ hence
$$d_T(P,Q)=|y_1-y_2|=|f(P)-f(Q)|$$

\medskip
Case 2:
For the nonvertical line $L_{m,b}$ let $f(x,y)=(1+|m|)x$.
As a first step, let us show that $f$ is surjective.
The range of $f$ is $R$ so
let $t\in R$ be an arbitrary real number.
We want to show that for every $t$ we can find an $x_1$ and $y_1$
such that $f(x_1,y_1)=t$.
We have
$$f(x_1,y_1)=(1+|m|)x_1=t$$
Solving for $x_1$ we have
$$x_1=t/(1+|m|)$$
On the Cartesian plane we have
$$y_1=mx_1+b$$
Substituting for $x_1$ we have
$$y_1=mt/(1+|m|)+b$$
Now that we have found $x_1$ and $y_1$ we can put
$$f(x_1,y_1)=f\big[t/(1+|m|),mt/(1+|m|)+b\big]=t$$
Therefore $f$ is surjective.
%
Next we want to show that
$$|f(P)-f(Q)|=d_T(P,Q)$$
%
For $P=(x_1,y_1)$ and $Q=(x_2,y_2)$ we have
$$|f(P)-f(Q)|=|f(x_1,y_1)-f(x_2,y_2)|=\big|(1+|m|)x_1-(1+|m|)x_2\big|$$
The quantity $1+|m|$ is always positive so we can extract it from the
absolute value and put
$$\eqalign{
|f(P)-f(Q)|&=(1+|m|)|x_1-x_2|\cr
&=|x_1-x_2|+|m||x_1-x_2|\cr
}$$
On the Cartesian plane we have
$$m={y_2-y_1\over x_2-x_1}$$
hence
$$\eqalign{
|f(P)-f(Q)|&=|x_1-x_2|+\left|{y_2-y_1\over x_2-x_1}\right||x_1-x_2|\cr
&=|x_1-x_2|+|y_1-y_2|\cr
&=d_T(P,Q)\cr
}$$

\medskip
In both cases 1 and 2 we have shown that there is an $f$ that is surjective and
$$|f(P)-f(Q)|=d_T(P,Q)$$
Hence by Lemma 2.24 we have the result that the Cartesian plane
with a taxicab distance is a metric geometry.

\beginsection Problem 4.

In the Euclidean Plane, (i) find the coordinate $(2,3)$ with respect
to the line $x=2$;
(ii) find the coordinate of $(2,3)$ with respect to the line $y=-4x+11$.

\medskip
Solution: The book says to use the standard ruler function.

\medskip
(i) The line $x=2$ is the vertical line $L_2$.
The standard ruler function is $f(a,y)=y$.
Hence the coordinate of $(2,3)$ is $f(2,3)=3$

\medskip
(ii) The line $y=-4x+11$ is the nonvertical line $L_{-4,11}$.
The standard ruler function is $f(x,y)=x\sqrt{1+m^2}$.
Hence the coordinate of $(2,3)$ is $f(2,3)=2\sqrt{17}$.

\beginsection Problem 5.

Find the coordinate of $(2,3)$ with respect to the line
$y=-4x+11$ for the Taxicab Plane.

\medskip
Solution: 
The line $y=-4x+11$ is the nonvertical line $L_{-4,11}$.
The standard ruler function is $f(x,y)=(1+|m|)x$.
Hence the coordinate of $(2,3)$ is $f(2,3)=10$.

\beginsection Problem 7.

Find the Poincare distance between
\item{i.} $(1,2)$ and $(3,4)$.
\item{ii.} $(2,1)$ and $(4,3)$.

\medskip
Solution: For case (i) the points are on a type II line.
$$(x_1,y_1)=(1,2)\qquad(x_2,y_2)=(3,4)$$
$$c={y_2^2-y_1^2+x_2^2-x_1^2\over2(x_2-x_1)}={4^2-2^2+3^2-1^2\over2(3-1)}
={16-4+9-1\over4}=5$$
$$r=\sqrt{(x_1-c)^2+y_1^2}=\sqrt{(1-5)^2+2^2}=\sqrt{20}=2\sqrt5$$
The Poincare distance is
$$
\left|
\ln
\left({
\displaystyle{x_1-c+r\over y_1}
\over
\displaystyle{x_2-c+r\over y_2}
}\right)
\right|
=
\left|
\ln
\left({
\displaystyle{1-5+2\sqrt2\over2}
\over
\displaystyle{3-5+2\sqrt2\over4}
}\right)
\right|
\approx0.9624
$$

For case (ii) the points are also on a type II line, as in case (i).
$$(x_1,y_1)=(2,1)\qquad(x_2,y_2)=(4,3)$$
$$c={y_2^2-y_1^2+x_2^2-x_1^2\over2(x_2-x_1)}={3^2-1^2+4^2-2^2\over2(4-2)}
={9-1+16-4\over4}=5$$
$$r=\sqrt{(x_1-c)^2+y_1^2}=\sqrt{(2-5)^2+1^2}=\sqrt{10}$$
The Poincare distance is
$$
\left|
\ln
\left({
\displaystyle{x_1-c+r\over y_1}
\over
\displaystyle{x_2-c+r\over y_2}
}\right)
\right|
=
\left|
\ln
\left({
\displaystyle{2-5+\sqrt{10}\over1}
\over
\displaystyle{4-5+\sqrt{10}\over3}
}\right)
\right|
\approx1.491
$$

\vfill
\eject

\beginsection Problem 8.

Show that the function $g:{}_aL\rightarrow R$ given by $g(a,y)=\ln(y)$ is a bijection
and that it satisfies the Ruler Equation. Show that the inverse of $g$ is given by
$g^{-1}(t)=(a,e^t)$.
\medskip
Solution: The Poincare Plane is an incidence geometry.
Knowing this, if we show that $g$ is surjective and satsifies the Ruler Equation,
then by Lemma 2.24 we have that $g$ is a bijection.
Let us begin by showing that $g$ is surjective.
The range of $g$ is $R$ so let $t\in R$ be an arbitrary real number.
We want to show that for every $t$ we can find a $y_1$ such that
$g(a,y_1)=t$.
We have
$$g(a,y_1)=\ln(y_1)=t$$
Solving for $y_1$ we have
$$y_1=e^t$$
Now that we have found $y_1$ we can put
$$g(a,e^t)=t$$
This solution demonstrates that $g$ is surjective.
Now let us show that $g$ satisfies the Ruler Equation.
The Poincare distance for two points on the line ${}_aL$ is
$$d_H(P,Q)=\left|\ln\left({y_2\over y_1}\right)\right|$$
We have
$$|g(P)-g(Q)|=|\ln y_1-\ln y_2|=|\ln y_2-\ln y_1|=\left|\ln\left({y_2\over y_1}\right)\right|
=d_H(P,Q)$$
Therefore $g$ satisfies the Ruler Equation and by Lemma 2.24 we have $g$ is surjective.
Finally, let us show that the inverse of $g$ is given by
$g^{-1}(t)=(a,e^t)$.
Let $P=(a,y_1)$. We have
$$g^{-1}(g(P))=g^{-1}(\ln y_1)=(a,y_1)=P$$
Hence $g^{-1}(t)=(a,e^t)$ is the inverse of $g(a,y)=\ln(y)$.

\vfill
\eject

\beginsection Problem 10.

Find a point $P$ on the line $L_{2,-3}$ in the Taxicab Plane whose coordinate is $-2$.
\medskip
Solution: The standard ruler for the line is
$$f(x,y)=(1+|m|)x$$
We have $f(P)=-2$, $m=2$, and $b=-3$.
From the standard ruler we have
$$-2=(1+|2|)x$$
Solving for $x$ yields
$$x=-2/3$$
Now solve for $y$ to obtain
$$y=mx+b=(2)(-2/3)-3=-13/3$$
Therefore $P=(-2/3,-13/3)$.

\beginsection Problem 11.

Find a point $P$ on the line ${}_{-3}L_{\sqrt7}$ in the Poincare Plane
whose coordinate is $\ln2$.
\medskip
Solution: The standard ruler for the line is
$$f(x,y)=\ln\left({x-c+r\over y}\right)$$
We have $f(P)=\ln2$, $c=-3$, and $r=\sqrt7$.
From the standard ruler we have
$$\ln2=\ln\left({x+3+\sqrt7\over y}\right)$$
Solving for $x$ we have
$$x=2y-3-\sqrt7$$
Next, use the equation for $r$ to solve for $y$. We have
$$0=(x-c)^2+y^2-r^2=(2y-\sqrt7)^2+y^2-7
=5y^2-4\sqrt7y$$
The solution $y=0$ is not on the Poincare Plane.
Instead, divide through by $y$ and obtain
$$0=5y-4\sqrt7$$
Hence we have
$$y={4\sqrt7\over5}$$
Now solve for $x$.
$$x=2y-3-\sqrt7={8\sqrt7\over5}-3-\sqrt7={3\sqrt7\over5}-3$$
Therefore $P=(3\sqrt7/5-3,4\sqrt7/5)$.

\vfill
\eject

\beginsection Problem 14.

We shall define a new distance $d^*$ on $R^2$ by using $d_E$.
Specifically:
$$d^*(P,Q)=\cases{
d_E(P,Q) & if $d_E(P,Q)\le1$ (case 1)\cr
1 & if $d_E(P,Q)>1$ (case 2)\cr
}$$

\beginsection 14. (i)

Prove that $d^*$ is a distance function.
\medskip
Solution: We have $d^*(P,Q)\in R$ for both cases 1 and 2 which is necessary
for a distance function.
Next, verify conditions (i), (ii), (iii)
of Definition 2.11.

\item{(i)} $d^*(P,Q)\ge0$

True for case 1 by definition of $d_E$.

True for case 2 because $d^*(P,Q)=1$ implies $d^*(P,Q)\ge0$.

\item{(ii)} $d^*(P,Q)=0$ if and only if $P=Q$

True for case 1 by definition of $d_E$.

True for case 2 because $d^*(P,Q)=1$ implies $d_E(P,Q)\ge1$
which implies $P\ne Q$.


\item{(iii)} $d^*(P,Q)=d^*(Q,P)$

True for case 1 by definition of $d_E$.

True for case 2 because $d^*(P,Q)=1$ and $d^*(Q,P)=1$.

\beginsection 14. (ii)

Find and sketch all points $P\in R^2$ such that $d^*((0,0),P)\le2$.
\medskip
Solution: The solution set $\{P\}$ is all $R^2$ since by definition $d^*(P,Q)\le1$.

\beginsection 14. (iii)

Find and sketch all points $P\in R^2$ such that $d^*((0,0),P)=2$.
\medskip
Solution: The solution set $\{P\}$ is the empty set since by definition $d^*(P,Q)\le1$.

\vfill
\eject

\beginsection Problem 17. (extra credit)

If $\{S,L,d\}$ is a metric geometry and $P\in S$, prove that for any $r>0$
there is a point in $S$ at distance $r$ from $P$.
\medskip
Solution: A metric geometry implies
the existence of a ruler $f$.
Define $t\in R$ such that $t=f(P)-r$. Then we have
$$f(P)-t=r$$
Since $r>0$ we have $|r|=r$ so we can write
$$|f(P)-t|=r$$
A ruler is a bijection hence for any $t\in R$ we can find a $Q\in S$ such that
$$f(Q)=t$$
Hence
$$|f(P)-f(Q)|=r$$
By the ruler equation we have
$$|f(P)-f(Q)|=d(P,Q)=r$$
Hence for every $r>0$ there is a $Q\in S$ that $d(P,Q)=r$.

\vfill
\eject

\beginsection Problem 18.

Define the max distance (or supremum distance), $d_S$, on $R^2$ by
$$d_S(P,Q)=max\{|x_1-x_2|,|y_1-y_2|\}$$
where $P=(x_1,y_1)$, $Q=(x_2,y_2)$.

\beginsection 18. (i)

Show that $d_S$ is a distance function.
\medskip
Solution: Note that $d_S(P,Q)\in R$ which is necessary for a distance function.
Now verify conditions (i), (ii) and (iii) of Definition 2.11.

\item{(i)} Verify that $d(P,Q)\ge0$.
True because $d(P,Q)$ is always the absolute value of something,
i.e., either $d_S(P,Q)=|x_1-y_1|$ or $d(P,Q)=|y_1-y_2|$.

\item{(ii)} Verify that $d_S(P,Q)=0$ if and only if $P=Q$.
If $P=Q$ then $x_1=x_2$ and $y_1=y_2$.
Then by computation we have $d_S(P,Q)=0$.
Therefore $P=Q$ implies that $d_S(P,Q)=0$.
Now let us prove the converse.
Let $d_S(P,Q)=0$.
Then $|x_1-x_2|=0$ and $|y_1-y_2|=0$.
Hence $x_1=x_2$ and $y_1=y_2$ from which it follows that $P=Q$.
Therefore $d_S(P,Q)=0$ implies that $P=Q$.

\item{(iii)} Verify that $d(P,Q)=d(Q,P)$.
We have $|x_1-x_2|=|x_2-x_1|$ and $|y_1-y_2|=|y_2-y_1|$. Hence
$$d(P,Q)=max\{|x_1-x_2|,|y_1-y_2|\}=max\{|x_2-x_1|,|y_2-y_1|\}=d(Q,P)$$

\beginsection 18. (ii)

Show that $\{R^2,L_E,d_S\}$ is a metric geometry.
\medskip
Solution: We know already that $\{R^2,L_E\}$ is an incidence geometry.
Now, using $d_S$, we need to show that every line $l\in L_E$ has a ruler.
Let us begin by considering a vertical line.
Let $P$ and $Q$ lie on the vertical line $L_{a}$.
Then $x_1=x_2$ which leads to $d_S(P,Q)=|y_1-y_2|$.
By the ruler equation we have
$$|f(P)-f(Q)|=d_S(P,Q)=|y_1-y_2|$$
It follows that the ruler is $f(a,y)=y$
which is a standard ruler.
Now let us consider nonvertical lines of the form $L_{m,b}$.
We define the following ruler.
$$f(x,y)=\cases{
y,&$|m|\ge1$\cr
x,&$|m|<1$\cr
}$$
Note that $f$ is a standard ruler in both cases and is therefore bijective.
For $|m|\ge1$ we have
$$\left|{y_2-y_1\over x_2-x_1}\right|\ge1$$
which leads to
$$|y1-y_2|\ge|x_1-x_2|$$
Hence $d_S(P,Q)=|y_1-y_2|$ for $|m|\ge1$.
%
For $|m|<1$ we have
$$\left|{y_2-y_1\over x_2-x_1}\right|<1$$
which leads to
$$|y1-y_2|<|x_1-x_2|$$
Hence $d_S(P,Q)=|x_1-x_2|$ for $|m|<1$.
Putting it all together we have
$$\eqalign{
|f(P)-F(Q)|=|y_1-y_2|=d_S(P,Q),&\qquad|m|\ge1\cr
|f(P)-F(Q)|=|x_1-x_2|=d_S(P,Q),&\qquad|m|<1\cr
}$$
hence $f(x,y)$ is indeed a ruler.
Therefore $\{R^2,L_E,d_S\}$ is a metric geometry.


\vfill
\eject

\beginsection 19.

In a metric geometry $\{S,L,d\}$ if $P\in S$ and $r>0$, then the circle
with center $P$ and radius $r$ is $C=\{Q\in S\mid d(P,Q)=r\}$.
Draw a picture of the circle of radius 1 and center $(0,0)$ in the
$R^2$ plane for each of the distances $d_E$, $d_T$, and $d_S$.

\medskip
For $d_E$ we have
$$d_E(P,Q)=\sqrt{(x_1-x_2)^2+(y_1-y_2)^2}=\sqrt{x_2^2+y_2^2}=1$$
In this case the drawing is a unit circle centered about the origin.

\medskip
For $d_T$ we have 
$$d_T(P,Q)=|x_1-x_2|+|y_1-y_2|=|x_2|+|y_2|=1$$
In this case the drawing is a diamond centered about
the origin with vertices at $(1,0)$, $(0,1)$,
$(-1,0)$, and $(0,-1)$.

\medskip
For $d_S$ we have
$$d_S=max\{|x_1-x_2|,|y_1-y_2|\}=max\{|x_2|,|y_2|\}=1$$
In this case the drawing is a square centered about the origin
with vertices at $(1,1)$, $(-1,1)$, $(-1,-1)$, and $(1,-1)$.

\medskip
A scanned image in PDF format can be found in the digital dropbox.

\beginsection 20. (extra credit)

Let $\{S,L,d\}$ be a metric geometry, let $P\in S$, let $l\in L$ with $P\in l$,
and let $C$ be a circle with center $P$.
Prove that $l\cap C$ contains exactly two points.

\medskip
Let $r\in R$, $r>0$ be the radius of the circle.
Then $C=\{Q\in S\mid d(P,Q)=r\}$.
Since $\{S,L,d\}$ is a metric geometry, there exists a ruler $f$ for $l$.
Since $f$ is a bijection on $l$ we have that $f$ is both surjectve
and injective.
By the surjective property we can always find
a $Q\in l$
and a $T\in l$ such that
$$\eqalign{
f(P)-f(Q)&=r\cr
f(P)-f(T)&=-r\cr
}$$
%By subtraction
%we get $f(T)-f(Q)=2r$ proving that $Q\ne T$.
Next, take the absolute value of both sides.
$$\eqalign{
|f(P)-f(Q)|&=|r|\cr
|f(P)-f(T)|&=|r|\cr
}$$
Since $r>0$ we can drop the absolute value around $r$.
Now, by the ruler equation we have
$$\eqalign{
|f(P)-f(Q)|&=d(P,Q)=r\cr
|f(P)-f(T)|&=d(P,T)=r\cr
}$$
From this we conclude that $Q,T\in C$.
Therefore $Q,T\in l\cap C$.
Next we want to show that any point in $l\cap C$ is either $Q$ or $T$.
Let $W\in l\cap C$.
It follows that
$$|f(P)-f(W)|=d(P,W)=r$$
We have two cases. Either $f(P)-f(W)=r$ or $f(P)-f(W)=-r$.
If $f(P)-f(W)=r$ then $f(W)=f(Q)$ which implies that $W=Q$
by the injective property of $f$.
Similarly, if $f(P)-f(W)=-r$ then $f(W)=f(T)$ which implies that $W=T$.
Therefore $l\cap C=\{Q,T\}$ which contains exactly two points.

\beginsection 21.

Find the circle of radius 1 with center $(0,e)$ in the Poincare Plane.
\medskip
Solution: Let $P=(0,e)$. Then the circle is the set of points
$$\{Q\in H\mid d_H(P,Q)=1\}$$
When $P$ and $Q$ lie on a type II line we have
$$d_H(P,Q)=\left|\ln\left(
{\displaystyle{x_1-c+r\over y_1}\over\displaystyle{x_2-c+r\over y_2}}
\right)\right|=1
$$
Hence
$${\displaystyle{x_1-c+r\over y_1}\over\displaystyle{x_2-c+r\over y_2}}=e$$
For $x_1=0$ and $y_1=e$ we have
$${\displaystyle{-c+r\over e}\over\displaystyle{x_2-c+r\over y_2}}=e$$
Hence
$${-c+r\over e^2}={x_2-c+r\over y_2}$$
It follows that the equation of a circle in the Poincare Plane is
$$\left({-c+r\over e^2}\right)y_2=x_2-c+r$$

Now let us consider the Euclidean Plane.
We deduce that the top of the circle is at $Q=(0,e^2)$ and the bottom
of the circle is at $T=(0,1)$, i.e.,
$$\eqalign{
|f(P)-f(Q)|&=|\ln e-\ln e^2|=1\cr
|f(P)-f(T)|&=|\ln e-\ln 1|=1\cr
}$$
Therefore on the Euclidean Plane the radius of the circle is
$${Q(y)-T(y)\over2}={e^2-1\over2}$$
and the $y$-coordinate of the center of the circle is
$$Q(y)+{Q(y)-T(y)\over2}=1+{e^2-1\over2}={e^2+1\over2}$$
Hence the equation of the circle is
$$x^2+\left(y-{e^2+1\over2}\right)^2=\left({e^2-1\over2}\right)^2$$
Solving for $x^2$ we have
$$x^2=\left({e^2-1\over2}\right)^2-y^2+(e^2+1)y-\left({e^2+1\over2}\right)^2$$
which simplifies to
$$x^2=-y^2+(e^2+1)y-e^2$$

\end