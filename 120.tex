\beginsection 2.


\item{\S1.} Life of Leibniz

\itemitem{1.}
Born 1646 in Germany.
As a schoolboy, Leibniz taught himself Latin and Greek.
He entered the University of Leipzig at the age of 14.
After he graduated at the age of 20, Leibniz obtained
a doctorate in law from the University of Altdorf.
Suprisingly, Leibniz' career was not in academia.
He was mainly employed as a diplomat.

\itemitem{2.}
In 1672 at the age of twenty-six Leibniz took up residence in Paris.
He had drafted a plan to get the French to leave Germany and
for the next four years he was involved in negotiations
to implement the plan.
Ultimately the plan did not work out but Leibniz learned
a lot about mathematics while living in Paris.
He met the Dutch mathematician Huygens there.
With Huygens as mentor, Leibniz taught himself
advanced mathematics and physics.
It was after this that he began making contributions
in both areas.

\itemitem{3.}
In 1676 Leibniz traveled to London and
demonstrated a calculating machine
he had built. Members of the Royal Society were
impressed and made Leibniz an external member.
While in London, Leibniz previwed some
of Newton's work.
Later that year he returned home from Paris.
He began working for the House of Hanover and kept the
job for the next 40 years.
He also worked for the Duke of Brunswick.
For the latter job he was commissioned to write a
history of the House of Brunswick.
Lebniz traveled extensively doing the necessary research.

\itemitem{4.}
Leibniz never married. He died in 1716 at the age of 70.

\item{\S2.} Work of Leibniz

\itemitem{1.}
Leibniz published his work on calculus in 1684.
He invented the notation we use today, that is,
$dx$ and $\int$.
He also derived formulas for the calculus
of products, quotients and powers.

\itemitem{2.}
Leibniz knew a lot about philosophy so he
worked on symbolic logic quite a bit,
hoping to make philosophical arguments
computable.
He thought that if this were possible then
all philosophical questions could be
answered definitively through symbolic
computation.
Due to this work, Leibniz ended up making many contributions
in the field of symbolic logic.

\itemitem{3.}
He discovered the binary number system.
Leibniz spent a lot of time investigating concepts
for building calculating machines.
His work predates that of well-known pioneers in the field
of machine logic.

\itemitem{4.}
Lebniz made many contributions in the area of linear
algebra.
He figured out that systems of linear equations could
be organized into arrays of coefficients.
He derived a formula for the determinant.


\item{\S3.} References
\itemitem{1.} {\tt http://en.wikipedia.org/wiki/Leibniz}
\itemitem{2.} {\tt http://www.kirjasto.sci.fi/leibnitz.htm}
\itemitem{3.} {\it A History of Mathematics} by Carl B. Boyer

\vfill
\eject

\beginsection 4.

$$4\cdot\left(1
-{1\over3}
+{1\over5}
-{1\over7}
+{1\over9}
-{1\over11}
+{1\over13}
-{1\over15}
+{1\over17}
-{1\over19}
\right)=3.04184$$
$${\pi-3.04184\over\pi}=0.032$$
Hence the approximation for $\pi$ is 3.2\% low.

\beginsection 5. (a)

Prove (21).
$$\ln2=\int_0^1{1\over1+x}\,dx\eqno(21)$$
Solution: Ley $y=1+x$. Then $dy=dx$ and we have
$$\int_0^1{1\over1+x}\,dx=\int_1^2{1\over y}\,dy=\ln y\bigg|_1^2=\ln2-\ln1=\ln2$$

\beginsection 5. (b)

Use (21) to derive
$$\ln2=1-{1\over2}+{1\over3}-{1\over4}+\cdots=\sum_{k=1}^\infty{(-1)^{k-1}\over k}$$
Step 1. Use the geometric series with $r=-x$ to obtain
$${1\over1+x}=\sum_{k=0}^\infty(-x)^k=1-x+x^2-x^3+\cdots+(-1)^kx^k+(-1)^{k+1}{x^{k+1}\over1+x}$$
Step 2. From (21) and the previous result we can write
$$\eqalign{
\ln2&=\int_0^1{1\over1+x}\,dx\cr
&=\int_0^1\left(1-x+x^2-x^3+\cdots+(-1)^kx^k+(-1)^{k+1}{x^{k+1}\over1+x}\right)\,dx\cr
&=\int_0^1dx-\int_0^1x\,dx+\int_0^1x^2\,dx-\int_0^1x^3\,dx+\cdots
+(-1)^k\int_0^1x^k\,dx
+(-1)^{k+1}\int_0^1{x^{k+1}\over1+x}\,dx\cr
&=\left(1-{1\over2}+{1\over3}-{1\over4}+\cdots+{(-1)^k\over k+1}\right)
+(-1)^{k+1}\int_0^1{x^{k+1}\over1+x}\,dx\cr
}$$
Step 3. Now show that as $k\rightarrow\infty$ the last integral above disappears.
For $0\le x\le 1$ we have
$${1\over2}\le{1\over1+x}\le1$$
Hence
$$0\le\int_0^1{x^{k+1}\over1+x}\,dx\le\int_0^1x^{k+1}\,dx={1\over k+2},
\qquad\lim_{k\rightarrow\infty}{1\over k+2}=0$$
Step 5. We have
$$\ln2=1-{1\over2}+{1\over3}-{1\over4}+\cdots=\sum_{k=0}^\infty{(-1)^k\over k+1}$$
Now make the substitution $k-1\rightarrow k$.
For the starting index we have $k-1=0$, hence the starting index becomes $k=1$
and we have
$$\ln2=1-{1\over2}+{1\over3}-{1\over4}+\cdots=\sum_{k=1}^\infty{(-1)^{k-1}\over k}$$

\end