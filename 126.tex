\item{5.} The number of red blood cells per square unit
visible under a microscope follows a Poisson distribution
with average number 4.

\bigskip
{\bf (a) Find the probability of more than 5 blood cells
visible from a square unit.}

\bigskip
Solution: $\lambda=4$
$$P(X>5)=1-\sum_{x=0}^5{4^xe^{-4}\over x!}=0.2149$$
Check with Excel {\tt =1-POISSON(5,4,TRUE)}

\bigskip
{\bf (b) Find the probability that exactly 4 blood
cells are visible from two square units.}

\bigskip
Solution: The trick here is to use the independence
property. (See lecture 37, slide 1.)
In other words, the average for 2 square units is
twice the average of 1 square unit,
hence $\lambda=8$.
$$p(4)={8^4e^{-8}\over4!}=0.0573$$
Check with Excel {\tt =POISSON(4,8,FALSE)}

\bigskip
{\bf (c) Find the probability that less than 2 blood cells
are visible from a half square unit.}

\bigskip
Solution: As above, $\lambda=2$.
$$P(X<2)=p(0)+p(1)=0.4060$$

\bigskip
{\bf (d) Find the probability that exactly 6 blood cells are visible from a square unit.}

\bigskip
Solution: $\lambda=4$
$$p(6)={4^6e^{-4}\over6!}=0.1042$$


\end
