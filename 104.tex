\beginsection HW 3-1. (Lecture 4, exercise 2)

In 1202 Leonardo of Pisa published {\it Liber Abaci} with the following word problem.
\medskip
``A certain man put a pair of rabbits in a place surrounded on all sides by a wall.
How many pairs of rabbits can be produced from that pair in a year if it is supposed
that every month each pair begets a new pair which from the second month on becomes
productive?''

{\tt http://milan.milanovic.org/math/english/leonardo/leonardo.html}

\medskip
Let us see if we can work out this problem.
Suppose that at the end of a given month $k$ there are
$N_k$ pairs of productive rabbits and $n_k$ pairs of new rabbits.
Given this, how many pairs of rabbits will we have at the end of
the following month?
The $N_k$ rabbits will produce an equivalent number of offspring,
and the $n_k$ new rabbits will join the ranks of the productive rabbits.
Hence we have
$$\eqalign{
n_{k+1}&=N_k\cr
N_{k+1}&=N_k+n_k\cr
}$$

Now suppose that we are only interested in the total number of pairs of rabbits
at the end of each month.
Let $T_{k+2}$ be the total number of pairs of rabbits at the end of month $k+2$.
We have
$$\eqalign{
T_{k+2}&=N_{k+2}+n_{k+2}\cr
&=(N_{k+1}+n_{k+1})+N_{k+1}\cr
&=T_{k+1}+(N_{k}+n_{k})\cr
&=T_{k+1}+T_k
}$$

The initial condition is one pair of rabbits that produce
two offspring at the end of the first month.
Hence we have
$$\eqalign{
T_0&=1\cr
T_1&=2\cr
T_k&=T_{k-1}+T_{k-2}
}$$

Note that $T_n$ is different from the traditional Fibonacci sequence $F_n$.

\beginsection HW 3-2. (Lecture 4, exercise 3)

%It turns out that the Fibonacci sequence has the following closed-form solution.
%$$F_n={\phi^n-(1-\phi)^n\over\sqrt5}$$
%where
%$$\phi={1+\sqrt5\over2}$$
%It also turns out that the constant $\phi$ crops up all the time.
%The quantity $\phi$ is called the {\it golden ratio.}
%
In many cases, the number of flower petals is a Fibonacci number.
For example,
buttercups have 5, lilies and iris have 3, corn marigolds have 13.\par
\noindent See {\tt http://www.mcs.surrey.ac.uk/Personal/R.Knott/Fibonacci/fibnat.html\#petals}

\medskip
The number of spirals on some pinecones is a Fibonacci number.

\medskip
A starfish has 5 arms, which is a Fibonacci number.

\medskip
Suppose we examine something from Nature with a characteristic feature that
is amenable to enumeration.
Suppose further that whatever we count is 10 or less.
Then, since 1, 2, 3, 5 and 8 are Fibonacci numbers,
we immediately have a 50\% chance that the number
is a Fibonacci number.

\beginsection HW 3-3. (Lecture 4, exercise 7)

Use (3) to prove (2).
$$F_n={1\over\sqrt5}\left(\phi^n-\left(-{1\over\phi}\right)^n\right)\eqno(3)$$
$$\lim_{n\rightarrow\infty}{F_{n+1}\over F_n}=\phi\eqno(2)$$
\bigskip
We have
$$\eqalign{
{F_{n+1}\over F_n}&=
{
{1\over\sqrt5}\left(\phi^{n+1}-\left(-{1\over\phi}\right)^{n+1}\right)
\over
{1\over\sqrt5}\left(\phi^{n}-\left(-{1\over\phi}\right)^{n}\right)
}\cr
&=
{
\phi^{n+1}-\left(-{1\over\phi}\right)^{n+1}
\over
\phi^{n}-\left(-{1\over\phi}\right)^{n}
}\cr
&=
{\left(-{1\over\phi}\right)^{n}\over\left(-{1\over\phi}\right)^{n}}\times
{
\phi^{n+1}-\left(-{1\over\phi}\right)^{n+1}
\over
\phi^{n}-\left(-{1\over\phi}\right)^{n}
}\cr
&=
{
(-1)^n\phi-\left(-{1\over\phi}\right)^{2n+1}
\over
(-1)^n-\left(-{1\over\phi}\right)^{2n}
}\cr
}$$
Now evaluate the limit for $\phi>1$.
$$\eqalign{
\lim_{n\rightarrow\infty}{F_{n+1}\over F_n}&=
{
\phi\lim_{n\rightarrow\infty}(-1)^n-\lim_{n\rightarrow\infty}\left(-{1\over\phi}\right)^{2n+1}
\over
\lim_{n\rightarrow\infty}(-1)^n-\lim_{n\rightarrow\infty}\left(-{1\over\phi}\right)^{2n}
}\cr
&=
{
\phi\lim_{n\rightarrow\infty}(-1)^n-0
\over
\lim_{n\rightarrow\infty}(-1)^n-0
}\cr
&=\phi
}$$

\vfill
\eject

\beginsection HW 3-4. (Lecture 4, exercise 8)

Prove (3).
$$F_n={1\over\sqrt5}\left(\phi^n-\left(-{1\over\phi}\right)^n\right)\eqno(3)$$

\bigskip
Proof by induction.
For $n=0$ we have $F_0=0$ which is correct.
For $n=1$ we have
$$\eqalign{
F_1&={1\over\sqrt5}\left(\phi+{1\over\phi}\right)\cr
&={1\over\sqrt5}\left({\phi^2+1\over\phi}\right)\cr
&={1\over\sqrt5}\left({(10+2\sqrt5)/4\over(1+\sqrt5)/2}\right)\cr
&={(2\sqrt5+2)/4\over(1+\sqrt5)/2}\cr
&=1
}$$
which is also correct.
Now show that $F_{n}+F_{n+1}=F_{n+2}$.
(The identity $\phi^{n+2}=\phi^{n+1}+\phi^n$
is used.)
$$\eqalign{
\sqrt5\,(F_n+F_{n+1})&=\phi^n-\left(-{1\over\phi}\right)^n
+\phi^{n+1}-\left(-{1\over\phi}\right)^{n+1}\cr
&=\phi^{n+2}
-\left(-{1\over\phi}\right)^{n+2}
\left[
\left(-{1\over\phi}\right)^{-2}
+
\left(-{1\over\phi}\right)^{-1}
\right]\cr
&=
\phi^{n+2}
-\left(-{1\over\phi}\right)^{n+2}
(\phi^2-\phi)\cr
&=\phi^{n+2}-\left(-{1\over\phi}\right)^{n+2}\cr
&=\sqrt5\,F_{n+2}
}$$
%
%$$\eqalign{
%\sqrt5\,(F_n+F_{n+1})&=\phi^n-\left(-{1\over\phi}\right)^n
%+\phi^{n+1}-\left(-{1\over\phi}\right)^{n+1}\cr
%&=\phi^{n+2}\left(\phi^{-2}+\phi^{-1}\right)
%-\left(-{1\over\phi}\right)^{n+2}
%\left[
%\left(-{1\over\phi}\right)^{-2}
%+
%\left(-{1\over\phi}\right)^{-1}
%\right]\cr
%&=
%\phi^{n+2}\left(\phi^{-2}+\phi^{-1}\right)
%-\left(-{1\over\phi}\right)^{n+2}
%(\phi^2-\phi)\cr
%&=\phi^{n+2}-\left(-{1\over\phi}\right)^{n+2}\cr
%&=\sqrt5\,F_{n+2}
%}$$
Hence by induction (3) is correct.

\vfill
\eject

\beginsection HW 3-5. (Lecture 4 exercise 9)

Use (1) to prove (4). For help, one can google ``fibonacci power series.''
$$F_n=F_{n-1}+F_{n-2}\eqno(1)$$
$$\sum_{n=1}^\infty F_nt^n={t\over1-t-t^2}\eqno(4)$$
We have
$$\eqalign{
\sum_{n=1}^\infty F_nt^n&=F_0+F_1t+\sum_{n=2}^\infty(F_{n-1}+F_{n-2})t^n\cr
&=t+t\sum_{n=2}^\infty F_{n-1}t^{n-1}+t^2\sum_{n=2}^\infty F_{n-2}t^{n-2}\cr
&=t+t\sum_{n=1}^\infty F_nt^n+t^2\sum_{n=0}^\infty F_nt^n\cr
&=t+t\sum_{n=1}^\infty F_nt^n+t^2\sum_{n=1}^\infty F_nt^n+t^2F_0\cr
}$$
Now letting $X=\sum_{n=1}^\infty F_nt^n$ we have
$$X=t+tX+t^2X$$
Hence
$$X(1-t-t^2)=t$$
and finally
$$X={t\over1-t-t^2}$$

\vfill
\eject

\beginsection HW 3-6. (Lecture 4, exercise 10)

Show that
$$\sum_{n=1}^\infty{F_n\over10^n}={10\over89}$$
Using equation (4) with $t=1/10$ we have
$$\eqalign{
\sum_{n=1}^\infty{F_n\over10^n}&=
{1/10\over1-1/10-(1/10)^2}\cr
&={10\over100-10-1}\cr
&={10\over89}
}$$

\end
