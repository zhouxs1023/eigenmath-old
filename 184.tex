\magnification=1200

\noindent
{\it George Weigt -- Geometry Homework \#6}

\beginsection Page 75, problem 3.

Let $\ell$ be a line in the Euclidean Plane and suppose that
$A\in H^+$ and $B\in H^-$.
Show that $\overline{AB}\cap\ell\ne\emptyset$ in the following way.
Let
$$g(t)=\langle A+t(B-A)-P,(Q-P)^\perp\rangle\qquad\hbox{if $t\in R$}$$
Evaluate $g(0)$ and $g(1)$, show that $g$ is continuous,
and then prove that $\overline{AB}\cap\ell\ne\emptyset$.

\medskip\hrule

\bigskip\noindent
We have
$$\eqalign{
g(0)&=\langle A-P,(Q-P)^\perp\rangle\cr
g(1)&=\langle B-P,(Q-P)^\perp\rangle\cr
}$$
By hypothesis $A\in H^+$ and $B\in H^-$ hence
$$\eqalign{
g(0)&>0\cr
g(1)&<0\cr
}$$
To show that $g$ is continuous, let us work out $g(t)$ in component notation.
$$\eqalign{
g(t)&=\langle A+t(B-A)-P,(Q-P)^\perp\rangle\cr
&=[A_x+t(B_x-A_x)-P_x](P_y-Q_y)+[A_y+t(B_y-A_y)-P_y](Q_x-P_x)\cr
}$$
We see that $g(t)$ is a linear function of $t$, therefore $g(t$) is continuous.
By the Intermediate Value Theorem,
there is a $c$ such that $0<c<1$ and $g(c)=0$.
Hence there is a point $C$ such that $C=A+c(B-A)$.
Since $0<c<1$ we have $A{-}C{-}B$, which implies $C\in\overline{AB}$.
In addition, $C\in\ell$ because $g(c)=0$.
Therefore $\overline{AB}\cap\ell\ne\emptyset$.











\vfill
\eject

\beginsection Page 75, problem 4.

If $\ell={}_aL$ is a type I line in the Poincare Plane then prove that
\item{a.} $H_+$ and $H_-$ as defined in Equation (2-10) are convex.
\item{b.} If $A\in H_+$ and $B\in H_-$ then $\overline{AB}\cap\ell\ne\emptyset$.
$$\eqalign{
H_+=\{(x,y)\in H\mid x>a\}\cr
H_-=\{(x,y)\in H\mid x<a\}\cr
}\eqno\hbox{(2-10)}
$$
\hrule

\bigskip\noindent
{\bf Part a.} Let $A=(x_1,y_1)$, $B=(x_2,y_2)$ and $X=(x_0,y_0)$ and let $A{-}X{-}B$.
The points $A$ and $B$ can lie on either a type I or a type II line
so we have to consider each case.

\medskip
\item{1.} If $A$, $B$ and $X$ lie on a type I line then
$$x_1=x_2=x_0$$
By the hypothesis that $A,B\in H_+$, we have
$$x_1=x_2=x_0>a$$
Hence $X\in H_+$ from which it follows that $\overline{AB}\subset H_+$.

\medskip
\item{2.} Let $A$, $B$ and $X$ lie on the type II line ${}_cL_r$.
Let $f(A)=t_1$, $f(B)=t_2$ and $F(X)=t_0$. Then we have
$$x_1=c+r\tanh t_1,\qquad x_0=c+r\tanh t_0,\qquad x_2=c+r\tanh t_2$$
Since $\tanh(t)$ is a strictly increasing function, $A{-}X{-}B$ implies $t_1*t_0*t_2$
which implies
$$x_1*x_0*x_2$$
By the hypothesis that $A,B\in H_+$, we have $x_1>a$ and $x_2>a$, hence $x_0>a$
Therefore we have $X\in H_+$ from whence it follows that $\overline{AB}\in H_+$.

\medskip\noindent
Therefore, by (1) and (2) above we have $H_+$ is convex.
The proof is similar for $H_-$, except use $x<a$.

\medskip\noindent
{\bf Part b.} Let $A=(x_1,y_1)$ and $B=(x_2,y_2)$.
We have by hypothesis that $x_1>a$ and $x_2<a$, hence $x_1\ne x_2$, hence
$\overleftarrow A\overrightarrow B$ is a type II line.
Let $A$ and $B$ lie on the type II line ${}_cL_r$.
Let $f(A)=t_1$ and $f(B)=t_2$.
Then we have
$$x_1=c+r\tanh t_1,\qquad x_2=c+r\tanh t_2$$
Since $\tanh(t)$ is a continuous function, by $x_1>a$ and $x_2<a$ and the
Intermediate Value Theorem we have a $t_0$ such that
$$x_2=c+r\tanh(t_0)=a$$
By the Ruler Postulate there is a point $X$ such that $f(X)=t_0$.
Note that since $x_1>x_0>x_2$ and the fact that $\tanh(t)$ is a strictly increasing function,
we have $t_1>t_0>t_2$ hence $f(A)>f(X)>f(B)$.
Therefore we have $A{-}X{-}B$ hence $X\in\overline{AB}$.
From $x_2=a$ we have $X\in{}_aL$.
Therefore $\overline{AB}\cap\ell\ne\emptyset$.











\vfill
\eject

\beginsection Page 75, problem 5.

For the Taxicab Plane $\{{\cal R}^2,{\cal L}_E,d_T\}$ prove that
\item{a.} If $A=(x_1,y_1)$, $B=(x_2,y_2)$ and $C=(x_3,y_3)$ are collinear but do not
lie on a vertical line then $A{-}B{-}C$ if and only if $x_1*x_2*x_3$.
\item{b.} The Taxicab Plane satisfies PSA.
\medskip
\hrule

\bigskip\noindent
{\bf Part a.} For a non-vertical ine in the Taxicab Plane we have $f(x,y)=x(1+|m|)$.
If $A{-}B{-}C$ then
$$f(A)*f(B)*f(C)$$
Divide through by $1+|m|$ to obtain
$$x_1*x_2*x_3$$
Now for the converse. Assume $x_1*x_2*x_3$.
Multiply all by $1+|m|$ to obtain
$$f(A)*f(B)*f(C)$$
Hence $x_1*x_2*x_3$ implies $A{-}B{-}C$.

\medskip\noindent
{\bf Part b.}
All we have to do is show is that $A{-}B{-}C$ in the Euclidean Plane if and only if
$A{-}B{-}C$ in the Taxicab Plane.

\medskip\noindent
%Let $A=(x_1,y_1)$, $B=(x_2,y_2)$, $C=(x_3,y_3)$.
For $A{-}B{-}C$ in the Euclidean Plane we have
$$(x_1\sqrt{1+m^2})*(x_2\sqrt{1+m^2})*(x_3\sqrt{1+m^2})$$
Divide through by $\sqrt{1+m^2}$ and multiply by $(1+|m|)$ to obtain
$$x_1(1+|m|)*x_2(1+|m|)*x_3(1+|m|)$$
which is $A{-}B{-}C$ in the Taxicab Plane.
For the converse, divide by $(1+|m|)$ and multiply by $\sqrt{1+m^2}$.









\vfill
\eject

\beginsection Page 80, problem 1.

(Peano's Axiom) Given a triangle $\triangle ABC$ in a metric geometry which satisfies PSA
and points $D$, $E$ with $B{-}C{-}D$ and $A{-}E{-}C$, prove there is a point
$F\in\overleftarrow D\overrightarrow E$ with $A{-}F{-}B$, and $D{-}E{-}F$.

\medskip\hrule

\bigskip\noindent
Let $\ell=\overleftarrow D\overrightarrow E$. Prove $A{-}F{-}B$ as follows.

\medskip
\item{1.} By Pasch's Postulate either $\ell\cap\overline{AB}\ne\emptyset$ or
$\ell\cap\overline{BC}\ne\emptyset$.
However, we must have $\ell\cap\overline{BC}=\emptyset$ because $\ell$ and
$\overleftarrow B\overrightarrow C$ are unique lines that already
intersect at $D$.
Therefore, there is no choice except to have $\ell\cap\overline{AB}\ne\emptyset$.

\medskip
\item{2.}
Let $F\in(\ell\cap\overline{AB})$.
%
We have $F\ne A$ because $\ell$ and
$\overleftarrow A\overrightarrow B$
are unique lines that already intersect at $E$.
%
We have $F\ne B$ because $\ell$ and
$\overleftarrow B\overrightarrow C$
are unique lines that already intersect at $D$.
%
Therefore we have $A{-}F{-}B$.

\medskip\noindent
Now prove $D{-}E{-}F$ as follows.
%
\medskip
\item{1.} By $B{-}C{-}D$ we have $B\in H_1$ and $D\in H_2$ for the line
$\overleftarrow A\overrightarrow C$.
%
By $A{-}F{-}B$ we have $F$ in the same half plane as $B$, hence $F\in H_1$.
%
Since $D\in H_2$ and $F\in H_1$, by theorem we have
$\overline{DF}\cap\overleftarrow A\overrightarrow C\ne\emptyset$.
%We also have $E\in\overline{AC}$ and $E\in\overleftarrow D\overrightarrow F$.
%

\medskip
\item{2.} Note that $\ell$ and $\overleftarrow A\overrightarrow C$ intersect at $E$.
Since distinct lines can intersect in at most one point, we must have
$\overline{DF}\cap\overleftarrow A\overrightarrow C=\{E\}$.

\medskip
\item{3.} Therefore we have $E\in\overline{DF}$, hence $D{-}E{-}F$.







\vfill
\eject

\beginsection Page 81, problem 2.

Given $\triangle ABC$ in a metric geometry which satisfies PSA and points
$D$, $F$ with $B{-}C{-}D$, $A{-}F{-}B$, prove there exists
$E\in\overleftarrow D\overrightarrow F$ with $A{-}E{-}C$ and $D{-}E{-}F$.

\medskip\hrule

\bigskip\noindent
Let $\ell=\overleftarrow D\overrightarrow F$. Prove $A{-}E{-}C$ as follows.

\medskip
\item{1.} By Pasch's Postulate either $\ell\cap\overline{AC}\ne\emptyset$ or
$\ell\cap\overline{BC}\ne\emptyset$.
However, we must have $\ell\cap\overline{BC}=\emptyset$ because $\ell$ and
$\overleftarrow B\overrightarrow C$ are unique lines that already
intersect at $D$.
Therefore, there is no choice except to have $\ell\cap\overline{AC}\ne\emptyset$.

\medskip
\item{2.}
Let $E\in(\ell\cap\overline{AC})$.
%
We have $E\ne A$ because $\ell$ and
$\overleftarrow A\overrightarrow B$
are unique lines that already intersect at $F$.
%
We have $E\ne C$ because $\ell$ and
$\overleftarrow B\overrightarrow C$
are unique lines that already intersect at $D$.
%
Therefore we have $A{-}E{-}C$.

\medskip\noindent
Now prove $D{-}E{-}F$ as follows.
%
\medskip
\item{1.} By $B{-}C{-}D$ we have $B\in H_1$ and $D\in H_2$ for the line
$\overleftarrow A\overrightarrow C$.
%
By $A{-}F{-}B$ we have $F$ in the same half plane as $B$, hence $F\in H_1$.
%
Since $D\in H_2$ and $F\in H_1$, by theorem we have
$\overline{DF}\cap\overleftarrow A\overrightarrow C\ne\emptyset$.
%We also have $E\in\overline{AC}$ and $E\in\overleftarrow D\overrightarrow F$.
%

\medskip
\item{2.} Note that $\ell$ and $\overleftarrow A\overrightarrow C$ intersect at $E$.
Since distinct lines can intersect in at most one point, we must have
$\overline{DF}\cap\overleftarrow A\overrightarrow C=\{E\}$.

\medskip
\item{3.} Therefore we have $E\in\overline{DF}$, hence $D{-}E{-}F$.






\vfill
\eject

\beginsection Page 91, problem 3.

Given $\triangle ABC$ and a point $P$ in a metric geometry which satisfies PSA
prove there is a line through $P$ that contains exactly two points of
$\triangle ABC$.

\medskip\hrule

\bigskip\noindent
Assume $P\not\in\overleftarrow A\overrightarrow B$.
Define $X$ such that $A{-}X{-}B$
and let $\ell=\overleftarrow P\overrightarrow X$.

\medskip\noindent
Since $\ell$ and $\overleftarrow A\overrightarrow B$ are unique lines
they can intersect in at most one point,
$\ell\cap\overline{AB}=\{X\}$.

\medskip\noindent
By Pasch's Postulate $\ell\cap\overline{AC}\ne\emptyset$ or $\ell\cap\overline{BC}\ne\emptyset$.

\medskip\noindent
Since $X\ne A$ we have $\ell\ne\overleftarrow A\overrightarrow C$.
Hence the cardinality of $\ell\cap\overline{AC}$ is at most one.

\medskip\noindent
Since $X\ne B$ we have $\ell\ne\overleftarrow B\overrightarrow C$.
Hence the cardinality of $\ell\cap\overline{BC}$ is at most one.

\medskip\noindent
If $\ell\cap\overline{AC}=\{C\}$ and $\ell\cap\overline{BC}=\{C\}$ then we are done,
$\ell\cap\triangle ABC=\{C,X\}$.

\medskip\noindent
Otherwise we have $X\ne A,B,C$.
By Pasch's Postulate and the theorem that $\ell$ cannot intersect all three sides,
we have either $\ell\cap\overline{AC}\ne\emptyset$ or $\ell\cap\overline{BC}\ne\emptyset$
but not both.

\medskip\noindent
Therefore by the cardinality conditions, $\ell$ contains exactly two points of $\triangle ABC$.

\medskip\noindent
A similar argument can be made for
$P\not\in\overleftarrow A\overrightarrow C$
and
$P\not\in\overleftarrow B\overrightarrow C$.
Since $A$, $B$ and $C$ are not collinear, at least one of these must be true.












\vfill
\eject
\beginsection Page 81, problem 4.

Prove that the Missing Strip Plane is an incidence geometry.

\medskip\hrule

\bigskip\noindent
It is known that the Cartesian Plane is an incidence geometry.
In the Missing Strip Plane we have ${\cal S}\subset R^2$.
Therefore every two points in $\cal S$ lie on a unique line.
Let $A=(-1,0)$, $B=(1,0)$ and $C=(1,1)$.
Then we have at least three points in $\cal S$ that are not collinear.










\vfill
\eject

\beginsection Page 81, problem 5.

Prove the following (Proposition 4.3.4).
If $\{{\cal S},{\cal L}\}$ is the Missing Strip Plane and $\ell=L_{m,b}$ then
$g_\ell:\ell\cap{\cal S}\rightarrow R$ is a bijection.
$$g_\ell(x,y)=\cases{
x\sqrt{1+m^2} & if $x<0$\cr
(x-1)\sqrt{1+m^2} & if $x\ge1$\cr
}$$

\hrule

\bigskip\noindent
First show that $g_\ell$ is injective, then show that $g_\ell$ is surjective.

\medskip\noindent
To show that $g_\ell$ is injective, suppose $g_\ell(A)=g_\ell(B)$, then show $A=B$.
Let $A=(x_1,y_1)$ and $B=(x_2,y_2)$.
Then $g_\ell(A)=g_\ell(B)$ implies the following:

\medskip
\item{i.} For $x_1,x_2<0$ we have $x_1\sqrt{1+m^2}=x_2\sqrt{1+m^2}$, hence $x_1=x_2$.

\medskip
\item{ii.} For $x_1,x_2\ge1$ we have $(x_1-1)\sqrt{1+m^2}=(x_2-1)\sqrt{1+m^2}$, hence $x_1=x_2$.

\medskip
\item{iii.} The case $x_1\ge1$ and $x_2<0$ does not apply
because
$(x_1-1)\sqrt{1+m^2}$ is zero or positive and
$x_2\sqrt{1+m^2}$ is negative.

\medskip
\item{iv.} The case $x_1<0$ and $x_2\ge1$ does not apply for reasons similar to (iii).

\medskip\noindent
Hence the only possible result of $g_\ell(A)=g_\ell(B)$ is that $x_1=x_2$.
By $y=mx+b$ we have $y_1=y_2$, hence $A=B$.
Therefore $g_\ell$ is injective.

\medskip\noindent
To show that $g_\ell$ is surjective, let $t\in R$ and show that there is an $A\in{\cal S}$
such that $g_\ell(A)=t$. We have $A=(x_1,y_1)$ where
$$x_1=\cases{
t/\sqrt{1+m^2} & $t<0$\cr
t/\sqrt{1+m^2}+1 & $t\ge0$\cr
}$$
and $y_1=mx_1+b$.
Hence for $t\in R$ we have $g_\ell(A)=t$. Therefore $g_\ell$ is surjective.

\medskip\noindent
Therefore $g_\ell$ is bijective.











\vfill
\eject

\beginsection Page 81, problem 7.

Given $\triangle ABC$ in a metric geometry and points $D$, $E$ with $A{-}D{-}B$ and
$C{-}E{-}B$, prove $\overleftarrow A\overrightarrow E\cap
\overleftarrow C\overrightarrow D\ne\emptyset$.

\medskip\hrule

\bigskip\noindent
Let $\ell=\overleftarrow A\overrightarrow E$.
Define half planes $H_1$ and $H_2$ for $\ell$.

\medskip\noindent
Assert $B\in H_1$ and $C\in H_2$ because $B{-}E{-}C$.

\medskip\noindent
Assert $D\in H_1$ because $A{-}D{-}B$ implies $D$ is in the same half plane as $B$.

\medskip\noindent
Assert $\overline{CD}\cap\ell\ne\emptyset$ because
$C\in H_2$, $D\in H_1$.

















\vfill
\eject

\beginsection Page 81, problem 8.

Let $R^3=\{(x,y,z)\mid x,y,z\in R\}$.
If $A,B\in R^3$ define $L_{A,B}=\{A+t(B-A)\mid t\in R\}$.
Let ${\cal L}=\{L_{A,B}\mid A,B\in R^3,A\ne B\}$.
If $A,B\in R^3$ let $d(A,B)=\|A-B\|$.
Prove that $\{R^3,{\cal L},d\}$ is a metric geometry but that
it does not satisfy PSA.

\medskip\hrule

\bigskip\noindent
Let ${\cal M}=\{R^3,{\cal L},d\}$.
First show that $\cal M$ is an abstract geometry.
By $L_{A,B}$ we have every two distinct points $A$ and $B$ belong to a line.
By $t\in R$ we have no line with only one point.

\medskip\noindent
Next show that $\cal M$ is an incidence geometry.
%Let $P=(x_1,y_1,z_1)$ and $Q=(x_2,y_2,z_2)$.
(i) Assume that $P$ and $Q$ lie on two distinct lines $\ell_1$ and $\ell_2$.
Then
$$\eqalign{
\ell_1&=\{P+t(Q-P)\mid t\in R\}\cr
\ell_2&=\{P+t(Q-P)\mid t\in R\}\cr
}$$
However, $\ell_1=\ell_2$ hence the assumption is false.
Therefore, every two distinct points lie on a unique line.
(ii) Let $A=(0,0,1)$, $B=(0,1,0)$ and $C=(1,0,0)$.
There is no linear combination of $A$ and $B$ that equals $C$.
Hence $A$, $B$ and $C$ are not collinear.
Therefore, by (i) and (ii), $\cal M$ is an incidence geometry.

\medskip\noindent
Next show that $d$ is a distance function. We have
$$d=\sqrt{(x_1-x_2)^2+(y_1-y_2)^2+(z_1-z_2)^2}$$
(i) $d(P,Q)\ge0$ by the square root function.
(ii) If $d(P,Q)=0$ then $x_1=x_2$, $y_1=y_2$ and $z_1=z_2$ hence $P=Q$.
If $P=Q$ then $x_1=x_2$, $y_1=y_2$ and $z_1=z_2$ hence $d(P,Q)=0$.
Therefore $d(P,Q)=0$ if and only if $P=Q$.
(iii) $x_1-x_2=x_2-x_1$ and so on. Therefore $d(P,Q)=d(Q,P)$.
Hence by (i), (ii) and (iii), $d$ is a distance function.

\medskip\noindent
When $z_1\ne z_2$ let the ruler function be
$$f(x,y,z)=z\sqrt{m_1^2+m_2^2+1}$$
Then we have
$$|f(A)-f(B)|=d(A,B)\eqno(1)$$
To show that $f$ is surjective, define an arbitrary $t\in R$.
Then for $A=(x,y,z)$ with
$$z={t\over\sqrt{m_1^2+m_2^2+1}}$$
we have $f(A)=t$.
Hence $t$ is surjective and by (1) it is also bijective.
For $z_1=z_2$ use the Euclidean ruler which is known to be bijective.
Therefore, $\cal M$ is a metric geometry.

\medskip\noindent
$\cal M$ does not satisfy PSA because it fails Pasch's Postulate.
For example, let $A=(-1,0,0)$, $B=(1,0,0)$ and $C=(0,1,0)$.
Let $\ell=(0,0,t)$.
Then $\ell$ intersects $\triangle ABC$ at just one point, $X=(0,0,0)$,
and $X\ne A,B,C$.

\end
