\beginsection 1.

$${9801\over2\sqrt2\cdot1103}\approx3.14159$$
This number approximates the famous mathemtical constant $\pi$.

\beginsection 2.

The quadratic formula is
$$x={-b\pm\sqrt{b^2-4ac}\over2a}$$
For $x^2-x-1=0$ we have $a=1$, $b=-1$ and $c=-1$.
$$x={1\pm\sqrt5\over2}$$

\beginsection 3.

For $x^2+2ax-1=0$ the trick is to not confuse the coefficient $2a$ with the
$a$ in the quadratic formula. We have
$$x={-2a\pm\sqrt{4a^2+4}\over2}=-a\pm\sqrt{a^2+1}$$

\beginsection 4.

$$\lim_{x\to0}{\sin x\over x}=1$$
Although not a rigorous proof, the result can be grasped intuitively
when one considers that $\sin x\approx x$ for small $x$.

\beginsection 5.

$$\int \sin x\,dx=-\cos x+C$$

\beginsection 6.

$$\int_0^1\left(1+x+{x^2\over2}+{x^3\over3}+{x^4\over4}+{x^5\over5}\right)\,dx
=\left(x+{x^2\over2}+{x^3\over6}+{x^4\over12}+{x^5\over20}+{x^6\over30}\right)
\bigg|_0^1={11\over6}$$

\beginsection 7.

\indent{\tt http://www-history.mcs.st-and.ac.uk/history/}\par
{\tt http://mathworld.wolfram.com/topics/History.html}\par
{\tt http://www.wilkiecollins.demon.co.uk/roman/front.htm}\par
{\tt http://www.claymath.org/euclid/}\par
{\tt http://front.math.ucdavis.edu/math.HO}

\beginsection 8.

Descartes (1596--1650),
Fermat (1601--1665), Euler (1707-1783), Lagrange (1736-1813),
Legendre (1752--1833).

\beginsection 9.

Here is a list of the first 100 prime numbers.\par
  2   3   5   7  11  13  17  19  23  29\par
 31  37  41  43  47  53  59  61  67  71\par
 73  79  83  89  97 101 103 107 109 113\par
127 131 137 139 149 151 157 163 167 173\par
179 181 191 193 197 199 211 223 227 229\par
233 239 241 251 257 263 269 271 277 281\par
283 293 307 311 313 317 331 337 347 349\par
353 359 367 373 379 383 389 397 401 409\par
419 421 431 433 439 443 449 457 461 463\par
467 479 487 491 499 503 509 521 523 541\par

\beginsection 10.

$$\int_0^\pi\sin^4x\,dx={1\over8}\int_0^\pi(3-4\cos 2x+\cos 4x)\,dx$$
Let $y=2x$, then $dx=dy/2$ and
$$\int\cos2x\,dx={1\over2}\int\cos y\,dy=(\sin y)/2=(\sin 2x)/2$$
Let $y=4x$, then $dx=dy/4$ and
$$\int\cos4x\,dx={1\over4}\int\cos y\,dy=(\sin y)/4=(\sin 4x)/4$$
We have
$${1\over8}\int_0^\pi(3-4\cos 2x+\cos 4x)\,dx=
\left({3x\over8}-{\sin2x\over4}+{\sin4x\over32}\right)\bigg|_0^\pi
={3\pi\over8}$$

\beginsection 11.

Like Gauss' solution...
$$1+2+\cdots+1000=500\times1001=500{,}500$$

\beginsection 12.

$$1+3+5+\cdots+997+999=(1+999)+(3+997)+\cdots+(499+501)=250\times1000=250{,}000$$
because there are 250 odd integers between 0 and 500.

\beginsection 13.

$$1^2+2^2+3^2+\cdots+1000^2=333{,}833{,}500$$

\end
