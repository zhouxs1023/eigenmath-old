\noindent{\it 176.tex}

\beginsection Try out 4.1

Write the meanings of $a*b*c$ and $a*b*d$ from Definition 4.1
and show that neither $c<b<d$, nor $d<b<c$ can occur at the same
time with $a*b*c$ and $a*b*d$.

\medskip
$$a*b*c=a<b<c\hbox{\ \ or\ \ }c<b<a$$
$$a*b*d=a<b<d\hbox{\ \ or\ \ }d<b<a$$
There are two cases to consider.
\item{i.}
The condition $a<b<c$ implies $a<b<d$.
In this case, $b$ is less than both $c$ and $d$.
Hence $b$ is not between $c$ and $d$.
Therefore $c<b<d$ is not true and $d<b<c$ is not true.
\item{ii.}
The condition $a>b>c$ implies $a>b>d$.
In this case, $b$ is greater than both $c$ and $d$.
Hence $b$ is not between $c$ and $d$.
Therefore $c<b<d$ is not true and $d<b<c$ is not true.

\medskip\noindent
{\it Alternate proof.}
From $a*b*c$ we have
$$|a-b|+|b-c|=|a-c|\eqno(1)$$
From $a*b*d$ we have
$$|a-b|+|b-d|=|a-d|\eqno(2)$$
Subtract equation (2) from equation (1) to obtain
$$|b-c|-|b-d|=|a-c|-|a-d|$$
Not completed yet.

\beginsection Try out 4.2

Show that $\overline{AB}\subset\overleftarrow A\overrightarrow B$.

\medskip\noindent
We need to show that $C\in\overline{AB}$ implies $C\in\overleftarrow A\overrightarrow B$.
Let $C\in\overline{AB}$. Then there are three cases.
\item{i.} If $C=A$ then $C\in\overleftarrow A\overrightarrow B$.
\item{ii.} If $C=B$ then $C\in\overleftarrow A\overrightarrow B$.
\item{iii.} If $A{-}C{-}B$ then $A$, $B$, and $C$ are collinear hence
$C\in\overleftarrow A\overrightarrow B$.

\noindent
Therefore $\overline{AB}\subset\overleftarrow A\overrightarrow B$.

\beginsection Try out 4.3

Show that the function
$c+r\tanh t$
is a strictly increasing function by showing that its derivative is strictly positive.
\medskip\noindent
$$\eqalign{
{d\over dt}(c+r\tanh t)&={d\over dt}\left({\sinh t\over\cosh t}\right)\cr
&={\cosh t\over\cosh t}-{\sinh t\sinh t\over(\cosh t)^2}\cr
&={1\over(\cosh t)^2}\cr
}$$

\beginsection Try out 4.4

Write down the proof of part (i) of the previous theorem in the case $A{-}B{-}C$,
that is
\item{i.} if $C\in\overrightarrow{AB}$ and $C\ne A$, then
$\overrightarrow{AC}=\overrightarrow{AB}$.
\medskip\noindent
Let $A{-}B{-}C$. We want to show that $\overrightarrow{AC}=\overrightarrow{AB}$.
Hence, we will show that (1) $\overrightarrow{AC}\subset\overrightarrow{AB}$
and (2) $\overrightarrow{AC}\supset\overrightarrow{AB}$.

\medskip\noindent
\item{1.} {\it Proof by contradiction.}
Suppose there is a point $X$ such that $X\in\overrightarrow{AC}$
and $X\not\in\overrightarrow{AB}$.
Then we must have $X{-}A{-}B$ because otherwise we would have
$X\in\overrightarrow{AB}$.
%
Suppose $x<a<b$. Then by $A{-}B{-}C$ we must have $a<b<c$, hence $x<a<c$.
This implies $X{-}A{-}C$ which in turn implies
$X\not\in\overrightarrow{AC}$ which contradicts the assumption.
%
Suppose $x>a>b$. Then by $A{-}B{-}C$ we must have $a>b>c$, hence $x>a>c$.
This implies $X{-}A{-}C$ which in turn implies
$X\not\in\overline{AC}$ which contradicts the assumption.
%
Hence for any $X\in\overrightarrow{AC}$ we must also have $X\in\overrightarrow{AB}$.
Therefore $\overrightarrow{AC}\subset\overrightarrow{AB}$.

\medskip\noindent
\item{2.} As above.

\end