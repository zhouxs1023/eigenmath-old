\magnification=2000
\nopagenumbers

$$c={y_2^2-y_1^2+x_2^2-x_1^2\over2(x_2-x_1)}$$

$$r=\sqrt{(x_1-c)^2+y_1^2}$$

$$f(a,y)=\ln y$$

$$f(x,y)=\ln\left({x-c+r\over y}\right)$$

\vfill
\eject

\noindent
{\bf Convexity} --- 5.17

\bigskip
\noindent
In a metric geometry $\{S,L,d\}$

\bigskip
\noindent
Regarding a subset $S_1$ of $S$

\bigskip
\noindent
$S_1$ is convex if $\overline{PQ}\subset S_1$ for all $P,Q\in S_1$.

\vfill
\eject

\noindent
{\bf Plane separation axiom} --- 5.20

\bigskip
\noindent
A metric geometry $\{S,L,d\}$ satisfies PSA if

\bigskip
\noindent
For every $\ell\in L$ there are $H_1,H_2\subset S$ such that

\bigskip
\item{(i)} $S-\ell=H_1\cup H_2$

\bigskip
\item{(ii)} $H_1$, $H_2$ convex and $H_1\cap H_2=\emptyset$

\bigskip
\item{(iii)} If $A\in H_1$ and $B\in H_2$ then $\overline{AB}\cap\ell\ne\emptyset$.

\vfill
\eject

\noindent
{\bf Pasch's postulate} --- 6.9

\bigskip
\noindent
A metric geometry satisfies PP if

\bigskip
\noindent
For any line $\ell$, $\triangle ABC$, $D\in\ell$ such that $A{-}D{-}B$

\bigskip
\noindent
Either $\ell\cap\overline{AC}\ne\emptyset$ or $\ell\cap\overline{BC}\ne\emptyset$.

\vfill
\eject

\noindent
{\bf Interior of an angle} --- 8.6

\bigskip
\noindent
In a Pash geometry

\bigskip
\noindent
$\mathop{\hbox{int}}(\angle ABC)$ is the intersection of

\bigskip
\item{} the side of $\overleftarrow A\overrightarrow B$ that contains $C$

\bigskip
\item{} the side of $\overleftarrow B\overrightarrow C$ that contains $A$

\vfill
\eject

\noindent
{\bf Interior of a triangle} --- 8.14

\bigskip
\noindent
In a Pasch geometry

\bigskip
\noindent
$\mathop{\hbox{int}}(\triangle ABC)$ is the intersection of

\bigskip
\item{} the side of $\overleftarrow A\overrightarrow B$ that contains $C$

\bigskip
\item{} the side of $\overleftarrow B\overrightarrow C$ that contains $A$

\bigskip
\item{} the side of $\overleftarrow C\overrightarrow A$ that contains $B$

\vfill
\eject

\noindent
{\bf Angle measure} --- 9.1

\bigskip
\noindent
In a Pasch geometry

\bigskip
\noindent
Let $r_0$ be a fixed positive real number

\bigskip
\noindent
$m$ is an angle measure if

\bigskip
\item{(i)} $0<m(\angle ABC)<r_0$

\bigskip
\item{(ii)} There is a unique ray such that $m(\angle ABC)=\theta$

\bigskip
\item{(iii)} For $D\in\mathop{\hbox{int}}(\angle ABC)$ we have
$$m(\angle ABD)+m(\angle DBC)=m(\angle ABC)$$

\vfill
\eject

\noindent
{\bf Protractor geometry} --- 9.3

\bigskip
\noindent
Is a Pasch geometry with an angle measure.

\vfill
\eject

\noindent
{\bf Euclidean angle measure} --- 9.4

$$m_E(\angle ABC)=\cos^{-1}\left(
{\langle A-B,C-B\rangle\over
\|A-B\|\cdot\|C-B\|}\right)$$

\vfill
\eject

\noindent
{\bf Euclidean tangent vector} --- 9.7

$$T_{PQ}=\cases{
(0,y_Q-y_P) & type I line\cr
(y_P,c-x_P) & type II line, $x_P<x_Q$\cr
-(y_P,c-x_P) & type II line, $x_P>x_Q$\cr
}$$

\vfill
\eject

\noindent
{\bf Poincare angle measure} --- 9.8

$$m_H=\cos^{-1}\left({
\langle T_{BA},T_{BC}\rangle
\over
\|T_{BA}\|\cdot\|T_{BC}\|
}\right)$$

\bigskip
\bigskip
\noindent
$$m_H(\angle ABC)=m_E(\angle A'BC')$$
$$A'=B+T_{BA}$$
$$C'=B+T_{BC}$$

\vfill
\eject

\noindent
{\bf Pasch's theorem} --- 6.10

\bigskip
\noindent
A metric geometry that satisfies PSA also satisfies PP.

\bigskip
\bigskip
\noindent
{\bf Converse of Pasch's theorem} --- 7.2

\bigskip
\noindent
A metric geometry that satisfies PP also satisfies PSA.

\vfill
\eject

\noindent
{\bf Crossbar theorem} --- 8.11

\bigskip
\noindent
In a Pasch geometry

\bigskip
\noindent
If $P\in\mathop{\hbox{int}}(\angle ABC)$

\bigskip
\noindent
Then $\overrightarrow{BP}$ intersects $\overline{AC}$ at a unique point $F$

\bigskip
\noindent
Such that $A{-}F{-}C$.

\vfill
\eject

\noindent
{\bf Linear pair theorem} --- 9.14

\bigskip
\noindent
If $\angle ABC$ and $\angle CBD$ form a linear pair

\bigskip
\noindent
Then the angles are supplementary

\bigskip
\noindent
$m(\angle ABC)+m(\angle CBD)=180$.

\vfill
\eject

\noindent
{\bf Perpendicular bisector} --- 9.19

\bigskip
\noindent
In a protractor geometry

\bigskip
\noindent
Every line segment $\overline{AB}$ has a perpendicular bisector

\bigskip
\noindent
That is, a line $\ell$ such that $\ell\perp\overline{AB}$

\bigskip
\noindent
With $\ell\cap\overline{AB}=\{M\}$ where $M$ is the midpoint of $\overline{AB}$.



\end
