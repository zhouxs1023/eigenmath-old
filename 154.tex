\parindent=0pt

\beginsection 1. (a)

What is the distribution of $\overline X$?

\bigskip
Solution: See the book, p. 281.
$$\overline X\sim N(\mu,\,\sigma^2/n)$$
We have
$$n=9,\qquad\mu=100,\qquad\sigma^2=64$$
Hence
$$\overline X\sim N(100,\,64/9)$$

\beginsection 1. (b)

What is the distribution of $S_X^2/8$?

\bigskip
Solution: See Note 6 p. 9.
$${(n-1)S^2\over\sigma^2}\sim\chi_{n-1}^2$$
We have
$$n-1=8,\qquad\sigma^2=64$$
Hence
$${S_X^2\over8}\sim\chi_8^2$$

\beginsection 1. (c)

What is the distribution of $3(\overline X-100)/S_X$?
\bigskip
Solution: See Note 6 p. 14.
$${\sqrt n(\overline X-\mu)\over S}\sim t_{n-1}$$
We have
$$n=9,\qquad\mu=100$$
Hence
$${3(\overline X-100)\over S_X}\sim t_8$$

\beginsection 1. (d)

What is the distribution of $(\overline Y-80)/5$?
\bigskip
Solution: See the book, p. 292.
$${\overline X-\mu\over\sigma/\sqrt n}\sim N(0,1)$$
We have
$$n=4,\qquad\mu=80,\qquad\sigma^2=100$$
therefore
$${\overline Y-80\over5}\sim N(0,1)$$

\beginsection 1. (e)

What is the distribution of $S_2^2/25$?
\bigskip
Solution: See Note 6 p. 11.
$$\sum_{i=1}^n\left({X_i-\mu\over\sigma}\right)^2\sim\chi_n^2$$
We have
$$\mu=80,\qquad\sigma^2=100$$
hence
$${S_2^2\over25}={\sum_{j=1}^4(Y_j-80)^2\over 4\times25}
=\sum_{j=1}^4\left({Y_j-80\over10}\right)^2\sim\chi_4^2$$

\beginsection 1. (f)

What is the distribution of $0.140625 S_1^2+0.03S_Y^2$?
\bigskip
Solution: See Note 6 page 8 about adding $\chi^2$ distributions (first theorem on the page.)
See also bottom of page 11.
We have
$$0.140625S_1^2={9S_1^2\over64}\sim\chi_9^2\qquad
\hbox{(using the formula containing $X-\mu$)}$$
$$0.03S_Y^2={3S_Y^2\over100}\sim\chi_3^2\qquad
\hbox{(using the formula containing $X-\overline X$)}$$
hence
$$0.140625 S_1^2+0.03S_Y^2\sim\chi_{12}^2$$

\beginsection 1. (g)

What is the distribution of $0.64S_2^2/S_X^2$?
\bigskip
Solution: See Note 6 page 17.
From problems (b) and (e) we have
$$
{S_X^2\over8}\sim\chi_8^2,\qquad
{S_2^2\over25}\sim\chi_4^2
$$
therefore
$$0.64{S_2^2\over S_X^2}={(S_2^2/25)/4\over(S_X^2/8)/8}\sim F_{4,8}$$




\end
