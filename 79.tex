\nopagenumbers
\parindent=0pt
{\bf Problem A.}
Let $a,b\in N$ and $c,x,r\in Z$.
Show that if $a|b$, $a|c$ and $c=bx+r$ then $a|r$.
\bigskip
{\bf Proof.}
Let $a,b\in N$ and $c,x,r\in Z$.
Assume that $c=bx+r$.
Then $c-bx=r$.
Assume that $a|b$ and $a|c$.
Then by $N\subset Z$ and the definition of divisibility
there are $s,t\in Z$ such that $b=as$ and $c=at$.
Thus $c-bx=at-(as)x=a(t-sx)$.
Therefore $a(t-sx)=r$.
Since $t-sx\in Z$ it follows that $a|r$.

% On this one it turned out to be better to introduce $c=bx+r$ first.
% I started out introducing $a|b$ and $a|c$ first but it did not flow right.

\bigskip
\bigskip

{\bf Problem B.}
Let $a,b\in Z$ such that $a\ne0$.
Show that if $15|a$ and $a|b$ then $5|(3a-19b)$.
\bigskip
{\bf Proof.}
Let $a,b\in Z$ such that $a\ne0$.
Assume $15|a$ and $a|b$.
Then there are $s,t\in Z$ such that $a=15s$ and $b=at=15st$.
Thus $3a-19b=45s-285st=5(9s-57st)$.
Since $9s-57st\in Z$ it follows that $5|(3a-19b)$.

\bigskip
\bigskip

{\bf Problem C.}
Let $a,b\in Z$ such that $a\ne0$.
Determine the truth value of the statement:
If $2a|8b$ then $a|b$.
\bigskip
False by $a=4$, $b=1$.

\bigskip
\bigskip

{\bf Problem D.}
Show that for any $n\in N$, $4|(5^n+7)$.
\bigskip
{\bf Proof.}
For $n=1$ we have $4|12$ which is true.
Assume that $4|(5^n+7)$ for a fixed $n\in N$.
Then there is a $k\in Z$ such that $5^n+7=4k$.
For $n+1$ we have
$$5^{n+1}+7=5\cdot5^n+5\cdot7-5\cdot7+7
=5(5^n+7)-28=5(4k)-4\cdot7=4(5k-7)$$
Since $5k-7\in Z$ we have $4|(5^{n+1}+7)$.
Hence by induction $4|(5^n+7)$ for any $n\in N$.

\end
