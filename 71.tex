\parindent=0pt

Statistical Methods

\bigskip
6a) The parameter in question ($\mu$, $\sigma^2$, $p$) has
a $1-\alpha$ chance of falling within the confidence
interval. Note that we cannot know the value of the
parameter for sure unless
we sample the entire population.

\bigskip
6b) It is true that the parameter is either within the
interval or it is not. However, since we cannot really know
for sure, we have to rely on the probability of the
confidence interval.

\bigskip
6c) There is no variance associated with a constant.
The distribution of a constant is a delta function.

\bigskip
6d) There are two ways of looking at $1-\alpha$.
For example, suppose we have a 95\% confidence interval
for $\mu$.
Then we can say that there is a 95\% chance that
$\mu$ falls within the confidence interval.
We can also say that if we repeat the experiment 100 times
and calculate a new confidence interval each time, then
approximately 95 of
the intervals will contain $\mu$ and 5 will not.

\end
