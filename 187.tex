\magnification=1200

\noindent
{\it George Weigt -- Advanced Calculus Homework \#7}

\beginsection Question 1.

Show that
$$\sum_{n=0}^\infty {(-1)^n\over(n+1)^2}=\int\!\!\!\int_D{1\over1+xy}\,dx\,dy$$
where
$$D=\left\{\matrix{
0\le x\le 1\cr
0\le y\le 1\cr
}\right\}
$$

\medskip
\hrule

\bigskip
\noindent
The substitution $-u=x$, $-du=dx$ will be useful
for converting the integrand to a geometric series.
Hence
$$\eqalign{
\int\!\!\!\int_D{1\over1+xy}\,dx\,dy
&=-\int_0^{-1} du\int_0^1 dy \, {1\over1-uy}\cr
&=-\int_0^{-1} du\int_0^1 dy \sum_{n=0}^\infty(uy)^n\cr
%
&=-\sum_{n=0}^\infty \left[ \int_0^{-1} du\int_0^1 dy\,(uy)^n \right]\cr
%
&=-\sum_{n=0}^\infty \left[ \int_0^{-1} du
\left({u^n y^{n+1}\over n+1}\;\bigg|_{y=0}^{y=1}\right)\right]\cr
%
&=-\sum_{n=0}^\infty \left[ \int_0^{-1} du \, {u^n\over n+1}\right]\cr
%
&=-\sum_{n=0}^\infty \left[{u^{n+1}\over(n+1)^2}\;\bigg|_{u=0}^{u=-1}\right]\cr
%
&=-\sum_{n=0}^\infty{(-1)^{n+1}\over(n+1)^2}\cr
}$$
We can remove the leading minus sign by noting that
$$-(-1)^{n+1}=(-1)^{n+2}=(-1)^n$$
Hence
$$\int\!\!\!\int_D{1\over1+xy}\,dx\,dy=\sum_{n=0}^\infty{(-1)^n\over(n+1)^2}$$

\vfill
\eject

\beginsection Question 2.

Compute the Jacobian for
$$\eqalign{
x&=u-v\cr
y&=u+v\cr
}$$

\bigskip
\noindent
Solution:
$${\partial(x,y)\over\partial(u,v)}=\left|
\matrix{
{\partial x\over\partial u} & {\partial x\over\partial v}\cr
\cr
{\partial y\over\partial u} & {\partial y\over\partial v}\cr
}\right|
=
\left|\matrix{
1 & -1\cr
1 & 1\cr
}\right|
=2
$$

\beginsection Question 3.

Rewrite the integrand
$${1\over1+xy}$$
in terms of $u$ and $v$.

\bigskip
\noindent
Solution:
$${1\over 1+xy}={1\over 1+(u-v)(u+v)}={1\over1+u^2-v^2}$$

\beginsection Question 4.

The four lines are
$$\eqalign{
(0,0)&\longrightarrow(1/2,1/2)\cr
(0,0)&\longrightarrow(1/2,-1/2)\cr
(1/2,-1/2)&\longrightarrow(1,0)\cr
(1/2,1/2)&\longrightarrow(1,0)\cr
}$$
The equations for the four lines are
$$\eqalign{
v&=u\cr
v&=-u\cr
v&=u-1\cr
v&=-u+1\cr
}$$

\beginsection Question 5.

$$
D_1^*=\left\{
\matrix{
0\le v\le 1/2\cr
v\le u\le 1-v\cr
}\right\}
\qquad
D_2^*=\left\{
\matrix{
-1/2\le v\le 0\cr
-v\le u\le 1+v\cr
}\right\}
$$

\vfill
\eject

\beginsection Question 6.

%$$I_+=\int\!\!\!\int_{D^*}{2\over1+u^2-v^2}\,du\,dv$$

$$I_1=2\int_0^{1/2}dv\int_v^{1-v} du\,{1\over1+u^2-v^2}$$
Integrating over $u$ we have
$$\eqalign{
\int_v^{1-v}du\,{1\over1+u^2-v^2}
&={1\over\sqrt{1-v^2}}\tan^{-1}{u\over\sqrt{1-v^2}}\bigg|_{u=v}^{u=1-v}\cr
&={1\over\sqrt{1-v^2}}\left(\tan^{-1}{1-v\over\sqrt{1-v^2}}-\tan^{-1}{v\over\sqrt{1-v^2}}\right)
}$$
In order to integrate the above result over $v$, let $v=\sin\theta$, $dv=\cos\theta\,d\theta$,
$0=\sin0$, $1/2=\sin(\pi/6)$.
Then we have
$$\eqalign{
I_1=
2\int_0^{\pi/6}\cos\theta\, d\theta \, {1\over\sqrt{1-\sin^2\theta}}
\left(\tan^{-1}{1-\sin\theta\over\sqrt{1-\sin^2\theta}}
-\tan^{-1}{\sin\theta\over\sqrt{1-\sin^2\theta}}
\right)\cr
}$$
$$\eqalign{
&=2\int_0^{\pi/6}\cos\theta\, d\theta \,
{1\over\cos\theta}
\left(\tan^{-1}{1-\sin\theta\over\cos\theta}
-\tan^{-1}{\sin\theta\over\cos\theta}
\right)\cr
&=2\int_0^{\pi/6}d\theta\,\left(\tan^{-1}\tan({\pi\over4}-{\theta\over2})-\theta \right)\cr
&=2\int_0^{\pi/6}d\theta\,\left({\pi\over4}-{3\over2}\theta\right)\cr
&=2\left({\pi\over4}\theta-{3\over4}\theta^2\right)\bigg|_0^{\pi/6}\cr
&={\pi^2\over24}\cr
}$$
Now for $I_2$.
$$I_2=2\int_{-1/2}^0dv\int_{-v}^{1+v} du\,{1\over1+u^2-v^2}$$
Integrating over $u$ we have
$$\eqalign{
\int_{-v}^{1+v}du\,{1\over1+u^2-v^2}
&={1\over\sqrt{1-v^2}}\tan^{-1}{u\over\sqrt{1-v^2}}\bigg|_{u=-v}^{u=1+v}\cr
&={1\over\sqrt{1-v^2}}\left(\tan^{-1}{1+v\over\sqrt{1-v^2}}+\tan^{-1}{v\over\sqrt{1-v^2}}\right)
}$$
In order to integrate the above result over $v$, let $v=-\sin\theta$, $dv=-\cos\theta\,d\theta$,
$0=-\sin0$, $-1/2=-\sin(\pi/6)$.
Then we have
$$\eqalign{
I_2=
-2\int_{\pi/6}^0\cos\theta\, d\theta \, {1\over\sqrt{1-\sin^2\theta}}
\left(\tan^{-1}{1-\sin\theta\over\sqrt{1-\sin^2\theta}}
-\tan^{-1}{\sin\theta\over\sqrt{1-\sin^2\theta}}
\right)\cr
}$$
Note that we can remove the leading minus sign by reversing the limits of integration.
Then we have an expression that is identical to $I_1$, hence
$$I_1=I_2={\pi^2\over24}$$

\beginsection Question 7.

$$I_+=I_1+I_2={\pi^2\over12}$$
Hence
$$\sum_{n=0}^\infty{(-1)^n\over(n+1)^2}={\pi^2\over12}$$


\end
