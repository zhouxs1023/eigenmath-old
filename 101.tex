\beginsection 1.

\item{\S 1.}
Archimedes' Life
\itemitem{1.}
Born circa 287 B.C. in the port city of Syracuse, Sicily.
\itemitem{2.}
Educated in Alexandria at the school of mathematics founded by Euclid.
Continued to correspond with friends in Alexandria long after he returned
to Syracuse.
\itemitem{3.}
Became famous for creating astonishing machines that helped
defend Syracuse during the Second Punic War, 214-212 B.C.
\itemitem{4.}
Killed by a Roman soldier during the sack of Syracuse in 212 B.C.

\bigskip
\item{\S 2.} Archimedes' Work
\itemitem{1.} Angle Trisection.
What Archimedes figured out is that a spiral is like
an analog computer.
To see this, consider the spiral curve
$$r=\theta$$
This curve starts at the origin and spirals out counterclockwise.
At every point on the curve, $r=\theta$ so the spiral can be
uused to convert angles ($\theta$) to distance ($r$) and back again.
To trisect an angle, Archimedes draws a spiral that intersects the
slanted line subtending $\theta$.
The point of intersection provides a distance $r$ equivalent to $\theta$.
Divide this distance by 3.
With a compass draw a circle about the origin passing through $r/3$.
The circle intersects the original spiral at $\theta/3$.
A ray through this point trisects $\theta$.
\itemitem{2.}
Calculation of $\pi$.
The calculation boils down to finding the circumference of a circle.
Archimedes realized that the circumference is bounded by
the perimeters of circumscribed and inscribed polygons.
He devised an algorithm that permits the bound to
be made arbitrarily small by iteratively doubling of number
of polygon sides.

\itemitem{3.}
Archimedes was the first mathematician to use infinitesimals, the
forerunner of differential calculus.

\itemitem{4.}
Discovered the principles of density and buoyancy while taking a bath.

\itemitem{5.}
Calculated the area of a parabolic segment.

\itemitem{6.}
Devised a proof relating the volume of a sphere to its radius.

\itemitem{7.}
Devised a proof of the Law of the Lever based on static principles.

\itemitem{8.}
Used the method of exhaustion to devise many proofs.
The method of exhaustion is the forerunner of integral calculus.

\bigskip
\item{\S 3.}
References

\itemitem{1.}
``Archimedes,'' {\tt http://en.wikipedia.org/wiki/Archimedes}

\itemitem{2.}
``Archimedes,'' {\tt http://fermatslasttheorem.blogspot.com/2006/04/archimedes.html}

\itemitem{3.}
``Infinitesimal,'' {\tt http://en.wikipedia.org/wiki/Infinitesimals}

\itemitem{4.}
"Lecture 3: Archimedes and $\pi$,'' {\tt chan.hei-chi@uis.edu}

\itemitem{5.}
``A History of Mathematics'' by Carl B. Boyer

\vfill
\eject

\beginsection 2.

Prove the geometric series
$$1+r+r^2+r^3+\cdots={1\over1-r}$$
First, prove
$$\sum_{k=0}^nr^k={1-r^{n+1}\over1-r}$$
For $n=0$ we have $r^0=1=(1-r)/(1-r)$ so the equation is true for $n=0$.
Now show that the equation is true for $n+1$ whenever it is true for $n$.
We have
$$\eqalign{
\left(\sum_{k=0}^nr^k\right)+r^{n+1}&={1-r^{n+1}\over1-r}+r^{n+1}\cr
&={1-r^{n+1}\over1-r}+{(1-r)r^{n+1}\over 1-r}\cr
&={1-r^{n+2}\over1-r}
}$$
In addition we have
$$\left(\sum_{k=0}^nr^k\right)+r^{n+1}=\sum_{k=0}^{n+1}r^k$$
Hence
$$\sum_{k=0}^{n+1}r^k={1-r^{n+2}\over1-r}$$
and by induction the original identity is true, i.e.
$$\sum_{k=0}^nr^k={1-r^{n+1}\over1-r}$$
For $|r|<1$ we have
$$\lim_{n\rightarrow\infty}r^{n+1}=0$$
Consequently
$$\sum_{k=0}^\infty r^k={1\over1-r}$$

\vfill
\eject

\beginsection 3.

{\tt\obeylines
6.0000000000 < Circumference < 6.9282032303
6.2116570825 < Circumference < 6.4307806183
6.2652572266 < Circumference < 6.3193198842
6.2787004061 < Circumference < 6.2921724303
6.2820639018 < Circumference < 6.2854291993
6.2829049446 < Circumference < 6.2837461000
6.2831152158 < Circumference < 6.2833254941
6.2831677843 < Circumference < 6.2832203532
6.2831809265 < Circumference < 6.2831940686
6.2831842120 < Circumference < 6.2831874975
\bigskip
3.0000000000 < $\pi$ < 3.4641016151
3.1058285412 < $\pi$ < 3.2153903092
3.1326286133 < $\pi$ < 3.1596599421
3.1393502030 < $\pi$ < 3.1460862151
3.1410319509 < $\pi$ < 3.1427145996
3.1414524723 < $\pi$ < 3.1418730500
3.1415576079 < $\pi$ < 3.1416627471
3.1415838921 < $\pi$ < 3.1416101766
3.1415904632 < $\pi$ < 3.1415970343
3.1415921060 < $\pi$ < 3.1415937488
}
\end
