\beginsection 1.


\item{\S1.} Life of Wallis

\itemitem{1.} John Wallis was born in 1616 in England.
He was a smart kid and went off to college at the age of sixteen.
He graduated four years later with a BA then stayed on for another
two years to earn a Master's degree.
At that time mathematics was not an organized subject in
college so he studied to be a medical doctor.
However, when he graduated he entered the ministry instead.
He then left the ministry after two years when he got married
in 1645.
John Wallis was a rich man. In 1643 he inherited an
estate in Kent when his mother died.

\itemitem{2.} John Wallis essentially taught himself
mathematics.
In 1647 he obtained a book, {\it Clavis Mathematicae,} and worked
through the whole thing in the space of a couple of weeks.
He then proceeded to make many contributions to various
areas of mathematics.

\itemitem{3.}
In 1649, at the age of 33, he was appointed to the Savilian Chair
at Oxford,
which is a department head.
He held the job for over 50 years, until his death in 1703.
He was also appointed a second prestigious position
at Oxford in 1657.

\itemitem{4.}
Like Viete, Wallis was an expert cryptographer.
When the government intercepted a coded message from a foreign power,
they would bring it to Wallis to decrypt.
At that time people just came up with different methods to scramble
a text.
John Wallis had the insight to realize that a much more secure
method would be to use a standard algorithm in combination with a
secret key.

\item{\S2.} Wallis' Work

\itemitem{1.} In his book {\it Arithmetica Infinitorum,}
Wallis was the first to define conic sections as analytic curves.
For example, Wallis defines the parabola in terms of $x^2$.
This made the whole field of conic sections much more understandable.
Also in {\it Arithmetica Infinitorum,} Wallis defined
the meaning of fractional and negative exponents.
For example, he defined $x^{-1}=1/x$, $x^{1/2}=$ square root of $x$, etc.
It is interesting to realize that this simple notation is actually an invention,
and the inventor lived over 400 years ago.
Before Wallis mathematicians normally just thought of exponents as natural numbers.
Then folks begin to wonder, ``What does it mean if the exponent is a fraction or
a negative number?''
No one really knew for sure until Wallis
solved the problem.

\itemitem{2.}
John Wallis made many contributions in the area of calculus.
What Wallis was interested in was finding the area under a curve.
He figured out the solution for finding the area
under a curve when
the curve can be expressed as a sum of powers of $x$.
The solution that he found is the one we use today to
integrate polynomials, although the modern proof
differs from Wallis' proof.

\itemitem{3.}
Wallis hit upon his famous expression for $\pi$ while trying
to find the area under a hemisphere.
The Wallis product was the first analytic expression for $\pi$
that did not involve radicals.
That is the unique thing about the Wallis product, all of the
factors are rational numbers.

\itemitem{4.} John Wallis was as a founding member of the Royal Society.
It started out as just an informal club.
People who were interested in math and science just got together
and talked about things.

\itemitem{5.} Selected Publications\hfill\break
{\it De sectionibus conicis} (1655)
Wallis introduces the symbol $\infty$ to mean infinity.
\hfill\break
{\it Arithmetica Infinitorum} (1656)
The properties of rational powers are defined.\hfill\break
{\it Mathesis Universalis} (1657)\hfill\break
{\it Tractatus de Sectionibus Conicis} (1659)\hfill\break
{\it Mechanica, sive Tractatus de Motu} (1669--71, 3 parts)\hfill\break
{\it Treatise on Algebra} (1685)\hfill\break
{\it Opera Mathematica} (1693-99, 3 vols.) In volume 1, Wallis introduces
the term ``continued fraction.''

\item{\S 3.} References

\itemitem{1.} {\tt http://www.maths.tcd.ie/pub/HistMath/People/Wallis/RouseBall/RB\_Wallis.html}
\itemitem{2.} {\tt http://www.nndb.com/people/599/000087338/}
\itemitem{3.} {\tt http://www.1911encyclopedia.org/John\_Wallis}
\itemitem{4.} {\tt http://en.wikipedia.org/wiki/John\_wallis}
\itemitem{5.} {\tt http://www-history.mcs.st-andrews.ac.uk/Biographies/Wallis.html}
\itemitem{6.} {\tt http://www-history.mcs.st-andrews.ac.uk/HistTopics/Infinity.html\#s83}


\beginsection 2.

Use (3) repeatedly to evaluate $I_{16}$.
$$I_n={n-1\over n}I_{n-2}\eqno(3)$$
In addition,
$$\eqalign{
I_0&={\pi\over2}\cr
I_1&=1\cr
}$$

Solution:
$$\eqalign{
I_{16}&={15\over16}\cdot I_{14}\cr
&={15\over16}\cdot{13\over14}\cdot I_{12}\cr
&={15\over16}\cdot{13\over14}\cdot{11\over12}\cdot I_{10}\cr
&={15\over16}\cdot{13\over14}\cdot{11\over12}\cdot{9\over10}\cdot I_{8}\cr
&={15\over16}\cdot{13\over14}\cdot{11\over12}\cdot{9\over10}\cdot{7\over8}\cdot I_{6}\cr
&={15\over16}\cdot{13\over14}\cdot{11\over12}\cdot{9\over10}\cdot{7\over8}
\cdot{5\over6}\cdot I_{4}\cr
&={15\over16}\cdot{13\over14}\cdot{11\over12}\cdot{9\over10}\cdot{7\over8}
\cdot{5\over6}\cdot{3\over4}\cdot I_{2}\cr
&={15\over16}\cdot{13\over14}\cdot{11\over12}\cdot{9\over10}\cdot{7\over8}
\cdot{5\over6}\cdot{3\over4}\cdot{1\over2}\cdot I_{0}\cr
&={15\over16}\cdot{13\over14}\cdot{11\over12}\cdot{9\over10}\cdot{7\over8}
\cdot{5\over6}\cdot{3\over4}\cdot{1\over2}\cdot{\pi\over2}\cr
}$$

\beginsection 3.

$$\eqalign{
{I_2\over I_3}&=1.1781\cr
{I_4\over I_5}&=1.1045\cr
{I_6\over I_7}&=1.0738\cr
{I_8\over I_9}&=1.0570\cr
{I_{10}\over I_{11}}&=1.0464\cr
}$$

%CAS code for the above calculation...
%\medskip
%{\obeylines\tt
%f(x) = product(k,1,x/2,(2k-1)/(2k))
%g(x) = product(k,1,x/2,(2k)/(2k+1))
%h(x) = pi/2*f(x)/g(x)
%float(h(2))
%float(h(4))
%float(h(6))
%float(h(8))
%float(h(10))
%}

\beginsection 4. (c)

Use (9) and (3) to show that
$$1\le{I_{2n}\over I_{2n+1}}\le{2n+2\over2n+1}\eqno(10)$$

Solution: By (9) we have
$$I_{2n+1}\le I_{2n}$$
Hence
$$1\le{I_{2n}\over I_{2n+1}}$$
Again by (9) we have
$$I_{2n+2}\le I_{2n+1}$$
Hence
$$\left({2n+1\over2n+2}\right)I_{2n}\le I_{2n+1}$$
Rearrange...
$${I_{2n}\over I_{2n+1}}\le{2n+2\over2n+1}$$

\beginsection 4. (d)

Finally, use (10) to show that
$$\lim_{n\rightarrow\infty}{I_{2n}\over I_{2n+1}}=1$$

Solution: We have
$$\lim_{n\rightarrow\infty}{I_{2n}\over I_{2n+1}}=
\lim_{n\rightarrow\infty}{2n+2\over2n+1}
=\lim_{n\rightarrow\infty}\left({1\over1+1/(2n)}+{2\over2n+1}\right)=1$$

\beginsection 5.

$$2
\cdot{2\over1}
\cdot{2\over3}
\cdot{4\over3}
\cdot{4\over5}
\cdot{6\over5}
\cdot{6\over7}
\cdot{8\over7}
\cdot{8\over9}
\cdot{10\over9}
\cdot{10\over11}
\cdot{12\over11}
\cdot{12\over13}
={2097152\over693693}=3.02317
$$


\end