\parindent=0pt
Statistical Methods

\bigskip
5a) Mean = 20, variance = 16, sample size = 16.
\medskip
See lecture 34, slide 21.
$$\eqalign{
P\{\bar X>18.75\}&=P\{Z>(18.75-20)/\sqrt{16/16}\}\cr
&=P\{Z>-1.25\}\cr
&=0.1056
}$$

\bigskip
5b) The distribution of the ratio of sample mean and sample
variance is a $t$-distribution.
\medskip
$$\eqalign{
P\{\bar Y<10-0.62 S_Y\}&=P\{\bar Y-\mu<10-0.62 S_Y-10\}\cr
&=P\{t_8<-15.5/(25/\sqrt{9}\}\cr
&=P\{t_8<-1.86\}\cr
&=1-P\{t_8>1.86\}\cr
&=1-0.05\cr
&=0.95
}$$

\bigskip
5c) See p. 24 of Note03.
The distribution of sample variance is a chi-square distribution.
$$\eqalign{
P\{S_X^2>9.1168\}&=P\{(n-1)\times S_X^2/\sigma^2>(16-1)\times9.1168/16\}\cr
&=P\{\chi_{15}^2>8.5470\}\cr
&=0.90
}$$

\bigskip
5d) See p. 33 of Note03. The distribution of the ratio
of two sample variances is an $F$-distribution.
$$\eqalign{
P\{S_Y^2<6.25S_X^2\}&=P\{S_Y^2/S_X^2<6.25\}\cr
&=P\{F_{8,15}<6.25\times16/25\}\cr
&=P\{F_{8,15}<4\}
}$$
From table 9, $a=0.01$ but per lecture 34, slide 51 (case 2),
we use $1-a=0.99$.
$$P\{F_{8,15}<4\}=0.99$$

\bigskip
5e) The distribution of the ratio of sample mean and sample
variance is a $t$-distribution.
$$\eqalign{
P\{|4\bar X-80|<1.341 S_X\}&=1-2P\{4\bar X-80>1.341 S_X\}\cr
&=1-2P\{\bar X>20+0.3353 S_X\}\cr
&=1-2P\{\bar X-\mu>20+0.3353 S_X-20\}\cr
&=1-2P\{t_{15}>5.364/(16/\sqrt4)\}\cr
&=1-2P\{t_{15}>1.341\}\cr
&=1-2(0.1)\cr
&=0.8
}$$

\end
