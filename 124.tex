\item{3.} A certain experiment is to be performed until a successful result
is obtained.
The trials are independent and the probability of success on a single trial
is 0.25.
The cost of performing the experiment is \$25{,}000; however, if a failure
results, it costs \$5{,}000 to ``set-up'' for the next trial.

\bigskip
{\bf (a) What is the expected cost of the project?}

\bigskip
Solution: Here we are interested in the number of trials before a success
so it must be a geometric distribution.
The key word above is ``expected'' which means ``average'' or mean.
From lecture 35 slide 10 we have
$$\mu={1-p\over p}$$
In this case we have
$$\mu={1-0.25\over0.25}=3$$
So on average the experiment will fail 3 times.
That means the experiment must be done 4 times
so the expected cost is $4\times25{,}000+3\times5{,}000=115{,}000$
dollars.

\bigskip
{\bf (b) Suppose the experimenter has a maximum of \$500{,}000, what is the
probability that the experimental work would cost more than this amount?}

\bigskip
Solution: First we need to find $x$ such that
$$25{,}000(x+1)+5{,}000x=500{,}000$$
We have
$$x={475{,}000\over30{,}000}=15.83$$
Since $x$ is discrete, 16 failures or
more will break the budget.
Now we need to compute $P(X\ge16)$.
Fortunately, lecture 35 slide 16 gives us a distribution function
that makes it easy.
$$P(X>15)=(1-0.25)^{16}=0.0100$$

\end