\beginsection 1. Skin (dermal layer)

Separates and protects an organism from the environment.
Outer layer of skin is the epidermis.

\beginsection 2. Respiratory (pulmonary) system

Consists of the lungs and upper airways.
Primary function is the exchange of gasses.
Oxygen goes from air into blood.
Carbon dioxide goes from blood into air.
Pulmonary edema -- fluid in the lungs
Fibrosis -- lung tissue changed to scar tissue.
Emphysema -- destruction of alveoli

\beginsection 3. Digestive system

\beginsection 4. Circulatory system

\beginsection 5. Liver (hepatic system)

Helps maintain blood sugar levels.
Stores excess calories.
Converts foreign compounds (xenobiotics) into metabolites
by the process of biotransformation.

\beginsection 6. Kidneys and bladder (renal system)

Removes wastes from the body.
Regulates body fluid volume and blood pressure.

\beginsection What is homeostasis?

The process by which individual organs function together.

\beginsection What is a mutagen?

A chemical that causes changes to DNA.

\beginsection What is a carcinogen?

A chemical that causes cancer.

\beginsection What is a teratogen?

A chemical that causes birth defects.

\beginsection What is a ``target tissue''?

The tissue in the body that is most adversely affected by a chemical.

\beginsection What is edema?

Pulmonary edema is the accumulation of fluid in the lungs.

\beginsection What is irritation?

Reddening of the skin, often with some swelling.

\beginsection What does LD50 mean?

LD50 is the dose that kills 50\% of the animals tested.
A lower LD50 means the chemical is more toxic.

\beginsection What does LC50 mean?

The concentration of a chemical that kills 50% of
the test organisms in 24 hours.
LC50 is used instead of LD50 for gaseous chemicals
and chemicals that are toxic to aquatic animals.





\end
