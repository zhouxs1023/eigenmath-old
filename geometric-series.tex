\section{Details}
\subsection{User-defined symbols}

A geometric series converges according to the formula
$$\sum_{k=0}^\infty a^k={1\over1-a},\qquad|a|<1$$
If we use $a=-1/2$ and for practical purposes only count up to nine instead of infinity,
we should have
$$\sum_{k=0}^9\left(-{1\over2}\right)^k\approx{2\over3}$$
The above calculation can be done in one line of code using the $sum$ function.

\medskip
\verb$sum(k,0,9,(-0.5)^k)$
$$0.666016$$

\medskip
\noindent
The following example uses an intermediate variable.

\medskip
\verb$f=sum(k,0,9,a^k)$

\verb$f$
$$f=1+a+a^2+a^3+a^4+a^5+a^6+a^7+a^8+a^9$$

\verb$eval(f,a,-1/2)$
$$341\over512$$

\verb$float(last)$
$$0.666016$$

\medskip
\noindent
As seen on the first line, no result is printed when a symbol is defined.
When you do in fact want to see the value of a symbol,
just enter it as shown on the second line.

\medskip
\noindent
When a result is displayed, it is also stored in the symbol $last$.

\subsection{User-defined functions}

\noindent
The following example shows how to define a function.

\medskip
\verb$f(a)=sum(k,0,9,a^k)$

\verb$f(-1/2)$
$$341\over512$$

\verb$f(-0.5)$
$$0.666016$$

