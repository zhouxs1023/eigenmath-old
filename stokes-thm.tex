
\newpage

\noindent
Stokes' theorem identifies a special equivalence of line and surface
integrals.

\bigskip
\noindent
$\displaystyle{\oint_C P\,dx+Q\,dy+R\,dz}$
$$=
\int\!\!\!\int_S
\left({\partial Q\over\partial x}-{\partial P\over\partial y}\right)\,dx\,dy
+
\left({\partial R\over\partial y}-{\partial Q\over\partial z}\right)\,dy\,dz
+
\left({\partial P\over\partial z}-{\partial R\over\partial x}\right)\,dz\,dx
$$

\noindent
Curve $C$ is the perimeter around $S$.
The theorem can be also be written as
$$\oint_C {\bf F}\cdot d{\bf r}
=\int\!\!\!\int_S\mathop{\rm curl}{\bf F}\cdot{\bf n}\,dS
$$
where ${\bf F}=(P,Q,R)$ and $d{\bf r}=(dx,dy,dz)$.
In many cases, converting an integral according to
Stokes' theorem can turn a difficult problem into an easy one.

