{\it George Weigt -- Geometry Homework \#3}

\beginsection Problem 6.

Find the coordinates in $\cal H$ of $(2,3)$ (i) with respect to the
line $(x-1)^2+y^2=10$; (ii) with respect to the line $x=2$.

\medskip\noindent
(i) By inspection we have $c=1$ and $r=\sqrt{10}$ hence the line
is the type II line ${}_1L_{\sqrt{10}}$.
Using the standard ruler for this line we have
$$f(2,3)=\ln\left({2-1+\sqrt{10}\over 3}\right)
\approx0.32745$$
(ii) The line $x=2$ corresponds to the type I line ${}_2L$.
Note that $(2,3)$ is indeed on this line.
We have
$$f(2,3)=\ln3\approx1.09861$$

\beginsection Problem 11.

Find a point $P$ on the line ${}_{-3}L_{\sqrt7}$ in the Poincare Plane
whose coordinate is $\ln2$.
\medskip
Solution: We have $c=-3$, $r=\sqrt7$ and $t=\ln2$. Hence
$$x_0=r\tanh(t)+c=\sqrt7\left({2-1/2\over2+1/2}\right)-3={3\sqrt7\over5}-3$$
$$y_0=r\mathop{\rm sech}(t)=\sqrt7\left({2\over2+1/2}\right)={4\sqrt7\over5}$$
Therefore $P=(3\sqrt7/5-3,4\sqrt7/5)$.

\beginsection Problem 1.

In the Euclidean Plane find a ruler $f$ with $f(P)=0$ and $f(Q)>0$ for
the given pair
\item{i.} $P=(2,3)$, $Q=(2,-5)$
\item{ii.} $P=(2,3)$, $Q=(4,0)$.

\medskip\noindent
(i) We have $x_1=x_2$ so this is a vertical line and the standard ruler is
$$f(a,y)=y$$
We have $f(P)=3$, $f(Q)=-5$.
Since $f(Q)<f(P)$ we have $\varepsilon=-1$.
Therefore the ruler we seek is
$$f(a,y)=-y+3$$
(ii) We have
$$m={y_2-y_1\over x_2-x_1}={0-3\over4-2}=-{3\over2}$$
$$b=y_2-mx_2=0+{3\over2}\cdot4=6$$
The standard ruler is
$$f(x,y)=x\sqrt{1+m^2}=x\sqrt{13}/2$$
Hence
$$f(P)=f(2,3)=2\sqrt{13}/2=\sqrt{13}$$
$$f(Q)=f(4,0)=4\sqrt{13}/2=2\sqrt{13}$$
The ruler we seek is
$$f(x,y)=x\sqrt{13}/2-\sqrt{13}$$

\beginsection Problem 2.

In the Poincare Plane find a ruler $f$ with $f(P)=0$ and $f(Q)>0$ for
the given pair
\item{i.} $P=(2,3)$, $Q=(2,1)$
\item{ii.} $P=(2,3)$, $Q=(-1,6)$.

\medskip\noindent
(i)
$$f(a,y)=\ln y,\qquad f(P)=\ln3,\qquad f(Q)=\ln1=0$$
The ruler we seek is
$$f(a,y)=\ln3-\ln y=\ln{3\over y}$$
(ii) For $x_1=2$, $y_1=3$, $x_2=-1$, and $y_2=6$ we have
$$c={y_2^2-y_1^2+x_2^2-x_1^2\over2(x_2-x_1)}
={36-9+1-4\over2(-1-2)}=-4$$
$$r=\sqrt{(x_1-c)^2+y_1^2}=\sqrt{36+9}=3\sqrt{5}$$
Using the standard ruler
$$f(P)=\ln\left({x_1-c+r\over y_1}\right)
=\ln\left({2+4+3\sqrt5\over3}\right)
=\ln(2+\sqrt5)\approx1.4436$$
$$f(Q)=\ln\left({x_1-c+r\over y_1}\right)
=\ln\left({-1+4+3\sqrt5\over6}\right)
=\ln\left({1+\sqrt5\over2}\right)\approx0.4812$$
The ruler we seek is
$$f(x,y)=\ln(2+\sqrt5)-\ln\left({x+4+3\sqrt5\over y}\right)
%=\ln\left[{y(2+\sqrt5)\over x+4+3\sqrt5}\right]
$$

\beginsection Problem 4.

Let $P$ and $Q$ be points in a metric geometry.
Show that there is a point $M$ such that $M\in\overleftarrow P\overrightarrow Q$
and $d(P,M)=d(M,Q)$.
\medskip
Solution: In a metric geometry there exists a ruler $f$.
Let $t\in R$ such that
$$f(P)-f(Q)=t\eqno(1)$$
Because $f$ is surjective and $t/2\in R$, there must be a point $M$ on
the line $\overleftarrow P\overrightarrow Q$
such that
$$f(P)-f(M)=t/2\eqno(2)$$
Subtract equation (2) from equation (1) and obtain
$$f(M)-f(Q)=t/2$$
Hence
$$f(P)-f(M)=f(M)-f(Q)$$
Take the absolute value of both sides to obtain
$$|f(P)-f(M)|=|f(M)-f(Q)|$$
Therefore by the ruler equation we have
$$d(P,M)=d(M,Q)$$

\beginsection Problem 5.

Prove that a line in a metric geometry has infinitely many points.
\medskip
Solution: In a metric geometry there exists a ruler $f$
for every line.
By the bijective property of $f$ there is a one-to-one
correspondence between the set of points on a line
and the set of real numbers.
Since there are an infinite number of reals, there
must be an infinite number of points on every line.

\beginsection Problem 5.

Show that the ruler in Proposition 3.1.4 is a coordinate system with $A$
as origin and $B$ positive.
\medskip
Solution: The parametric equation for the line is
$$X=A+t(B-A)$$
For $X=A$ we have $t=0$.
Then from the ruler in Proposition 3.1.4 we have
$$t\,\|A-B\|=0\,\|A-B\|=0$$
For $X=B$ we have $t=1$ hence
$$t\,\|A-B\|=1\,\|A-B\|>0$$

\beginsection Problem 7.

Prove that the Taxicab distance $d_T$ satisfies the triangle inequality.
\medskip
Solution: The Taxicab distance is
$$d_T(P,Q)=|x_1-x_2|+|y_1-y_2|$$
Let $A=(x_1,y_1)$, $B=(x_2,y_2)$, and $C=(x_3,y_3)$.
A basic property of absolute value is that $|a+b|\le|a|+|b|$.
Hence we have
$$\eqalign{
|x_1-x_3|\le|x_1-x_2|+|x_2-x_3|\cr
|y_1-y_3|\le|y_1-y_2|+|y_2-y_3|\cr
}$$
Adding the above equations we have
$$|x_1-x_3|+|y_1-y_3|\le|x_1-x_2|+|y_1-y_2|+|x_2-x_3|+|y_2-y_3|$$
Therefore
$$d_T(A,C)\le d_T(A,B)+d_T(B,C)$$

\beginsection Problem 9.

Define a function $d_F$ for points $P$ and $Q$ in $R^2$ by
$$d_F(P,Q)=\cases{
0 & if $P=Q$\cr
d_E(P,Q) & if $L_{PQ}$ is not vertical\cr
3d_E(P,Q) & if $L_{PQ}$ is vertical\cr
}$$
{\bf a.} Prove that $d_F$ is a distance function on $R^2$
and that $\{R^2,{\cal L}_E,d_F\}$ is a metric geometry.

\medskip
Solution: It is already known that $d_E$ is a distance function
and has a ruler.
Therefore we shall only check $d_F$ for the case of vertical lines.

\medskip
First, verify that $3d_E$
satisfies the conditions for a distance function.
\item{(i)} $3d_E(P,Q)\ge0$.
We have $d_E(P,Q)\ge0$. Multiply both sides by three to obtain
$3d_E(P,Q)\ge0$.

\item{(ii)} $3d_E(P,Q)=0$ if and only if $P=Q$.
If $P=Q$ then $P$ and $Q$ are not on any line.
That is why there is the explicit $P=Q$ case in the definition of $d_F$.
Hence by definition of $d_F$, $P=Q$ implies $d_F(P,Q)=0$.
Now let us consider the converse.
If $P$ and $Q$ are on a vertical line then $P\ne Q$ therefore we must have
$3d_E(P,Q)\ne0$.
This condition is ensured because it is already true that $d_E(P,Q)\ge0$ when $P\ne Q$.
Multiply both sides by three to obtain $3d_E(P,Q)\ge0$.

\item{(iii)} $3d_E(P,Q)=3d_E(Q,P)$.
It is already true that $d_E(P,Q)=d_E(Q,P)$. Multiply both sides
by three to obtain $3d_E(P,Q)=3d_E(Q,P)$

\medskip
To show that this is a metric geometry, we need to show that every vertical
line has a ruler.
Let $f(a,y)=3y$.
Let $t\in R$ be an arbitrary real number.
We need to find a point $P=(x_1,y_1)$ such that $f(P)=t$.
The solution is $P(x_1,t/3)$
hence $f$ is surjective.
Next, for vertical lines we have
$$3d_E(P,Q)=3\sqrt{(y_1-y_2)^2}=3|y_1-y_2|$$
$$|f(P)-f(Q)|=|3y_1-3y_2|=3|y_1-y_2|=3d_E(P,Q)$$
Hence $f(a,y)=3y$ preserves the distance between points.
Since $f$ is surjective and preserves the distance between points, it is a ruler.

\medskip
Therefore, since $d_F$ is a distance function and a ruler exists,
we have $\{R^2,{\cal L}_E,d_F\}$ is a metric geometry.

\bigskip\noindent
{\bf b.} Prove that the triangle inequality is not satisfied for this distance, $d_F$.
\medskip
Solution: Let $A=(0,0)$, $B=(1,0)$, and $C=(0,1)$.
Then
$$\eqalign{
d(A,B)&=1\cr
d(B,C)&=\sqrt2\cr
d(A,C)&=3\cr
}$$
We have $3>1+\sqrt2$ hence
$$d(A,C)\not\le d(A,B)+d(B,C)$$
\end