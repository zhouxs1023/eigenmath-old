\item{\S1.} John Machin (1680-1751)

\itemitem{1.} John Machin was employed as a tutor for Brook Taylor in 1701.
This is the Taylor of Taylor series fame.
The two corresponded frequently during their careers.

\itemitem{2.} In 1713 John Machin was appointed professor of astronomy at Gresham College
in London.
He eventually became the chair of the department and held this post until his death in 1751.

\itemitem{3.} Machin was elected a fellow of the Royal Society in 1710.
He served as Secretary of the Royal Society from 1718--1747.

\itemitem{4.} He was a member of the Royal Society commitee that
attempted to resolve the calculus priority dispute between Leibniz and Newton in 1712.

\item{\S2.} His work.

\itemitem{1.} Devised a mathematical series that quickly converges to $\pi$.
He was able to calculate $\pi$ to 100 digits.
According to Wikipedia, Machin's formula for $\pi$ was not improved upon
until the computer era.

\itemitem{2.} He worked out a solution to Kepler's equations for the motion of the Moon.
Netwon included Machin's solution in the third edition of {\it Principia.}

\itemitem{3.} Selected publications:

\itemitem{} {\it The solution of Kepler's problem} published in
{\it Philosophical Proceedings of the Royal Society} in 1738.

\itemitem{} {\it Quadrature of the Circle} published as an appendix
in a book by Maseres in 1758.

\item{\S3.} References
\itemitem{1.} {\it John Machin,} Wikipedia, {\tt http://en.wikipedia.org/wiki/John\_Machin}
\itemitem{2.} {\tt http://www-history.mcs.st-andrews.ac.uk/Biographies/Machin.html}
\itemitem{3.} {\it Lecture 10: Machin and $\pi$,} {\tt chan.hei-chi@uis.edu}

\beginsection Lecture 10, Exercise 2

Prove the addition formula for tangent
$$\tan(x+y)={\tan x + \tan y\over1-\tan x \tan y}$$

\bigskip
Solution: Use the following identities
$$\eqalign{
\sin(x+y)&=\sin x\cos y +\cos x\sin y\cr
\cos(x+y)&=\cos x\cos y - \sin x\sin y\cr
}$$
Hence
$$\tan(x+y)={\sin(x+y)\over\cos(x+y)}
={\sin x\cos y+\cos x\sin y\over\cos x\cos y-\sin x\sin y}
$$
Divide everything by $\cos x\cos y$.
$${
{\sin x\cos y\over\cos x\cos y}
+
{\cos x\sin y\over\cos x\cos y}
\over
{\cos x\cos y\over\cos x\cos y}
-
{\sin x\sin y\over\cos x\cos y}
}
=
{\tan x+\tan y\over1-\tan x\tan y}$$

\beginsection Lecture 10, Exercise 3

Prove (4)
$$\tan4\beta={120\over119}$$
where $\beta=1/5$.

\bigskip
Given $\tan2\beta=5/12$ we have
$$\tan4\beta={2\tan2\beta\over1-\tan^22\beta}
={10/12\over1-25/144}={120/144\over119/144}={120\over119}
$$

\beginsection
Lecture 10, Exercise 4

Derive
$$\tan(4\beta-\pi/4)={\tan4\beta-\tan(\pi/4)\over1+\tan4\beta\tan(\pi/4)}={1\over239}$$

\bigskip
Solution: From the previous exercise we have $\tan4\beta=120/119$.
The angle $\pi/4$ is a $45^\circ$ angle.
For that angle the rise equals the run therefore $\tan(\pi/4)=\hbox{rise/run}=1$.
Hence
$$\tan(4\beta-\pi/4)={\tan4\beta-\tan(\pi/4)\over1+\tan4\beta\tan(\pi/4)}
={120/119-1\over1+(120/119)(1)}
={1/119\over239/119}
={1\over239}$$

\beginsection Lecture 10, Exercise 6

Prove
$${\pi\over4}=\arctan\left(1\over2\right)+\arctan\left(1\over3\right)$$

\bigskip
Solution: Let
$$\beta=\arctan\left({1\over2}\right)$$
Then $\tan\beta=1/2$ and
$$\tan(\pi/4-\beta)={\tan(\pi/4)-\tan\beta\over1+\tan(\pi/4)\tan\beta}
={1-1/2\over1+(1)(1/2)}={1/2\over3/2}={1\over3}$$
Hence
$$\pi/4-\beta=\arctan\left({1\over3}\right)$$
Substituting for $\beta$ we have
$${\pi\over4}-\arctan\left({1\over2}\right)=\arctan\left({1\over3}\right)$$

\beginsection Lecture 10, Exercise 7

Solution:
$$\pi\approx{76528487109180192540976\over24359780855939418203125}\approx3.14159$$

\end