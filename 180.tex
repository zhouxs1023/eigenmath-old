\noindent
{\it 180.tex Notes for exam \#1}

\beginsection 1.1 Abstract Geometry

An abstract geometry is a set of points $\cal S$ and a set of lines $\cal L$.
Each line is a subset of $\cal S$.
For every two points $A,B\in\cal S$ there must be a line
$\ell\in\cal L$ such that $A,B\in\ell$. Every line has at
least two points.

\beginsection 1.2 Cartesian Plane

${\cal L}_E$ is the set of all vertical and non-vertical lines.
$$\eqalign{
L_a&=\{(x,y)\in R^2\mid x=a\}\cr
L_{m,b}&=\{(x,y)\in R^2\mid y=mx+b\},
\qquad m={y_2-y_1\over x_2-x_1},\qquad b=y_2-mx_2\cr
}$$

\beginsection 1.3 Poincare Plane

${\cal L}_H$ is the set of all type I and type II lines.
$$\eqalign{
{}_aL&=\{(x,y)\in H\mid x=a\}\cr
{}_cL_r&=\{(x,y)\in H\mid(x-c)^2+y^2=r^2\},
\qquad c={y_2^2-y_1^2+x_2^2-x_1^2\over 2(x_2-x_1)},
\qquad r=\sqrt{(x_1-c)^2+y_1^2}\cr
}$$

\beginsection 1.4 Incidence Geometry

An abstract geometry is an incidence geometry if
\item{i.} Every two points in $\cal S$ lie on a unique line.
\item{ii.} There are at least three points that are not collinear.

\beginsection 1.5 Parallel Lines

In an abstract geometry, $\ell_1$ and $\ell_2$ are parallel if
either $\ell_1=\ell_2$ or $\ell_1\cap\ell_2=\emptyset$.

\beginsection 1.6 Theorem

In an incidence geometry, if $\ell_1\cap\ell_2$ has two or more
points, then $\ell_1=\ell_2$.

\medskip
\noindent{\it Proof.}
Let $P,Q\in\ell_1\cap\ell_2$ and $P\ne Q$.
In an incidence geometry, two points determine a unique line.
Therefore $\ell_1=\ell_2$.

\beginsection 1.7 Corollary

In an incidence geometry, two lines are either parallel, or they intersect
in exactly one point.

\beginsection 2.1 Surjective

A function $f$ is surjective if for every $t\in T$ there is an $s\in S$ such that $f(s)=t$.
In other words, $Image(f)=Range(f)$.

\medskip\noindent
To prove, let $t$ be an arbitrary $t\in R$. Then find $g$ such that $f(g(t))=t$.

\beginsection 2.2 Injective

A function is injective (one-to-one) if the condition $f(s_1)=f(s_2)$ implies that $s_1=s_2$.

\medskip\noindent
To prove, let $f(s_1)=f(s_2)$ then show that $s_1=s_2$.

\beginsection 2.3 Distance Function

A function $d$ is a distance function if for every $P,Q\in\cal S$ we have
\item{i.} $d(P,Q)\ge 0$
\item{ii.} $d(P,Q)=0$ if and only if $P=Q$
\item{iii.} $d(P,Q)=d(Q,P)$

\beginsection 2.4 Ruler

A function $f:\ell\rightarrow R$ is a ruler (coordinate system) if
\item{i.} $f$ is a bijection
\item{ii.} for each $P,Q\in\ell$ the following Ruler Equation is true
$$|f(P)-f(Q)|=d(P,Q)$$

\beginsection 2.5 Lemma

In an incidence geometry, all we have to prove is that $f$ is surjective and
satisfies the Ruler Equation.

\beginsection 2.6 Metric Geometry

A metric geometry is an incidence geometry in which every line has a ruler.

\beginsection 3.1 Hyperbolic Functions

$$\sinh t={e^t-e^{-t}\over2},\qquad \cosh t={e^t+e^{-t}\over2}$$
To remember which is which, recall that
$$\sinh0=0,\qquad\cosh0=1$$

\beginsection 3.2 Poincare Plane

This is how to convert the result of a standard ruler back to a point. Given
$$t=\ln\left({x_0-c+r\over y_0}\right)$$
we have
$$x_0=r\tanh t+c,\qquad y_0=r\mathop{\hbox{sech}}t$$

\beginsection 3.3 Standard Ruler, Euclidean Plane

$$\eqalign{
f(a,y)&=y\cr
f(x,y)&=x\sqrt{1+m^2}\cr
}$$

\beginsection 3.4 Standard Ruler, Taxicab Plane

$$\eqalign{
f(a,y)&=y\cr
f(x,y)&=x(1+|m|)\cr
}$$

\beginsection 3.5 Standard Ruler, Poincare Plane

$$\eqalign{
f(a,y)&=\ln y\cr
f(x,y)&=\ln\left({x-c+r\over y}\right)\cr
}$$

\beginsection 3.6 Special Coordinate Systems

{\it Theorem.}
A ruler can be translated and/or reflected and the result is still a ruler.
$$g(P)=\pm(f(P)-a)$$
\medskip\noindent
{\it Theorem (Ruler Placement Theorem).}
Let $P,Q\in\ell$. Then there is a ruler $g$ with $g(P)=0$ and $g(Q)>0$.
$$g(X)=\cases{
f(X)-f(P),& for $f(Q)>f(P)$\cr
f(P)-f(X),& for $f(Q)<f(P)$\cr
}$$

\beginsection 3.7 Cartesian Plane Redux

\item{i.} $A+B=(x_1+x_2,y_1+y_2)$
\item{ii.} $rA=(rx_1,rx_2)$
\item{iii.} $A-B=(x_1-x_2,y_1-y_2)$
\item{iv.} $\langle A,B\rangle=x_1x_2+y_1y_2$
\item{v.} $\|A\|=\sqrt{\langle A,A\rangle}$

\medskip\noindent
Hence
$$d_E(P,Q)=\|P-Q\|$$

\medskip\noindent
The line through $A$ and $B$ can be described parametrically as
$$L_{A,B}=\{X\in R^2\mid X=A+t(B-A)\hbox{\ for some $t\in R$}\}$$

\medskip\noindent
Alternate ruler, not necessarily the standard ruler.
$$f(C)=f(A+t(B-A))=t\|A-B\|$$

\beginsection 3.8 Triangle Inequality

$$d(A,C)\le d(A,B)+d(B,C)$$

\beginsection 3.9 Betweeness

$A{-}B{-}C$ if and only if $A$, $B$, $C$ are collinear and $d(A,C)=d(A,B)+d(B,C)$.

\beginsection 4.1 Definition

For $x,y,z\in R$, $y$ is between $x$ and $z$, written $x*y*z$,
if either $x<y<z$ or $z<y<x$.

\beginsection 4.2 Theorem

$A{-}B{-}C$ if and only if $f(A)*f(B)*f(C)$.

\beginsection 4.3 Corollary

Given three distinct points on a line, only one is between the other two.

\beginsection 4.4 Proposition

In the Euclidean Plane, we have $A{-}B{-}C$ if and only if there is
a $t\in R$ such that $0<t<1$ and $B=A+t(C-A)$.

\beginsection 4.5 Definition

$A{-}B{-}C{-}D$ implies the following.
\item{i.} $A{-}B{-}C$
\item{ii.} $A{-}B{-}D$
\item{iii.} $A{-}C{-}D$
\item{iv.} $B{-}C{-}D$

\beginsection 4.6 Proposition

If $A{-}B{-}C{-}D$ then the points are collinear.

\beginsection 4.7 Proposition

$A{-}B{-}C$ and $B{-}C{-}D$ imply $A{-}B{-}C{-}D$.

\beginsection 4.8 Proposition

Suppose we have $A{-}B{-}C$ and $A{-}B{-}D$.
Then either $A{-}B{-}C{-}D$ or $A{-}B{-}D{-}C$.

\beginsection 4.9 Line Segment

$$\overline{AB}=\{C\in{\cal S}\mid A{-}C{-}B\hbox{, or $C=A$ or $C=B$}\}$$

\beginsection 4.10 Try Out

A function is strictly increasing if its derivative is strictly positive.

\beginsection 4.11 Definition

Let $X$ be a subset of a metric geometry.
A point $P\in X$ is called a passing point of $X$ if there are $A,B\in X$
such that $A{-}P{-}B$.
Otherwise, $P$ is called an extreme point of $X$.

\beginsection 4.12 Theorem

For the line segment $\overline{AB}$, the extreme points are $A$ and $B$.
All other points on $\overline{AB}$ are passing points.

\beginsection 4.13 Corollary

If $\overline{AB}=\overline{CD}$, then either $A=C$, $B=D$, or $A=D$, $B=C$.

\beginsection 4.14 Definition

$A$ and $B$ are the endpoints (pr vertices) of the line segment $\overline{AB}$.
The length of $\overline{AB}$ is $AB=d(A,B)$.

\beginsection 4.15 Lemma

If $A$ and $B$ are distinct points in a metric geometry, then there is a point
$C$ such that $A{-}B{-}C$.

\beginsection 4.16 Definition

$$\overrightarrow{AB}=\overline{AB}\cup\{C\in S\mid A{-}B{-}C\}$$

\beginsection 4.17 Theorem

In a metric geometry
\item{* i.} if $C\in\overrightarrow{AB}$ and $C\ne A$, then $\overrightarrow{AC}=\overrightarrow{AB}$.
\item{ii.} If $\overrightarrow{AB}=\overrightarrow{CD}$, then $A=C$.

\beginsection 4.18 Definition

The point $A$ is called the vertex, or initial point of the ray $\overrightarrow{AB}$.

\vfill
\eject

\noindent
{\bf Hints for solving problems}

\medskip\noindent
1. To show that something is a subset of something else, use the arbitrary member trick.

\medskip\noindent
2. To show the equivalence of sets, use the arbitrary member trick in both directions.

\medskip\noindent
3. Consider using proof by contradiction in cases involving rays.

\medskip\noindent
4. In the Poincare Plane,
$$f^{-1}(t)=(c+r\tanh t, r\mathop{\hbox{sech}}t)$$

\medskip\noindent
5. In the Euclidean Plane,
$$\eqalign{
\overline{AB}&=\{C\in R^2\mid C=A+t(B-A)\hbox{\ \ for some $t$ with $0\le t\le1$}\}\cr
\overrightarrow{AB}&=\{C\in R^2\mid C=A+t(B-A)\hbox{\ \ for some $t\ge 0$}\}\cr
}$$

\medskip\noindent
6. In the Euclidean Plane,
$$d_E=\sqrt{(x_1-x_2)^2+(y_1-y_2)^2}$$


\end