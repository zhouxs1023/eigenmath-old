\magnification=1200

\noindent
{\it George Weigt -- Advanced Calculus Homework \#6}

\beginsection 1. Kaplan, p. 234, problem 1a.

Evaluate the following integral.
$$\int\!\!\!\int_R (x^2+y^2)\,dx\,dy$$
where $R$ is the triangle with vertices $(0,0)$, $(1,0)$, $(1,1)$.

\medskip
\hrule

\bigskip
\noindent
From the picture on p. 94 of Lecture 06, we have
$$\eqalign{
0&\le x\le 1\cr
0&\le y\le x\cr
}$$
Therefore
$$\eqalign{
\int_0^1 dx\int_0^x dy\,(x^2+y^2)
&=\int_0^1 dx\left[\left(x^2y+{1\over3}y^3\right)\bigg|_{y=0}^{y=x}\right]\cr
&=\int_0^1 dx\,{4\over3}x^3\cr
&={1\over3}x^4\bigg|_0^1\cr
&={1\over3}\cr
}$$

\vfill
\eject

\beginsection 2. Kaplan, p. 234, problem 1b.

Evaluate the following integral.
$$\int\!\!\!\int\!\!\!\int_Ru^2v^2w\,du\,dv\,dw$$
where $R$ is the region $u^2+v^2\le1$, $0\le w\le1$.

\medskip
\hrule

\bigskip
\noindent
%From the picture on p. 95 we have
%$$\eqalign{
%-1&\le u\le 1\cr
%-\sqrt{1-u^2}&\le v\le\sqrt{1-u^2}\cr
%0&\le w\le1\cr
%}$$
It turns out to be more convenient to integrate over the circle
using polar coordinates.
We have
$u=r\cos\theta$,
$v=r\sin\theta$ and
$$
u^2v^2
=r^4\cos^2\theta\sin^2\theta
={1\over4}r^4(1+\cos2\theta)(1-\cos2\theta)
={1\over4}r^4(1-\cos^22\theta)
={1\over4}r^4\sin^22\theta
$$
$$du\,dv=r\,dr\,d\theta$$
Hence the original integral becomes
$$\int_0^{2\pi}d\theta\int_0^1dr\int_0^1dw\left({1\over4}wr^5\sin^22\theta\right)$$
%
Let us do the integrals one at a time, first over $w$.
$$\int_0^1{1\over4}wr^5\sin^22\theta\,dw={1\over8}w^2r^4\sin^22\theta\bigg|_0^1
={1\over8}r^5\sin^22\theta$$
%
Next, over $r$.
$$\int_0^1{1\over8}r^5\sin^22\theta\,dr={1\over48}r^6\sin^22\theta\bigg|_0^1
={1\over48}\sin^22\theta$$
Finally, over $\theta$.
$$\int_0^{2\pi}{1\over48}\sin^22\theta\,d\theta
=\left({\theta\over96}-{\sin4\theta\over384}\right)\bigg|_0^{2\pi}
={\pi\over48}
$$

\bigskip\noindent
Trivia: Using the original formula, we obtain the following by integrating over
$v$ and $w$.
$$\int_{-1}^1{1\over3}u^2(1-u^2)^{3/2}\,du$$
This is a difficult integral but a certain web-based tool ({\tt integrals.wolfram.com})
can solve it.
Then we have
$$\left({u\sqrt{1-u^2}(-8u^4+14u^2-3)\over144}+{\sin^{-1}u\over48}\right)\bigg|_{-1}^1
={\pi\over48}$$

\vfill
\eject
\beginsection 3. Kaplan, p. 234, problem 1c.

Evaluate the following integral.
$$\int\!\!\!\int_R r^3 \cos\theta\,dr\,d\theta$$
where $R$ is the region $1\le r\le2$, ${\pi\over4}\le\theta\le\pi$.

\medskip
\hrule

\bigskip
\noindent
Let us do the integrals one at a time. First over $\theta$.
$$\int_{\textstyle{\pi\over4}}^\pi r^3\cos\theta\,d\theta
=r^3\sin\theta\bigg|_{\textstyle{\pi\over4}}^\pi
=-{r^3\over\sqrt2}$$
Then over $r$.
$$\int_1^2 -{r^3\over\sqrt2}\,dr=-{r^4\over4\sqrt2}\bigg|_1^2
=-{15\over4\sqrt2}$$

\vfill
\eject
\beginsection 4. Kaplan, p. 235, problem 5a.

In the following integral, find the region $R$ and then interchange the order of integration.
$$\int_{1/2}^1\int_0^{1-x}f(x,y)\,dy\,dx$$

\hrule
\bigskip

\noindent
$$\matrix {
x & y=1-x\cr
\cr
0.5 & 0.5\cr
0.6 & 0.4\cr
0.7 & 0.3\cr
0.8 & 0.2\cr
0.9 & 0.1\cr
1.0 & 0.0\cr
}$$
Therefore
$$\int_{1/2}^1\int_0^{1-x}f(x,y)\,dy\,dx=\int_0^{1/2}\int_{1/2}^{1-y}f(x,y)\,dx\,dy$$

\vfill
\eject
\beginsection 5.

Evaluate
$$\int\!\!\!\int_D\lfloor x+y\rfloor\,dx\,dy$$
where $D=\{0\le x\le 1, 0\le y\le 1\}$.

\medskip
\hrule
\bigskip

\noindent
The integrand is 1 when
$$1\le x+y<2$$
and zero otherwise. Hence
$$0\le x\le 1,\qquad 1-x\le y\le1$$
We do not have to worry about the extreme point $(1,1)$ because even though
$\lfloor x+y\rfloor=2$
at that point, the area under a single point is zero.
Therefore
$$\int\!\!\!\int_D\lfloor x+y\rfloor\,dx\,dy
=\int_0^1 dx \int_{1-x}^1 1\,dy={1\over2}$$
%This is what we expect because it is one half the area of a unit square.


\vfill
\eject
\beginsection 6.

Evaluate
$$\int\!\!\!\int\!\!\!\int_V \,dx\,dy\,dz$$
where $V$ is the solid ball centered at $(1,2,3)$ with radius 4.

\medskip
\hrule
\bigskip

\noindent
Since the integrand does not depend on $x$, $y$ or $z$
it does not really matter
where the ball is centered.
Hence the integral is just the volume of the ball.
$${4\over3}\pi r^3={256\over3}\pi$$

\vfill
\eject
\beginsection 7.

Evaluate
$$\int\!\!\!\int\!\!\!\int_V z\,dx\,dy\,dz$$
where $V$ is the solid box with corners at
$(0,0,0)$,
$(1,0,0)$,
$(1,1,0)$,
$(0,1,0)$,
$(0,0,1)$,
$(1,0,1)$,
$(1,1,1)$, and
$(0,1,1)$.

\medskip
\hrule
\bigskip

\noindent
We have
$$\int_0^1 dy\int_0^1 dx=1$$
Hence
$$\int\!\!\!\int\!\!\!\int_V \,dx\,dy\,dz
=\int_0^1 z\,dz={1\over2}z^2\bigg|_0^1={1\over2}$$

\vfill
\eject
\beginsection 8.

Evaluate
$$\int\!\!\!\int\!\!\!\int_Vz\,dx\,dy\,dz$$
where $V$ is the solid ball centered at the origin with radius 1.

\medskip
\hrule
\bigskip

\noindent
The result is zero since $z$ is antisymmetrical across the $xy$ plane, i.e.
$$\int_{-a}^az\,dz={z^2\over2}\bigg|_{-a}^a=0$$
Therefore the top and bottom hemispheres cancel out.

\vfill
\eject
\beginsection 9.

Evaluate
$$\int\!\!\!\int\!\!\!\int_V z\,dx\,dy\,dz$$
where $V$ is the solid tetrahedron bounded by the four planes
$x=0$, $y=0$, $z=0$, and $x+y+z=2$.

\medskip
\hrule
\bigskip

\noindent
If we let $z$ range from 0 to 2, then $y$ and $x$ are constrained as follows.
$$\eqalign{
0&\le z\le 2\cr
0&\le y\le 2-z\cr
0&\le x\le 2-z-y\cr
}$$
Hence
$$\int\!\!\!\int\!\!\!\int_V z\,dx\,dy\,dz
=\int_0^2 z\,dz\int_0^{2-z}\,dy\int_0^{2-z-y} dx
$$
First, integrate over $x$.
$$\int_0^{2-z-y} dx=2-z-y$$
Next, integrate over $y$.
$$\int_0^{2-z}(2-z-y)\,dy=\left(2y-zy-{1\over2}y^2\right)\bigg|_{y=0}^{y=2-z}
={1\over2}z^2-2z+2$$
Finally, over $z\,dz$.
$$\int_0^2\left({1\over2}z^3-2z^2+2z\right)\,dz
=\left({1\over8}z^4-{2\over3}z^3+z^2\right)\bigg|_0^2
={2\over 3}
$$

\vfill
\eject
\beginsection 10.

If $f$ is a continuous function and $a$ is a positive number, show that
$$\int\!\!\!\int\!\!\!\int_V f(x)\,dx\,dy\,dz={1\over2}\int_0^a (a-x)^2 f(x)\,dx$$
where
$$V=\left\{\matrix{
0\le z\le a\cr
0\le y\le z\cr
0\le x\le y\cr
}\right\}
$$
\hrule
\bigskip

\noindent
We have
$$0\le x\le y\le z\le a$$
To change the order of the integrals,
we can let $x$ range from 0 to $a$ and then let $y$ and $z$ fall into line.
$$V=\left\{\matrix{
0\le x\le a\cr
x\le y\le a\cr
y\le z\le a\cr
}\right\}
$$
Since $f(x)$ does not depend on $y$ or $z$, we can rewrite the integral as follows.
$$\int\!\!\!\int\!\!\!\int_V f(x)\,dx\,dy\,dz
=\int_0^a f(x)\,dx\int_x^a dy\int_y^a dz$$
We have
$$\int_y^a dz=z\bigg|_y^a=a-y$$
Next,
$$\int_x^a (a-y)\,dy=\left(ay-{1\over2}y^2\right)\bigg|_x^a=
a^2-{1\over2}a^2-ax+{1\over2}x^2={1\over2}(a-x)^2$$
Hence
$$\int\!\!\!\int\!\!\!\int_V f(x)\,dx\,dy\,dz
%=\int_0^a f(x)\,dx\int_x^a dy\int_y^a dz
={1\over2}\int_0^a (a-x)^2f(x)\,dx
$$

\end
