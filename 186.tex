\magnification=1200

\def\INT{\mathop{\rm int}\nolimits}

{\it George Weigt -- Geometry Homework \#7}

\beginsection Page 85, problem 1.

Prove that in a metric geometry, $\INT(\overrightarrow{AB})$ and
$\INT(\overline{AB})$ are convex sets

\medskip
\hrule
\bigskip

\noindent
First, prove that $\INT(\overrightarrow{AB})$ is convex.

\medskip
\item{$\scriptstyle1$}
Let $P,Q\in \overrightarrow{AB}-\{A\}$.

\medskip
\item{$\scriptstyle2$}
Then we have $A{-}P{-}Q$ or $A{-}Q{-}P$.

\medskip
\item{$\scriptstyle3$}
In either case $A\not\in\overline{PQ}$.

\medskip
\item{$\scriptstyle4$}
Therefore for every $P{-}X{-}Q$ we have
$X\in\overrightarrow{AB}-\{A\}$.

\medskip
\item{$\scriptstyle5$}
Hence $\overline{PQ}\subset\overrightarrow{AB}-\{A\}$.

\medskip
\item{$\scriptstyle6$}
Therefore $\INT(\overrightarrow{AB})$ is convex.

\medskip
\noindent
Next, prove that $\INT(\overline{AB})$ is convex.

\medskip
\item{$\scriptstyle1$}
Let $P,Q\in \overline{AB}-\{A,B\}$.

\medskip
\item{$\scriptstyle2$}
Then we have $A{-}P{-}Q{-}B$ or $A{-}Q{-}P{-}B$.

\medskip
\item{$\scriptstyle3$}
In either case $A,B\not\in\overline{PQ}$.

\medskip
\item{$\scriptstyle4$}
Therefore for every $P{-}X{-}Q$ we have
$X\in\overline{AB}-\{A,B\}$.

\medskip
\item{$\scriptstyle5$}
Hence $\overline{PQ}\subset\overline{AB}-\{A,B\}$.

\medskip
\item{$\scriptstyle6$}
Therefore $\INT(\overline{AB})$ is convex.

\vfill
\eject

\beginsection Page 85, problem 5.

In a Pasch geometry, if $P\in \INT(\angle ABC)$ prove
$\INT(\overrightarrow{BP})\subset \INT(\angle ABC)$.

\medskip
\hrule

\bigskip
\item{$\scriptstyle1$}
We have $\INT(\overrightarrow{BP})\cap\overleftarrow A\overrightarrow B=\emptyset$.

\medskip
\item{$\scriptstyle2$}
Then by Theorem 8.4, all of $\INT(\overrightarrow{BP})$ lies on the same side of
$\overleftarrow A\overrightarrow B$.

\medskip
\item{$\scriptstyle3$}
We have $\INT(\overrightarrow{BP})\cap\overleftarrow B\overrightarrow C=\emptyset$.

\medskip
\item{$\scriptstyle4$}
Then by Theorem 8.4, all of $\INT(\overrightarrow{BP})$ lies on the same side of
$\overleftarrow B\overrightarrow C$.

\medskip
\item{$\scriptstyle5$}
Let $\INT(\angle ABC)=H_1\cap H_2$ such that $A\in H_1$ and $C\in H_2$.

\medskip
\item{$\scriptstyle6$}
By $P\in \INT(\angle ABC)$ we have $P\in H_1$ and $P\in H_2$.

\medskip
\item{$\scriptstyle7$}
By (2) and (6) we have
$\INT(\overrightarrow{BP})\subset H_2$.

\medskip
\item{$\scriptstyle8$}
By (4) and (6) we have
$\INT(\overrightarrow{BP})\subset H_1$.

\medskip
\item{$\scriptstyle9$}
Hence by (7) and (8) we have $\INT(\overrightarrow{BP})\subset H_1\cap H_2$.

\medskip
\item{$\scriptstyle10$}
Therefore
$\INT(\overrightarrow{BP})\subset \INT(\angle ABC)$.

\vfill
\eject

\beginsection Page 85, problem 6.

In a Pasch geometry, given $\triangle ABC$ and points $D$, $E$, $F$
such that $B{-}C{-}D$, $A{-}E{-}C$, and $B{-}E{-}F$,
prove that $F\in \INT(\angle ACD)$.

\medskip
\hrule

\bigskip
\item{$\scriptstyle1$}
By $B{-}C{-}D$ we have $B$ and $D$ on opposite sides of $\overleftarrow A\overrightarrow C$.

\medskip
\item{$\scriptstyle2$}
By $B{-}E{-}F$ we have $B$ and $F$ on opposite sides of $\overleftarrow A\overrightarrow C$.

\medskip
\item{$\scriptstyle3$}
Hence by (1) and (2) we have $D$ and $F$ on the same
side of $\overleftarrow A\overrightarrow C$.

\medskip
\item{$\scriptstyle4$}
By $A{-}E{-}C$ we have $A$ and $E$ on the same side of $\overleftarrow C\overrightarrow D$.

\medskip
\item{$\scriptstyle5$}
By $B{-}E{-}F$ we have $E$ and $F$ on the same side of $\overleftarrow C\overrightarrow D$.

\medskip
\item{$\scriptstyle6$}
Hence by (4) and (5) we have $A$ and $F$ on the same side of
$\overleftarrow C\overrightarrow D$.

\medskip
\item{$\scriptstyle7$}
Let $\INT(\angle ACD)=H_1\cap H_2$ such that $A\in H_1$ and $D\in H_2$.

\medskip
\item{$\scriptstyle8$}
By (3) we have $F\in H_2$.

\medskip
\item{$\scriptstyle9$}
By (6) we have $F\in H_1$.

\medskip
\item{$\scriptstyle10$}
Hence by (8) and (9) we have $F\in H_1\cap H_2$.

\medskip
\item{$\scriptstyle11$}
Therefore $F\in \INT(\angle ACD)$.

\vfill
\eject

\beginsection Page 85, problem 8.

In a Pasch geometry, if $P\in \INT(\angle ABC)$ and if
$D\in\overrightarrow{AP}\cap\overrightarrow{BC}$,
then prove that $A{-}P{-}D$.

\medskip
\hrule

\bigskip
\item{$\scriptstyle1$}
By $D\in\overrightarrow{BC}$ we have $\angle ABC=\angle ABD$.

\medskip
\item{$\scriptstyle2$}
Then by Theorem 8.8 we have $\INT(\angle ABC)=\INT(\angle ABD)$.

\medskip
\item{$\scriptstyle3$}
Hence $P\in \INT(\angle ABD)$.

\medskip
\item{$\scriptstyle4$}
By the Crossbar Theorem, $\overrightarrow{BP}$ intersects
$\overline{AD}$ at a unique point $F$ with $A{-}F{-}D$.

%\medskip
%\noindent
%Note that $P\in\overleftarrow B\overrightarrow P$
%and $P\in\overleftarrow A\overrightarrow D$.

\medskip
\item{$\scriptstyle5$}
Note that
$\overleftarrow B\overrightarrow P\cap\overleftarrow A\overrightarrow P=\{P\}$.

\medskip
\item{$\scriptstyle6$}
Since unique lines intersect in at most one point, we must have $F=P$.

\medskip
\item{$\scriptstyle7$}
By (4) and (6) we have $A{-}P{-}D$.

\vfill
\eject

\beginsection Page 85, problem 10.

Prove the following (Theorem 4.4.9).

\medskip
\noindent
In a Pasch geometry, if $A{-}B{-}D$ then $P\in \INT(\angle ABC)$
if and only if $C\in \INT(\angle DBP)$.

\medskip
\hrule
\bigskip

\noindent
First, prove the implication.
%Assume $P\in \INT(\angle ABC)$ then show that $C\in \INT(\angle DBP)$.

\medskip
\item{$\scriptstyle1$} Let $P\in \INT(\angle ABC)$.

\medskip
\item{$\scriptstyle2$}
By the Crossbar Theorem, there is an $F\in\overrightarrow{BP}$ such that $A{-}F{-}C$.

\medskip
\item{$\scriptstyle3$}
By $A{-}F{-}C$ we have $A$ and $C$ on opposite sides of $\overleftarrow B\overrightarrow P$.

\medskip
\item{$\scriptstyle4$}
By the assumption $A{-}B{-}D$ we have
$A$ and $D$ on opposite sides of $\overleftarrow B\overrightarrow P$.

\medskip
\item{$\scriptstyle5$}
By (3) and (4) we have $C$ and $D$ on the same side of $\overleftarrow B\overrightarrow P$.

\medskip
\item{$\scriptstyle6$}
By $P\in \INT(\angle ABC)$ we have $C$ and $P$ on the same side of
$\overleftarrow A\overrightarrow B$.

\medskip
\item{$\scriptstyle7$}
Let $\INT(\angle DBP)=H_1\cap H_2$ such that $D\in H_1$ and $P\in H_2$.

\medskip
\item{$\scriptstyle8$}
By (5) and (7) we have $C\in H_1$.

\medskip
\item{$\scriptstyle9$}
By (6) and (7) we have $C\in H_2$.

\medskip
\item{$\scriptstyle10$}
Hence $C\in H_1\cap H_2$.

\medskip
\item{$\scriptstyle11$}
Therefore $P\in \INT(\angle ABC)$ implies $C\in \INT(\angle DBP)$.

\medskip
\noindent
Next, prove the converse.
%Assume $P\in \INT(\angle ABC)$ then show that $C\in \INT(\angle DBP)$.

\medskip
\item{$\scriptstyle1$} Let $C\in \INT(\angle DBP)$.

\medskip
\item{$\scriptstyle2$}
By the Crossbar Theorem, there is an $F\in\overrightarrow{BC}$ such that $D{-}F{-}P$.

\medskip
\item{$\scriptstyle3$}
By $D{-}F{-}P$ we have $D$ and $P$ on opposite sides of $\overleftarrow B\overrightarrow C$.

\medskip
\item{$\scriptstyle4$}
By the assumption $A{-}B{-}D$ we have
$A$ and $D$ on opposite sides of $\overleftarrow B\overrightarrow C$.

\medskip
\item{$\scriptstyle5$}
By (3) and (4) we have $A$ and $P$ on the same side of $\overleftarrow B\overrightarrow P$.

\medskip
\item{$\scriptstyle6$}
By $C\in \INT(\angle DBP)$ we have $C$ and $P$ on the same side of
$\overleftarrow B\overrightarrow D$.

\medskip
\item{$\scriptstyle7$}
Let $\INT(\angle ABC)=H_1\cap H_2$ such that $A\in H_1$ and $C\in H_2$.

\medskip
\item{$\scriptstyle8$}
By (5) and (7) we have $P\in H_1$.

\medskip
\item{$\scriptstyle9$}
By (6) and (7) we have $P\in H_2$.

\medskip
\item{$\scriptstyle10$}
Hence $P\in H_1\cap H_2$.

\medskip
\item{$\scriptstyle11$}
Therefore $C\in \INT(\angle DBP)$ implies $P\in \INT(\angle ABC)$.

\vfill
\eject

\beginsection Page 85, problem 14.

In a Pasch geometry, if $\INT(\angle ABC)=\INT(\angle DEF)$,
prove $\angle ABC=\angle DEF$.

\medskip
\hrule

\bigskip
\noindent
Proof by contraposition, i.e. $(Q'\rightarrow P')\rightarrow(P\rightarrow Q$).
%We will show that
%$\angle ABC\ne\angle DEF$
%implies
%$\INT(\angle ABC)\ne\INT(\angle DEF)$.

\medskip
\item{$\scriptstyle1$}
Assume $\angle ABC\ne\angle DEF$.

%\medskip
%\item{$\scriptstyle2$}
%Then
%$
%\{
%\overleftarrow A\overrightarrow B,
%\overleftarrow B\overrightarrow C
%\}\ne
%\{
%\overleftarrow D\overrightarrow E,
%\overleftarrow E\overrightarrow F
%\}$.

\medskip
\item{$\scriptstyle2$}
By the contrapositive of Theorem 5.26, distinct lines imply distinct half planes, hence
$\INT(\angle ABC)\ne\INT(\angle DEF)$.

\medskip
\item{$\scriptstyle3$}
Therefore by contraposition we have
$\INT(\angle ABC)=\INT(\angle DEF)$
implies
$\angle ABC=\angle DEF$.

\vfill
\eject

\beginsection Page 86, problem 23

Prove that in a Pasch geometry if $\ell\cap\INT(\angle ABC)\ne\emptyset$,
then $\ell\cap\angle ABC\ne\emptyset$.

\medskip
\hrule

\bigskip
\noindent
Proof by contraposition.

\medskip
\item{$\scriptstyle1$}
Assume that $\ell\cap\angle ABC=\emptyset$.

\medskip
\item{$\scriptstyle2$}
Then all of $\overrightarrow{BA}$ lies on one side of $\ell$ and
all of $\overrightarrow{BC}$ lies on one side of $\ell$.

\medskip
\item{$\scriptstyle3$}
Since both $A$ and $C$ lie on the same side of $\ell$ as $B$, all of
$\overrightarrow{BA}$
and all of
$\overrightarrow{BC}$
lie on the same side of $\ell$, call it $H_1$.

\medskip
\item{$\scriptstyle4$}
Let $P\in\INT(\angle ABC)$.

\medskip
\item{$\scriptstyle5$}
Define $Q{-}P{-}R$ such that $Q\in\overrightarrow{BA}$
and $R\in\overrightarrow{BC}$.

\medskip
\item{$\scriptstyle6$}
By (3), both $Q$ and $R$ lie on $H_1$.

\medskip
\item{$\scriptstyle7$}
By Theorem 8.4 (i) all of $\overline{QR}$, including $P$, lie on $H_1$.

\medskip
\item{$\scriptstyle8$}
Since $P$ is arbitrary, all of $\INT(\angle ABC)$ lies on $H_1$.

\medskip
\item{$\scriptstyle9$}
Therefore, $\ell\cap\INT(\angle ABC)=\emptyset$.

\medskip
\item{$\scriptstyle10$}
Hence we have $\ell\cap\angle ABC=\emptyset$ implies $\ell\cap\INT(\angle ABC)=\emptyset$.

\medskip
\item{$\scriptstyle11$}
Therefore by contraposition we have $\ell\cap\INT(\angle ABC)\ne\emptyset$
implies $\ell\cap\angle ABC\ne\emptyset$.

\vfill
\eject

\beginsection Page 86, problem 24.

In a Pasch geometry, given $\triangle ABC$ and two points $P$, $Q$ with $A{-}P{-}B$
and $B{-}Q{-}C$, prove that if $R\in\overrightarrow{PQ}\cap\INT(\triangle ABC)$
then $P{-}R{-}Q$.

\medskip
\hrule

\bigskip
\item{$\scriptstyle1$}
Since $R\in\INT(\triangle ABC)$ and $P,Q\not\in\INT(\triangle ABC)$
we have $R\ne P$ and $R\ne Q$.

\medskip
\item{$\scriptstyle2$}
By (1) and by $R\in\overrightarrow{PQ}$ we have either $P{-}R{-}Q$ or $P{-}Q{-}R$.

\medskip
\item{$\scriptstyle3$}
By $R\in\INT(\triangle ABC)$ we have $A$ and $R$ on the same side of
$\overleftarrow B\overrightarrow C$.

\medskip
\item{$\scriptstyle4$}
By $A{-}P{-}B$ we have $A$ and $P$ on the same side of $\overleftarrow B\overrightarrow C$.

\medskip
\item{$\scriptstyle5$}
Hence $P$ and $R$ are on the same side of
$\overleftarrow B\overrightarrow C=\overleftarrow B\overrightarrow Q$.

\medskip
\item{$\scriptstyle6$}
Therefore $P{-}R{-}Q$.

\vfill
\eject

\beginsection Page 86, problem 26.

In a Pasch geometry if $P\in\INT(\angle ABC)$ prove that there is a line through $P$
which intersects both $\overrightarrow{BA}$ and $\overrightarrow{BC}$ but which does
not pass through $B$.

\medskip
\hrule

\bigskip
\item{$\scriptstyle1$}
Let there be points $Q$ and $R$ such that $Q\in\overrightarrow{BA}$ and
$R\in\overrightarrow{BC}$ and $P\in\INT(\triangle QBR)$.

\medskip
\item{$\scriptstyle2$}
Then by the Crossbar Theorem $\overrightarrow{QP}$ intersects $\overline{BR}$
at a unique point $F$ such that $B{-}F{-}R$.

\medskip
\item{$\scriptstyle3$}
Therefore the line $\ell=\overleftarrow Q\overrightarrow F$ passes through $P$
and intersects $\overrightarrow{BA}$ at $Q$ and intersects
$\overrightarrow{BC}$ at $F$ with $Q,F\ne B$.

\end
