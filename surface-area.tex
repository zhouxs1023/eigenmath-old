
\noindent
Let $P$ be a point on a surface parameterized by $x$ and $y$.
In other words, let $P=(x,y,z)$ where $z=f(x,y)$.
Draw a tiny rectangle on the surface with a corner at $P$.
The tangent lines at $P$ define the edges of the rectangle.
The area $a$ of the rectangle is given by the magnitude
of the cross product of the edges, hence
$$a=\left|{\partial P\over\partial x}\times{\partial P\over\partial y}\right|$$
By summing over all $a$ we obtain the total surface area $A$.
$$A=\int\!\!\!\int a\,dx\,dy$$
The following example computes the surface area of a unit disk
parallel to the $xy$ plane.

\medskip
\verb$z=2$

\verb$P=(x,y,z)$

\verb$a=abs(cross(d(P,x),d(P,y)))$

\verb$defint(a,y,-sqrt(1-x^2),sqrt(1-x^2),x,-1,1)$

$$\pi$$

\medskip
\noindent
The result is $\pi$, the area of a unit circle, which is what we expect.
The following example computes the surface area of $z=x^2+2y$ over
a unit square.

\medskip
\verb$z=x^2+2y$

\verb$P=(x,y,z)$

\verb$a=abs(cross(d(P,x),d(P,y)))$

\verb$defint(a,x,0,1,y,0,1)$

$${3\over2}+{5\over8}\log(5)$$

\medskip
\noindent
As a practical matter, $f(x,y)$ must be very simple in order
for Eigenmath to solve the double integral.

