\subsection{Surface area}
Let $S$ be a surface parameterized by $x$ and $y$.
That is, let $S=(x,y,z)$ where $z=f(x,y)$.
The tangent lines at a point on $S$ form a tiny parallelogram.
The area $a$ of the parallelogram is given by the magnitude of the cross product.
$$a=\left|{\partial S\over\partial x}\times{\partial S\over\partial y}\right|$$
By summing over all the parallelograms we obtain the total surface area $A$.
Hence
$$A=\int\!\!\!\int dA=\int\!\!\!\int a\,dx\,dy$$
The following example computes the surface area of a unit disk
parallel to the $xy$ plane.

\medskip
\verb$z=2$

\verb$S=(x,y,z)$

\verb$a=abs(cross(d(S,x),d(S,y)))$

\verb$defint(a,y,-sqrt(1-x^2),sqrt(1-x^2),x,-1,1)$

$$\pi$$

\medskip
\noindent
The result is $\pi$, the area of a unit circle, which is what we expect.
The following example computes the surface area of $z=x^2+2y$ over
a unit square.

\medskip
\verb$z=x^2+2y$

\verb$S=(x,y,z)$

\verb$a=abs(cross(d(S,x),d(S,y)))$

\verb$defint(a,x,0,1,y,0,1)$

$${3\over2}+{5\over8}\log(5)$$

\medskip
\noindent
As a practical matter, $f(x,y)$ must be very simple in order
for Eigenmath to solve the double integral.

\newpage

\noindent
Find the area of the spiral ramp defined by\footnote{
Williamson and Trotter, {\it Multivariable Mathematics,} p. 598.}
$$S=\left(\matrix{u\cos v\cr u\sin v\cr v}\right),\qquad 0\le u\le1,\qquad 0\le v\le3\pi$$
In this example, the coordinates $x$, $y$ and $z$ are all
functions of an independent parameter space.

\medskip
\verb$x=u*cos(v)$

\verb$y=u*sin(v)$

\verb$z=v$

\verb$S=(x,y,z)$

\verb$a=abs(cross(d(S,u),d(S,v)))$

\verb$defint(a,u,0,1,v,0,3pi)$

$${3\over2}\pi\log(1+2^{1/2})+{3\pi\over2^{1/2}}$$

\verb$float$

$$10.8177$$

