\parindent=0pt

{\bf A.} Are these equivalent?

\bigskip
{\bf 1.} ${\sim}\,(p\leftrightarrow q)$ and
$({\sim}\,p)\leftrightarrow({\sim}\,q)$.
$$
\matrix{
p & q & p\leftrightarrow q & {\sim}\,(p\leftrightarrow q) &
{\sim}\,p & {\sim}\,q & ({\sim}\,p)\leftrightarrow({\sim}\,q)\cr
\cr
T & T & T & F & F & F & T\cr
T & F & F & T & F & T & F\cr
F & T & F & T & T & F & F\cr
F & F & T & F & T & T & T\cr
}
$$
No, the expressions are not equivalent.

\bigskip
{\bf 2.} $p\rightarrow(q\land r)$ and $(p\rightarrow q)\land(p\rightarrow r)$.
$$\matrix{
p & q & r & q\land r & p\rightarrow(q\land r) & p\rightarrow q & p\rightarrow r
& (p\rightarrow q)\land(p\rightarrow r) \cr
\cr
T & T & T & T & T & T & T & T\cr
T & T & F & F & F & T & F & F\cr
T & F & T & F & F & F & T & F\cr
T & F & F & F & F & F & F & F\cr
\cr
F & T & T & T & T & T & T & T\cr
F & T & F & F & T & T & T & T\cr
F & F & T & F & T & T & T & T\cr
F & F & F & F & T & T & T & T\cr
}
$$
Yes, the expressions are equivalent.

\bigskip
{\bf 3.} $(p\lor q)\rightarrow r$ and
$(p\rightarrow r)\lor(q\rightarrow r)$.
$$\matrix{
p & q & r & p\lor q & (p\lor q)\rightarrow r &
p\rightarrow r & q\rightarrow r & (p\rightarrow r)\lor(q\rightarrow r)\cr
\cr
T & T & T & T & T & T & T & T\cr
T & T & F & T & F & F & F & F\cr
T & F & T & T & T & T & T & T\cr
T & F & F & T & F & F & T & T\cr
\cr
F & T & T & T & T & T & T & T\cr
F & T & F & T & F & T & F & T\cr
F & F & T & F & T & T & T & T\cr
F & F & F & F & T & T & T & T\cr
}$$
No, the expressions are not equivalent.

\bigskip
{\bf 4.} $p\lor(q\land r)$ and $(p\lor q)\land r$.
$$\matrix{
p & q & r & q\land r & p\lor(q\land r) & p\lor q &
(p\lor q)\land r\cr
\cr
T & T & T & T & T & T & T\cr
T & T & F & F & T & T & F\cr
T & F & T & F & T & T & T\cr
T & F & F & F & T & T & F\cr
\cr
F & T & T & T & T & T & T\cr
F & T & F & F & F & T & F\cr
F & F & T & F & F & F & F\cr
F & F & F & F & F & F & F\cr
}$$
No, the expressions are not equivalent.
\vfill\eject

{\bf B.} Negate the following statements and open statements.

\bigskip
{\bf 1.} $n$ is an integer, and $x=n$ is a solution to the equation
$x^2-x+1=0$.
\par
Either $n$ is not an integer, or $x=n$ is not a solution to the equation
$x^2-x+1=0$.

\bigskip
{\bf 2.} For any integer $n$, $n$ is an even number or $n$ is an odd number.
\par
There is an integer $n$ such that $n$ is not an even number and $n$ is not an
odd number.

\bigskip
{\bf 3.}
There is a negative integer $n$ such that $n$ is a solution to the equation
$x^2-x+1=0$.
\par
For all negative integer $n$, $n$ is not a solution to the equation
$x^2-x+1=0$.

\bigskip
{\bf 4.}
There is an integer $n$ such that $n\cdot x=0$ for any integer $x$.
\par
For all integer $n$, there is an integer $x$ such that $n\cdot x\ne0$.

\bigskip
{\bf 5.}
There is a unique real number $x$ such that $x^2+2x+1=0$.
\par
Either there is no real number $x$ such that $x^2+2x+1=0$,
or there is more than one $x$ such that $x^2+2x+1=0$.

\bigskip
{\bf 6.}
There is a real number $M$ such that $N<M$ for all natural number $N$.
\par
For all real number $M$, there is a natural number $N$ such that $N\ge M$.

\bigskip
{\bf 7.}
If a real number $\epsilon>0$, then there is a real number $\delta>0$
such that $\displaystyle{\left|{{1\over2+\delta}-{1\over2}}\right|<\epsilon}$.
\par
There is a real number $\epsilon>0$ such that for all real number
$\delta>0$,
$\displaystyle{\left|{{1\over2+\delta}-{1\over2}}\right|\ge\epsilon}$.


\end
