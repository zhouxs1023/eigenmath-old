\beginsection 11.

The current record is 1.24 trillion digits.\par
{\tt http://seattlepi.nwsource.com/national/98912\_pi07.shtml}

\beginsection 2.

Here are 10 different formulas for $\pi$.
$$\pi=4\sum_{k=1}^\infty{(-1)^{k+1}\over2k-1}$$
$$\pi=4\sum_{k=0}^\infty{(-1)^k\over2k+1}$$
$$\pi=2\sqrt3\sum_{k=0}^\infty{(-1/3)^k\over2k+1}$$
$$\pi=2\sqrt2\sum_{k=1}^\infty(-1)^{k+1}\left[
{1\over4k+1}
+
{1\over4k-3}
\right]$$
$$\pi=3\sum_{k=0}^\infty(-1)^k\left[
{1\over6k+1}
+
{1\over6k+5}
\right]$$
$$\pi^2=6\sum_{k=1}^\infty{1\over k^2}$$
$$\pi^2=8\sum_{k=1}^\infty{1\over(2k-1)^2}$$
$$\pi={3\sqrt3\over2}\sum_{k=0}^\infty{(k!)^2\over(2k+1)!}$$
$$\pi=\sum_{k=0}^\infty{(k!)^22^{k+1}\over(2k+1)!}$$
$$\pi=2\sum_{k=0}^\infty{k!\over(2k+1)!!}$$

\beginsection 3.

Here is a site that has an image of the Rhind Papyrus.

{\tt http://www.physics.utoledo.edu/\~{}ljc/rhind.html}

\beginsection 4.

If $x=2/3$, $y=1/2$, $z=1/3$ and $w=1/4$ then the proportions are correct.
However, $x+y+z+w=7/4$ so we have to multiply all of the proportions by
400 to get them to add up to 700 loaves. Consequently we have
$$x={800\over3},\qquad y=200,\qquad z={400\over3},\qquad w=100$$

\beginsection 5.

(a) The diameter of the circle is 12, the radius is 6.
Cut off $1/9$ of the diameter to get the
length $l$ of the square.
$$l=12-12/9={32\over3}$$
(b) Using Ahmes' method the area of the square approximates the area of the circle.
$$l^2={1024\over9}\approx36\pi$$
$$\pi\approx{256\over81}=3.16$$

\beginsection 6.

There are two passages in the Bible from which we can infer the value
of $\pi$ known to the ancient Israelites, 1 Kings 7:23 and 2 Chronicles 4:2.
\medskip
(1 Kings 7:23) And he made a molten sea, ten cubits from the one brim to the other:
it was round all about, and his height was five cubits:
and a line of thirty cubits did compass it round about.
\medskip
(2 Chronicles 4:2)
He made the Sea of cast metal, circular in shape, measuring ten cubits from rim to rim
and five cubits high. It took a line of thirty cubits to measure around it.
\medskip
The passages describe how Hiram builds a temple.

\beginsection 7.

$$\sum_{k=1}^n(2k-1)=n^2$$

\beginsection 8.

$$\sum_{k=1}^nk^2=n(n+1)(2n+1)/6$$

\end