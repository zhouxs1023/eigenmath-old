\newpage

\noindent
The following example demonstrates a technique for computing
a line integral when the path is already parameterized.
The task at hand is to compute\footnote{
Kaplan, {\it Advanced Calculus,} p. 312.}
$$\int_C z\,dx+x\,dy+y\,dz$$
where $C$ is the path
$$x=2t+1,\qquad y=t^2,\qquad z=1+t^3,\qquad 0\le t\le 1$$
The main idea is that we can rewrite the integrand with the following substitutions.
$$dx=\left({dx\over dt}\right)dt,\qquad
dy=\left({dy\over dt}\right)dt,\qquad
dz=\left({dz\over dt}\right)dt$$
Therefore in Eigenmath we have

\medskip
\verb$x=2t+1$

\verb$y=t^2$

\verb$z=1+t^3$

\verb$f=z*d(x,t)+x*d(y,t)+y*d(z,t)$

\verb$I=integral(f,t)$

\verb$eval(I,t,1)-eval(I,t,0)$

$$163\over30$$
