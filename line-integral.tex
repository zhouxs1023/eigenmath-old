\newpage

\noindent
Here are a few line integral examples.
The main idea is to convert $x$, $y$ and $z$ into functions of $t$.
Consequently, the measure changes as well.
For instance, $dx$ becomes $(dx/dt)\,dt$.

\medskip
\noindent
Evaluate $\int y^2\,dx$ along the straight
line from $(0,0)$ to $(2,2)$.\footnote{Kaplan,
{\it Advanced Calculus,} p. 278.}

\medskip
\verb$x=2t$

\verb$y=2t$

\verb$defint(y^2*d(x,t),t,0,1)$

$$8\over3$$

\medskip
\noindent
Evaluate $\int y\,dx$ along the straight line from
$(2,1)$ to $(1,2)$.

\medskip
\verb$x=2-t$

\verb$y=t+1$

\verb$defint(y*d(x),t,0,1)$

$$-{3\over2}$$

\medskip
\noindent
Evaluate $\int x\,dy$ along the straight line from
$(1,1)$ to $(2,1)$.

\medskip
\verb$x=t+1$

\verb$y=1$

\verb$defint(x*d(y),t,0,1)$

$$0$$

\medskip
\noindent
Evaluate $\int z\,dx+x\,dy+y\,dz$
along the path
$x=2t+1$, $y=t^2$, $z=1+t^3$, $0\le t\le 1$.\footnote{Ibid, p. 312.}

\medskip
\verb$x=2t+1$

\verb$y=t^2$

\verb$z=1+t^3$

\verb$f=z*d(x)+x*d(y)+y*d(z)$

\verb$defint(f,t,0,1)$

$$163\over30$$

