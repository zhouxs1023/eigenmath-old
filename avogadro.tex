\section{}

There is a proposal to define Avogadro's constant as exactly $84446886^3$
atoms or molecules.\footnote{Fox, Ronald and Theodore Hill.
``An Exact Value for Avogadro's Number.''
{\it American Scientist} 95 (2007): 104--107.
The proposed number in the article is actually $84446888^3$.
In a subsequent addendum the authors reduced it to $84446886^3$ to make the
number divisible by 12. See {\tt www.physorg.com/news109595312.html}}
Of course, this number corresponds to a perfect cube of material with 84446886
atoms along each edge.
Let us check the difference between the proposed value and the measured value
of $(6.0221415\pm0.0000010)\times10^{23}$ atoms.

\medskip
\verb$A=84446886^3$

\verb$B=6.0221415*10^23$

\verb$A-B$

$$-5.17173\times10^{16}$$

\verb$0.0000010*10^23$

$$1\times10^{17}$$

\medskip
\noindent
We see that the proposed value is within the experimental error.
Just for the fun of it, let us factor the proposed value.

\medskip
\verb$factor(A)$

$$2^3\times3^3\times1667^3\times8443^3$$

