\documentclass[11pt]{article}
\title{Linux Hacks}
\date{\today}
\begin{document}
\maketitle

\newpage

\tableofcontents

\newpage

\section{Creating a ram disk}
Here is the code for creating a ram disk.

\begin{verbatim}
dd if=/dev/zero of=ramdisk bs=1024 count=32768
/sbin/mkfs.ext2 ramdisk
mkdir -p /mnt/ramdisk
mount -o loop ramdisk /mnt/ramdisk
\end{verbatim}

\section{Notes on CVS}
Use cvs init to create a repository.

\begin{verbatim}
export CVSROOT=/cvsroot
mkdir -p /cvsroot
cvs init
\end{verbatim}

\noindent
Use cvs import to check-in a directory tree for the first time.
The following commands create a project foo under /cvsroot that
contains everything in foobar.

\begin{verbatim}
cd foobar
cvs import -m "" foo vendor_tag release_tag
\end{verbatim}

\noindent
The following command adds a version tag.

\begin{verbatim}
cvs rtag version1 foo
\end{verbatim}

\section{Power-up call trace}

Summary of power-up call trace

\begin{verbatim}
	_start (arch/ppc/kernel/head.S)
		early_init (arch/ppc/kernel/setup.c)
	start_here (arch/ppc/kernel/head.S)
		machine_init (arch/ppc/kernel/setup.c)
			platform_init (arch/ppc/platforms/gigateak.c)
	start_kernel (init/main.c)
		setup_arch (arch/ppc/kernel/setup.c)
			gigateak_setup_arch (arch/ppc/platforms/gigateak.c)
				gigateak_setup_bridge
				gigateak_setup_peripherals
				gigateak_setup_ethernet
				gigateak_enable_ipmi
\end{verbatim}

Notes:

1. U-boot jumps to address \_start.
Normally \_start is at address 0. See System.map

2. The call to gigateak\_setup\_arch() is made via the function pointer
ppc\_md.setup\_arch().
This function pointer is initialized in platform\_init().

\end{document}
