\magnification=1200

\noindent
{\it George Weigt --- Advanced Calculus Homework \#11}

\beginsection Question 1.

Evaluate the following surface integral.
$$\int\!\!\!\int_S x\,dy\,dz + y\,dz\,dx + z\,dx\,dy$$
where $S$ is the triangle with vertices $(1,0,0)$,
$(0,1,0)$, $(0,0,1)$ and the normal points away from $(0,0,0)$.

\bigskip
\noindent
Solution: Per HW11 (and Eq. 5.79 in the book) the above integral can be written as
$$\int\!\!\!\int_S({\bf G}\cdot{\bf n})\,d\sigma$$
where
$${\bf G}=\langle x,y,z\rangle$$
For the triangle we have
$$z=1-x-y,\qquad
r=\langle x,y,1-x-y\rangle,\qquad
D=\left\{\matrix{0\le x\le 1\cr 0\le y\le 1-x}\right\}$$
$${\partial r\over\partial x}=\langle 1,0,-1\rangle,\qquad
{\partial r\over\partial y}=\langle 0,1,-1\rangle,\qquad
{\partial r\over\partial x}\times{\partial r\over\partial y}=\langle 1,1,1\rangle$$
The direction of the normal vector is correct. We have
$$\eqalign{
\int\!\!\!\int_S({\bf G}\cdot{\bf n})\,d\sigma
&=\int\!\!\!\int_D \langle x,y,z\rangle\langle 1,1,1\rangle\,dx\,dy\cr
&=\int\!\!\!\int_D (x+y+1-x-y)\,dy\cr
&=\int_0^1 dx\int_0^{1-x}dy\cr
&=\int_0^1(1-x)\,dx\cr
&=\left(x-{x^2\over2}\right)\bigg|_0^1\cr
&={1\over2}
}$$

\vfill
\eject

\beginsection Question 2.

Evaluate the following surface integral.
$$\int\!\!\!\int_S dy\,dz+dz\,dx+dx\,dy$$
where $S$ is the hemisphere $z=\sqrt{1-x^2-y^2}$, $x^2+y^2\le1$,
and the normal is the upper normal.

\bigskip
\noindent
Solution: Per the HW11, this is the same as
$$\int\!\!\!\int_S({\bf G}\cdot{\bf n})\,d\sigma$$
where
$${\bf G}=\langle 1,1,1\rangle$$
Using the parameterization from Worked Example \#2 with $R=1$ we have
$$r=\langle\sin\phi\cos\theta,\sin\phi\sin\theta,\cos\phi\rangle,\qquad
D=\left\{\matrix{0\le\phi\le\pi/2\cr0\le\theta\le2\pi}\right\}$$
$${\partial r\over\partial\phi}\times{\partial r\over\partial\theta}
=\langle \sin^2\phi\cos\theta,\sin^2\phi\sin\theta,\sin\phi\cos\phi\rangle$$
Per HW11, the normal points in the correct direction. We have
$$\eqalignno{
\int\!\!\!\int_S({\bf G}\cdot{\bf n})\,d\sigma
&=\int\!\!\!\int_D \langle 1,1,1\rangle
\langle \sin^2\phi\cos\theta,\sin^2\phi\sin\theta,\sin\phi\cos\phi\rangle\,d\phi\,d\theta\cr
&=\int_0^{\pi/2}d\phi\int_0^{2\pi}
(\sin^2\phi\cos\theta+\sin^2\phi\sin\theta+\sin\phi\cos\phi)\,d\theta\cr
&=\int_0^{\pi/2}d\phi\,
(\sin^2\phi\sin\theta-\sin^2\phi\cos\theta+\theta\sin\phi\cos\phi)
\bigg|_{\theta=0}^{\theta=2\pi}\cr
&=2\pi\int_0^{\pi/2}\sin\phi\cos\phi\,d\phi\cr
\noalign{\hbox{By $\sin\phi\cos\phi=(\sin2\phi)/2$ we have}}
&=\pi\int_0^{\pi/2}\sin2\phi\,d\phi\cr
&=-{\pi\over2}(\cos2\phi)\bigg|_0^{\pi/2}\cr
&=\pi\cr
}$$

\vfill
\eject

\beginsection Question 3.

Evaluate the following surface integral.
$$\int\!\!\!\int_S (x\cos\alpha+y\cos\beta+z\cos\gamma)\,d\sigma$$
where $S$ is the hemisphere $z=\sqrt{1-x^2-y^2}$, $x^2+y^2\le1$, and the normal
is the upper normal.
\bigskip
\noindent
Solution: Per the HW11, this is the same as
$$\int\!\!\!\int_S({\bf G}\cdot{\bf n})\,d\sigma$$
where
$${\bf G}=\langle x,y,z\rangle$$
From Worked Examples 11 we have
$${\bf G}=\langle \sin\phi\cos\theta,\sin\phi\sin\theta,\cos\phi\rangle$$
$${\partial r\over\partial\phi}\times{\partial r\over\partial\theta}
=\langle \sin^2\phi\cos\theta,\sin^2\phi\sin\theta,\sin\phi\cos\phi\rangle$$
$${\bf G}\cdot\left({\partial r\over\partial\phi}\times{\partial r\over\partial\theta}\right)
=\sin\phi$$
$$\eqalign{
\int\!\!\!\int_S ({\bf G}\cdot{\bf n})\,d\sigma
&=\int_0^{\pi/2}d\phi\int_0^{2\pi}\sin\phi\,d\theta\cr
&=2\pi\int_0^{\pi/2}\sin\phi\,d\phi\cr
&=2\pi(-\cos\phi)\bigg|_0^{\pi/2}\cr
&=2\pi
}$$

\vfill
\eject

\beginsection Question 4.

Solution: We have
$$P(x,y)={\partial\tan^{-1}(y/x)\over\partial x}=-{y\over x^2+y^2}\qquad
Q(x,y)={\partial\tan^{-1}(y/x)\over\partial y}={x\over x^2+y^2}$$
$${\partial P\over\partial y}={2y^2\over(x^2+y^2)^2}-{1\over x^2+y^2}
={y^2-x^2\over(x^2+y^2)^2}\qquad
{\partial Q\over\partial x}={1\over x^2+y^2}-{2x^2\over(x^2+y^2)^2}
={y^2-x^2\over(x^2+y^2)^2}$$
The path $C$ does not enclose the origin where $x^2+y^2=0$.
Therefore by Green's Theorem
$$\int_C P\,dx+Q\,dy=\int\!\!\!\int_D\left(
{\partial Q\over\partial x}-{\partial P\over\partial y}\right)\,dx\,dy=0$$

\bigskip
\noindent
Let us check this result by doing the line integral.
Start by parametrizing the path using polar coordinates.
$$x=r\cos\theta+5,\qquad y=r\sin\theta$$
Then for loop $C$ we have
$$r^2\cos^2\theta+r^2\sin^2\theta=(2r^2\cos^2\theta+2r^2\sin^2\theta-r\cos\theta)^2$$
which reduces to
$$r=2r^2-r\cos\theta$$
This works out to
$$r={1+\cos\theta\over2}$$
Hence
$$x={1\over2}\cos^2\theta+{1\over2}\cos\theta+5\qquad
y={1\over2}\sin\theta+{1\over2}\sin\theta\cos\theta$$
$${dx\over d\theta}=-{1\over2}\sin\theta-\sin\theta\cos\theta\qquad
{dy\over d\theta}={1\over2}\cos\theta+{1\over2}\cos^2\theta-{1\over2}\sin^2\theta$$
$P(\theta)\,d\theta$ and $Q(\theta)\,d\theta$ are going to be very complicated so I'll stop here.
Thank goodness for Green's Theorem!

\end
