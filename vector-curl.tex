\section*{Vector curl}
The curl of a vector function can be expressed in tensor form as
$$\mathop{\rm curl}{\bf F}=\epsilon_{ijk}\,{\partial F_k\over\partial x_j}$$
where $\epsilon_{ijk}$ is the Levi-Civita tensor.
The following script demonstrates that this formula is equivalent
to computing curl the old fashioned way.
First, define $\epsilon_{ijk}$.

\medskip
{\tt
epsilon=zero(3,3,3)

epsilon[1,2,3]=1

epsilon[2,3,1]=1

epsilon[3,1,2]=1

epsilon[3,2,1]=-1

epsilon[1,3,2]=-1

epsilon[2,1,3]=-1
}

\medskip
\noindent
Next, define a generic vector function $\bf F$ and
then compute $A=\epsilon_{ijk}\,\partial F_k/\partial x_j$.
The first summation is over $k$ which corresponds to indices 3 and 4.
The second summation is over $j$ which (with $k$ out of the way)
corresponds to indices 2 and 3.

\medskip
{\tt
F=(FX(),FY(),FZ())

A=outer(epsilon,d(F,(x,y,z)))

A=contract(A,3,4)

A=contract(A,2,3)
}

\medskip
\noindent
Now compute curl the old fashioned way and check for equality.

\medskip
{\tt
B=(

\ d(F[3],y)-d(F[2],z),

\ d(F[1],z)-d(F[3],x),

\ d(F[2],x)-d(F[1],y)

)

\medskip
A-B
}

$$\left(\matrix{0\cr0\cr0}\right)$$

\newpage

\noindent
The following is a variation on the previous script.
The product $\epsilon_{ijk}\,\partial F_k/\partial x_j$
is computed in just one line of code.
In addition, there is an optimization.
The outer product and the first contraction have been replaced with a
dot product.

\medskip
\verb$F=(FX(),FY(),FZ())$

\medskip
\verb$epsilon=zero(3,3,3)$

\verb$epsilon[1,2,3]=1$

\verb$epsilon[2,3,1]=1$

\verb$epsilon[3,1,2]=1$

\verb$epsilon[3,2,1]=-1$

\verb$epsilon[1,3,2]=-1$

\verb$epsilon[2,1,3]=-1$

\medskip
\verb$A=contract(dot(epsilon,d(F,(x,y,z))),2,3)$

\medskip
\verb$B=($

\verb$ d(F[3],y)-d(F[2],z),$

\verb$ d(F[1],z)-d(F[3],x),$

\verb$ d(F[2],x)-d(F[1],y)$

\verb$)$

\medskip
\verb$--Are A and B equal? Subtract to find out.$

\medskip
\verb$A-B$

