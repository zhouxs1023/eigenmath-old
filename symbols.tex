
\subsection{Defining symbols}
As we saw earlier, Eigenmath uses the same syntax as beloved Fortran.

\medskip
\verb$N=212^17$

\medskip
\noindent
No result is printed when a symbol is defined.
To see a symbol's value, just evaluate it.

\medskip
\verb$N$

$$3529471145760275132301897342055866171392$$

\medskip
\noindent
Beyond its prosaic syntax, Eigenmath does have a few tricks up its sleeve.
For example, a symbol can have a subscript.

\medskip
\verb$NA=6.02214*10^23$

\verb$NA$

$$N_A=6.02214\times10^{23}$$

\medskip
\noindent
A symbol can be the name of a Greek letter.

\medskip
\verb$xi=1/2$

\verb$xi$

$$\xi=\hbox{$1\over2$}$$

\medskip
\noindent
Since xi is $\xi$, how is $x_i$ entered?
Well, that is an issue that may get resolved in the future.
For now, xi is always $\xi$.

\medskip
\noindent
Eigenmath exhaustively evaluates symbolic subexpressions.

\medskip
\verb$A=B+C$

\verb$A$

$$A=B+C$$

\verb$B=D$

\verb$C=D$

\verb$A$

$$A=2D$$

\medskip
\noindent
In the above example, the symbol $A$ evaluates to $2D$ but internally its
binding is still $A=B+C$.
The $binding$ function can be used to show this.

\medskip
\verb$binding(A)$

$$B+C$$

\medskip
\noindent
One final detail should be noted.
When a symbol is defined, the right hand side is evaluated.
This leads to the following difference from the previous example.

\medskip
\verb$B=D$

\verb$C=D$

\verb$A=B+C$

\verb$binding(A)$

$$2D$$

\medskip
\noindent
In this example, since $B$ and $C$ were already defined, the binding for $A$
is $2D$ instead of $B+C$.
To define a literal binding, use $quote$.

\medskip
\verb$A=quote(B+C)$

\verb$binding(A)$

$$B+C$$

\verb$A$

$$2D$$

\medskip
\noindent
The functions $quote$ and $binding$ are mentioned here mainly to provide insight
into what is happening belowdecks.
Normally you should not really need to use these functions.
However, one notable exception is the use of $quote$ to clear a symbol.

\medskip
\verb$x=3$

\verb$x$

$$x=3$$

\verb$x=quote(x)$

\verb$x$

$$x$$

