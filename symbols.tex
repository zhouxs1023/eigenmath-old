
\subsection{Defining symbols}
As we saw earlier, Eigenmath uses the same syntax as beloved Fortran.

\medskip
\verb$N=212^17$

\medskip
\noindent
No result is printed when a symbol is defined.
To see a symbol's value, just evaluate it.

\medskip
\verb$N$

$$3529471145760275132301897342055866171392$$

\medskip
\noindent
Beyond its prosaic syntax, Eigenmath does have a few tricks up its sleeve.
For example, a symbol can have a subscript.

\medskip
\verb$NA=6.02214*10^23$

\verb$NA$

$$N_A=6.02214\times10^{23}$$

\medskip
\noindent
A symbol can be the name of a Greek letter.

\medskip
\verb$xi=1/2$

\verb$xi$

$$\xi=\hbox{$1\over2$}$$

\medskip
\noindent
Since xi is $\xi$, how is $x_i$ entered?
Well, that is an issue that may get resolved in the future.
For now, xi is always $\xi$.

\medskip
\noindent


