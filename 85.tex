\nopagenumbers
\parindent=0pt

{\bf Problem A.} Show that $\root3\of{-3}$ is not a rational number.
\medskip
The technique is to construct a polynomial $p(x)$ such that
$p(\root3\of{-3})=0$ and then
show that $p(x)$ has no rational roots.
\medskip
{\bf Proof.}
Let $p(x)=x^3+3$ and
$s,t\in Z$.
Assume $gcd(s,t)=1$ and $p(s/t)=0$.
Then $s|3$ and $t|1$.
So the possible values for $s/t$ are $\pm1$ and $\pm3$.
By exhaustive calculation we have
$p(1)=4$, $p(-1)=2$, $p(3)=30$ and $p(-3)=-24$.
None of these are zero therefore there are no rational solutions to $p(x)=0$.
However, by construction $p(\root3\of{-3})=0$.
Hence $\root3\of{-3}$ is not rational.

\bigskip
\bigskip

{\bf Problem B.} Find a nonzero $Q$-polynomial $p(x)$ so that
$x=\sqrt5-\sqrt2$ is a zero of $p(x)$.
\medskip
$$x=\sqrt5-\sqrt2$$
$$x^2=5-2\sqrt5\sqrt2+2=7-2\sqrt5\sqrt2$$
$$x^2-7=-2\sqrt5\sqrt2$$
$$(x^2-7)^2=40$$
$$x^4-14x^2+49=40$$
$$x^4-14x^2+9=0$$
Answer: $p(x)=x^4-14x^2+9$.

\bigskip
\bigskip

{\bf Problem C.} (1) Show that if
$p(x)=a_nx^n+\cdots+a_1x+a_0\in Q[x]$ then $p(2x)\in Q[x]$.
\medskip
{\bf Proof.} Let $p(x)=a_nx^n+\cdots+a_1x+a_0\in Q[x]$.
Then $a_n,\ldots,a_1,a_0\in Q$.
For $p(2x)$ we have
$$\eqalign{
p(2x)&=a_n(2x)^n+\cdots+a_1(2x)+a_0\cr
&=2^na_nx^n+\cdots+2a_1x+a_0
}$$
By simple closure $2^na_n,\ldots,2a_1\in Q$.
Hence $p(2x)\in Q[x]$.

\bigskip
\bigskip
{\bf Problem C cont'd.} (2) Show that if $a\in R$ is an algebraic number
over $Q$, then $a/2$ is an algebraic number over $Q$.
\medskip
{\bf Proof.} Let $a\in R$. Assume that $a$ is an algebraic number over $Q$.
Then there exists a nonzero rational polynomial $p(x)$ such that $p(a)=0$.
By (1) above $p(2x)$ is also a rational polynomial.
We have $p(2\cdot a/2)=p(a)=0$. Hence $a/2$ is an algebraic number over $Q$.

\end
