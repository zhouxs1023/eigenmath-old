{\it George Weigt -- Homework \#2}

\beginsection Question A1.

$$\int_0^\pi\cos^2x\,dx=\int_0^\pi{1+\cos2x\over2}\,dx
=\left({x\over2}+{\sin 2x\over4}\right)\bigg|_0^\pi
={\pi\over2}$$

\beginsection Question A2.

$$g(x)=\lfloor x\rfloor-2\lfloor x/2\rfloor$$
Sketch the graph of $g(x)$ for $0\le x\le5$.

$g(0)=0-0=0$

$g(1)=1-0=1$

$g(2)=2-2=0$

$g(3)=3-2=1$

$g(4)=4-4=0$

$g(5)=5-4=1$

{\obeylines\obeyspaces\tt
\    ****    ****    *


****    ****    ****
0   1   2   3   4   5
}

\beginsection Question A3.

$$\int_0^2g(x)\,dx=1$$

\beginsection Question A4.

Integration by parts. Evaluate
$$\int_0^1\ln x\,dx$$
\medskip
Solution: Let $u=\ln x$ and $dv=dx$. Then $du=dx/x$ and $v=x$ hence
$$\int\ln x\,dx=uv-\int v\,du=x\ln x-\int x\cdot{dx\over x}=x\ln x-x+C$$
Then
$$(x\ln x-x+C)\bigg|_0^1=-1$$

\beginsection Question A5.

Integration by parts. Evaluate
$$\int_0^1x\ln x\,dx$$
\medskip
Solution: Let $u=\ln x$ and $dv=x\,dx$.
Then $du=dx/x$ and $v=x^2/2$ hence
$$\int x\ln x\,dx=uv-\int v\,du={x^2\ln x\over2}-{x^2\over4}+C$$
Then
$$\left({x^2\ln x\over2}-{x^2\over4}+C\right)\bigg|_0^1=-{1\over4}$$

\beginsection Question A6.

Integration by parts. Evaluate
$$\int_0^1\tan^{-1}x\,dx$$
\medskip
Solution: Let $u=\tan^{-1}x$ and $dv=dx$.
Then
$$du={1\over x^2+1}\,dx,\qquad v=x$$
Hence
$$\int\tan^{-1}x\,dx=uv-\int v\,du=x\tan^{-1}x-\int{x\over x^2+1}\,dx$$
Use the substitution $y=x^2+1$. Then $dy=2x\,dx$ hence $dx=dy/2x$.
We have
$$\int{x\over x^2+1}\,dx=\int{x\over y}\cdot{dy\over2x}
={1\over2}\int {dy\over y}={1\over2}\ln y={1\over2}\ln(x^2+1)=\ln\sqrt{x^2+1}$$
Therefore
$$\int\tan^{-1}x\,dx=x\tan^{-1}x-\ln\sqrt{x^2+1}+C$$
We have
$$\int_0^1\tan^{-1}x\,dx=\left(x\tan^{-1}x-\ln\sqrt{x^2+1}+C\right)\bigg|_0^1
={\pi\over4}-\ln \sqrt2$$

\beginsection Question A7.

Evaluate the double integral
$$\int\!\!\int_D(x^2+y^2)\,dx\,dy$$
where $D$ is the region in the $xy$-plane given by
$$D=\{(x,y)\mid0\le x\le1,\,2\le y\le3\}$$
\medskip
Solution: When the region $D$ is a rectangle, as it is here, we do not
have to parameterize the limits of integration.
First, integrate over $x$.
$$\int_0^1(x^2+y^2)\,dx=\left({x^3\over3}+xy^2\right)\bigg|_0^1={1\over3}+y^2$$
Now integrate the result over $y$.
$$\int_2^3\left({1\over3}+y^2\right)\,dy=\left({y\over3}+{y^3\over3}\right)\bigg|_2^3
={3\over3}+{27\over3}-{2\over3}-{8\over3}={20\over3}$$
Therefore
$$\int\!\!\int_D(x^2+y^2)\,dx\,dy={20\over3}$$

\beginsection Question A8.

Evaluate the double integral
$$\int\!\!\int_D(x^2+y^2)\,dx\,dy$$
where $D$ is the region in the $xy$-plane
bounded by the line $y=x$ and $y=x^2$.
\medskip
Solution: For each $x$, integrate $y$ from $x^2$ to $x$.
$$\int_{x^2}^x(x^2+y^2)\,dy=\left(yx^2+{y^3\over3}\right)\bigg|_{y=x^2}^{y=x}=
x^3+{x^3\over3}-x^4-{x^6\over3}$$
$$\int_0^1\left({4x^3\over3}-x^4-{x^6\over3}\right)\,dx
=\left({x^4\over3}-{x^5\over5}-{x^7\over21}\right)\bigg|_0^1
={1\over3}-{1\over5}-{1\over21}={3\over35}$$
Therefore
$$\int\!\!\int_D(x^2+y^2)\,dx\,dy={3\over35}$$

\beginsection Question B1.

Evaluate exactly
$$\int_{-\infty}^{\infty}e^{-x^2}\,dx$$
\medskip
Solution: The trick is to use to polar coordinates.
$$\eqalign{
\left(\int_{-\infty}^{\infty}e^{-x^2}\,dx\right)^2
&=\left(\int_{-\infty}^{\infty}e^{-x^2}\,dx\right)
\left(\int_{-\infty}^{\infty}e^{-y^2}\,dy\right)\cr
%
&=\int_{-\infty}^{\infty}\int_{-\infty}^{\infty}
e^{-(x^2+y^2)}\,dx\,dy\cr
%
&=\int_0^{2\pi}\int_0^\infty re^{-r^2}\,dr\,d\theta\cr
&=\int_0^{2\pi}\left(-{e^{-r^2}\over2}\bigg|_0^\infty\right)\,d\theta\cr
&=\int_0^{2\pi}{1\over2}\,d\theta\cr
&={\theta\over2}\bigg|_0^{2\pi}\cr
&=\pi
}$$
Therefore
$$\int_{-\infty}^{\infty}e^{-x^2}\,dx=\sqrt\pi$$

\beginsection Question B2.

Let $f(x)$ be a continuous function. Define
$$J(a)={1\over2}\int_0^a(a-x)^2f(x)\,dx$$
Find $dJ(a)/da$ and $d^2J(a)/da^2$.
\medskip
Solution: Expanding the quadratic term we have
$$J(a)={a^2\over2}\int_0^af(x)\,dx-a\int_0^axf(x)\,dx+{1\over2}\int_0^ax^2f(x)\,dx$$
Then
$$\eqalign{
{dJ(a)\over da}&=
{d\over da}\left({a^2\over2}\int_0^af(x)\,dx\right)
-{d\over da}\left(a\int_0^axf(x)\,dx\right)
+{d\over da}\left({1\over2}\int_0^ax^2f(x)\,dx\right)\cr
%
}$$
Now work out the derivatives one at a time.
$${d\over da}\left({a^2\over2}\int_0^af(x)\,dx\right)
=a\int_0^af(x)\,dx+{a^2\over2}f(a)$$
$$-{d\over da}\left(a\int_0^axf(x)\,dx\right)
=-\int_0^axf(x)\,dx-a^2f(a)$$
$${d\over da}\left({1\over2}\int_0^ax^2f(x)\,dx\right)
={a^2\over2}f(a)$$
Note that the $a^2$ terms cancel. Putting it all together we have
$${dJ(a)\over da}=a\int_0^af(x)\,dx-\int_0^axf(x)\,dx
=\int_0^a(a-x)f(x)\,dx$$
Next, for the second derivative we have
$$\eqalign{
{d^2J(a)\over da^2}&={d\over da}\left(a\int_0^af(x)\,dx\right)
-{d\over da}\left(\int_0^axf(x)\,dx\right)\cr
&=\int_0^af(x)\,dx+af(a)-af(a)\cr
&=\int_0^af(x)\,dx
}$$


\end