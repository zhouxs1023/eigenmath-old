An urn contains 50 tickets, each with a different number from
1 to 50. If 7 tickets are drawn at random without replacement,
what is the probability that the median of these seven numbers
20?

\bigskip
Possible ways of selecting 7 tickets:
${}_{50}C_7$

\medskip
Possible ways of selecting 3 tickets less than 20:
${}_{19}C_3$

\medskip
Possible ways of selecting 3 tickets greater than 20:
${}_{30}C_3$

\medskip
Possible ways of selecting 1 ticket that is 20:
${}_1C_1$

\medskip
Solution:
$${{}_{19}C_3\times{}_{30}C_3\times{}_1C_1
\over{}_{50}C_7}
={969\times4060\times1\over99884400}=0.0394$$

\bigskip
Discussion

\medskip
\item{1.} The problem is not about finding a distribution,
contrary to what the word ``median'' might suggest.
Instead, think about what it takes for the median of 7 tickets
to be 20.
Since 7 is an odd number, exactly one ticket must be 20.
Then 3 tickets have to be less than 20 and 3 have to be greater
than 20.
There are 19 tickets that are less than 20 and 30 tickets that
are greater than 20.

\medskip
\item{2.} What about the ${}_1C_1$ term?
It doesn't seem right somehow.
Shouldn't the probability of drawing
ticket number 20 be on the order of $1/50$?
The solution is to realize that the other two terms,
${}_{19}C_3$ and ${}_{30}C_3$, cover all the
cases of ticket numbers except 20, i.e.
$19+30=49$.
So the only possibility left is for the 7th ticket
to be exactly 20.
Another way of looking at it is the $1/50$ factor is already
in the denominator, ${}_{50}C_7$.
Putting something in the numerator increases the probability.
So in a way the ${}_1C_1$ in the numerator lowers the
probability because it is a small number.
Anything else would increase the overall probability.


\end
