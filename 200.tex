\documentclass[12pt,openany]{report}
\usepackage{graphicx}
\begin{document}

\noindent
{\it George Weigt --- Geometry Homework \#13}

\section*{Page 177, problem 2.}

Given two lines and a transversal in a protractor geometry,
prove that a pair of alternate interior angles are congruent
if and only if a pair of corresponding angles are congruent.

\bigskip
\noindent
\includegraphics[scale=1.0]{102.png}

\bigskip
\noindent
Solution: First prove the implication.

\begin{itemize}

\item[]
By definition, $\angle DEB$ and $\angle CBE$ are alternate interior angles.

\item[]
By definition, $\angle ABG$ and $\angle DEB$ are corresponding angles.

\item[]
By hypothesis, $\angle ABG\cong\angle DEB$.

\item[]
By the Vertical Angle Theorem, $\angle ABG\cong\angle CBE$.

\item[]
Hence $\angle DEB\cong\angle CBE$.

\end{itemize}

\noindent
Now prove the converse.

\begin{itemize}

\item[]
By hypothesis, $\angle DEB\cong\angle CBE$.

\item[]
By the Vertical Angle Theorem, $\angle CBE\cong\angle ABG$.

\item[]
Hence $\angle DEB\cong\angle ABG$

\end{itemize}

\newpage

\section*{Page 177, problem 4.}

In a neutral geometry, if $\overleftarrow B\overrightarrow C$ is a common
perpendicular of $\overleftarrow A\overrightarrow B$
and $\overleftarrow C\overrightarrow D$, prove
that if $\ell$ is a transversal of $\overleftarrow A\overrightarrow B$
and $\overleftarrow C\overrightarrow D$ that contains the midpoint
of $\overline{BC}$ then a pair of alternate interior angles for $\ell$
are congruent.

\bigskip
\noindent
\includegraphics[scale=0.5]{104.png}

\bigskip
\noindent
Solution:

\begin{itemize}

\item[]
Let $M$ be the midpoint of $\overline{BC}$.

\item[]
Let the line $\ell$ traverse
$\overleftarrow A\overrightarrow B$
and
$\overleftarrow C\overrightarrow D$
such that $\ell\cap\overleftarrow B\overrightarrow C=\{M\}$.

\item[]
If $\ell=\overleftarrow B\overrightarrow C$ then we are done
since the alternate interior angles are right angles
and therefore congruent.

\item[]
Otherwise, let $\ell\cap\overleftarrow A\overrightarrow B=\{P\}$ and
$\ell\cap\overleftarrow C\overrightarrow D=\{Q\}$.

\item[]
By $B{-}M{-}C$ we have $P{-}M{-}Q$.

\item[]
By the Vertical Angle Theorem $\angle BMP\cong\angle CMQ$.

\item[]
By hypothesis $\overline{BM}\cong\overline{CM}$ and $\angle MBP\cong\angle MCQ$.

\item[]
Therefore by ASA $\triangle BMP\cong\triangle CMQ$.

\item[]
Therefore $\angle BPM\cong\angle CQM$.

\end{itemize}

\newpage

\section*{Page 177, problem 5.}

Give an example of the following in the Poincare Plane:
Two lines $\ell_1$ and $\ell_2$ which have a common perpendicular
and a transversal $\ell$ for which a pair of alternate interior
angles are not congruent.

\bigskip
\noindent
\includegraphics[scale=0.5]{105.png}

\bigskip
\noindent
Solution:

\begin{itemize}

\item[]
Let $\ell_1={}_0L_{\sqrt2}$ and $\ell_2={}_0L_{\sqrt5}$.

\item[]
The common perpendicular is ${}_0L$.

\item[]
Let $\ell={}_1L$.

\item[]
Then $\ell\cap\ell_1=\{(1,1)\}$ and $\ell\cap\ell_2=\{(1,2)\}$.

\item[]
Let $A=(0,\sqrt2)$, $B=(1,1)$, $C=(1,2)$.
Then $m(\angle ABC)=45$.

\item[]
Let $D=(0,\sqrt5)$, $E=(1,2)$, $F=(1,3)$.
Then $m(\angle DEF)\approx63$.

\item[]
By the Vertical Angle Theorem, the alternate interior angle
for $\angle ABC$ is congruent to $\angle DEF$.

\item[]
Hence the alternate interior angles are not congruent.

\end{itemize}

\newpage

\section*{Page 177, problem 6.}

In the Poincare Plane show that two distinct type I lines are parallel
but do not have a common perpendicular.

%\bigskip
\noindent
\includegraphics[scale=0.5]{106.png}

%\bigskip
\noindent
Solution:

\begin{itemize}

\item[]
Let $\ell_1={}_aL$ and $\ell_2={}_bL$ with $a\ne b$.
We have $\ell_1=\{(x,y)\in H\mid x=a\}$ and $\ell_2=\{(x,y)\in H\mid x=b\}$.
Consequently, $\ell_1\cap\ell_2=\emptyset$ hence $\ell_1\|\ell_2$.

\item[]
Let $\ell={}_cL_r$ be perpendicular to $\ell_1$ such that $\ell\cap\ell_1=\{B\}$.

\item[]
Let $A=(a,y_A)$, $B=(a,y_B)$ and $C\in{}_cL_r$.

\item[]
We have $T_{BA}=(0,y_A-y_B)$ and $T_{BC}=\pm(y_B,c-a)$.

\item[]
Since $\ell$ is perpendicular we have $\langle T_{BA},T_{BC}\rangle=0$.

\item[]
Hence $\pm(y_A-y_B)(c-a)=0$ which implies $c=a$.

\item[]
Therefore $\ell={}_aL_r$.

\item[]
By the same logic any line perpendicular to $\ell_2$ is ${}_bL_r$.

\item[]
Therefore $\ell_1$ and $\ell_2$ do not have a common perpendicular.

\end{itemize}

\newpage

\section*{Page 177, problem 10.}

In $H$ let $\ell={}_2L_5$ and let $P=(1,2)$.
Find a line $\ell'$ through $P$ parallel to $\ell$.

\bigskip
\noindent
Solution:

\begin{itemize}

\item[]
A vertical line through $P$ intersects $\ell$ so let us try
the type II line ${}_2L_r$.

\item[]
We have $1^2+2^2=r^2$, hence $r=\sqrt5$.

\item[]
Therefore $\ell'={}_2L_{\sqrt5}$.

\end{itemize}

\newpage

\section*{Page 177, problem 11.}

Let $\{S,L,d,m\}$ be a neutral geometry that satisfies EPP.
Prove that if $\ell_1\|\ell_2$ and $\ell$ is a traversal of $\ell_1$ and $\ell_2$,
then a pair of alternate interior angles are congruent.

\bigskip
\noindent
\includegraphics[scale=0.5]{111.png}

\bigskip
\noindent
Solution:

\begin{itemize}

\item[]
Let $\angle ABC$ and $\angle BCD$ be a pair of alternate interior angles.
\item[]
Define $E$ such that $A{-}B{-}E$ hence $D$ and $E$ are on the same side of $\ell$.

\item[]
Suppose $\angle ABC\not\cong\angle BCD$.

\item[]
Then either $\angle ABC>\angle BCD$ or $\angle ABC<\angle BCD$.

\item[]
Assume $\angle ABC>\angle BCD$.

\item[]
Then by the Linear Pair Theorem we have
$$m(\angle ABC)=180-m(\angle CBE)>m(\angle BCD)$$

\item[]
Consequently $180>m(\angle CBE)+m(\angle BCD)$.

\item[]
By Theorem 14.13 $\{S,L,d,m\}$ satsifies EFP.

\item[]
Then by EFP we have $\ell_1\not\parallel\ell_2$.

\item[]
Therefore alternate interior angles that are not congruent imply $\ell_1\not\parallel\ell_2$.

\item[]
By the contrapositive, $\ell_1\parallel\ell_2$ implies that a pair of alternate
interior angles are congruent.

\end{itemize}

\newpage

\section*{Final problem.}

Prove that the Poincare Plane does not satisfy Euclid's Fifth Postulate.

\bigskip
\noindent
\includegraphics[scale=0.5]{final.png}

\bigskip
\noindent
Solution:

\begin{itemize}

\item[]
Let $\ell_1={}_{0}L$ and $\ell_2={}_1L$ and $\ell={}_0L_2$.

\item[]
Then $\ell_1\parallel\ell_2$ and $\ell$ traverses $\ell_1$ and $\ell_2$.

\item[]
Let $A=(0,3)$, $B=(0,2)$, $C=(1,\sqrt3)$ and $D=(1,3)$.

\item[]
Then $m(\angle ABC)=90$ and $m(\angle DCB)=60$.

\item[]
We have $m(\angle ABC)+m(\angle DCB)<180$.

\item[]
However, we have $\ell_1\parallel\ell_2$ which contradicts EFP.

\item[]
Therefore the Poincare Plane does not satisfy EFP.

\end{itemize}

\end{document}
