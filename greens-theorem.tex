\subsection{Green's theorem}
\index{Green's theorem}
Green's theorem tells us that
$$\oint P\,dx+Q\,dy=\int\!\!\!\int
\left({\partial Q\over\partial x}-{\partial P\over\partial y}\right)
dx\,dy$$

\noindent
Evaluate $\oint (2x^3-y^3)\,dx+(x^3+y^3)\,dy$ around the circle
$x^2+y^2=1$ using Green's theorem.\footnote{
Wilfred Kaplan, {\it Advanced Calculus, 5th Edition,} 287.}

\medskip
\noindent
It turns out that Eigenmath cannot solve the double integral over
$x$ and $y$ directly.
Polar coordinates are used instead.

\medskip
\verb$P=2x^3-y^3$

\verb$Q=x^3+y^3$

\verb$f=d(Q,x)-d(P,y)$

\verb$x=r*cos(theta)$

\verb$y=r*sin(theta)$

\verb$defint(f*r,r,0,1,theta,0,2pi)$

$${3\over2}\pi$$

\medskip
\noindent
The $defint$ integrand is $f{*}r$ because $r\,dr\,d\theta=dx\,dy$.

\medskip
\noindent
Now let us try computing the line integral side of Green's theorem
and see if we get the same result.
We need to use the trick of converting sine and cosine to exponentials
so that Eigenmath can find a solution.

\medskip
\verb$x=cos(t)$

\verb$y=sin(t)$

\verb$P=2x^3-y^3$

\verb$Q=x^3+y^3$

\verb$f=P*d(x,t)+Q*d(y,t)$

\verb$f=circexp(f)$

\verb$defint(f,t,0,2pi)$

$${3\over2}\pi$$

