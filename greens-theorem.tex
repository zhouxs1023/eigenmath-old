\index{Green's theorem}

\noindent
Green's theorem tells us that
$$\oint P\,dx+Q\,dy=\int\!\!\!\int
\left({\partial Q\over\partial x}-{\partial P\over\partial y}\right)
dx\,dy$$

\noindent
Evaluate $\oint (2x^3-y^3)\,dx+(x^3+y^3)\,dy$ around the circle
$x^2+y^2=1$ using Green's theorem.\footnote{
Wilfred Kaplan, {\it Advanced Calculus, 5th Edition,} 287.}

\medskip
\noindent
It turns out that Eigenmath cannot solve the double integral over
$x$ and $y$ directly.
Polar coordinates are used instead.

\medskip
\verb$P=2x^3-y^3$

\verb$Q=x^3+y^3$

\verb$f=d(Q,x)-d(P,y)$

\verb$x=r*cos(theta)$

\verb$y=r*sin(theta)$

\verb$f=eval(f)$

\verb$defint(f*r,r,0,1,theta,0,2pi)$

$${3\over2}\pi$$

\medskip
\noindent
A few words of explanation are in order.
The line $f=eval(f)$ is necessary to update $f$ with the polar
substitutions for
$x$ and $y$.
The $defint$ integrand is $f{*}r$ because $r\,dr\,d\theta=dx\,dy$.

