\documentclass[11pt]{report}
\usepackage{mathrsfs}
\begin{document}

\noindent
1a. Derive the Laplace transform for the following function.
$$f(t)=tu(t)$$

\bigskip
\noindent
We do not have to worry about the unit step function $u(t)$ because all
it does is ensure that $f(t) = 0$ for $t<0$ which is required by the
Laplace transform
(Ogata p. 13.)
That leaves us with $t$ and according to the table on p. 17
the Laplace transform of $t$ is $s^{-2}$.
Hence
$${\mathscr L}[tu(t)]={1\over s^2}$$

\bigskip
\noindent
1b. Derive the Laplace transform for the following function.
$$f(t)=\sin\omega t\,u(t)$$

\bigskip
\noindent
Again, we can ignore $u(t)$.
From the table on p. 17 we obtain
$${\mathscr L}[\sin\omega t\,u(t)]={\omega\over s^2+\omega^2}$$

\bigskip
\noindent
2. Derive the Laplace transform for the following function.
$$f(t)=e^{-at}\cos\omega t\,u(t)$$

\bigskip
\noindent
The unit step function $u(t)$ is implicit in the Laplace transform tables.
By the table on p. 17 we have
$$F(s)={\mathscr L}[\cos\omega t\,u(t)]={s\over s^2+\omega^2}$$
Then by the complex shifting theorem (Lecture 2, p. 12) we have
$${\mathscr L}[e^{-at}\cos\omega t\,u(t)]=F(s+a)={s+a\over (s+a)^2+\omega^2}$$


\end{document}
