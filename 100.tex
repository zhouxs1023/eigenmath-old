1. Consider three urns, A, B, C. Urn A contains 6 white balls
and 4 black balls, urn B contains 2 white balls and 6 black balls,
and urn C is empty. Two balls are drawn at random from each of A and B
and placed in C. Then a ball is drawn at random from urn C.
What is the probability the ball drawn from urn C is black?

\bigskip
Solution: Use Bayes' Theorem. Let $b$ stand for $black$, $w$ for $white$.

\bigskip
First, compute the probabilities for drawing two balls from urn A.
$$\eqalign{
P(A,b,b)&=(4/10)(3/9)=2/15\cr
P(A,b,w)&=(4/10)(6/9)=4/15\cr
P(A,w,b)&=(6/10)(4/9)=4/15\cr
P(A,w,w)&=(6/10)(5/9)=1/3\cr
}$$
$$2/15+4/15+4/15+1/3=1$$
Next, compute the probabilities for drawing two balls from urn B.
$$\eqalign{
P(B,b,b)&=(6/8)(5/7)=15/28\cr
P(B,b,w)&=(6/8)(2/7)=3/14\cr
P(B,w,b)&=(2/8)(6/7)=3/14\cr
P(B,w,w)&=(2/8)(1/7)=1/28\cr
}$$
$$15/28+3/14+3/14+1/28=1$$
Next, compute $P(n)$ where $n$ is the number of black balls
in urn C.
$$\eqalign{
P(4)&=(2/15)(15/28)=1/14\cr
P(3)&=(2/15)(3/7)+(8/15)(15/28)=12/35\cr
P(2)&=(2/15)(1/28)+(8/15)(3/7)+(1/3)(15/28)=173/420\cr
P(1)&=(8/15)(1/28)+(1/3)(3/7)=17/105\cr
P(0)&=(1/3)(1/28)=1/84\cr
}$$
$$1/14+12/35+173/420+17/105+1/84=1$$
Using Bayes' theorem,
$$\eqalign{
P(b)&=P(4)P(b|4)+P(3)P(b|3)+P(2)P(b|2)+P(1)P(b|1)+P(0)P(b|0)\cr
&=(1/14)(1)+(12/35)(3/4)+(173/420)(1/2)+(17/105)(1/4)+(1/84)(0)\cr
&={23\over40}
}$$

\vfill
\eject

2. Consider three events A, B and C with probabilities
$P(A)=0.3$, $P(B)=0.6$, $P(C)=0.5$, $P(A\cap B)=0.2$,
$P(A\cap C)=0$, and $P(B\cap C)=0.4$. Find the
probability of the following events.

\bigskip
a) $A\cup B$
$$P(A\cup B)=P(A)+P(B)-P(A\cap B)=0.3+0.6-0.2=0.7$$

\bigskip
b) $A\cup C$
$$P(A\cup C)=P(A)+P(C)-P(A\cap C)=0.3+0.5-0=0.8$$

\bigskip
c) $A\cap B'$
$$P(A\cap B')=P(A)-P(A\cap B)=0.3-0.2=0.1$$

\bigskip
d) $A|B$
$$P(A|B)=P(A\cap B)/P(B)=0.2/0.6=0.33$$

\bigskip
e) $A'\cup B'$
$$P(A'\cup B')=P((A\cap B)')=1-P(A\cap B)=1-0.2=0.8$$

\bigskip
f) $B|A'$
$$P(B|A')=P(B\cap A')/P(A')=(P(B)-P(A\cap B))/(1-P(A))=(0.6-0.2)/(1-0.3)=0.57$$

\bigskip
g) $A|C'$
$$P(A|C')=P(A\cap C')/P(C')=(P(A)-P(A\cap C))/(1-P(C))=(0.3-0)/(1-0.5)=0.6$$

\bigskip
h) $A\cup B\cup C$
$$\eqalign{
P(A\cup B\cup C)&=P(A)+P(B)+P(C)-P(A\cap B)-P(A\cap C)-P(B\cap C)+P(A\cap B\cap C)\cr
&=0.3+0.6+0.5-0.2-0-0.4+0\cr
&=0.8\cr
}$$

\vfill
\eject

3. Find an example of two events for each of the following cases.

\bigskip
Notes:\par
Mutually exclusive: $A\cap B=\emptyset$\par
Independent: $P(A\cap B)=P(A)P(B)$\par
$P(\emptyset)=0$.

\bigskip
Example: A computer generates a random number 1-5.\par
$S=\{0,1,2,3,4,5\}$, $P(0)=0$, $P(1)=P(2)=P(3)=1/4$, $P(4)=P(5)=1/8$.

\bigskip
a) Find events that are mutually exclusive and independent.
Note that if
$A\cap B=\emptyset$ then $P(A\cap B)=0$ therefore we must have
$P(A)=0$ and/or $P(B)=0$ in order for the events to be independent.\par
$A=\{0\}$\par
$B=\{1\}$\par
Mutually exclusive: $A\cap B=\emptyset$\par
Independent: $P(A\cap B)=0=P(A)P(B)$

\bigskip
b) They are not mutually exclusive and independent.\par
$A=\{1,2\}$\par
$B=\{1,3\}$\par
Not mutually exclusive: $A\cap B=\{1\}$\par
Independent: $P(A\cap B)=1/4=(1/2)(1/2)=P(A)P(B)$

\bigskip
c) They are mutually exclusive and dependent.\par
$A=\{1\}$\par
$B=\{2\}$\par
Mutually exclusive: $A\cap B=\emptyset$\par
Dependent: $P(A\cap B)=0\ne(1/4)(1/4)=P(A)P(B)$

\bigskip
d) They are non-mutually exclusive and dependent.\par
$A=\{1,2\}$\par
$B=\{1,4\}$\par
Not mutually exclusive: $A\cap B=\{1\}$\par
Dependent: $P(A\cap B)=1/4\ne(1/2)(3/8)=P(A)P(B)$.

\vfill
\eject

4. Bowl A contains 6 red chips and 4 blue chips
Four of these 10 chips are selected at random and put
into bowl B, which contains 2 blue chips originally.
One chip is then drawn at random from bowl B.
If a red chip is selected, what is the probability
that 2 red chips and 2 blue chips were transferred
from bowl A to bowl B?

\bigskip
Let $P(n)$ be the probability of selecting $n$ red
chips from bowl A.
$$\eqalign{
P(0)&={{}_6C_0\times{}_4C_4\over{}_{10}C_4}=1/210\cr
P(1)&={{}_6C_1\times{}_4C_3\over{}_{10}C_4}=4/35\cr
P(2)&={{}_6C_2\times{}_4C_2\over{}_{10}C_4}=3/7\cr
P(3)&={{}_6C_3\times{}_4C_1\over{}_{10}C_4}=8/21\cr
P(4)&={{}_6C_4\times{}_4C_0\over{}_{10}C_4}=1/14\cr
}$$
Now determine the conditional probability for selecting
a red chip, $P(r|n)$, where $n$ is the number of red chips
from bowl A.
$$\eqalign{
P(r|0)&=0/6\cr
P(r|1)&=1/6\cr
P(r|2)&=2/6\cr
P(r|3)&=3/6\cr
P(r|4)&=4/6\cr
}$$
Now use Bayes' theorem to compute $P(2|r)$.

$$\eqalign{
P(2|r)&={P(2)P(r|2)\over\sum_nP(n)P(r|n)}\cr
&={(3/7)(2/6)\over(1/210)(0/6)+(4/35)(1/6)+(3/7)(2/6)+(8/21)(3/6)+(1/14)(4/6)}\cr
&=5/14
}$$

\vfill
\eject
5.
Event C: machine is correctly adjusted.
$$P(C)=0.8$$
Event D: a defect is observed.
$$P(D|C)=0.1$$
$$P(D|C')=0.3$$

\bigskip
5a. What is the probability of observing a good part on the first sample?
$$P(D')=1-P(C)P(D|C)-P(C')P(D|C')=1-(0.8)(0.1)-(0.2)(0.3)=0.86$$

\bigskip
5b. Compute the probability that the machine is incorrectly adjusted
after observing that the first sample is defective.
$$P(C'|D)={P(C')P(D|C')\over P(C)P(D|C)+P(C')P(D|C')}
={(0.2)(0.3)\over(0.8)(0.1)+(0.2)(0.3)}=0.4286$$

\bigskip
5c. A second part is tested and found to be good.
What is the probability that the machine is incorrectly adjusted?
Use revised probability from 5b.
$$\eqalign{
P(C)&=0.5714\cr
P(C')&=0.4286
}$$
$$\eqalign{
P(C'|D')&={P(C')P(D'|C')\over P(C)P(D'|C)+P(C')P(D'|C')}\cr
&={(0.4286)(1-0.3)\over(0.5714)(1-0.1)+(0.4286)(1-0.3)}\cr
&=0.3684\cr
}$$

\end