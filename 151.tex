\beginsection 1.

The joint probability mass function of random variables $X$
and $Y$ is defined as

$$p(x,y)=\bordermatrix{
& y=1 & y=2 & y=3\cr
x=3 & 0.0625 & 0.1875 & 0.0625\cr
x=4 & 0.1875 & 0.0000 & 0.1875\cr
x=5 & 0.0625 & 0.1875 & 0.0625\cr
}$$

\beginsection 1. (a)

Find the value of $Cov(X,Y)$.
\medskip\noindent
Solution: See lecture 52, slide 15.
$$E(X)=\sum_{x,y}xp(x,y)=3(0.3125)+4(0.3750)+5(0.3125)=4$$
$$E(Y)=\sum_{x,y}yp(x,y)=1(0.3125)+2(0.3750)+3(0.3125)=2$$
$$E(XY)=\sum_{x,y}xyp(x,y)=0.0625(3+5+9+15)+0.1875(4+6+10+12)=8$$
$$Cov(X,Y)=E(XY)-E(X)E(Y)=0$$

\beginsection 1. (b)

Find the value of $P(X\le2Y)$.
\medskip\noindent
Solution:
$$P(X\le2Y)=p(3,2)+p(3,3)+p(4,2)+p(4,3)+p(5,3)=0.5$$

\beginsection 1. (c)

Find the marginal probability mass function of $Y$.
$$p_Y(y)=\sum_xp(x,y)=\cases{
0.3125,&$y=1$\cr
0.3750,&$y=2$\cr
0.3125,&$y=3$\cr
}$$

\beginsection 1. (d)

Find the value of $P(Y{\ge}2|X{=}4)$.
\medskip\noindent
Solution:
$$P(Y{\ge}2|X{=}4)={P(Y{\ge}2\cap X{=}4)\over P(X{=}4)}
={p(4,2)+p(4,3)\over p(4,1)+p(4,2)+p(4,3)}
={0.0000+0.1875\over0.1875+0.0000+0.1875}=0.5$$

\beginsection 1. (e)

Find the value of $E(2X-3|Y{=}1)$.
\medskip\noindent
Solution:
$$p_{X|Y{=}1}(x)={p(x,y)\over p_Y(y)}\bigg|_{y=1}={p(x,1)\over0.3125}$$
$$E(2X-3|Y{=}1)=\sum_x{(2x-3)p(x,1)\over0.3125}
={(3)(0.0625)+(5)(0.1875)+(7)(0.0625)\over0.3125}=5$$

\vfill\eject

\beginsection 2.

Let the j.p.m.f. of random variables $X$ and $Y$ be defined as
$$p(x,y)={xy\over60},\qquad x=1,2,3,\quad y=1,2,3,4$$

\beginsection 2. (a)

Find the marginal p.m.f. of $X$.
\medskip\noindent
Solution: For each $x$, sum over $y$.
$$p_X(x)=\sum_yp(x,y)=\cases{
1/60+2/60+3/60+4/60=1/6,&$x=1$\cr
2/60+4/60+6/60+8/60=1/3,&$x=2$\cr
3/60+6/60+9/60+12/60=1/2,&$x=3$\cr
}$$

\beginsection 2. (b)

Find the correlation coefficient of $X$ and $Y$.
\medskip\noindent
Solution: Follow along with the example on page 23 of Note 5.
$$E(X)=\sum_{x,y}{x^2y\over60}={7\over3}$$
$$E(X^2)=\sum_{x,y}{x^3y\over60}=6$$
$$Var(X)=E(X^2)-E(X)^2={5\over9}$$
$$E(Y)=\sum_{x,y}{xy^2\over60}=3$$
$$E(Y^2)=\sum_{x,y}{xy^3\over60}=10$$
$$Var(Y)=E(Y^2)-E(Y)^2=1$$
$$E(XY)=\sum_{x,y}{x^2y^2\over60}=7$$
$$Cov(X,Y)=E(XY)-E(X)E(Y)=0$$
$$\rho={Cov(X,Y)\over\sqrt{Var(X)Var(Y)}}=0$$

\beginsection 2. (c)

Find the conditional p.m.f. of $Y$ given $X=3$.
\medskip\noindent
Solution: We need to use the marginal p.m.f. of $X$ from 2a.
$$p_{Y|X{=}3}(y)={p(x,y)\over p_X(x)}\bigg|_{x=3}={y/20\over1/2}={y\over10}$$

\beginsection 2. (d)

Find the m.g.f. of $Y$ given $X=1$.
\medskip\noindent
Solution: First compute the conditional probability mass function.
$$p_{Y|X{=}1}(y)={p(x,y)\over p_X(x)}\bigg|_{x=1}={y/60\over1/6}={y\over10}$$
Next, sum over the exponential.
$$M(t)=E(e^{tY})=\sum_y{ye^{ty}\over10}={e^t+2e^{2t}+3e^{3t}+4e^{4t}\over10}$$

\beginsection 2. (e)

Find the value of $Var(Y|X{=}3)$.
\medskip\noindent
Solution: From 2c we have
$$p_{Y|X{=}3}(y)={y\over10}$$
Next we need $E(Y)$ and $E(Y^2)$ to compute $Var$.
$$E(Y)=\sum_yy^2/10=(1+4+9+16)/10=3$$
$$E(Y^2)=\sum_yy^3/10=(1+8+27+64)/10=10$$
$$Var=E(Y^2)-E(Y)^2=1$$

\vfill\eject

\beginsection 3.

Consider the random variables $X$ and $Y$ with j.p.m.f. as
$$p(x,y)=C(10,x)C(10-x,y)(0.4)^x(0.3)^{10-x},\qquad
x,y\in Z,\quad x\ge0,\quad y\ge0,\quad x+y\le10$$

\beginsection 3. (a)

Find the m.g.f. of $X$.
\medskip\noindent
Solution: We have
$$p_X=0.4$$
$$M(t)=(1-p_X-p_Xe^t)^n=(0.6-0.4e^t)^{10}$$

\beginsection 3. (b)

Find the value of $Cov(X,Y)$.
\medskip\noindent
Solution
$$Con(X,Y)=-np_Xp_Y=-(10)(0.4)(0.3)=-1.2$$

\beginsection 3. (c)

Find the conditional p.m.f. of $X$ given $Y=4$.
\medskip\noindent
Solution:
$$p_Y(4)={10!\over4!6!}(0.3)^4(0.7)^6=0.2001$$
$$p_{X|Y{=}4}={p(x,y)\over p_Y(y)}\bigg|_{y=4}={
C(10,x)C(10-x,4)(0.4)^x(0.3)^{10-x}
\over
0.2001
},\qquad0\le x\le 6$$

\beginsection 3. (d)

Find the m.g.f. of $Y$ given $X=1$.
\medskip\noindent
$$p_X(1)=C(10,1)(0.4)^1(0.6)^9$$
$$p_{Y|X{=}1}={C(10,1)C(9,y)(0.4)^1(0.3)^y(0.3)^{9-y}\over C(10,1)(0.4)^1(0.6)^9}
={1\over(0.6)^9}C(9,y)(0.3)^y(0.3)^{9-y},
\qquad0\le y\le9$$
$$M(t)=(0.7+0.3e^t)^9$$

\beginsection 3. (e)

Find the value of $P(X{=}4|Y{=}4)$
\medskip\noindent
$$P(X{=}4|Y{=}4)={p(4,4)\over p_Y(4)}
={C(10,4)C(6,4)(0.4)^4(0.3)^6\over C(10,4)(0.3)^4(0.7)^6}=0.2938$$

\vfill
\eject

\beginsection 4.

A box contains 20 red, 10 white and 30 blue balls.
Eight balls are drawn without replacement.

\beginsection 4. (a)

Find the probability that the total number of red balls and white
balls is less than 6.
\medskip\noindent
Solution: Find the probability that the number of blue balls is 0, 1 or 2
and subtract from 1.
$$p(x)=C(30,x)C(30,8-x)/C(60,8)$$
$$1-p(0)-p(1)-p(2)=0.8729$$

\beginsection 4. (b)

Find the correlation coefficient between the number of red balls and
the number of white balls.
\medskip\noindent
Solution:
$$\rho=-\sqrt{20\times10\over(60-20)\times(60-10)}=-0.3162$$

\beginsection 4. (c)

Find the probability of two red balls in the sample when there
are four white balls in the sample.
\medskip\noindent
Solution: When there are 4 white, then the remaining 4 must have
been drawn from the population of 20 red and 30 blue.
$$P(2r|4w)=C(20,2)C(30,2)/C(50,4)=0.3589$$

\beginsection 4. (d)

Find the probability of two red balls in the sample when there
are four blue balls in the sample.
\medskip\noindent
Solution: When there are 4 blue, the remaining 4 must have been
drawn from the population of 20 red and 10 white.
$$P(2r|4b)=C(20,2)C(10,2)/C(30,4)=0.3120$$

\vfill
\eject

\beginsection 5.

Consider the random variables $X$ and $Y$ with joint probability
density function as
$$f(x,y)=\cases{
c,&for $-1<x<1$, $0<y<\sqrt{1-x^2}$\cr
0,&otherwise
}$$

\beginsection 5. (a)

Find the value of $c$.
\medskip\noindent
Solution: Since the domain is a half circle, we have
$$\int\int f(x,y)=c\pi/2$$
Therefore $c=2/\pi$.

\beginsection 5. (b)

Find the value of $P(X+Y\le1)$.
\medskip\noindent
Solution: We have the area of a quarter circle plus a triangle.
$$P(X+Y\le1)={2\over\pi}\left({\pi\over4}+{1\over2}\right)={1\over2}+{1\over\pi}=0.8183$$

\beginsection 5. (c)

Find the marginal p.d.f. of $X$.
\medskip\noindent
Solution: Sum $f(x,y)$ over $y$.
$$f_X(x)=\int_0^{\sqrt{1-x^2}}{2\over\pi}\,dy={2y\over\pi}\bigg|_0^{\sqrt{1-x^2}}
={2\sqrt{1-x^2}\over\pi}
$$

\beginsection 5. (d)

Find the conditional p.d.f. of $Y$ given $X=0$.
\medskip\noindent
Solution:
$$f_{Y|X{=}0}(y)={f(x,y)\over f_X(x)}\bigg|_{x=0}
={2/\pi\over2\sqrt{1-x^2}/\pi}\bigg|_{x=0}=1$$

\vfill
\eject

\beginsection 6.

Consider the random variables $X$ and $Y$ with j.p.d.f. as
$$f(x,y)=\cases{
24xy,&for $0<x<1$, $0<y<1$, $0<x+y<1$\cr
0,&therwise\cr
}$$

\beginsection 6. (a)

Find the value of $P(Y>0.5)$.
\medskip\noindent
Solution: The trick here is how to parameterize the domain.
The domain is the right triangle consisting of all the points below the line $x=1-y$
and greater than $x=0$ and $y=0.5$.
$$\eqalign{
P(Y>0.5)&=\int_{0.5}^1\int_0^{1-y}24xy\,dx\,dy\cr
&=\int_{0.5}^1\left(12x^2y\bigg|_{x=0}^{x=1-y}\right)\,dy\cr
&=\int_{0.5}^1(12y^3-24y^2+12y)\,dy\cr
&=(3y^4+8y^3+6y^2)\bigg|_{0.5}^1\cr
&={5\over16}
}$$

\beginsection 6. (b)

Find the value of $P(X<Y)$.
\medskip\noindent
Solution: Let $x$ range from 0 to $1/2$.
Then $y$ ranges from $x$ to $1-x$.
$$P(X<Y)=\int_0^{1/2}\int_x^{1-x}24xy\,dy\,dx={1\over2}$$

\beginsection 6. (c)

Find the marginal p.d.f. of $X$.
\medskip\noindent
Solution: Given $x$ we have to integrate over all possible values of $y$.
For each $x$, $y$ ranges from 0 to $1-x$.
$$f_X(x)=\int_0^{1-x}24xy\,dy=12x^3-24x^2+12x$$

\beginsection 6. (d)

Find the conditional p.d.f. of $X$ given $Y=0.5$.
\medskip\noindent
Solution: For each $y$, $x$ ranges from 0 to $1-y$.
$$f_Y(y)=\int_0^{1-y}24xy\,dx=12y^3-24y^2+12y$$
$$f_{X|Y{=}0.5}(x)={f(x,y)\over f_Y(y)}\bigg|_{y=0.5}
={24xy\over 12y^3-24y^2+12y}\bigg|_{y=0.5}=8x,\qquad0<x<0.5$$

\beginsection 6. (e)

Find the value of $E(Y|X{=}0.5)$.
\medskip\noindent
Solution: By the symmetry of $f(x,y)$ we obtain from 6d
$$f_{Y|X{=}0.5}(y)=8y,\qquad0<y<0.5$$
$$E(Y|X{=}0.5)=\int_0^{0.5}8y^2\,dy={1\over3}$$

\vfill
\eject

\beginsection 7.

A professor has noticed that grades on each of two quizzes have a bivariate
normal distribution with the following parameters:
$$\mu_X=75\qquad \sigma_X^2=25\qquad \mu_Y=83
\qquad \sigma_Y^2=16\qquad Cov(X,Y)=16$$

\beginsection 7. (a)

If a student receives a grade of 80 on the first quiz ($X$),
what is the probability that she will do better on the
second one ($Y$)?
\medskip\noindent
Solution:
$$\rho={Cov(X,Y)\over\sigma_X\sigma_Y}=0.8$$
$$E(Y|X{=}80)=\mu_Y+\rho(\sigma_Y/\sigma_X)(x-\mu_X)=83+(0.8)(4/5)(80-75)=86.2$$
$$\sigma^2_{Y|X}=\sigma_Y^2(1-\rho^2)=5.76$$
$$P(Y{>}80|X{=}80)=\hbox{\tt 1-NORMDIST(80,86.2,SQRT(5.76),TRUE)}=0.9951$$

\beginsection 7. (b)

If a student receives a grade of 70 on the first quiz ($X$), what is the
probability that she will do worse on the second one ($Y$)?
\medskip\noindent
$$E(Y|X{=}70)=\mu_Y+\rho(\sigma_Y/\sigma_X)(x-\mu_X)=83+(0.8)(4/5)(70-75)=79.8$$
$$P(Y{<}70|X{=}70)=\hbox{\tt NORMDIST(70,79.8,SQRT(5.76),TRUE)}=0.000022$$

\end
