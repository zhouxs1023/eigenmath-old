\beginsection Lecture 12, Exercise 3

What is the so-called fundamental theorem of arithmetic (or,
the unique factorization theorem)?
\medskip\noindent
The fundamental theorem of arithmetic states that every natural
number is either a prime number or a unique product of prime numbers.
Dr. John Baez has an interesting perspective on prime numbers that
sort of echoes the unique product idea:
``The spectrum of Z is the set of prime numbers.''
See {\tt http://math.ucr.edu/home/baez/week218.html}.
The fundamental theorem of arithmetic leads to an interesting
semantic problem. What about the number 1?
The number 1 is not a prime number.
In order to handle the problem of its factorization,
we need the empty product rule.
The empty product rule states that the product of no
factors is 1.
So the product of no numbers,
and by extension no prime numbers, is 1.

\beginsection Lecture 12, Exercise 4

What is the fundamental theorem of algebra?
\medskip\noindent
Officially, it is ``A complex polynomial has at least one complex root.''
However, there is a semantic problem since the reals are a subset
of complex numbers. So, an equivalent and far more meaningful expression is
``Every polynomial of degree $n$ has $n$ roots.''
And of course we mean this over the field of complex numbers.

\beginsection Lecture 12, Exercise 5

What is the definition of a hypergeometric series?
\medskip\noindent
A hypergeometric series is a power series in which the ratio of
successive coefficients is a rational function of the power of the term.
For example,
$$e^x=\sum_{k=0}^\infty {1\over k!}x^k$$
In this case the ratio of successive coefficients is
$${1/(k+1)!\over1/k!}={k!\over(k+1)!}={1\over k+1}$$

\beginsection Lecture 12, Exercise 6

What are the five postulates of Euclidean geometry?
\medskip\noindent
From {\tt http://en.wikibooks.org/wiki/Geometry/Five\_Postulates\_of\_Euclidean\_Geometry}
\item{1.} A straight line may be drawn from any given point to any other.
\item{2.} A straight line may be extended to any finite length.
\item{3.} A circle may be described with any given point as its center and any distance
as its radius.
\item{4.} All right angles are equal.
\item{5.} If a straight line intersects two other straight lines, and so makes the two
interior angles on one side of it together less than two right angles, then the other
straight lines will meet at a point if extended far enough on the side on which the
angles are less than two right angles.

\beginsection Lecture 12, Exercise 7

What is Gauss' AGM algorithm?
\medskip\noindent
It is an iterative algorithm. According to Mathworld,
the arithmetic geometric mean of $a$ and $b$ is calculated as follows.
$$a_0=a,\quad b_0=b$$
$$a_{n+1}={a_n+b_n\over2},\quad b_{n+1}=\sqrt{a_nb_n}$$
The calculation is repeated until $|a_n-b_n|<\delta$ where
$\delta$ is the desired precision.

\vfill
\eject

\beginsection Lecture 12, Exercise 8

Prove (4).
$$I(a,b)=I\left({a+b\over 2},\sqrt{ab}\right)\eqno(4)$$
\medskip\noindent
Solution: Start with (3)
$$I(a,b)={2\over\pi}\int_0^\infty{dt\over\sqrt{(a^2+t^2)(b^2+t^2)}}\eqno(3)$$
Let
$$u={1\over2}\left(t-{ab\over t}\right)$$
then
$${du\over dt}={1\over2}\left(1+{ab\over t^2}\right)$$
and
$$dt={2\,du\over1+{ab/t^2}}$$
Let
$$a'={a+b\over2},\qquad b'=\sqrt{ab}$$
then
$$(1+ab/t^2)^2(a^2+t^2)(b^2+t^2)=16(a'^2+u^2)(b'^2+u^2)$$
which can be checked by expanding all the terms.
Note that in changing over to $u$ the lower limit of integration becomes $-\infty$ hence
$$\eqalign{
\int_0^\infty{dt\over\sqrt{(a^2+t^2)(b^2+t^2)}}
&=\int_{-\infty}^\infty{2\,du\over(1+ab/t^2)\sqrt{(a^2+t^2)(b^2+t^2)}}\cr
&=\int_{-\infty}^\infty{du\over2\sqrt{(a'^2+u^2)(b'^2+u^2)}}\cr
&=\int_0^\infty{du\over\sqrt{(a'^2+u^2)(b'^2+u^2)}}
}$$
The last step is allowed since the radicand is symmetric about $u$.
Hence
$${2\over\pi}\int_0^\infty{dt\over\sqrt{(a^2+t^2)(b^2+t^2)}}
={2\over\pi}\int_0^\infty{du\over\sqrt{(a'^2+u^2)(b'^2+u^2)}}$$
and
$$I(a,b)=I(a',b')$$

\end
