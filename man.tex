\documentclass[12pt,openany]{report}
\title{Eigenmath Manual}
\author{George Weigt}
\date{March 18, 2007}
\pagestyle{plain}
\usepackage{graphicx}

\begin{document}
\maketitle

\newpage

\section*{The Big Picture}
In simplest terms, when you type something into Eigenmath,
it gets evaluated and the result is printed.
For a first example, let us try entering $5+7$.

\medskip
\noindent
\begin{itemize}
\item[$\scriptstyle1$]{\tt 5+7}
\item[$\scriptstyle2$]\hspace{50pt} $12$
\end{itemize}

\medskip
\noindent
No surprise here of course, but Eigenmath has other features
that should make things a little more interesting as we go along.

\medskip
\noindent
Now is a good time to say a few words about the examples in this manual.
Courier font (line 1) is used to show what should be entered.
Results are shown with some space in front (line 2).
The line numbers themselves are just for reference.
They are not entered or displayed.

\newpage

\noindent
Here is a simple example that draws the graph of $y=mx+b$.

\medskip
\noindent
\begin{itemize}
\item[$\scriptstyle1$]{\tt y=m*x+b}
\item[$\scriptstyle2$]{\tt m=1/2}
\item[$\scriptstyle3$]{\tt b=-3}
\item[$\scriptstyle4$]{\tt draw(y)}
\item[$\scriptstyle5$]
\item[]\includegraphics[scale=0.5]{1.png}
\end{itemize}

\newpage

\noindent
Now suppose that we want to draw a graph
using different values for $m$ and $b$.
We could type in lines 2--4 all over again, but it would be easier
in the long run to pause for a moment and write a script.
Then we can go back and quickly change $m$ and $b$ as many times as we want.

\medskip
\noindent
To prepare a script, click on the Edit Script button.
Then enter the script commands, one per line.

\medskip
\noindent
\begin{itemize}
\item[$\scriptstyle1$]{\tt y=m*x+b}
\item[$\scriptstyle2$]{\tt m=1/2}
\item[$\scriptstyle3$]{\tt b=-3}
\item[$\scriptstyle4$]{\tt draw(y)}
\end{itemize}

\medskip
\noindent
Next, click on the Run Script button to see the graph.

\medskip
\noindent
\begin{itemize}
\item[]\includegraphics[scale=0.5]{1.png}
\end{itemize}

\medskip
\noindent
Eigenmath runs a script by stepping through it line by line.
Each line is evaluated just like a regular command.
This continues until the end of the script is reached.
Then you can click on the Edit Script button to go back and change something.

\newpage

\noindent
You may have noticed that when we define a symbol
like $y=mx+b$, no result is printed.
Eigenmath works this way so the screen does not get cluttered
up with intermediate results.
When you do in fact want to see the value of a symbol, just enter it.

\medskip
\noindent
\begin{itemize}
\item[$\scriptstyle1$]{\tt y}
\item[$\scriptstyle2$]\hspace{50pt} $y=mx+b$
\end{itemize}

\medskip
\noindent
Symbols can also be defined as functions of one or more variables.
In the following example, we define $y$ as a function of $x$.
Just like before, we can enter $y$ to see its value and
also draw $y$.
The main difference is that now we can easily evaluate $y$ for
a specific value of $x$.

\medskip
\noindent
\begin{itemize}
\item[$\scriptstyle1$]{\tt y(x)=m*x+b}
\item[$\scriptstyle2$]{\tt y}
\item[$\scriptstyle3$]\hspace{50pt} $y=mx+b$
\item[$\scriptstyle4$]{\tt y(0)}
\item[$\scriptstyle5$]\hspace{50pt} $-3$
\end{itemize}
\noindent

\medskip
\noindent
Here is an example of a function of two variables.
\begin{itemize}
\item[$\scriptstyle1$]{\tt r(x,y)=(x{\char94}2+y{\char94}2){\char94}(1/2)}
\item[$\scriptstyle2$]{\tt r}
\item[$\scriptstyle3$]\hspace{50pt} $r=(x^2+y^2)^{1/2}$
\item[$\scriptstyle4$]{\tt r(1,2)}
\item[$\scriptstyle5$]\hspace{50pt} $5^{1/2}$
\end{itemize}

\medskip
\noindent
We are now ready to move on so
it is time to clear out the old definitions
with the following command.

\medskip
\noindent
\begin{itemize}
\item[$\scriptstyle1$]{\tt clear}
\end{itemize}

\medskip
\noindent
Clicking on the Clear button does the same thing.
It should also be pointed out that Eigenmath automatically does a clear before
running a script.

\newpage

\section*{Vectors and matrices}
Vectors and matrices are entered like this.
\begin{itemize}
\item[$\scriptstyle1$]{\tt X=(x1,x2,x3,x4)}
\item[$\scriptstyle2$]{\tt A=((1,2,3),(4,5,6),(7,8,9))}
\item[$\scriptstyle3$]{\tt A}
\item[$\scriptstyle4$]\hspace{50pt} $\left(\matrix{1&2&3\cr4&5&6\cr7&8&9\cr}\right)$
\end{itemize}

\medskip
\noindent
Eigenmath's dot function is used to multiply vectors and matrices.
The following example shows how to use dot and inv to solve for $\bf X$ in $\bf AX=B$.
%\footnote{
%Numerical values from {\it HP-15C Owner's Handbook,} p. 138.}
\begin{itemize}
\item[$\scriptstyle1$]{\tt A=((3.8,7.2),(1.3,-0.9))}
\item[$\scriptstyle2$]{\tt B=(16.5,-22.1)}
\item[$\scriptstyle3$]{\tt X=dot(inv(A),B)}
\item[$\scriptstyle4$]{\tt X}
\item[$\scriptstyle5$]\hspace{50pt} $\left(\matrix{-11.2887\cr8.24961}\right)$
\end{itemize}

\newpage

\label{d}

\section*{Derivatives}
The function d($f,x$) returns the derivative of $f$ with respect to $x$.
The $x$ can be omitted for expressions in $x$.
\begin{itemize}
\item[$\scriptstyle1$]{\tt d(x{\char94}2)}
\item[$\scriptstyle2$]\hspace{50pt} $2x$
\end{itemize}

\noindent
Multi-derivatives can be obtained by extending the argument list.
\begin{itemize}
\item[$\scriptstyle1$]{\tt r=sqrt(x{\char94}2+y{\char94}2)}
\item[$\scriptstyle1$]{\tt d(r,x,y)}
\item[$\scriptstyle2$]\hspace{50pt} $\displaystyle{-{xy\over(x^2+y^2)^{3/2}}}$
\end{itemize}

\noindent
A gradient can be obtained by using a vector for the second argument.
\begin{itemize}
\item[$\scriptstyle1$]{\tt r=sqrt(x{\char94}2+y{\char94}2)}
\item[$\scriptstyle2$]{\tt d(r,(x,y))}
\item[$\scriptstyle3$]\hspace{50pt}
$\left(\matrix{
\displaystyle{{x\over(x^2+y^2)^{1/2}}}\cr
\cr
\displaystyle{{y\over(x^2+y^2)^{1/2}}}\cr
}\right)$
\end{itemize}

\newpage

\section*{Derivatives (cont'd)}
Here is a trick.
The function $f$ in d($f$) does not have to be defined.
Eigenmath will check the argument list
of $f$ to figure out what to do.
\begin{itemize}
\item[$\scriptstyle1$]{\tt d(f(x),(x,y))}
\item[$\scriptstyle2$]\hspace{50pt}
$\left(\matrix{
\partial(f(x),x)\cr
\cr
0\cr
}\right)$
\item[$\scriptstyle3$]{\tt d(f(),(x,y))}
\item[$\scriptstyle4$]\hspace{50pt}
$\left(\matrix{
\partial(f(),x)\cr
\cr
\partial(f(),y)\cr
}\right)$
\end{itemize}
As the second example shows,
if the argument list is empty then $f$ is assumed to depend
on any variable that d encounters.

\medskip
\noindent
This ``generic'' function property is useful for experimenting with
div, grad, curl, etc.
%\footnote{
%Schey, H. M. {\it Div, grad, curl, and all that:
%an informal text on vector calculus.} New York: W. W. Norton and Company, Inc., 1992.}
There is just one more detail to add, and that has to do with vector functions.
In the previous examples, $f$ was a scalar function.
In the following example, line 1 shows how to create a vector function.
The rest of the example computes the gradient of $\bf F$.
\begin{itemize}
\item[$\scriptstyle1$]{\tt F=(F1(),F2())}
\item[$\scriptstyle2$]{\tt X=(x1,x2)}
\item[$\scriptstyle3$]{\tt d(F,X)}
\item[$\scriptstyle4$]\hspace{50pt}
$\left(\matrix{
\partial(F_1(),x_1) & \partial(F_1(),x_2)\cr
\cr
\partial(F_2(),x_1) & \partial(F_2(),x_2)\cr
}\right)$
\end{itemize}

\newpage

\label{integral}

\section*{Integrals}
The function integral($f,x$) returns the integral of $f$ with respect to $x$.
The $x$ can be omitted for expressions in $x$.
A multi-integral can be obtained by extending the argument list.
\begin{itemize}
\item[$\scriptstyle1$]{\tt integral(x{\char94}2)}
\item[$\scriptstyle2$]\hspace{50pt} ${1\over3}x^3$
\item[$\scriptstyle3$]{\tt integral(x{\char94}2,x,x)}
\item[$\scriptstyle4$]\hspace{50pt} ${1\over12}x^4$
\end{itemize}
The eval function can be used to compute definite integrals.
The following example computes the integral of $x^2$
over a half circle.
\begin{itemize}
\item[$\scriptstyle1$]{\tt I=integral(x{\char94}2,y)}
\item[$\scriptstyle2$]{\tt I=eval(I,y,sqrt(1-x{\char94}2))-eval(I,y,0)}
\item[$\scriptstyle3$]{\tt I=integral(I,x)}
\item[$\scriptstyle4$]{\tt eval(I,x,1)-eval(I,x,-1)}
\item[$\scriptstyle5$]\hspace{50pt} ${1\over8}\pi$
\end{itemize}

\newpage

\section*{That's all for now}
Please check the following web site for more examples.

\medskip
{\tt eigenmath.blogspot.com}

\newpage

\section*{abs}
abs($x$) returns the absolute value or vector length of $x$.
The mag function should be used for complex $x$.
\begin{itemize}
%\item[$\scriptstyle1$]{\tt abs(b-a)}
%\item[$\scriptstyle2$]\hspace{50pt} $\mathop{\hbox{abs}}(a-b)$
\item[$\scriptstyle1$]{\tt P=(x,y)}
\item[$\scriptstyle2$]{\tt abs(P)}
\item[$\scriptstyle3$]\hspace{50pt} $(x^2+y^2)^{1/2}$
\end{itemize}

\section*{adj}
adj($m$) returns the adjunct of matrix $m$.
\begin{itemize}
\item[$\scriptstyle1$]{\tt A=((a,b),(c,d))}
\item[$\scriptstyle2$]{\tt adj(A)}
\item[$\scriptstyle3$]\hspace{50pt} $\left[\matrix{d & -b\cr -c & a\cr}\right]$
\end{itemize}

\section*{and}
and($a,b,\ldots$) returns the logical ``and'' of predicate expressions.

\section*{arccos}
arccos($x$) returns the inverse cosine of $x$.

\section*{arccosh}
arccosh($x$) returns the inverse hyperbolic cosine of $x$.

\section*{arcsin}
arcsin($x$) returns the inverse sine of $x$.

\section*{arcsinh}
arcsinh($x$) returns the inverse hyperbolic sine of $x$.

\section*{arctan}
arcttan($x$) returns the inverse tangent of $x$.

\section*{arctanh}
arctanh($x$) returns the inverse hyperbolic tangent of $x$.

\section*{arg}
arg($z$) returns the angle of complex $z$.
%\begin{itemize}
%\item[$\scriptstyle1$]{\tt arg(1+exp(i*pi/3))}
%\item[$\scriptstyle2$]\hspace{50pt} ${1\over6}\pi$
%\end{itemize}

\section*{ceiling}
ceiling($x$) returns the smallest integer not less than $x$.

\section*{check}
check($x$) In a script, if the predicate $x$ is true then continue, else stop.

\section*{choose}
choose($n,k$) returns $\displaystyle\left({n \atop k}\right)$
%\begin{itemize}
%\item[$\scriptstyle1$]{\tt choose(n,k)}
%\item[$\scriptstyle2$]\hspace{50pt} $\displaystyle{n!\over k!(-k+n)!}$
%\end{itemize}

\section*{coeff}
coeff($p,x,n$) returns the coefficient of $x^n$ in polynomial $p$.

\section*{cofactor}
cofactor($m,i,j$) returns of the cofactor of matrix $m$ with respect to row $i$ and column $j$.

\section*{conj}
conj($z$) returns the complex conjugate of $z$.
%\begin{itemize}
%\item[$\scriptstyle1$]{\tt conj(3+4*i)}
%\item[$\scriptstyle2$]\hspace{50pt} $3-4i$
%\end{itemize}

\section*{contract}
contract($a,i,j$) returns tensor $a$ summed over indices $i$ and $j$.

\section*{cos}
cos($x$) returns the cosine of $x$.
%If $x$ is a floating point number then $\cos(x)$ is evaluated numerically.

\section*{cosh}
cosh($x$) returns the hyperbolic cosine of $x$.
%If $x$ is a floating point number then $\cosh(x)$ is evaluated numerically.

\section*{d}
d($f,x$) returns the derivative of $f$ with respect to $x$.
Please see page \pageref{d} for additional details.

\section*{deg}
deg($p,x$) returns the degree of polynomial $p$ in $x$.
%\begin{itemize}
%\item[$\scriptstyle1$]{\tt p=(x-a){\char94}5}
%\item[$\scriptstyle2$]{\tt deg(p,x)}
%\item[$\scriptstyle3$]\hspace{50pt} $5$
%\end{itemize}

\section*{denominator}
denominator($x$) returns the denominator of expression $x$.
%\begin{itemize}
%\item[$\scriptstyle1$]{\tt denominator(a/b+b/a)}
%\item[$\scriptstyle2$]\hspace{50pt} $ab$
%\end{itemize}

\section*{det}
det($m$) returns the determinant of matrix $m$.

\section*{display}
display($x$) prints expression $x$.

\section*{do}
do($a,b,\ldots$) evaluates the argument list from left to right.
Returns the result of the last argument.

\section*{dot}
dot($a,b,\ldots$) returns the dot product of tensors.
%\begin{itemize}
%\item[$\scriptstyle1$]{\tt A=(a,b)}
%\item[$\scriptstyle2$]{\tt B=(c,d)}
%\item[$\scriptstyle3$]{\tt dot(A,B)}
%\item[$\scriptstyle4$]\hspace{50pt} $ac+bd$
%\end{itemize}

\section*{draw}
draw($f,x$) draws the function $f$ with respect to $x$.

\section*{erf}
erf($x$) returns the error function of $x$.

\section*{erfc}
erf($x$) returns the complementary error function of $x$.

\section*{eval}
eval($f,x,n$) returns $f$ evaluated at $x=n$.
Please see page \pageref{integral} for an example.

\section*{exp}
exp($x$) returns $e^x$.

\section*{expcos}
expcos($x$) returns the cosine of $x$ in exponential form.
%\begin{itemize}
%\item[$\scriptstyle1$]{\tt expcos(x)}
%\item[$\scriptstyle2$]\hspace{50pt} ${1\over2}\exp(-ix)+{1\over2}\exp(ix)$
%\end{itemize}

\section*{expsin}
expsin($x$) returns the sine of $x$ in exponential form.
%\begin{itemize}
%\item[$\scriptstyle1$]{\tt expsin(x)}
%\item[$\scriptstyle2$]\hspace{50pt} ${1\over2}i\exp(-ix)-{1\over2}i\exp(ix)$
%\end{itemize}

\section*{factor}
factor($n$) factors the integer $n$.
\begin{itemize}
\item[$\scriptstyle1$]{\tt factor(12345)}
\item[$\scriptstyle2$]\hspace{50pt} $3\times 5\times 823$
\end{itemize}
%
factor($p,x$) factors polynomial $p$ in $x$.
The last argument can be omitted for polynomials in $x$.
\begin{itemize}
\item[$\scriptstyle1$]{\tt factor(125*x{\char94}3-1)}
\item[$\scriptstyle2$]\hspace{50pt} $(5x-1)(25x^2+5x+1)$
\end{itemize}

\section*{factorial}
For example,
\begin{itemize}
\item[$\scriptstyle1$]{\tt 10!}
\item[$\scriptstyle2$]\hspace{50pt} $3628800$
\end{itemize}

\section*{filter}
filter($f,a,b,\ldots$) returns $f$ with terms involving $a$, $b$, etc. removed.
\begin{itemize}
\item[$\scriptstyle1$]{\tt 1/a+1/b+1/c}
\item[$\scriptstyle2$]\hspace{50pt} $\displaystyle{{1\over a}+{1\over b}+{1\over c}}$
\item[$\scriptstyle3$]{\tt filter(last,a)}
\item[$\scriptstyle4$]\hspace{50pt} $\displaystyle{{1\over b}+{1\over c}}$
\end{itemize}

\section*{float}
float($x$) converts $x$ to a floating point value.
\begin{itemize}
\item[$\scriptstyle1$]{\tt sum(n,0,20,(-1/2){\char94}n)}
\item[$\scriptstyle2$]\hspace{50pt} $699051\over1048576$
\item[$\scriptstyle3$]{\tt float(last)}
\item[$\scriptstyle4$]\hspace{50pt} 0.666667
\end{itemize}

\section*{floor}
floor($x$) returns the largest integer not greater than $x$.

\section*{for}
for($i,j,k,a,b,\ldots$) For $i$ equals $j$ through $k$ evaluate $a$, $b$, etc.
\begin{itemize}
\item[$\scriptstyle1$]{\tt x=0}
\item[$\scriptstyle2$]{\tt y=2}
\item[$\scriptstyle3$]{\tt for(k,1,9,x=sqrt(2+x),y=2*y/x)}
\item[$\scriptstyle4$]{\tt float(y)}
\item[$\scriptstyle5$]\hspace{50pt} 3.14159
\end{itemize}

\section*{gcd}
gcd($a,b,\ldots$) returns the greatest common divisor.

\section*{hermite}
hermite($x,n$) returns the $n$th Hermite polynomial in $x$.

\section*{hilbert}
hilbert($n$) returns a Hilbert matrix of order $n$.

\section*{imag}
imag($z$) returns the imaginary part of complex $z$.

\section*{inner}
inner($a,b,\ldots$) returns the inner product of tensors.
Same as the dot product.

\section*{integral}
integral($f,x$) returns the integral of $f$ with respect to $x$.
Please see page \pageref{integral} for additional details.

\section*{inv}
inv($m$) returns the inverse of matrix $m$.

\section*{isprime}
isprime($n$) returns 1 if $n$ is prime, zero otherwise.
%\begin{itemize}
%\item[$\scriptstyle1$]{\tt isprime(2{\char94}53-111)}
%\item[$\scriptstyle2$]\hspace{50pt} 1
%\end{itemize}

\section*{laguerre}
laguerre($x,n,a$) returns the $n$th Laguerre polynomial in $x$.
If $a$ is omitted then $a=0$ is used.

\section*{lcm}
lcm($a,b,\ldots$) returns the least common multiple.

\section*{legendre}
legendre($x,n,m$) returns the $n$th Legendre polynomial in $x$.
If $m$ is omitted then $m=0$ is used.

\section*{log}
log($x$) returns the natural logarithm of $x$.

\section*{mag}
mag($z$) returns the magnitude of complex $z$.

\section*{mod}
mod($a,b$) returns the remainder of $a$ divided by $b$.

\section*{not}
not($x$) negates the result of predicate expression $x$.

\section*{numerator}
numerator($x$) returns the numerator of expression $x$.
%\begin{itemize}
%\item[$\scriptstyle1$]{\tt numerator(a/b+b/a)}
%\item[$\scriptstyle2$]\hspace{50pt} $a^2+b^2$
%\end{itemize}

\section*{or}
or($a,b,\ldots$) returns the logical ``or'' of predicate expressions.

\section*{outer}
outer($a,b,\ldots$) returns the outer product of tensors.

\section*{polar}
polar($z$) converts complex $z$ to polar form.

\section*{prime}
prime($n$) returns the $n$th prime number, $1\le n\le10{,}000$.

\section*{product}
product($i,j,k,f$) returns $\displaystyle\prod_{i=j}^k f$

\section*{quote}
quote($x$) returns expression $x$ unevaluated.

\section*{quotient}
quotient($p,q,x$) returns the quotient of polynomials in $x$.

\section*{rank}
rank($a$) returns the number of indices that tensor $a$ has.

\section*{rationalize}
rationalize($x$) puts everything over a common denominator.
\begin{itemize}
\item[$\scriptstyle1$]{\tt rationalize(a/b+b/a)}
\item[$\scriptstyle2$]\hspace{50pt} $\displaystyle{a^2+b^2\over ab}$
\end{itemize}

\section*{real}
real($z$) returns the real part of complex $z$.

\section*{rect}
rect($z$) returns complex $z$ in rectangular form.

\section*{roots}
roots($p,x$) returns the values of $x$ such that the polynomial $p(x)=0$.
The polynomial should be factorable over integers.

\section*{simplify}
simplify($x$) returns $x$ in a simpler form.

\section*{sin}
sin($x$) returns the sine of $x$.

\section*{sinh}
sinh($x$) returns the hyperbolic sine of $x$.

\section*{sqrt}
sqrt($x$) returns the square root of $x$.

\section*{stop}
In a script, it does what it says.

\section*{subst}
subst($a,b,c$) substitutes $a$ for $b$ in $c$ and returns the result.

\section*{sum}
sum($i,j,k,f$) returns $\displaystyle\sum_{i=j}^k f$

\section*{tan}
tan($x$) returns the tangent of $x$.

\section*{tanh}
tanh($x$) returns the hyperbolic tangent of $x$.

\section*{taylor}
taylor($f,x,n,a$) returns the Taylor expansion of $f$ of $x$ at $a$.
The argument $n$ is the degree of the expansion.
If $a$ is omitted then $a=0$ is used.
\begin{itemize}
\item[$\scriptstyle1$]{\tt taylor(1/cos(x),x,4)}
\item[$\scriptstyle2$]\hspace{50pt} ${5\over24}x^4+{1\over2}x^2+1$
\end{itemize}

\section*{test}
test($a,b,c,d,\ldots$)
If $a$ is true then $b$ is returned else if $c$ is true then $d$ is returned, etc.
If the number of arguments is odd then the last argument is returned when all else fails.

\section*{trace}
trace($m$) returns the trace of matrix $m$.

\section*{transpose}
transpose($a,i,j$) returns the transpose of tensor $a$ with respect to indices $i$ and $j$.
If $i$ and $j$ are omitted then 1 and 2 are used.
Hence a matrix can be transposed with a single argument.
\begin{itemize}
\item[$\scriptstyle1$]{\tt A=((a,b),(c,d))}
\item[$\scriptstyle2$]{\tt transpose(A)}
\item[$\scriptstyle3$]\hspace{50pt} $\left[\matrix{a & c\cr b & d\cr}\right]$
\end{itemize}



\end{document}
