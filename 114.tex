\beginsection 2.

Consider the continuous random variable $X$ with probability
density function as
$$f(x)=\cases{
0.5(1+x^2), & $-1<x<0$\cr
0.5(1-x^2), & $0<x<1$\cr
0, & otherwise\cr
}$$

\bigskip
2a) Find the distribution of $X$.
Solution: The trick is, for $0<t<1$ we need to add $F(0)$ to the integral.
For a quick check, verify that $F(1)=1$.
$$F(t)={1\over2}\int_{-1}^t(1+x^2)\,dx
=\left({x\over2}+{x^3\over6}\right)\bigg|_{-1}^t
={t\over2}+{t^3\over6}+{2\over3},\qquad-1<t<0$$
$$F(t)=F(0)+{1\over2}\int_0^t(1-x^2)\,dx
={2\over3}+\left({x\over2}-{x^3\over6}\right)\bigg|_0^t
={t\over2}-{t^3\over6}+{2\over3},\qquad0<t<1$$
Hence
$$F(t)=\cases{
0, & $t<-1$\cr
\cr
\displaystyle{{t\over2}+{t^3\over6}+{2\over3}}, & $-1\le t < 0$\cr
\cr
\displaystyle{{t\over2}-{t^3\over6}+{2\over3}}, & $0\le t < 1$\cr
\cr
1, & $1\le t$\cr
}$$

\bigskip
2b) Find the variance of $X$.
Solution: First find $\mu=\int xf(x)$ then find $\sigma^2=\int(x-\mu)^2f(x)$.
The trick is we have to integrate $f(x)$ piecewise over $R.$
$$\eqalign{
\mu&={1\over2}\int_{-1}^0x(1+x^2)\,dx+{1\over2}\int_0^1x(1-x^2)\,dx\cr
&={1\over2}\left({x^2\over2}+{x^4\over4}\right)\bigg|_{-1}^0
+{1\over2}\left({x^2\over2}-{x^4\over4}\right)\bigg|_0^1\cr
&={1\over2}\left(-{1\over2}-{1\over4}+{1\over2}-{1\over4}\right)\cr
&=-{1\over4}
}$$
$$\eqalign{
\sigma^2&={1\over2}\int_{-1}^0(x+1/4)^2(1+x^2)\,dx+{1\over2}\int_0^1(x+1/4)^2(1-x^2)\,dx\cr
&={1\over2}\int_{-1}^0\left(x^4+{1\over2}x^3+{17\over16}x^2+{1\over2}x+{1\over16}\right)\,dx
+{1\over2}\int_0^1\left(-x^4-{1\over2}x^3+{15\over16}x^2+{1\over2}x+{1\over16}\right)\,dx\cr
&={1\over2}\left({1\over5}x^5+{1\over8}x^4+{17\over48}x^3+{1\over4}x^2+{1\over16}x\right)
\bigg|_{-1}^0
+{1\over2}\left(-{1\over5}x^5-{1\over8}x^4+{15\over48}x^3+{1\over4}x^2+{1\over16}x\right)
\bigg|_0^1\cr
&={1\over2}\left({1\over5}-{1\over8}+{17\over48}-{1\over4}+{1\over16}
-{1\over5}-{1\over8}+{15\over48}+{1\over4}+{1\over16}\right)\cr
&={13\over48}\cr
&=0.2708
}$$
Oops, it would have been a lot easier to calculate $E[X^2]$
then $\sigma^2=E[X^2]-(E[X])^2$.

\vfill
\eject

\bigskip
2c) Find the third moment about the mean of $X$.
Solution: Decompose $\int(x-\mu)^3f(x)\,dx$ into a manageable form.
$$\eqalign{
E[(X-\mu)^3]&=\int(x-\mu)^3f(x)\,dx\cr
&=\int(x^3-3\mu x^2+3\mu^2x-\mu^3)f(x)\,dx\cr
&=\int x^3f-3\mu\int x^2f+3\mu^2\int xf-\mu^3\int f\cr
&=E[X^3]-3E[X]E[X^2]+3(E[X])^2E[X]-(E[X])^3\cr
&=E[X^3]-3E[X]E[X^2]+2(E[X])^3
}$$
Find $E[X^3]$ by integrating piecewise over $R$.
$$\eqalign{
E[X^3]&={1\over2}\int_{-1}^0x^3(1+x^2)\,dx+{1\over2}\int_0^1x^3(1-x^2)\,dx\cr
&={1\over2}\left({1\over4}x^4+{1\over6}x^6\right)\bigg|_{-1}^0
+{1\over2}\left({1\over4}x^4-{1\over6}x^6\right)\bigg|_0^1\cr
&={1\over2}\left(-{1\over4}-{1\over6}+{1\over4}-{1\over6}\right)\cr
&=-{1\over6}
}$$
From the previous problem we have $E[X]=-1/4$, $E[X^2]=13/48+(-1/4)^2=1/3$.
Hence
$$\eqalign{
E[(X-\mu)^3]&=E[X^3]-3E[X]E[X^2]+2(E[X])^3\cr
&=-1/6-3(-1/4)(1/3)+2(-1/4)^3\cr
&=5/96\cr
&=0.0521\cr
}$$

\vfill
\eject

2d) Find the median of $X$. Solution: Find $t$ such that $F(t)=0.5$.
$$F(t)=\cases{
0, & $t<-1$\cr
\cr
\displaystyle{{t\over2}+{t^3\over6}+{2\over3}}, & $-1\le t < 0$\cr
\cr
\displaystyle{{t\over2}-{t^3\over6}+{2\over3}}, & $0\le t < 1$\cr
\cr
1, & $1\le t$\cr
}$$
First try $-1\le t < 0$.
$${t\over2}+{t^3\over6}+{2\over3}={1\over2}$$
Use the cubic root calculator at {\tt http://home.att.net/$\sim$srschmitt/script\_cubic.html}
to find
$$t=-0.32219$$

\end