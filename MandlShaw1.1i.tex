\beginsection{Mandl and Shaw Problem 1.1(i)}

The radiation field inside a cubic enclosure, which contains no charges,
is specified by the state
$$|\psi\rangle=\exp(-\hbox{$1\over2$}|c|^2)\sum_{n=0}^\infty{c^n\over\sqrt{n!}}
\,|n\rangle$$
where $c$ is any complex number and $|n\rangle$ is the state
in which there are $n$ photons.
Show that $\langle \psi|\psi\rangle=1$.

\beginsection{Solution}

Let's review what is going on here.
The state $|\psi\rangle$ is a mixture of photon states $|n\rangle$ where $n$
is the number of photons.
Each photon state $|n\rangle$ is a basis vector with infinite dimension, like
this:
$$
|0\rangle=\left[\matrix{1\cr0\cr0\cr0\cr0\cr\vdots}\right]
\quad
|1\rangle=\left[\matrix{0\cr1\cr0\cr0\cr0\cr\vdots}\right]
\quad
|2\rangle=\left[\matrix{0\cr0\cr1\cr0\cr0\cr\vdots}\right]
\quad
|3\rangle=\left[\matrix{0\cr0\cr0\cr1\cr0\cr\vdots}\right]
$$
Fortunately we don't have to deal with these vectors directly,
we only have to compute vector dot products.
Since the vectors are orthonormal, the dot product is either 0 or 1.
Here are a few examples.
$$\matrix{
\langle0|0\rangle=1&\langle0|1\rangle=0&\langle0|2\rangle=0\cr
\langle1|0\rangle=0&\langle1|1\rangle=1&\langle1|2\rangle=0\cr
\langle2|0\rangle=0&\langle2|1\rangle=0&\langle2|2\rangle=1\cr
}$$
%
OK, now let's go ahead and do the multiply and see what happens.
We have
$$\eqalign{
\langle\psi|\psi\rangle&=
\left(\exp(-\hbox{$1\over2$}|c|^2)\sum_{n=0}^\infty{(c^*)^n\over\sqrt{n!}}\,\langle n|\right)
\left(\exp(-\hbox{$1\over2$}|c|^2)\sum_{n=0}^\infty{c^n\over\sqrt{n!}}\,|n\rangle\right)
\cr
&=\exp(-|c|^2)\left[{(c^*)^0c^0\over0!}\langle0|0\rangle
+{(c^*)^1c^1\over1!}\langle1|1\rangle
+{(c^*)^2c^2\over2!}\langle2|2\rangle+\cdots\right]\cr
&=\exp(-|c|^2)\exp(|c|^2)\cr
&=1
}$$
Now we see what the $\sqrt{n!}$ is all about.
It's all cooked up so that after we multiply we get a series expansion
for an exponential.
Note also that we used the relation
$$(c^*)^nc^n=(c^*\times c^* \times c^*\times\cdots)(c\times c\times c\times\cdots)
=(c^*c)^n=|c|^{2n}$$
The following script demonstrates the above calculation.
The main idea is that we limit $n$ to a finite value so we can have a
representation for $|n\rangle$.

\vfill
\eject

{\tt\obeylines

clear()
\# This is the number of terms to use for psi.
N = 6
\# This returns basis vector n.
I = unit(N)
ket(n) = I[n + 1]
\# This is an arbitrary complex number.
c = r exp(i phi)
\# Calculate psi.
psi = exp(-1/2 r\char94 2) sum(n, 0, N - 1, c\char94 n / sqrt(n!)~ket(n))
\# Calculate < psi | psi >
P = dot(conj(psi), psi)
P = condense(P)
display(P)
"Replace the truncated series with an exponential."
EXP(x) = eval(taylor(exp(x), x, N - 1))
P = subst(exp(r\char94 2), EXP(r\char94 2), P)
P = eval(P)
display(P)

}

