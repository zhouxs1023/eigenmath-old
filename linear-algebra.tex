\section*{Linear algebra}
\index{linear algebra}
$dot$ is used to multiply vectors and matrices.
The following example shows how to use $dot$ and $inv$ to solve for
$\bf X$ in $\bf AX=B$.

\medskip
{\tt A=((3.8,7.2),(1.3,-0.9))}

{\tt B=(16.5,-22.1)}

{\tt X=dot(inv(A),B)}

{\tt X}

$$\left(\matrix{-11.2887\cr8.24961}\right)$$

\medskip
\noindent
One might wonder why the $dot$ function is necessary.
Why not simply use $X=inv(A)*B$ like scalar multiplication?
The reason is that the software normally reorders factors internally to optimize processing.
For example, $inv(A)*B$ in symbolic form is changed to $B*inv(A)$ internally.
Since the dot product is not commutative, this reordering would give the wrong result.
Using a function to do the multiply avoids the problem because
function arguments are not reordered.

\medskip
\noindent
It should be noted that $dot$ can have more than two arguments.
For example, $dot(A,B,C)$ can be used for the dot product of three tensors.

\newpage

\noindent
The following example demonstrates the relation
${\bf A}^{-1}=\mathop{\rm adj}{\bf A}/\mathop{\rm det}{\bf A}$.

\medskip
\verb$A=((a,b),(c,d))$

\medskip
\verb$inv(A)$
$$\left(\matrix{
\displaystyle{d\over ad-bc} & \displaystyle{-{b\over ad-bc}}\cr
\cr
\displaystyle{-{c\over ad-bc}} & \displaystyle{a\over ad-bc}\cr
}\right)$$

\medskip
\verb$adj(A)$
$$\left(\matrix{
d & -b\cr
-c & a\cr
}\right)$$

\medskip
\verb$det(A)$
$$ad-bc$$

\medskip
\verb$inv(A)-adj(A)/det(A)$
$$\left(\matrix{
0 & 0\cr
0 & 0\cr
}\right)$$

\medskip
\noindent
Sometimes a calculation will be simpler if it can be reorganized to use $adj$ instead of $inv$.
The main idea is to try to prevent the determinant from appearing as a divisor.
For example, suppose for matrices $\bf A$ and $\bf B$ you want to check that
$${\bf A}-{\bf B}^{-1}=0$$
Depending on the complexity of $\mathop{\rm det}\bf B$, the software
may not be able to find a simplification that yields zero.
Should that occur, the following alternative can be tried.
$$(\mathop{\rm det}{\bf B})\cdot{\bf A}-\mathop{\rm adj}{\bf B}=0$$

\newpage

\noindent
The adjunct of a matrix is related to the cofactors as follows.

\medskip
\verb$A=((a,b),(c,d))$

\verb$C=((0,0),(0,0))$

\verb$C[1,1]=cofactor(A,1,1)$

\verb$C[1,2]=cofactor(A,1,2)$

\verb$C[2,1]=cofactor(A,2,1)$

\verb$C[2,2]=cofactor(A,2,2)$

\verb$C$

$$C=\left(\matrix{d&-c\cr -b&a}\right)$$

\verb$adj(A)-transpose(C)$

$$\left(\matrix{0&0\cr0&0\cr}\right)$$
