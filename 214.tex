\noindent
6.2 Show that if $\alpha$ and $\beta$ are Dedekind cuts, then so is
$\alpha+\beta=\{r_1+r_2:r_1\in\alpha, r_2\in\beta\}$.

\bigskip
\noindent
The requirements for $\alpha+\beta$ to be a Dedekind cut are

\medskip
\item{(i)} $\alpha+\beta\ne Q$ and $\alpha+\beta$ is not empty,
\item{(ii)} if $r\in\alpha+\beta$, $s\in Q$ and $s<r$, then $s\in\alpha+\beta$,
\item{(iii)} $\alpha+\beta$ contains no largest rational.

\medskip
\noindent
First show that $\alpha+\beta$ satisfies the two conditions in (i).

\medskip
\item{1.} Let $a,b\in Q$
\item{2.} $a,b\in Q$ implies $a+b\in Q$
\item{3.} Let $\alpha=a^*$ and $\beta=b^*$
\item{4.} $\alpha=a^*$ and $\beta=b^*$ implies $\alpha+\beta=a^*+b^*$
\item{5.} $\alpha+\beta=a^*+b^*$ implies $\alpha+\beta=(a+b)^*$
by exercise 6.1 (c)
\item{6.} By definition, $a+b\not\in(a+b)^*$
\item{7.} $a+b\not\in(a+b)^*$ implies $a+b\not\in\alpha+\beta$
\item{8.} $a+b\not\in\alpha+\beta$ and $a+b\in Q$ implies $\alpha+\beta\ne Q$

\medskip
\item{1.} Let $s<a$ and $t<b$
\item{2.} $s<a$ and $t<b$ implies $s+t<a+b$
\item{3.} $s+t<a+b$ implies $s+t\in(a+b)^*$
\item{4.} $s+t\in(a+b)^*$ implies $s+t\in\alpha+\beta$
\item{5.} $s+t\in\alpha+\beta$ implies $\alpha+\beta\ne\emptyset$

\medskip
\noindent
Next show that $\alpha+\beta$ satisfies condition (ii).

\medskip
\item{1.} Let $r\in\alpha+\beta$
\item{2.} $r\in\alpha+\beta$ implies $r\in(a+b)^*$
\item{3.} $r\in(a+b)^*$ implies $r<a+b$
\item{4.} Let $s\in Q$ and $s<r$
\item{5.} $s<r$ implies $s<a+b$
\item{6.} $s<a+b$ implies $s\in(a+b)^*$
\item{7.} $s\in(a+b)^*$ implies $s\in\alpha+\beta$

\end

