\item{\S1.} Leonhard Euler 1707--1783

\itemitem{1.} Euler was born in Basel, Switzerland.
He entered the University of Basel at the age of 14 and completed a
Master's degree in philosophy three years later.
After receiving his degree, he stayed at the University and studied
mathematics for an additional three years.
For the next 14 years, from 1727 until 1741, Euler lived in St. Petersburg,
Russia, and taught mathematics and physics at the
St. Petersburg Academy of Sciences.
In 1741, at the age of 34, Euler moved to Berlin to teach at
the Academy of Science there.
He lived and worked in Berlin for the next 25 years when, in 1766 at the age of 59 he returned
to St. Petersburg and its Academy.

\itemitem{2.} In 1734, while living in St. Petersburg the first time,
Euler married.
Over the years, he and his wife had thirteen children, five of
whom survived childhood.

\itemitem{3.} In 1735 Euler nearly died from a fever.
In 1738 he lost sight in his right eye from what he believed
was due to eye strain from his work in cartography.
In 1766 he lost sight in his left eye as well and was
blind for the last 17 years of his life.
However, his work continued undiminished.

\item{\S2.} His work.

\itemitem{1.} Euler was the most prolific writer in the history of mathematics.
He is in the {\it Guinness Book of Records.}
All of his work combined is estimated to fill 60 to 80 volumes.
On average, he produced 800 pages of manuscript per year.

\itemitem{2.}
He founded graph theory with his solution in 1736 of the
seven bridges problem.
He also discovered theorems that would later become the basis
of the science of topology.

\itemitem{3.}
He developed many of the standard notations that are used in mathematics today,
such as $f(x)$, $e$, $i$, $\pi$ and $\Sigma$.

\itemitem{4.}
Euler advanced calculus by merging the work of Leibniz and Newton into
a coherent theory.
This work is the foundation of modern day analysis.
He also worked out power series expressions for the exponential,
sine, and cosine functions.

\itemitem{5.}
Euler's most popular work was also the least technical.
His {\it Letters of Euler on Different Subjects in Natural
Philosophy Addressed to a German Princess} was very popular when
it was published.
It explained many scientific ideas in layman's terms.

\itemitem{6.}
Euler was also a physicist.
He won the Paris Academy Prize Problem a dozen times.
Many of the problems that he won were astrophysics problems.
His doctoral dissertation dealt with the physics of sound.
Throughout his career he also made contributions to the theory of light.

\item{\S3.} References

\itemitem{1.}
http://www-history.mcs.st-andrews.ac.uk/Biographies/Euler.html

\itemitem{2.}
http://en.wikipedia.org/wiki/Euler

\itemitem{3.}
{\it A History of Mathematics} by Carl B. Boyer.

\beginsection Lecture 11, Exercise 2

Give two different definitions of the Gamma function (denoted by $\Gamma(x)$).
$$\Gamma(x)=\int_0^\infty t^{x-1}e^{-t}\,dt$$
$$\Gamma(x)=\int_0^1\left(\ln{1\over t}\right)^{x-1}\,dt$$

\beginsection Lecture 11, Exercise 3

State two properties satisfied by the Gamma function $\Gamma(x)$.
$$\Gamma(1+x)=x\Gamma(x)$$
$$\Gamma(1/2)=\sqrt\pi$$

\beginsection Lecture 11, Exercise 4

What is the Konigsberg bridge problem?
What is Euler's solution to this problem?

\bigskip
In the city of Konigsberg, Prussia, there were two islands connected
to each other and the mainland by seven bridges.
Does a path exist that traverses each
bridge just once and returns to the starting point?
Euler reduced the problem to a network of vertices that represent the land
and arcs that connect to the vertices and
represent the bridges.
Then he defined a property of the vertices, odd or even,
depending on the number of arcs connected to it.
Then he proved that if there were more than two
odd vertices, then a path did not exist.
It turned out that for Konigsberg there were more
than two odd vertices so a path did not exist.

\beginsection Lecture 11, Exercise 7

Prove (1).
$$e^{i\theta}=\cos\theta+i\sin\theta\eqno(1)$$

\bigskip
This proof is from Wikipedia, {\tt http://en.wikipedia.org/wiki/Euler's\_formula}.
Define the function $f(x)$ as
$$f(x)={\cos x+i\sin x\over e^{ix}}=(\cos x)(e^{-ix})+i(\sin x)(e^{-ix})$$
Next, take the derivative of $f$.
$$\eqalign{
df/dx&=d(\cos x)(e^{-ix})+(\cos x)d(e^{-ix})+id(\sin x)(e^{-ix})+i(\sin x)d(e^{-ix})\cr
&=(-\sin x)(e^{-ix})-i(\cos x)(e^{-ix})+i(\cos x)(e^{-ix})+(\sin x)(e^{-ix})\cr
&=0\cr
}$$
Since $df/dx=0$, the function $f(x)$ is constant.
Hence
$$f(x)=f(0)=1$$
Therefore
$$1={\cos x+i\sin x\over e^{ix}},\qquad
e^{ix}=\cos x+i\sin x$$

\beginsection Lecture 11, Exercise 8(a)

Let
$$a=x^2/\pi^2,\qquad b=x^2/4\pi^2,\qquad c=x^2/9\pi^2,\qquad d=x^2/16\pi^2$$
Then for $Bx^4$ we have that $B$ is the sum of all the
combinations of $a$, $b$, $c$, $d$ taken two at a time.
$$B=ab+ac+ad+bc+bd+cd={1\over\pi^4}\left(
{1\over4}+
{1\over9}+
{1\over16}+
{1\over36}+
{1\over64}+
{1\over144}
\right)$$

\beginsection Lecture 11, Exercise 8(b)

We have
$$A=-\left({1\over\pi^2}+{1\over4\pi^2}+{1\over9\pi^2}+{1\over16\pi^2}\right)$$
$$\eqalign{
A^2&={1\over\pi^4}\left(
1+
{1\over4}+
{1\over9}+
{1\over16}+
{1\over4}+
{1\over16}+
{1\over36}+
{1\over64}+
{1\over9}+
{1\over36}+
{1\over81}+
{1\over144}+
{1\over16}+
{1\over64}+
{1\over144}+
{1\over256}
\right)\cr
&={1\over\pi^4}\left(
1+
{2\over4}+
{2\over9}+
{3\over16}+
{2\over36}+
{2\over64}+
{1\over81}+
{2\over144}+
{1\over256}
\right)\cr
}$$
$$A^2-2B=
{1\over\pi^4}+
{1\over16\pi^4}+
{1\over81\pi^4}+
{1\over256\pi^4}
$$

\beginsection Lecture 11, Exercise 8(c)

From (5) we have
$$\eqalign{
A&=-{1\over6}\cr
B&={1\over120}\cr
}$$
Hence
$$A^2-2B={1\over36}-{1\over60}={1\over90}$$
By analogy with the result of 8(b) above we have
$${1\over90}=
{1\over\pi^4}+
{1\over16\pi^4}+
{1\over81\pi^4}+
{1\over256\pi^4}+\cdots
$$
Hence
$${\pi^4\over90}=
1+
{1\over2^4}+
{1\over3^4}+
{1\over4^4}+\cdots=\sum_{n=1}^\infty{1\over n^4}
$$



\end
