\section{Introduction}
The following is an excerpt from Vladimir Nabokov's
autobiography {\it Speak, Memory.}
\begin{quote}
A foolish tutor had explained logarithms to me much too early, and I had
read (in a British publication, the {\it Boy's Own Paper}, I believe)
about a certain Hindu calculator who in exactly two seconds could find the
seventeenth root of, say,
3529471145 760275132301897342055866171392
(I am not sure I have got this right; anyway the root was 212).
\end{quote}
We can check Nabokov's arithmetic by typing the following into Eigenmath.

\medskip
\verb$212^17$

\medskip
\noindent
After pressing the return key, Eigenmath displays the following result.
$$3529471145760275132301897342055866171392$$
So Nabokov did get it right after all.
We can enter {\it float} or click on the float button to scale the number
down to size.

\medskip
\verb$float$
$$3.52947\times10^{39}$$

\medskip
\noindent
Now let us see if Eigenmath can find the
seventeenth root of this number, like the Hindu calculator could.

\medskip
\verb$N=212^17$

\verb$N$
$$N=3529471145760275132301897342055866171392$$

\verb$N^(1/17)$
$$212$$

\medskip
\noindent
It is worth mentioning that when a symbol is assigned a value,
no result is printed.
To see the value of a symbol, just evaluate it by putting it on a line by
itself.

\medskip
\verb$N$
$$N=3529471145760275132301897342055866171392$$

\newpage

\subsection{Negative exponents}
Eigenmath requires parentheses around negative exponents.
For example,

\medskip
\verb$10^(-3)$

\medskip
\noindent
instead of

\medskip
\verb$10^-3$

\medskip
\noindent
The reason for this is that the binding of the negative sign is not always
obvious.
For example, consider

\medskip
\verb$x^-1/2$

\medskip
\noindent
It is not clear whether the exponent should be $-1$ or $-1/2$.
So Eigenmath requires

\medskip
\verb$x^(-1/2)$

\medskip
\noindent
which is unambiguous.

\medskip
\noindent
Now a new question arises.
Never mind the minus sign, what is the binding of the caret symbol itself?
The answer is, it binds to the first symbol that follows it and nothing else.
For example, the following is parsed as $(x^1)/2$.

\medskip
\verb$x^1/2$

$$\hbox{$1\over2$}x$$

\medskip
\noindent
So in general, parentheses are needed when the exponent is an expression.

\medskip
\verb$x^(1/2)$

$$x^{1/2}$$

